% Look!  An Appendix!

Appendices are a good idea for almost any thesis.  Your main thesis body will likely contain perhaps 40-60 pages of text and figures.  You may well write a larger document than this, but chances are that some of the information contained therein, while important, does \emph{not} merit a place in the main body of the document.  This sort of content - peripheral clarifying details, computer code, information of use to future students but not critical to understanding your work \ldots - should be allocated to one or several appendices.  


\section{About the bibliography}
What follows this is the bibliography.  This has its own separate environment and syntax; check out the comments in the .tex files for details.  Worth nothing, though, is that you may find it helpful to use automated bibliography management tools.  BibTeX will automatically generate a bibliography from you if you create a database of references.  Other software - for example JabRef on a pc - can be used to make managing the reference database easy.  Regardless, once you've created a .bib file you can cite it in the body of your thesis using the \texttt{\textbackslash cite} tag.  For example, one might wish to cite a reference by Bermudez \cite{Bermudez}.  If you use BibTeX, you can put the relevant information into a referencedatabase (called bibliography.bib here), and then BibTeX will compile the references into a .bbl file ordered appropriately for your thesis based on when the citations appear in the main document.