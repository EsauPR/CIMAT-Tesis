\section{De RNN's a Transformers}

Las \textbf{Recurrent Neural Networks, RNN} (Redes Neuronales Recurrentes) basadas en el
trabajo de \citeauthor{Rumelhart} datan del año 1986. Este tipo de redes están especializadas
en el procesamiento de datos que contienen información temporal, mejorando los resultados obtenidos
por otros tipos de redes como \textit{FeedForward Neural Networks} (Redes neuronales de propagación
hacia adelante) o \textit{Convolutional Neural Networks} (Redes neuronales convolucionales).

La idea principal detrás de estos modelos de redes es el concepto de \textit{parámetros compartidos}
encontrado en la literatura de idioma ingles como \textit{Parameter Sharing}.
Usando parámetros compartidos, un modelo puede generalizar mejor cuando la información
está contenida en diferentes partes de una secuencia. Así, el modelo no necesita aprender
independientemente todas las reglas que forman las secuencias, sino que ahora, la salida para cada
elemento perteneciente a un tiempo $t$ está determinada por la salida del elemento anterior $t-1$.
Esto resulta en una recurrencia con las mismas reglas de actualización aplicadas a cada elemento en el tiempo.
La ecuación \ref{eq:rnnh} representa este proceso; $h^{(t)}$ es el estado de la recurrencia definida
por una función $f$ sobre un elemento $x^{(t)}$ de la secuencia $X$ en el tiempo $t$ y $\theta$ son
los parámetros compartidos.

\begin{equation}
    h^{(t)} = f(x^{(t)}, h^{(t-1)}; \theta)
    \label{eq:rnnh}
\end{equation}

En una \textit{RNN} vista como un \textit{grafo computacional dirigido y acíclico}, cada nodo
representa un estado en la recurrencia y procesa la información de la secuencia $X$ con los mismos
parámetros $\theta$ en cada paso. Observe la figura \ref{fig:rnn_cg}.

\begin{figure}[!ht]
\centering
\includegraphics[width=.8\textwidth]{Chapters/2. Transformer/Figures/rnn/rnn_cgraph.png}
\caption{Grafo computacional generado por una \textit{RNN} al
        \quotes{desenrollar} la recurrencia. Usando los parámetros compartidos en cada nodo
        y con cada elemento $x^{(t)}$ de la secuencia genera un nuevo estado oculto $h^{(t)}$
        para retroalimentar nuevamente la entrada del siguiente nodo.}
\label{fig:rnn_cg}
\end{figure}

% Redes Neuronales Recurrentes más comunes
\subsection{Redes Neuronales Recurrentes más comunes}

Existen diversas formas como construir \textit{Redes Neuronales Recurrentes}, estas pueden producir una
salida en cada paso de tiempo o tener solo una al final de la recurrencia  o tener
conexiones entre unidades ocultas. La manera más común de implementar una \textit{RNN} está ilustrada en la
figura \ref{fig:rnn_cfga}. En esta figura, cada etapa de la recurrencia es retroalimentada por la
activación del estado oculto previo. Así, $h^{(t)}$ contiene información codificada de elementos
previos de la secuencia que puede ser usada en el futuro para obtener una salida $O^{(t+1)}$. En la
figura \ref{fig:rnn_cfgb} se
cambia la retroalimentación de $h^{(t)}$ por $o^{(t)}$. Nótese que en este caso, la red es entrenada
para obtener un valor en específico $o^{(t)}$ lo que provocaría que gran parte de la información de
los estados ocultos pasados $h^{(t-1)}, h^{(t-2)}, ...$ no se transmita.
La diferencia entre los dos esquemas anteriores es que
la red \ref{fig:rnn_cfga} es entrenada para decidir que información debe transmitir en el futuro a través
de los estados ocultos, en cambio, en la figura \ref{fig:rnn_cfgb} cada estado esta conectado con el
pasado a través de la predicción del paso anterior, perdiendo así gran parte de la información
codificada en cada estado oculto $h^{(t)}$. Este no sería un problema si la salida $O^{(t-1)}$ fuese
lo suficientemente enriquecedora y en altas dimensiones.

% \begin{figure}[!ht]
% \centering
% \includegraphics[width=.8\textwidth]{Chapters/2. Transformer/Figures/rnn/rnn_cfg.png}
% \caption[RNN - CFG]{Distintos tipos de RNNs. \textbf{a)} Las activaciones de las capas ocultas $h^{(t)}$
% alimentan al nodo siguiente, cada etapa de la recurrencia genera una salida $o^{(t)}$ \textbf{b)}
% Cada nodo es alimentado por las salidas de cada estado oculto $o^{(t)}$. \textbf{c)} Al final de la
% recurrencia solo tiene una salida $o^{(T)}$, puede ser usada para resumir/predecir un valor de una
% secuencia. \textbf{d)} Teacher Forcing. En modo de entrenamiento cada nodo en el tiempo $t$ es
% retroalimentado por la salida correcta $y^{(t-1)}$, en modo evaluación es retroalimentado por las
% salidas del modelo $O^{(t-1)}$}
% \label{fig:rnn_cfg}
% \end{figure}

\begin{figure}[!ht]
    \centering
    \begin{subfigure}[b]{0.49\textwidth}
        \centering
        \includegraphics[width=\textwidth]{Chapters/2. Transformer/Figures/rnn/rnn_cfga.png}
        \caption{Las activaciones de las capas ocultas $h^{(t)}$
        alimentan al nodo siguiente, cada etapa de la recurrencia genera una salida $o^{(t)}$.}
        \label{fig:rnn_cfga}
    \end{subfigure}
    \hfill
    \begin{subfigure}[b]{0.4\textwidth}
        \centering
        \includegraphics[width=\textwidth]{Chapters/2. Transformer/Figures/rnn/rnn_cfgb.png}
        \caption{Cada nodo es alimentado por las salidas de cada estado oculto $o^{(t)}$.}
        \label{fig:rnn_cfgb}
    \end{subfigure}

    \begin{subfigure}[b]{0.4\textwidth}
        \centering
        \includegraphics[height=0.63\textwidth]{Chapters/2. Transformer/Figures/rnn/rnn_cfgc.png}
        \caption{Al final de la recurrencia solo tiene una salida $o^{(T)}$, puede ser usada para
        resumir/predecir un valor de una secuencia.}
        \label{fig:rnn_cfgc}
    \end{subfigure}
    \hfill
    \begin{subfigure}[b]{0.49\textwidth}
        \centering
        \includegraphics[height=0.6\textwidth]{Chapters/2. Transformer/Figures/rnn/rnn_cfgd.png}
        \caption{Teacher Forcing. En modo de entrenamiento cada nodo en el tiempo $t$ es
        retroalimentado por la salida correcta $y^{(t-1)}$, en modo evaluación es retroalimentado por las
        salidas del modelo $O^{(t-1)}$.}
        \label{fig:rnn_cfgd}
    \end{subfigure}

    \caption[RNN - CFG]{Distintos tipos de RNNs.}
    \label{fig:three graphs}
\end{figure}


Por otro lado, la \textit{RNN} representada en la figura \ref{fig:rnn_cfgc} tiene una sola salida al
final de la recurrencia.
Al contrario de las anteriores, este tipo de redes pueden ser usadas para resumir
información contenida en la secuencia para finalmente predecir un único valor final.
El \textit{Análisis de Sentimiento} en textos es una tarea común que puede ser representada con este esquema
de red. En la figura \ref{fig:rnn_cfgd} vemos un modelo de \textit{RNN} entrenado mediante el proceso de
Teacher Forcing; durante el entrenamiento la red es retroalimentada con las salidas
esperadas del modelo $y^{(t)}$ en el tiempo $t+1$. La ventaja de esta red es que al ser eliminadas
las conexiones entre estados ocultos, las funciones de pérdida basadas en comparar la predicción en
el tiempo $t$ con el valor objetivo $y^{(t)}$ pueden ser desacopladas. Por tanto, el entrenamiento
puede ser paralelizado al calcular el gradiente para cada tiempo $t$ por separado, puesto que ya
tenemos el valor ideal para esta salida.

Finalmente, en la figura \ref{fig:rnn_cfge} la \textit{Red Neuronal Recurrente} es modificada para esta vez
no procesar una secuencia, sino, procesar un solo vector en cada paso. El estado oculto
previo $h^{(t-1)}$ retroalimenta al siguiente paso $t$ así como la predicción esperada $y^{(t)}$, que
a su vez, es usada para calcular la función de costo del paso anterior $L^{(t-1)}$. Esta estructura de
red puede ser implementada en tareas como \textit{Image Captioning}, en donde la entrada es una imagen y la salida
una secuencia de palabras que describen esta misma.

\begin{figure}[!ht]
\centering
\includegraphics[width=.4\textwidth]{Chapters/2. Transformer/Figures/rnn/rnn_cfge.png}
\caption[RNN - Image Captioning]{Modelo usado para tareas de \textit{Image Captioning}, la entrada es una
sola imagen y la red predice una secuencia de palabras que describen dicha imagen La salida esperada
$y^{(t)}$ sirve como objetivo para la función de costo del paso anterior y como entrada en cada paso.}
\label{fig:rnn_cfge}
\end{figure}


Los modelos ejemplificados anteriormente son construidos de forma \textit{causal}, es
decir, la secuencia es procesada en un solo sentido en donde la información pasada es transmitida
hacia estados futuros. Sin embargo, este flujo de información puede ser insuficiente para resolver
todas las tareas. En \textit{Modelo de Lenguaje} se aprende la estructura estadística del lenguaje con el
que fue entrenado y su meta es predecir la siguiente palabra, n-grama o letra dado un contexto antes
visto. En otros términos, dada una secuencia de texto de longitud
$T$ $x^{(1)}, x^{(2)}, ..., x^{(T)}$
con $x \in \mathcal{R}^{1 \times d}$ donde $d$ es la dimensión de la codificación de las palabras,
la meta es predecir la probabilidad conjunta de la secuencia:

\begin{equation}
    P(x^{(1)}, x^{(2)}, ..., x^{(T)}) = \prod_{t=1}^{T} P(x^{(t)} | x^{(t)}, \dots , x^{(t-1)})
\end{equation}

Con ello, un modelo de lenguaje basado en \textit{Redes Neuronales Recurrentes} es capaz de predecir
un siguiente elemento $\hat x^{(t)}$ simplemente obteniéndolo de la secuencia mediante:

\begin{equation}
    \hat x^{(t)} \approx P(x^{(t)} | x^{(t-1)}, \dots, x^{(1)}) \approx P(x^{(t)} | h^{(t-1)})
\end{equation}

donde $h^{(t-1)}$ es el estado oculto que almacena la información pasada hasta el tiempo $t$
tal y como se definió en \ref{eq:rnnh}.

Sin embargo, la información previa de la secuencia codificada en $h^{(t)}$ no siempre contiene los
elementos necesarios para que el modelo pueda predecir correctamente el siguiente elemento,
observe la siguiente oración:

\begin{center}
    % \tiny
    << \textit{Ella estaba muy \_\_\_\_\_, después de que Alejandra vió el amanecer en la playa } >>
\end{center}

En la oración anterior, el espacio en blanco puede ser completado con algún adjetivo calificativo;
\textit{contenta}, \textit{enojada}, \textit{maravillada}, etc. Gracias a la información provista por
la parte final de la oración, podemos deducir que de las 3 opciones la menos probable de
elegir es \textit{enojada}. Es decir, usamos información del futuro que no pudo haber sido vista por
una red (que procesa la información en forma causal) para tomar la mejor elección. Una ligera
modificación fácilmente aplicable a estos modelos es que las secuencias sean procesadas
en ambas direcciones, las \textbf{Redes Neuronales Recurrentes Bidireccionales}
\cite{Schuster}.

Una \textit{RNN Bidireccional} procesa la secuencia en ambos sentidos (una \textit{RNN} en un
sentido y otra en el otro), capturando información del
pasado en el estado oculto $\overrightarrow{h}^{(t)}$  cuando la recurrencia es del inicio al final
de la secuencia e información del futuro en $\overleftarrow{h}  ^{(t)}$ cuando la recurrencia es del
final al inicio de la secuencia. Finalmente, el estado oculto $h^{(t)}$ es una concatenación de ambos
estados $\overrightarrow{h}^{(t)}$ y $\overleftarrow{h}^{(t)}$, vea la ecuación \ref{eq:rrnbi}.
Por lo cual, la salida $o^{(t)}$ ahora puede ser calculada con información tanto del futuro como del pasado
\ref{eq:rrn_out}.

\begin{equation}
\begin{split}
    \overrightarrow{h}^{(t)} = f(x^{(t)}, \overrightarrow{h}^{(t-1)}; \theta_f)\\
    \overleftarrow{h}^{(t)} = f(x^{(t)}, \overleftarrow{h}^{(t+1)}; \theta_b)\\
    h^{(t)} = Concat(\overrightarrow{h}^{(t)}, \overleftarrow{h}^{(t)})
\end{split}
\label{eq:rrnbi}
\end{equation}

\begin{equation}
        o^{(t)} = g(h^{(t)}; \theta_{out})
\label{eq:rrn_out}
\end{equation}


% Compuertas LSTM y GRU
\subsection{Compuertas LSTM y GRU}

Hasta el momento, se ha hecho mención de las salidas $o^{(t)}$ y estados ocultos $h^{(t)}$ solo como
el resultado de operaciones aplicadas por dos funciones; $g$ y $f$ respectivamente. Existen varias
alternativas de construir una \textit{RNN}, una de las maneras más comunes es usando
\ref{eq:rnn_Ht} y \ref{eq:rnn_Ot}:

\begin{equation}
    h^{(t)} = \phi(x^{(t)} W_{x} + h^{(t-1)} W_h + b)
    \label{eq:rnn_Ht}
\end{equation}

\begin{equation}
    o^{(t)} = x^{(t)} W_{out} + b
    \label{eq:rnn_Ot}
\end{equation}

Los parámetros compartidos de la red ahora son descritos por las matrices
$W_x \in \mathbb{R}^{d \times k}$, $W_h \in \mathbb{R}^{k \times k}$ y $
W_{out} \in \mathbb{R}^{k \times q}$ con $k$ como la dimensión del estado oculto, $q$ la dimensión
de las salidas $o^{(t)}$, $b \in \mathbb{R} ^ {1 \times q}$ el parámetro de sesgo y $\phi$ es la
función de activación. De esta manera, los pesos de los parámetros aprendidos en la matriz $W_h$
determinan cómo será usada la información del pasado, codificada en $h^{(t-1)}$. Posteriormente,
es incluida a la codificación de la información del tiempo actual $t$ calculada por $W_x$. La figura
\ref{fig:rnn_cell} representa gráficamente la lógica usada para calcular los estados ocultos y las
salidas de la red.


\begin{figure}[ht!]
\centering
\includegraphics[width=0.4 \textwidth]{Chapters/2. Transformer/Figures/rnn/rnn_cell.jpg}
\caption{Cómputo del estado oculto y salida de una Red Neuronal Recurrente.}
\label{fig:rnn_cell}
\end{figure}

Sin embargo el cálculo de los estados ocultos mediante \ref{eq:rnn_Ht} presenta algunos problemas.
La interacción entre la información del pasado y la actual siempre es \quotes{plana}, es decir, la información
fluye a través del tiempo de la misma manera sin forma de dar prioridad o ignorar parte de
esta. Por lo que resulta una tarea un poco más complicada preservar información relevante a en cada paso
o desechar información que ya no es util para la red. También, causado por este mismo flujo de los
datos, la información del pasado poco a poco es opacada por nueva información, impidiendo que se
puedan encontrar dependencias de información en secuencias largas en tiempos distantes;
comúnmente se hace referencia a este problema como \textit{The Short-term Memory Problem} en inglés
\cite{VanishinGradient2}, aunado a problemas como el
\textit{Desvanecimiento o Explosión del Gradiente} \cite{VanishinGradient} \cite{pmlr-v28-pascanu13},
y acentuándose aun más
debido a las matrices de pesos compartidos en la recurrencia. Dichas
multiplicaciones de las matrices en la recurrencia tienen similitud a las realizadas en el
\textit{método de potencia}, en donde cualquier
componente en la matriz inicial que no esté alineada con el vector propio asociado al mayor valor
propio son eventualmente descartados \cite[pp.~390-392]{GoodBengCour16}). Por ende, los resultados
de este producto tendrán a ser cercanos a cero (desvanecerse) o explotar dependiendo de la magnitud
de la matriz de pesos.

Una manera de solventar los problemas anteriores son las \textbf{Redes Neuronales con Compuertas},
creadas con la idea de crear conexiones a través del tiempo de tal manera de tener gradientes que no
se desvanezcan o exploten, convirtiéndose además en un mecanismo para olvidar información pasada y
decidiendo automáticamente cuándo y cuánto de la información debe prevalecer.

\subsubsection{LSTM}

\textbf{Long Short-Term Memory}, \textbf{LSTM} por sus siglas en inglés, fue propuesta en 1997 por
\citeauthor{LSTM}, como un método de preservar dependencias de información relevante
distantes a corto plazo. Las \textit{LSTM} introducen un nuevo componente la
\textit{Celda de Memoria} cuya función es guardar información a través del tiempo y es
controlada por distintas compuertas. Las compuertas aprenden a distinguir que información es relevante y
cual no. Hay 3 de ellas, la \textit{Compuerta de Entrada}, la \textit{Compuerta de Olvido}
y la \textit{Compuerta de Salida}. La \textit{Compuerta de Entrada} $I^{(t)}$ (véase la figura
\ref{fig:rnn_lstm}) determina cuanta información actual debe ser contemplada a través de la
\textit{Memoria Candidata} $\tilde C^{(t)}$ para actualizar la \textit{Celda de Memoria} $C^{(t)}$.
La \textit{Compuerta de Olvido} $F^{(t)}$ indica qué información del pasado debe ser desechada de la
\textit{Celda de Memoria} $C^{(t-1)}$ y la \textit{Compuerta de Salida} ayuda a determinar el nuevo
estado $h^{(t)}$ via la \textit{Celda de Memoria} actual $C^{(t)}$.

\begin{figure}[ht!]
\centering
\includegraphics[width=1.0 \textwidth]{Chapters/2. Transformer/Figures/rnn/lstm.jpg}
\caption{LSTM. La \textit{Compuerta de Entrada} $I^{(t)}$ determina cuánta información actual debe
         ser considerada a través de la \textit{Memoria Candidata} $\tilde{C}^{(t)}$ para actualizar
         la \textit{Celda de Memoria} $C^{(t)}$. La \textit{Compuerta de Olvido} $F^{(t)}$ indica
         qué información del pasado debe ser desechada de la \textit{Celda de Memoria} $C^{(t-1)}$,
         y la \textit{Compuerta de Salida} $O^{(t)}$ ayuda a determinar el nuevo estado $h^{(t)}$ a
         través de la \textit{Celda de Memoria} actual $C^{(t)}$.}
\label{fig:rnn_lstm}
\end{figure}

Las ecuaciones \ref{eq:comp} rigen el comportamiento de Compuertas de Entrada, Salida y Olvido,

\begin{equation}
    \begin{split}
        I^{(t)} =  \sigma(x^{(t)} W_{xi} + h^{(t-1)} W_{hi} + b_i)\\
        F^{(t)} =  \sigma(x^{(t)} W_{xf} + h^{(t-1)} W_{hf} + b_f)\\
        O^{(t)} =  \sigma(x^{(t)} W_{xo} + h^{(t-1)} W_{ho} + b_o)\\
    \end{split}
    \label{eq:comp}
\end{equation}

\noindent donde $W_{xi}, W_{xf}, W_{xo} \in \mathbb{R}^{d \times k}$,
$W_{hi}, W_{hf}, W_{ho} \in \mathbb{R}^{k \times k}$ y $b_i, b_f, b_o \in \mathbb{R}^{1xk}$

La Memoria Candidata y la Celda de memoria son actualizadas mediante:

\begin{equation}
    \begin{split}
        \tilde C^{(t)} =  \tanh(x^{(t)} W_{xi} + h^{(t)} W_{hc} + b_c)\\
        C^{(t)} =  F^{(t)} \odot C^{(t-1)} + I^{(t)} \odot \tilde C^{(t)} \\
    \end{split}
\end{equation}

\noindent donde $W_{xi} \in \mathbb{R}^{d \times k}$,
$W_{hc} \in \mathbb{R}^{k \times k}$ y $b_c \in \mathbb{R}^{1xk}$

\noindent y finalmente el estado oculto $h^{(t)}$ esta dado por:

\begin{equation}
    \begin{split}
        h^{(t)} =  O^{(t)} \odot \tanh(C^{(t)}) \\
    \end{split}
\end{equation}

\noindent $\sigma$ y $\odot$ denotan la función de activación sigmoide y la multiplicación uno a uno
respectivamente.


\subsubsection{GRU}

\textbf{Gated Recurrent Units} o \textbf{GRU} por sus siglas en inglés, fueron propuestas en 2014
\cite{GRU1} como una alternativa computacionalmente más rápida y con similar rendimiento que las
\textit{LSTM} \cite{GRU2}. A diferencia de anteriores, las \textit{GRU} prescinden de la
\textit{Celda de Memoria} y utilizan un par de compuertas (la \textit{Compuerta de Actualización} y
la de \textit{Olvido}) para decidir que información aún es necesaria que esté codificada dentro del
estado oculto, véase la ecuación \ref{eq:gru_gates}.
La \textit{Compuerta de Olvido} permite decidir que del pasado aún debe ser transmitido a futuros
estados o de otro modo ser desechada. La \textit{Compuerta de Actualización} indica que información nueva es relevante y
necesita ser incorporada al no estar codificada dentro del estado oculto,
véase la ecuación \ref{eq:gru_out}.

\begin{equation}
    \begin{split}
        R^{(t)} = \sigma(x^{(t)} W_{xR} + h^{(t-1)} W_{hR} + b_R)\\
        Z^{(t)} = \sigma(x^{(t)} W_{xZ} + h^{(t-1)} W_{hZ} + b_Z)
    \end{split}
    \label{eq:gru_gates}
\end{equation}

\begin{equation}
    \begin{split}
        \tilde h^{(t)} = \tanh(x^{(t)} W_{xh} + ( R^{(t)} \odot h^{(t-1)}) W_{hh} + b_h)\\
        h^{(t)} =  Z^{(t)} \odot h^{(t-1)} + (1 - Z^{(t)}) \odot \tilde h^{(t-1)} \\
    \end{split}
    \label{eq:gru_out}
\end{equation}

\begin{figure}[ht!]
\centering
\includegraphics[width=1.0 \textwidth]{Chapters/2. Transformer/Figures/rnn/GRU.jpg}
\caption{GRU. A diferencia de las \textit{LSTM}, las \textit{GRU} prescinden de la
         \textit{Celda de Memoria} y utilizan un par de compuertas (la \textit{Compuerta de
         Actualización} y la \textit{Compuerta de Olvido}) para decidir qué información es necesaria
         que esté codificada dentro del estado oculto.}
\label{fig:rnn_gru}
\end{figure}


% Modelos de atención
\subsection{Mecanismos de Atención} \label{section:att}

Una de las arquitecturas comunes vistas previamente es la mostrada en la figura \ref{fig:rnn_cfgc} cuya información
procesada es resumida en una sola salida. Este tipo de red es usada como parte de las soluciones en
tareas de reconocimiento de voz (\textit{Speech Recognition}), traducción de lenguaje
(\textit{Machine Translation}) o asistencia en respuestas automáticas (\textit{Question Answering}), entre
otros,
típicamente bajo modelos Secuencia a Secuencia (\textit{Sequence to Sequence, Seq2Seq})
\cite{DBLP:journals/corr/ChoMGBSB14}. Los modelos
\textit{Seq2Seq} están formados por dos redes neuronales como la mostrada en \ref{fig:seq2seq}. La
primera se comporta como un \textit{codificador} al resumir la entrada y producir un vector de salida
de tamaño fijo llamado \textit{vector de contexto}. La segunda red se comporta como un
\textit{decodificador}, este es inicializado y condicionado con el
\textit{vector de contexto} para obtener una transformación de la entrada no necesariamente del
mismo tamaño de secuencia, debido a que en tareas como traducir una oración de un lenguaje a otro
donde la traducción no siempre contiene las misma cantidad de palabras usadas en el idioma original.

\begin{figure}[ht!]
    \centering
    \includegraphics[width=1.0 \textwidth]{Chapters/2. Transformer/Figures/rnn/seq2seq.jpg}
    \caption{Descripción.}
    \label{fig:seq2seq}
\end{figure}

Por ejemplo, en tareas de \textit{Machine Translation} el \textit{codificador} esta formado por una
\textit{RNN} Bidireccional que lee y procesa un conjunto de
vectores $X = (x^{(1)}, x^{(2)}, \dots, x^{(T_x)})$ para obtener un vector de contexto $C$. La forma
más común es como en \ref{eq:s2s_simple}:

\begin{equation}
    \begin{split}
        h^{(t)} = f_{bi}(x^{(t)}, h^{(t-1)}; \theta_{f}, \theta_{b}) \\
        C = q({h^{(1)}, h^{(2)}, \dots, h^{(T)}})
    \end{split}
    \label{eq:s2s_simple}
\end{equation}

Recordemos que $h^{(t)}$ es el estado oculto generado por la concatenación de los dos estados ocultos
generados por la \textit{RNN Bidireccional}, $f_{bi}$ y $q$ son funciones no lineales, ya sea,
una \textit{LSTM} para $f_{bi}$ y $q({h^{(1)}, h^{(2)}, \dots, h^{(T)}}) = h^{(T)}$, equivalente a
tomar solo el ultimo estado oculto como vector de contexto $C$. El \textit{decodificador} es entrenado
para predecir la siguiente palabra $y^{(t')}$ dado el vector de contexto $C$ y todas las palabras
previas predichas. En otras palabras, el decodificador define la probabilidad conjunta modelada por una
\textit{RNN}:

\begin{equation}
    p(Y) = \prod_{t=1}^{T_y} p(y^{(t)} | \{y^{(1)}, \dots , y^{(t-1)}\}, C)
\end{equation}
\begin{equation}
    p(y^{(t)} | \{y^{(1)}, \dots , y^{(t-1)}\}, C) = g(y^{(t-1)}, s^{(t)}, C; \theta_g)
\end{equation}

donde $g$ es una función no lineal que emite la probabilidad de $y^{(t)}$ y $s^{(t)}$ es el estado oculto
del \textit{decodificador} \ref{eq:sqs_s}.

\begin{equation}
    s^{(t)} = f(s^{(t)}, y^{(t-1)}, C; \theta_s)
    \label{eq:sqs_s}
\end{equation}


Sin embargo, cuando las secuencias son bastante largas el \textit{vector de contexto} emitido por el
\textit{codificador} no es lo suficientemente grande como para resumir correctamente la secuencia y
por tanto, la información inicial de la entrada es olvidada, teniendo escasa presencia en estados
ocultos más lejanos. En 2015 \citeauthor{bahdanau2016neural} observaron estos efectos y
propusieron una forma de minimizarlos, los \textbf{Mecanismos de Atención}.

La función principal de los \textbf{Mecanismos de Atención} es permitir que el \textit{decodificador}
pueda acceder al historial completo de los estados ocultos del \textit{codificador}, así, ahora podrá
contar con un mecanismo
para selectivamente centrarse en las distintas partes de la secuencia que tienen mayor influencia sobre
una la salida esperada a cierto tiempo.

Por tanto, las palabras predichas no son calculadas por un único \textit{vector de contexto} generado por
el \textit{codificador}, sino que para cada objetivo $y^{(t)}$ se calcula un nuevo \textit{vector de contexto} $c^{(t)}$:

\begin{equation}
    p(y^{(t)} | \{y^{(1)}, \dots , y^{(t-1)}\}, c^{(t)}) = g(y^{(t-1)}, s^{(t)}, c^{(t)}; \theta_g)
\end{equation}

\begin{equation}
    s^{(t)} = f(s^{(t)}, y^{(t-1)}, c^{(t)}; \theta_s)
\end{equation}

Dado que cada estado oculto $h^{(t)}$ contiene mucho mejor la información que se encuentran alrededor
del t-ésimo término, se puede generar cada vector de contexto como una suma pesada de sobre los
estados ocultos del \textit{codificador}. Estos pesos nos ayudan a determinar que tan importante es la
información codificada por cada estado oculto y al momento de obtener la salida del t-ésimo valor
\quotes{prestar atención} a aquellos que son más relevantes para esta predicción:

\begin{equation}
    c^{(t)} = \sum_{i=1}^{T_x} \alpha_{t,i} h^{(i)}
\end{equation}

aquí cada peso $\alpha_{t,i}$ indica que tan bien se \quotes{alinean} los términos $y^{(t)}$ y $x^{(i)}$,
y son calculados por una \textit{función de alineamiento} que denota que tan importante es el estado
oculto del \textit{codificador} $h^{(t)}$ para el estado oculto del decodificador $s^{(i)}$.

\begin{equation}
    \alpha_{t,i} = align(y^{(t)}, x^{(i)}) = \frac{\exp(score(s^{(t-1)}, h^{(i)}))}{\sum_{k=1}^{T_x} \exp(score(s^{(t-1)}, h^{(k)}))}
    \label{eq:b_align}
\end{equation}

\textit{Bahdanau} propone aprender esta alineación usando una \textit{Red feed-forward} con una sola
capa oculta y la función $\tanh$ como activación:

\begin{equation}
    score(s^{(t)}, h^{(i)}) = v^\top_a \tanh(W_a[s^{(t)};h^{(i)}])
    \label{eq:concat}
\end{equation}

con $v_a$ y $W_a$ como matrices de pesos a aprender durante el entrenamiento, $[s^{(t)};h^{(i)}]$
representa una concatenación de los estados ocultos del \textit{codificador} y decodificador. En la figura
\ref{fig:att} podemos ver gráficamente el modelo usado por \textit{Bahdanau}.

\begin{figure}[ht!]
    \centering
    \includegraphics[width=0.8 \textwidth]{Chapters/2. Transformer/Figures/rnn/attention.png}
    \caption{Modelo seq2seq propuesto por \citeauthor{bahdanau2016neural} con \textit{Additive/Concat Attention}}
    \label{fig:att}
\end{figure}

Los modelos de atención pueden ser vistos de manera más general como un mapeo de una secuencia de
llaves $k$ hacia una distribución de atención $\alpha$ de acuerdo a una consulta $q$ aplicándose a un
conjunto de valores $V$ para selectivamente propagar la información contenida en $V$.
Si bien, los términos de consulta, llaves y valores (query, keys, values)
son en ámbitos de los \textit{Sistemas de Recuperación de Información} su relación en términos de
la atención aplicada por Bahdanau es muy similar; las llaves son los estados ocultos del \textit{codificador}
y la consulta es el estado oculto del decodificador en cuestión, en este caso el mapeo de entre llaves
y valores es la misma:

\begin{equation}
    A(q, K, V) = \sum_i p(a(K-i, q)) * v_i
    \label{eq:att_general}
\end{equation}

En la ecuación \ref{eq:att_general}, $p$ es una función de distribución que mapea los puntajes de la
función de alineación $a$ a pesos de atención. Comúnmente se usan las funciones \textit{softmax} o
\textit{logistic sigmoid} puesto que nos aseguran que los pesos de atención producidos estarán dentro
del rango $[0,1]$ y la suma de ellos es igual a $1$, por lo que los pesos pueden ser interpretados como
una probabilidad que indica que tan relevante es cierto elemento. Algunas variaciones en donde se
consideran solo los términos relevantes como \textit{sparsemax} \citeauthor{DBLP:journals/corr/MartinsA16}
o \textit{sparse entmax} \citeauthor{DBLP:journals/corr/abs-2006-07214} permiten trabajar y enfocarse
en solo algunas relaciones de alineamiento. \citeauthor{NEURIPS2019_16fc18d7} proponen una función de
distribución de pesos $M = \tanh(E) \odot sigmoid(N)$ con $E$ como una matrix en donde cada entrada
representa la similaridad entre estados ocultos y $N$ una medida negativa (disimilaridad), por lo que
podemos usar
$sigmoid(N)$ como información para \quotes{de-atender} los alineamientos de $E$.

Las funciones de alineamiento se encargan de comparar y extraer la relación entre las representaciones de las llaves (keys) y
consultas (queries), por ejemplo usando el producto punto y el coseno como función de similaridad.
Bahdanau calcula esta relación a través de una red neuronal \ref{eq:b_align}, lo que evita asumir
que ambas representaciones están en el mismo espacio, como lo hace las funciones de alineación
como el producto punto o la similaridad coseno. La tabla \ref{Tab:att} muestra una
recopilación de funciones de alineamiento.


\begin{table}[ht!]
\begin{center}
\resizebox{\textwidth}{!}{
\begin{tabular}{@{}lll@{}}
\toprule
\textbf{Nombre} & \textbf{Función de Alineación} & \textbf{Cita} \\
\midrule
Similarity / Content-Base & $a(k_i, q) = sim(k_i, q)$ & \citeauthor{DBLP:journals/corr/GravesWD14} \\ \\
Dot Product\textsuperscript{1} & $a(k_i, q) = q^\top k_i$ & \citeauthor{DBLP:journals/corr/LuongPM15} \\ \\
Scaled Dot Product & $a(k_i, q) = \frac{q^\top k_i}{\sqrt{d_k}}$ & \citeauthor{DBLP:journals/corr/VaswaniSPUJGKP17} \\ \\
General & $a(k_i, q) = q^\top W k_i$ & \citeauthor{DBLP:journals/corr/LuongPM15} \\ \\
Biased General & $a(k_i, q) = k_i (Wq + b )$ & \citeauthor{DBLP:journals/corr/SordoniBB16} \\ \\
Activated General & $a(k_i, q) = act(q^\top W k_i + b)$ & \citeauthor{DBLP:journals/corr/abs-1709-00893} \\ \\
Generalized Kernel & $a(k_i, q) = \phi(q)^\top \phi(k_i)$ & \citeauthor{DBLP:journals/corr/abs-2009-14794} \\ \\
Additive\textbackslash Concat \textsuperscript{2} & $a(k_i, q) = v^\top act(W[q;k_i]+ b)$ & \citeauthor{bahdanau2016neural}, \citeauthor{DBLP:journals/corr/LuongPM15} \\ \\
Deep & $a(k_i, q) = v^\top E^{(L-1)} + b^L$ & \citeauthor{Pavlopoulos} \\
    & $E(l) = act(W_l E^{(l-1)} + b^l)$ &  \\
    & $E(1) = act(W_0k_i + W_1q + b^l)$ &  \\ \\
Location-based & $a(k_i, q) = act(W q)$ & \citeauthor{DBLP:journals/corr/LuongPM15} \\ \\
Feature-based & $a(k_i, q) = v^\top act(W_0 \phi(K) + W_1 \phi(K) + b)$ & \citeauthor{DBLP:journals/corr/abs-1810-10126} \\
\bottomrule
\end{tabular}}
\end{center}
\caption{Distintos tipos de funciones de alineación. (Tabla basada en \cite{DBLP:journals/corr/abs-1904-02874} y \cite{weng2018attention}). \\
$a(k_i, q)$ representa la función de alineación entre $k_i$ y $q$ y $act$ es una función de activación. \\
$sim$ es una función de similaridad, \citeauthor{DBLP:journals/corr/GravesWD14} propone la función coseno.\\
Los parámetros $v, W, W_1, W_2$ son parámetros aprendidos por la red neuronal.\\
\textsuperscript{1} El factor de escala $\frac{1}{\sqrt{d_k}}$ ayuda a estabilizar cuando el
gradiente es muy pequeño. $d_k$ es el tamaño de la cabeza de atención.\\
\textsuperscript{2} La función de activación propuesta por \citeauthor{bahdanau2016neural} es la función $tanh$ como se ve en \ref{eq:concat} \\
\label{Tab:att}}
\end{table}

De acuerdo a como es aplicado los distintos tipos de atención \citeauthor{DBLP:journals/corr/abs-1904-02874}
los dividen en 4 grandes grupos; por número de secuencias, por nivel de abstracción, por número de
posiciones y por número de representaciones. Estos grupos no son mutualmente excluyentes por tanto
una aplicación de atención puede pertenecer a más de una.

En la categoría \textit{por número de secuencias} se identifican 3 tipos, el primero de ellos,
\textbf{Distintivos} (\textit{Distinctive}) es cuando la clave (key) y valor (value) pertenecen a
distintas secuencias de entrada y salida respectivamente, como es el caso del modelo propuesto por
\citeauthor{bahdanau2016neural}. El segundo tipo, \textbf{Co-Atención} (\textit{co-attention}) utiliza
distintos secuencias al mismo tiempo para conocer los pesos de atención entre estas entradas. Por ejemplo,
en tareas en donde se necesita trabajar con datos multi-modales como procesar imágenes y texto
simultáneamente. En tareas como \textit{Visual Question Answering}
se puede aplicar un mecanismo de atención conjunto tanto para las imágenes y el texto para identificar
las regiones de la imagen y los palabras del texto que son más relevantes. El
tercer tipo es \textbf{auto-atención} (\textit{Self Attention}), fue propuesto por \citeauthor{yang2016hierarchical} y es uno
de los puntos claves para los modelos \textit{Transformers} \cite{DBLP:journals/corr/VaswaniSPUJGKP17}.
Es comúnmente usada en tareas que solo requieren una salida resumen y no una secuencia como \textit{
Clasificación de texto}. La clave (key) y valor (value) pasan a ser las mismas y la atención es
calculada sobre los mismos elementos pertenecientes a la secuencia de entrada, buscando
así, encontrar las relaciones entre las palabras de la misma oración.

La segunda Categoría agrupa la atención por el nivel de abstracción en la que es aplicada, a un
\textbf{solo nivel} o en \textbf{múltiples niveles}. La información a procesar muchas veces puede ser
representada en distintos niveles de abstracción, es decir, en texto, podemos separar los datos a
nivel de letras, n-gramas, palabras, oraciones, párrafos, etc., por tanto, es posible atender
de manera jerárquica a las palabras que forman una oración para posteriormente prestar atención a
las oraciones que conformar un texto más largo. \citeauthor{yang2016hierarchical} utiliza este procedimiento
para generar un vector de características usado posteriormente en un etapa de clasificación.

En la tercer categoría la atención es realizada en diversas partes de la secuencia; la suma pesada sobre
todos los puntajes de las entradas usada por \citeauthor{bahdanau2016neural} se le denomina \textbf{
Atención suave} (\textit{Soft-Attention}). Una alternativa es la \textbf{Atención dura}
(\textit{Hard-Attention}) \cite{DBLP:journals/corr/XuBKCCSZB15} que calcula la atención no sobre todas
los puntajes de alineamiento sino en una parte de estos, para ello se usa una distribución multinoulli
parametrizada por los pesos de la atención. A pesar de que es más eficiente que la
\textit{atención suave} resulta difícil de entrenar al no ser completamente diferenciable. Otra opción
a la \textit{atención dura} es la \textbf{Atención Local} (\textit{Local Attention}) cuya idea es aplicar
atención sobre una ventana elegida ya sea centrada con respecto a la tiempo actual (alineamiento monotónico)
o predicha por una función (alineamiento predictivo). La \textit{atención local} fue propuesta por
\citeauthor{DBLP:journals/corr/LuongPM15} así como la \textit{Atención Global} la cual es similar a la
\textit{atención suave}.

La última categoría divide los modelos de atención por las formas de representación de las entradas
sobre las que la atención es aplicada. Distintos modelos pueden beneficiarse de procesar los datos
creando vectores de características distintos, cada uno de ellos deriva de algún tipo de representación
de la entrada. por tanto, es posible atendera a diferentes representaciones y formar un vector
final usando una combinación pesada de estos a través de dichos pesos de atención.
\citeauthor{DBLP:journals/corr/abs-1904-02874} llama a este tipo de modelos de atención como
\textbf{multi-representational AM}. En la segunda categoría,
\textbf{milti-dimensional attention}, la atención no es aplicada sobre los diversas vectores de
características sino a un nivel más interno,
sobre sus dimensiones. Pesando cada característica de un vector de características permite seleccionar
aquellas que mejor lo describen para un contexto dado. EN \textit{NLP}, resulta bastante útil cuando se trata
con \textit{polisemia}, en donde una palabra o frase puede tener más de un significado.


% Transformer
\section{El modelo Transformer}

A finales del año 2017 se presentó un nuevo modelo que vino a revolucionar el área de Procesamiento
de Lenguaje Natural, \textit{El Transformer} \cite{Vaswani}. Una de sus principales características
es la capacidad de procesar la información de una secuencia de forma paralela, a diferencia de las
Redes Neuronales Recurrentes, donde la información se procesa recurrentemente. Gracias a ello, la
capacidad de \textit{recuerdo} no se ve afectada por el problema de \textit{El desvanecimiento del
Gradiente}, especialmente cuando se trabaja con secuencias bastante largas.

\textit{El Transformer} puede ser visto como otro modelo \textit{seq2seq} (Secuencia a Secuencia)
\ref{fig:trans_seq2sqe}, formado por dos etapas: la primera encargada de codificar la información de
entrada y la segunda de decodificarla. Su principal característica es que aplica mecanismos de
\textit{Self-Attention} para capturar las dependencias globales entre la entrada y la salida. Dada
una secuencia de entrada $X = (x_1, x_2, \dots, x_n)$, con $n$ como el tamaño de la secuencia, el
codificador produce una representación intermedia $Z = (z_1, z_2, \dots, z_n)$, al igual que los
modelos \textit{seq2seq}. El decodificador usa la secuencia $Z$ para generar la secuencia de salida
$Y = (y_1, y_2, \dots, y_m)$ uno a la vez (en modo inferencia), con $m$ como el tamaño de la secuencia
de salida. Nótese que, al generar una salida a la vez, el decodificador tiene que ser auto-regresivo.
Usa la salida anterior $y_{i-1}$ como entrada adicional para generar la siguiente salida $y_i$. Por
ello, durante el entrenamiento, el modelo es alimentado con entradas y salidas desfasadas en el tiempo.

\begin{figure}[ht!]
    \centering
    \includegraphics[width=0.4 \textwidth]{Chapters/2. Transformer/Figures/transformer/t_seq2seq.jpg}
    \caption{Modelo Transformer generalizado como modelo Secuencia a Secuencia}
    \label{fig:trans_seq2sqe}
\end{figure}


\subsection{El Codificador y Decodificador}

El \textit{Modelo Transformer} está formado por múltiples codificadores y decodificadores apilados e
interconectados, como observamos en la figura \ref{fig:trans_seq2sqe}. El codificador consta de dos
capas: la primera de ellas aplica \textit{Self-Attention} múltiples veces sobre la misma entrada
(\textit{Multi-Head Self-Attention}) y la segunda capa está representada solo por una red
\textit{Feed-Forward}, cuya entrada es la salida de la capa anterior. Véase la figura \ref{fig:trans_encoder}.

\begin{figure}[ht!]
\centering
    \begin{minipage}{.4\textwidth}
        \centering
        \includegraphics[width=0.7 \textwidth]{Chapters/2. Transformer/Figures/transformer/encoder.jpg}
    \end{minipage}
    \begin{minipage}{.5\textwidth}
        \begin{equation*}
            \begin{split}
                mha = MHA(X, X, X)\\
                norm = Norm( mha + X)\\
                f = FeedForward(norm)\\
                Encoder(X) = Norm(f + norm)
            \end{split}
            \label{eq:trans_enc}
        \end{equation*}
    \end{minipage}
    \caption{Pseudocódigo - Etapa Codificadora del Modelo Transformer}
    \label{fig:trans_encoder}
\end{figure}


El decodificador tiene una estructura similar al codificador, con una etapa adicional intermedia
de \textit{Multi-Head Attention} aplicada sobre la salida de la pila de codificadores. También, la
primera capa de atención sufre un ligero cambio en su forma de operación, necesitando enmascarar (al
momento en que se realiza el entrenamiento) la atención prestada del pasado al futuro. Esto es debido
a que el decodificador se encarga de generar una secuencia (en modo inferencia) uno a la vez, usando
solamente la salida anterior y, por tanto, no tiene conocimiento de salidas futuras. Observe la figura
\ref{fig:trans_te}.


\begin{figure}[ht!]
\centering
    \begin{minipage}{.4\textwidth}
        \centering
        \includegraphics[width=1.0 \textwidth]{Chapters/2. Transformer/Figures/transformer/decoder.jpg}
    \end{minipage}
    \begin{minipage}{.5\textwidth}
        \begin{equation*}
            \begin{split}
                mha_1 = MHA(X, X, X)\\
                norm_1 = Norm( mha_1 + X)\\
                mha_2 = MHA(enc_{out}, enc_{out}, norm_1)\\
                norm_2 = Norm( mha_2 + X)\\
                f = FeedForward(norm_2)\\
                decoder(X) = Norm(f + norm_2)
            \end{split}
            \label{eq:trans_dec}
        \end{equation*}
    \end{minipage}
    \caption{Etapa Decodificadora del Modelo Transformer. Pseudocódigo}
    \label{fig:trans_decoder}
\end{figure}


\begin{figure}[ht!]
    \centering
    \begin{subfigure}[b]{0.49\textwidth}
        \centering
        \includegraphics[width=1.0 \textwidth]{Chapters/2. Transformer/Figures/transformer/train.jpeg}
        \caption{Transformer modo entrenamiento. Las entradas en el decodificador son recorridas un
                 elemento en el futuro, con el fin de que aprende a predecir la siguiente palabra
                 dado un contexto previo y las salidas actuales en el momento de la evaluación.}
        \label{fig:trans_train}
    \end{subfigure}
    \begin{subfigure}[b]{0.49\textwidth}
        \centering
        \includegraphics[width=1.0 \textwidth]{Chapters/2. Transformer/Figures/transformer/inference.jpg}
        \caption{Transformer modo inferencia. El decodificador funciona como un modelo auto-regresivo,
        usa sus predicciones hasta el tiempo $t$ para obtener el siguiente valor. En la primer iteración
        el decoder solo recibe el token de inicio de oración $<start>$ por lo que podrá predecir la primer
        palabra de la oración gracias a que fue entrenado con un desplazamiento hacia el futuro. En la siguiente
        iteración el nuevo token predicho es agregado como entrada al decodificador. El decodificador
        termina su predicción en el momento que el token $<end>$ es obtenido.}
        \label{fig:trans_eval}
    \end{subfigure}
    \caption{Esquema de entrenamiento e inferencia del modelo Transformer en un problema de
             Machine Translation.}
        \label{fig:trans_te}
\end{figure}


\subsection{Multi-Head Self-Attention} \label{section-mha}

En la sección \ref{section:att} se detalla una generalización de la atención y diversas variantes
usadas a lo largo de la literatura. El modelo original que introdujo a los Transformers usa en
especial la variante \textit{Scaled Dot-Product Attention} \cite{Vaswani}:


\begin{equation}
    Attention(q, k, v) = softmax(\frac{q k^\top}{\sqrt{d_k}}) v
    \label{eq:trans_att_gen}
\end{equation}

El Transformer está basado en la idea de de aplicar atención multiples veces, al usar varias cabezas
de atención, \textit{Multihead-Self-Attention} (MHA), permite  al modelo conjuntamente atender a información
en distintas posiciones desde $h$ diferentes subespacios de representación. \ref{eq:mha}

\begin{equation}
    mha(Q, K, V) = Concat(head_1,head_2,head_3,..., head_h)W^O
    \label{eq:mha}
\end{equation}

Todas las cabezas de atención son concatenadas y resumidas para ser devueltas a las dimensiones del
espacio de entrada original, principalmente para mantener consistencia en las dimensiones de usadas
en cada etapa de codificación y decodificación del modelo a través de $W^O \in \mathbb{R}^{hd_v \times d_m}$.
$W^O$ es entrenado conjuntamente para aprender a resumir la información capturada por cada cabeza de
atención. $Q, K \in \mathbb{R}^{n \times d_{m}}$ y $V \in \mathbb{R}^{n \times d_{v}}$ es la representación
consulta, clave y valor de los embeddings de entrada de cada capa de atención del codificador y
decodificador como se observa en las figuras \ref{fig:trans_encoder} \ref{fig:trans_decoder}.
$n$ es el tamaño de la secuencia, $d_m$ y $d_v$ son los tamaño del embedding y $h$ el número de
cabezas de atención.

En el caso del modelo transformer tenemos un conjunto embeddings sobre las cuales se aplica atención,
si bien, no representan necesariamente las consultas, llaves, y valores utilizados para la atención
generalizada, podemos obtener estas representaciones transportándolos a sus espacios respectivos a través de alguna
transformación aprendida conjuntamente con el entrenamiento del modelo.

Por tanto, para el conjunto de Embeddings  $E_Q \in \mathbb{R}^{n \times d_m}$,
$E_K \in \mathbb{R}^{n \times d_m}$ y $E_V \in \mathbb{R}^{n \times d_v}$ donde $n$ es el número
embeddings, $d_m$ y $d_v$ son las dimensiones de cada uno, la atención en cada cabeza $i$ se calcula
como:

\begin{equation}
    \begin{split}
        Q_i = E_Q W_i^Q\\
        K_i = E_K W_i^K\\
        V_i = E_V W_i^V\\
    \end{split}
\end{equation}

\begin{equation}
\begin{split}
    head_i = Attention(Q_i, K_i, V_i) = softmax\Big(\frac{Q_i K_i^T}{\sqrt{d_k}}\Big) V_i
    \label{eq:trans_att}
\end{split}
\end{equation}

\noindent donde $W_i^Q$, $W_i^K$ $\in \mathbb{R}^{d_m \times d_k}$, $W_i^V$ $\in \mathbb{R}^{d_m \times d_v}$
y $d_k=d_v=d_m/h$.

El término de escalamiento $\sqrt{d_k}$ ayuda a evitar que la magnitud de los productos punto calculados
entre cada consulta y llave crezcan demasiado, y que la función $softmax$ pueda ser más estable al evitar
regiones donde los gradientes son muy pequeños \cite{Vaswani}.



\subsection{Información Posicional}

En los modelos basados en \textit{Redes Recurrentes}, la información se procesa uno a uno en cada paso
de tiempo. Los modelos basados en \textit{Transformers} procesan la información en conjunto, perdiendo
la noción de la temporalidad de los datos. Una solución es agregar dicha información perdida a través
de vectores que codifiquen el tiempo/posición de los datos, sumándolos con los vectores de embeddings.
Estos vectores, llamados \textit{Positional Encodings} \cite{DBLP:journals/corr/GehringAGYD17}, siguen
un patrón específico que el modelo aprende a identificar y lo ayuda a determinar la posición de cada
elemento de la secuencia y, por tanto, calcular a qué distancia se encuentra cada uno de los demás.

Por lo regular, se usa una onda senoidal y cosenoidal para lugares pares e impares, formando una
progresión geométrica desde $2\pi$ hasta $10000 \cdot 2\pi$ \ref{eq:trans_pos_enc}:


\begin{equation}
    \begin{split}
        PE(pos, 2i) = \sin(pos/10000^{2i/d_m})\\
        PE(pos, 2i+1) = \cos(pos/10000^{2i/d_m})
    \end{split}
    \label{eq:trans_pos_enc}
\end{equation}


\begin{figure}[ht!]
    \centering
    \includegraphics[width=0.5 \textwidth]{Chapters/2. Transformer/Figures/transformer/pos_enc.png}
    \caption{2000 Vectores de Positional Encoding con dimensiones de embedding=500.}
    \label{fig:trans_pos_enc}
\end{figure}


\subsection{Problemas típicos en el entrenamiento de Transformers}

\subsubsection{Learning Rate WarmUp y Layer Normalization}

A pesar de que la arquitectura del modelo Transformer no es tan compleja, puesto que tanto el
codificador como el decodificador están formados por pilas de capas de atención y MLP, el
entrenamiento de este tipo de modelos muchas veces no resulta tan trivial. Regularmente requiere de
una combinación de técnicas para lograr su convergencia a valores aceptables y, en conjunto con una
gran cantidad de datos, tamaños de lotes de procesamiento grandes y una gran cantidad de tiempo de
procesamiento en GPU \cite{DBLP:journals/corr/abs-1804-00247}.

\textit{Learning Rate WarmUp} es una de las primeras técnicas usadas y descritas en el proceso de
entrenamiento por \citeauthor{Vaswani}. Usando el algoritmo Adam como optimizador, se varía el factor
de aprendizaje de acuerdo a la fórmula:


\begin{equation}
    lrate = d_{m}^{-0.5} \vdot \min\big(step\_num^{-0.5}, step\_num \vdot warmup\_steps^{-0.5} \big)
\end{equation}

En el esquema anterior, más conocido como \textit{Noam-Warmup}, el modelo original es entrenado
incrementando linealmente el factor de aprendizaje en los primeros $warmup\_steps=4000$ pasos.
Posteriormente, decrece proporcionalmente al inverso de la raíz cuadrada del paso $step\_num$
actual, véase la figura \ref{fig:noam}.

\begin{figure}[ht!]
    \centering
    \includegraphics[width=0.5 \textwidth]{Chapters/2. Transformer/Figures/transformer/noam.png}
    \caption{Noam-Warmup con $warmup\_steps=4000$ y $d_{m} = 512$}
    \label{fig:noam}
\end{figure}

Si bien la razón por la que funciona este tipo de técnica no está del todo clara, se presume que usar
\textit{Learning Rate WarmUp} ayuda a reducir la varianza del factor de aprendizaje adaptativo durante
las primeras etapas del entrenamiento del modelo. \citeauthor{DBLP:journals/corr/abs-1908-03265}
demostraron que el segundo momento del algoritmo de Adam durante etapas tempranas de optimización es
proporcional a una integral divergente, lo que provoca actualizaciones inestables, llevando al modelo
fuera de las regiones donde un mejor mínimo existe. Con esto en mente, \citeauthor{DBLP:journals/corr/abs-1908-03265}
proponen el algoritmo de optimización \textit{RAdam} (Rectified Adam) como una alternativa a usar
\textit{Learning Rate WarmUp} y mitigar este efecto durante la fase inicial del entrenamiento de los modelos.

El \textit{Learning Rate WarmUp} comúnmente es usado en conjunto con algoritmos de optimización
estocásticos como \textit{RMSprop} o \textit{Adam}. En vez de configurar el \textit{factor de
aprendizaje} $\alpha$ con un decremento constante, la estrategia de \textit{Learning Rate WarmUp}
configura este factor con valores muy pequeños en los primeros pasos de entrenamiento. Durante las
primeras etapas del entrenamiento, el factor de aprendizaje es incrementado hasta un límite que es
ligeramente superior o inferior al valor inicial de $\alpha$ del optimizador usado y, posteriormente,
es decrementado progresivamente hasta la convergencia del modelo.

Así, en cada paso del algoritmo de optimización, el cual está parametrizado por el factor de aprendizaje
$\alpha$, puede ser aplicado un factor de \textit{warmup} $\omega \in [0,1]$ que sirve para reducir
$\alpha$ y, a la vez, el paso de optimización en cada tiempo, reemplazando $\alpha_t = \alpha \omega_t$.
La forma más sencilla es usar un factor \textbf{linear warmup} parametrizado por un periodo de
\quotes{calentamiento} $\tau$.

\begin{equation}
    \omega_t^{linear, \tau} = \min\big(1, \frac{t}{\tau} \big)
    \label{eq:warn_linear}
\end{equation}

\citeauthor{DBLP:journals/corr/abs-1910-04209} proponen 3 formas de aplicar la técnica de \textit{warmup}:

\textbf{Exponential warmup} aplica un decaimiento exponencial

\begin{equation}
    \omega_t^{expo, \tau} = 1 - \exp(- \frac{1}{\tau} t)
    \label{eq:warn_expo}
\end{equation}

\noindent recomienda elegir $\tau = (1 - \beta_2)^{-1}$ tal que no se tan diferente del segundo momento de
corrección de bias del algoritmo de \textit{Adam} $\beta_2$.

\begin{equation}
    \omega_t^{expo, untuned} = 1 - \exp(- (1 - \beta_2) t)
    \label{eq:warn_expo_untened}
\end{equation}

Similar al decaimiento exponencial proponen usar \textit{linear warmup} sobre
$\tau = 2 (1 - \beta_2)^{-1}$ iteraciones para preservar un efecto similar de des-aceleración
con el paso del tiempo.

\begin{equation}
    \omega_t^{linear, untuned} = \min\big(1, \frac{1 - \beta_2}{2} t \big)
    \label{eq:warn_linear_untuned}
\end{equation}

\begin{figure}[ht!]
    \centering
    \includegraphics[width=0.5 \textwidth]{Chapters/2. Transformer/Figures/transformer/warmups.png}
    \caption{Learning rate sobre X 18000 iteraciones usando RAdam y lineal, exponencial warmup con Adam}
    \label{fig:warmup}
\end{figure}

Por otro lado, \citeauthor{pmlr-v119-huang20f} mencionan que usar la técnica de \textit{Learning Rate
WarmUp} para mitigar la varianza del optimizador Adam no es del todo la solución y que el problema
radica precisamente en la arquitectura del modelo Transformer, principalmente en las capas de
normalización \cite{DBLP:journals/corr/abs-1804-09849} \cite{DBLP:journals/corr/abs-2002-04745}. En
particular, \citeauthor{DBLP:journals/corr/abs-2002-04745} encuentran que para un modelo Transformer
de cualquier tamaño con capas de normalización entre bloques residuales (\textit{Post-LN Transformer}),
la escala de la norma del gradiente que incide en la última capa de normalización permanece igual al
no depender de la cantidad de bloques del Transformer. Por el contrario, si la capa de normalización
es colocada justo antes de la conexión residual (\textit{Pre-LN Transformer}), la magnitud de la norma
del gradiente decrece conforme el tamaño del modelo incrementa, guiando así al problema de
desvanecimiento de gradiente. \citeauthor{pmlr-v119-huang20f} proponen eliminar las capas de
normalización del modelo Transformer que, en conjunto con la inestabilidad del algoritmo de optimización
de Adam, provocan la dificultad de entrenamiento desde las primeras etapas. Para ello, estandarizan la
siguiente inicialización (\textit{T-Fixup}) de pesos del modelo, permitiendo evitar la etapa de
\textit{WarmUp} y las capas de normalización en el Transformer. La figura \ref{fig:t-fixup} muestra
una comparativa de los histogramas usando la inicialización \textit{T-Fixup} y usando el algoritmo de
Adam con y sin etapa de \textit{Warmup}:

\begin{itemize}
    \item Aplicar initialization tipo \textit{Xavier} para todos los pesos del modelo.
          Excepto el proceso de generación de embedding adecuados al tamaño del modelo $d_m$.
    \item Usar una inicialización tipo \textit{Gaussiana} con $\mathbb{N}(0, d_m^\frac{1}{2})$ para
          los pesos de generación de embeddings.
    \item Escalar las matrices $W_i^V$ y $W^O$ en cada bloque de atención en el decodificador, los
          pesos en de cada capa MLP del decodificador y los pesos de generación de embeddings tanto
          del codificador como decodificador por $9N^{-\frac{1}{4}}$ donde $N$ es el número de bloques
          del Transformer.
    \item Escalar las matrices $W_i^V$ y $W^O$ de cada bloque de atención del codificador y los pesos
          de cada capa de MLP del codificador por $0.67N^{-\frac{1}{4}}$
\end{itemize}

\begin{figure}[ht!]
    \centering
    \includegraphics[width=0.5 \textwidth]{Chapters/2. Transformer/Figures/transformer/tfixup.png}
    \caption{Histograma de gradientes del Algoritmo Adam con y sin etapa de \textit{WarmUp} y
    usando inicialización \textit{T-Fixup}. Imagen original de \citeauthor{pmlr-v119-huang20f}}
    \label{fig:t-fixup}
\end{figure}

\subsubsection{Cálculo de la Atención}

Además de lo específico y delicado del entrenamiento del Modelo Transformer, su costo en tiempo
computacional y de memoria también representa un serio problema a la hora de optimizar e inferir.
Esto se debe principalmente a que en el proceso de atención debe focalizar cada token con respecto
a todos los demás, lo que lleva a que su complejidad crezca cuadráticamente con respecto al tamaño
de la secuencia.

Varias técnicas han sido propuestas para reducir este problema. Muchas de ellas involucran reducir
la atención a vecindades de representaciones, aproximar la matriz de atención con otras matrices de
transformaciones a través de kernels o sustituir completamente la operación \textit{softmax} por otra función.


\textbf{Atención de vecindades}:

\citeauthor{DBLP:journals/corr/abs-1802-05751} particionan la información de las representaciones de
las consultas asignándolas a diferentes bloques de memoria, restringiéndose a vecindarios locales
alrededor de cada consulta, principalmente basados en cómo las redes convolucionales trabajan. Sin
embargo, esta solución es parcial y solo aplicable a secuencias de datos con relaciones cortas, como
imágenes.

\citeauthor{DBLP:journals/corr/abs-1904-10509} factorizan la matriz de atención para reducir su
complejidad de $O(n^2)$ a $O(n\sqrt{n})$ por medio de matrices ralas, separando la atención a través
de diferentes pasos al parametrizar la atención por una conectividad de distintos patrones elegidos
previamente bajo el supuesto de que las matrices de atención son ralas, puesto que no contienen
dependencias de relevancia sobre representaciones distantes, como se observa en la imagen \ref{fig:att-spar}.


\begin{figure}[ht!]
    \centering
    \includegraphics[width=0.7 \textwidth]{Chapters/2. Transformer/Figures/transformer/head_sparsity.png}
    \caption{Visualización de 8 cabezas de atención sobre una tarea de Machine-Translation. Las
    matrices de atención tienden a ser ralas al tener carencia de relaciones relevantes entre diversas
    representaciones a distancias lejanas.}
    \label{fig:att-spar}
\end{figure}

\citeauthor{DBLP:journals/corr/abs-1905-07799} proponen reducir el ancho de la atención basándose en
que cada representación no necesita prestar atención sobre todas las demás, sino que debería ser
adaptativa. Así, para cada cabeza de atención se agrega una función de enmascaramiento que controla
la flexibilidad del ancho de una ventana. La ventana formada cambia dinámicamente de tamaño
dependiendo de la representación en cuestión.

\citeauthor{DBLP:journals/corr/abs-2004-05150} siguen una estrategia similar, implementando
atención local con ventanas dilatadas distintas para cada cabeza, permitiendo atender contextos menos
locales en cada ocasión y atención global sobre localizaciones preseleccionadas. Dada la dificultad de
su implementación sin usar ciclos para iterar sobre los elementos seleccionados a atender, implementan
su propio kernel en \textit{CUDA} con las operaciones optimizadas para realizar esta tarea.

\citeauthor{DBLP:journals/corr/abs-1901-02860} mencionan que si el problema es el procesamiento de
grandes secuencias, ¿por qué no dividirlas en secuencias más pequeñas y procesarlas individualmente
para así evitar usar grandes cantidades de memoria en su procesamiento? El principal problema de este
enfoque es que cada secuencia es procesada individualmente y la información de secuencias previas es
ignorada, evitando que esta fluya a través de las próximas secuencias. Para solucionar este
inconveniente, introducen un mecanismo de recurrencia en la arquitectura del Transformer. Durante el
entrenamiento (véase la figura \ref{fig:trans-xl}), un estado oculto es calculado de las secuencias previas y guardado
en memoria para extender el contexto al momento de procesar la siguiente secuencia. Durante el
proceso de evaluación, el resultado de las operaciones del Transformer puede ser reutilizado y no
calculado nuevamente desde cero, permitiendo reducir considerablemente el tiempo de evaluación.

\begin{figure}[ht!]
    \centering
    \includegraphics[width=0.8 \textwidth]{Chapters/2. Transformer/Figures/transformer/trans-XL.png}
    \caption{Transformer-XL. Para tratar con secuencias largas divide el proceso en secuencias más
             cortas creando estados ocultos intermedios y usándolos en el cálculo de las próximas
             secuencias. Figura obtenida de \cite{DBLP:journals/corr/abs-1901-02860}.}
    \label{fig:trans-xl}
\end{figure}

\citeauthor{DBLP:journals/corr/abs-2001-04451} reducen el problema de realizar la operación de softmax
sobre toda la matriz $Q_i K_i^\top$ a calcularlo individualmente por cada consulta $q_j$, guardando
solo una vez en memoria este valor en cada iteración y recalculándolo cuando se necesite de nuevo
al utilizar \textit{Back-Propagation} usando capas reversibles. Si bien computacionalmente es
más costoso, permite usar mucho menos memoria que la solución original. Por otro lado, dado que el
resultado de la función softmax depende en mayor medida de los elementos dominantes de la matriz,
solo es necesario fijarse en las llaves más cercanas a la consulta en cuestión. \textit{LSH}
(\textit{Local Sensitive Hashing}) resuelve este problema permitiendo encontrar rápidamente los
vecinos más cercanos en espacios de altas dimensiones, con la restricción de que $W_i^Q = W_i^K$,
dado que se necesita conservar la similaridad entre consultas y llaves, algo que sería más difícil
si sus matrices de proyección $W_i^Q$ y $W_i^K$ fuesen muy distintas.


\textbf{Aproximaciones a la Atención original}:

También \citeauthor{DBLP:journals/corr/abs-1906-11024} proponen un modelo para compartir pesos de
capas adyacentes (Shared Attention Network - SAN). Cada $\pi$ capas continuas en el codificador
comparten la misma matriz de atención y en el decodificador se comparte la proyección de los pesos
de atención sobre la representación de los valores $V_i$. En otras palabras, se comparte directamente
la cabeza de atención $head_i$. Dado que no es tan fácil conocer qué capas deben compartir pesos,
establecen un proceso iterativo de entrenamiento basado en calcular qué tan diferentes son las capas
del transformer usando la \textit{Divergencia de Jensen-Shannon}. Si la similitud entre dos capas es
mayor a cierto umbral, se indica que dichas capas deben compartir pesos. Se repite un nuevo
entrenamiento y se calcula nuevamente la similitud entre capas, y así sucesivamente hasta la
convergencia. Podemos notar que este proceso de entrenamiento y ajuste de pesos es muy costoso, ya
que se requiere un nuevo entrenamiento por cada ajuste de compartición de pesos. Sin embargo, el
modelo resultante es menos complejo y el tiempo en modo de evaluación o inferencia se reduce
considerablemente.

\citeauthor{DBLP:journals/corr/abs-2006-04768}, bajo la hipótesis de que la matriz de atención tiene
un rango mucho menor que $n$, proponen obtener los valores de cada cabeza haciendo una aproximación
a ella. Para ello, se hace uso de dos matrices entrenables conjuntamente con el modelo, $E$ y $F$
$\in \mathbb{R}^{n \times k}$ con $k \ll n$, tal que,
$head_i = softmax(\frac{Q_i (E_i K_i)}{\sqrt{d_m}}) F_iV_i$. Con ello, la dimensión correspondiente
al tamaño de las secuencias se reduce bajo el supuesto de que podemos representar la información
secuencial en un espacio más pequeño sin gran pérdida de información.

\citeauthor{DBLP:journals/corr/abs-2009-14794}, por el contrario, descomponen la operación de
atención sobre los valores $head = softmax(\frac{QK^\top}{\sqrt{d_k}} V)$ en una multiplicación
matricial más simple $head = Q'k'^\top V$ con $Q'$ y $k'^\top \in \mathbb{R}^{n \times r}$ y $r \leq n$.
Para ello, construyen $Q'$ y $k'$ como dos matrices usando kernels tal que su producto forma una
aproximación a la función softmax aplicada al producto de $Q$ y $K$. Notemos que con ello podemos
reducir el costo computacional y en memoria simplemente reduciendo el producto $Q'(k'^\top V)$ de
derecha a izquierda.

Finalmente, autores como \citeauthor{DBLP:journals/corr/abs-2105-03824} reemplazan completamente
el bloque de \textit{Multihead Attention} con bloques que aplican operaciones de Transformada de
Fourier a lo largo de la dimensión de los embeddings y de las secuencias, demostrando que usar
\textit{FFT} (Fast Fourier Transform) es suficiente para abstraer y modelar las relaciones.
\citeauthor{DBLP:journals/corr/abs-2108-09084} cambian la atención entre todas las consultas
y llaves por una sola con todas las llaves. Para ello, a través de la atención, resumen todas las
consultas en una consulta global. Este proceso se repite con las llaves y valores, como se observa
en la figura \ref{fig:fast-former}.

\begin{figure}[ht!]
    \centering
    \includegraphics[width=0.6 \textwidth]{Chapters/2. Transformer/Figures/transformer/fastformer.png}
    \caption{Fast-Former. Remplaza la atención tradicional del  transformer por una iterativa. En
             cada paso crea una consulta y clave global usando atención sobre estos mismos.
             Figura obtenida de \cite{DBLP:journals/corr/abs-2108-09084}}
    \label{fig:fast-former}
\end{figure}


% \begin{figure}
%     \begin{center}
%         \scalebox{0.6}{%% Creator: Matplotlib, PGF backend
%%
%% To include the figure in your LaTeX document, write
%%   \input{<filename>.pgf}
%%
%% Make sure the required packages are loaded in your preamble
%%   \usepackage{pgf}
%%
%% Figures using additional raster images can only be included by \input if
%% they are in the same directory as the main LaTeX file. For loading figures
%% from other directories you can use the `import` package
%%   \usepackage{import}
%%
%% and then include the figures with
%%   \import{<path to file>}{<filename>.pgf}
%%
%% Matplotlib used the following preamble
%%
\begingroup%
\makeatletter%
\begin{pgfpicture}%
\pgfpathrectangle{\pgfpointorigin}{\pgfqpoint{10.000000in}{10.000000in}}%
\pgfusepath{use as bounding box, clip}%
\begin{pgfscope}%
\pgfsetbuttcap%
\pgfsetmiterjoin%
\pgfsetlinewidth{0.000000pt}%
\definecolor{currentstroke}{rgb}{1.000000,1.000000,1.000000}%
\pgfsetstrokecolor{currentstroke}%
\pgfsetstrokeopacity{0.000000}%
\pgfsetdash{}{0pt}%
\pgfpathmoveto{\pgfqpoint{0.000000in}{0.000000in}}%
\pgfpathlineto{\pgfqpoint{10.000000in}{0.000000in}}%
\pgfpathlineto{\pgfqpoint{10.000000in}{10.000000in}}%
\pgfpathlineto{\pgfqpoint{0.000000in}{10.000000in}}%
\pgfpathclose%
\pgfusepath{}%
\end{pgfscope}%
\begin{pgfscope}%
\pgfsetbuttcap%
\pgfsetmiterjoin%
\definecolor{currentfill}{rgb}{1.000000,1.000000,1.000000}%
\pgfsetfillcolor{currentfill}%
\pgfsetlinewidth{0.000000pt}%
\definecolor{currentstroke}{rgb}{0.000000,0.000000,0.000000}%
\pgfsetstrokecolor{currentstroke}%
\pgfsetstrokeopacity{0.000000}%
\pgfsetdash{}{0pt}%
\pgfpathmoveto{\pgfqpoint{0.763041in}{8.027083in}}%
\pgfpathlineto{\pgfqpoint{2.450346in}{8.027083in}}%
\pgfpathlineto{\pgfqpoint{2.450346in}{9.566628in}}%
\pgfpathlineto{\pgfqpoint{0.763041in}{9.566628in}}%
\pgfpathclose%
\pgfusepath{fill}%
\end{pgfscope}%
\begin{pgfscope}%
\pgfsetbuttcap%
\pgfsetroundjoin%
\definecolor{currentfill}{rgb}{0.000000,0.000000,0.000000}%
\pgfsetfillcolor{currentfill}%
\pgfsetlinewidth{0.803000pt}%
\definecolor{currentstroke}{rgb}{0.000000,0.000000,0.000000}%
\pgfsetstrokecolor{currentstroke}%
\pgfsetdash{}{0pt}%
\pgfsys@defobject{currentmarker}{\pgfqpoint{0.000000in}{-0.048611in}}{\pgfqpoint{0.000000in}{0.000000in}}{%
\pgfpathmoveto{\pgfqpoint{0.000000in}{0.000000in}}%
\pgfpathlineto{\pgfqpoint{0.000000in}{-0.048611in}}%
\pgfusepath{stroke,fill}%
}%
\begin{pgfscope}%
\pgfsys@transformshift{0.839736in}{8.027083in}%
\pgfsys@useobject{currentmarker}{}%
\end{pgfscope}%
\end{pgfscope}%
\begin{pgfscope}%
\definecolor{textcolor}{rgb}{0.000000,0.000000,0.000000}%
\pgfsetstrokecolor{textcolor}%
\pgfsetfillcolor{textcolor}%
\pgftext[x=0.839736in,y=7.929861in,,top]{\color{textcolor}\rmfamily\fontsize{10.000000}{12.000000}\selectfont \(\displaystyle {0.0}\)}%
\end{pgfscope}%
\begin{pgfscope}%
\pgfsetbuttcap%
\pgfsetroundjoin%
\definecolor{currentfill}{rgb}{0.000000,0.000000,0.000000}%
\pgfsetfillcolor{currentfill}%
\pgfsetlinewidth{0.803000pt}%
\definecolor{currentstroke}{rgb}{0.000000,0.000000,0.000000}%
\pgfsetstrokecolor{currentstroke}%
\pgfsetdash{}{0pt}%
\pgfsys@defobject{currentmarker}{\pgfqpoint{0.000000in}{-0.048611in}}{\pgfqpoint{0.000000in}{0.000000in}}{%
\pgfpathmoveto{\pgfqpoint{0.000000in}{0.000000in}}%
\pgfpathlineto{\pgfqpoint{0.000000in}{-0.048611in}}%
\pgfusepath{stroke,fill}%
}%
\begin{pgfscope}%
\pgfsys@transformshift{1.606693in}{8.027083in}%
\pgfsys@useobject{currentmarker}{}%
\end{pgfscope}%
\end{pgfscope}%
\begin{pgfscope}%
\definecolor{textcolor}{rgb}{0.000000,0.000000,0.000000}%
\pgfsetstrokecolor{textcolor}%
\pgfsetfillcolor{textcolor}%
\pgftext[x=1.606693in,y=7.929861in,,top]{\color{textcolor}\rmfamily\fontsize{10.000000}{12.000000}\selectfont \(\displaystyle {0.5}\)}%
\end{pgfscope}%
\begin{pgfscope}%
\pgfsetbuttcap%
\pgfsetroundjoin%
\definecolor{currentfill}{rgb}{0.000000,0.000000,0.000000}%
\pgfsetfillcolor{currentfill}%
\pgfsetlinewidth{0.803000pt}%
\definecolor{currentstroke}{rgb}{0.000000,0.000000,0.000000}%
\pgfsetstrokecolor{currentstroke}%
\pgfsetdash{}{0pt}%
\pgfsys@defobject{currentmarker}{\pgfqpoint{0.000000in}{-0.048611in}}{\pgfqpoint{0.000000in}{0.000000in}}{%
\pgfpathmoveto{\pgfqpoint{0.000000in}{0.000000in}}%
\pgfpathlineto{\pgfqpoint{0.000000in}{-0.048611in}}%
\pgfusepath{stroke,fill}%
}%
\begin{pgfscope}%
\pgfsys@transformshift{2.373650in}{8.027083in}%
\pgfsys@useobject{currentmarker}{}%
\end{pgfscope}%
\end{pgfscope}%
\begin{pgfscope}%
\definecolor{textcolor}{rgb}{0.000000,0.000000,0.000000}%
\pgfsetstrokecolor{textcolor}%
\pgfsetfillcolor{textcolor}%
\pgftext[x=2.373650in,y=7.929861in,,top]{\color{textcolor}\rmfamily\fontsize{10.000000}{12.000000}\selectfont \(\displaystyle {1.0}\)}%
\end{pgfscope}%
\begin{pgfscope}%
\definecolor{textcolor}{rgb}{0.000000,0.000000,0.000000}%
\pgfsetstrokecolor{textcolor}%
\pgfsetfillcolor{textcolor}%
\pgftext[x=1.606693in,y=7.750849in,,top]{\color{textcolor}\rmfamily\fontsize{16.000000}{19.200000}\selectfont FPR}%
\end{pgfscope}%
\begin{pgfscope}%
\pgfsetbuttcap%
\pgfsetroundjoin%
\definecolor{currentfill}{rgb}{0.000000,0.000000,0.000000}%
\pgfsetfillcolor{currentfill}%
\pgfsetlinewidth{0.803000pt}%
\definecolor{currentstroke}{rgb}{0.000000,0.000000,0.000000}%
\pgfsetstrokecolor{currentstroke}%
\pgfsetdash{}{0pt}%
\pgfsys@defobject{currentmarker}{\pgfqpoint{-0.048611in}{0.000000in}}{\pgfqpoint{-0.000000in}{0.000000in}}{%
\pgfpathmoveto{\pgfqpoint{-0.000000in}{0.000000in}}%
\pgfpathlineto{\pgfqpoint{-0.048611in}{0.000000in}}%
\pgfusepath{stroke,fill}%
}%
\begin{pgfscope}%
\pgfsys@transformshift{0.763041in}{8.097062in}%
\pgfsys@useobject{currentmarker}{}%
\end{pgfscope}%
\end{pgfscope}%
\begin{pgfscope}%
\definecolor{textcolor}{rgb}{0.000000,0.000000,0.000000}%
\pgfsetstrokecolor{textcolor}%
\pgfsetfillcolor{textcolor}%
\pgftext[x=0.418904in, y=8.048837in, left, base]{\color{textcolor}\rmfamily\fontsize{10.000000}{12.000000}\selectfont \(\displaystyle {0.00}\)}%
\end{pgfscope}%
\begin{pgfscope}%
\pgfsetbuttcap%
\pgfsetroundjoin%
\definecolor{currentfill}{rgb}{0.000000,0.000000,0.000000}%
\pgfsetfillcolor{currentfill}%
\pgfsetlinewidth{0.803000pt}%
\definecolor{currentstroke}{rgb}{0.000000,0.000000,0.000000}%
\pgfsetstrokecolor{currentstroke}%
\pgfsetdash{}{0pt}%
\pgfsys@defobject{currentmarker}{\pgfqpoint{-0.048611in}{0.000000in}}{\pgfqpoint{-0.000000in}{0.000000in}}{%
\pgfpathmoveto{\pgfqpoint{-0.000000in}{0.000000in}}%
\pgfpathlineto{\pgfqpoint{-0.048611in}{0.000000in}}%
\pgfusepath{stroke,fill}%
}%
\begin{pgfscope}%
\pgfsys@transformshift{0.763041in}{8.446959in}%
\pgfsys@useobject{currentmarker}{}%
\end{pgfscope}%
\end{pgfscope}%
\begin{pgfscope}%
\definecolor{textcolor}{rgb}{0.000000,0.000000,0.000000}%
\pgfsetstrokecolor{textcolor}%
\pgfsetfillcolor{textcolor}%
\pgftext[x=0.418904in, y=8.398734in, left, base]{\color{textcolor}\rmfamily\fontsize{10.000000}{12.000000}\selectfont \(\displaystyle {0.25}\)}%
\end{pgfscope}%
\begin{pgfscope}%
\pgfsetbuttcap%
\pgfsetroundjoin%
\definecolor{currentfill}{rgb}{0.000000,0.000000,0.000000}%
\pgfsetfillcolor{currentfill}%
\pgfsetlinewidth{0.803000pt}%
\definecolor{currentstroke}{rgb}{0.000000,0.000000,0.000000}%
\pgfsetstrokecolor{currentstroke}%
\pgfsetdash{}{0pt}%
\pgfsys@defobject{currentmarker}{\pgfqpoint{-0.048611in}{0.000000in}}{\pgfqpoint{-0.000000in}{0.000000in}}{%
\pgfpathmoveto{\pgfqpoint{-0.000000in}{0.000000in}}%
\pgfpathlineto{\pgfqpoint{-0.048611in}{0.000000in}}%
\pgfusepath{stroke,fill}%
}%
\begin{pgfscope}%
\pgfsys@transformshift{0.763041in}{8.796856in}%
\pgfsys@useobject{currentmarker}{}%
\end{pgfscope}%
\end{pgfscope}%
\begin{pgfscope}%
\definecolor{textcolor}{rgb}{0.000000,0.000000,0.000000}%
\pgfsetstrokecolor{textcolor}%
\pgfsetfillcolor{textcolor}%
\pgftext[x=0.418904in, y=8.748630in, left, base]{\color{textcolor}\rmfamily\fontsize{10.000000}{12.000000}\selectfont \(\displaystyle {0.50}\)}%
\end{pgfscope}%
\begin{pgfscope}%
\pgfsetbuttcap%
\pgfsetroundjoin%
\definecolor{currentfill}{rgb}{0.000000,0.000000,0.000000}%
\pgfsetfillcolor{currentfill}%
\pgfsetlinewidth{0.803000pt}%
\definecolor{currentstroke}{rgb}{0.000000,0.000000,0.000000}%
\pgfsetstrokecolor{currentstroke}%
\pgfsetdash{}{0pt}%
\pgfsys@defobject{currentmarker}{\pgfqpoint{-0.048611in}{0.000000in}}{\pgfqpoint{-0.000000in}{0.000000in}}{%
\pgfpathmoveto{\pgfqpoint{-0.000000in}{0.000000in}}%
\pgfpathlineto{\pgfqpoint{-0.048611in}{0.000000in}}%
\pgfusepath{stroke,fill}%
}%
\begin{pgfscope}%
\pgfsys@transformshift{0.763041in}{9.146752in}%
\pgfsys@useobject{currentmarker}{}%
\end{pgfscope}%
\end{pgfscope}%
\begin{pgfscope}%
\definecolor{textcolor}{rgb}{0.000000,0.000000,0.000000}%
\pgfsetstrokecolor{textcolor}%
\pgfsetfillcolor{textcolor}%
\pgftext[x=0.418904in, y=9.098527in, left, base]{\color{textcolor}\rmfamily\fontsize{10.000000}{12.000000}\selectfont \(\displaystyle {0.75}\)}%
\end{pgfscope}%
\begin{pgfscope}%
\pgfsetbuttcap%
\pgfsetroundjoin%
\definecolor{currentfill}{rgb}{0.000000,0.000000,0.000000}%
\pgfsetfillcolor{currentfill}%
\pgfsetlinewidth{0.803000pt}%
\definecolor{currentstroke}{rgb}{0.000000,0.000000,0.000000}%
\pgfsetstrokecolor{currentstroke}%
\pgfsetdash{}{0pt}%
\pgfsys@defobject{currentmarker}{\pgfqpoint{-0.048611in}{0.000000in}}{\pgfqpoint{-0.000000in}{0.000000in}}{%
\pgfpathmoveto{\pgfqpoint{-0.000000in}{0.000000in}}%
\pgfpathlineto{\pgfqpoint{-0.048611in}{0.000000in}}%
\pgfusepath{stroke,fill}%
}%
\begin{pgfscope}%
\pgfsys@transformshift{0.763041in}{9.496649in}%
\pgfsys@useobject{currentmarker}{}%
\end{pgfscope}%
\end{pgfscope}%
\begin{pgfscope}%
\definecolor{textcolor}{rgb}{0.000000,0.000000,0.000000}%
\pgfsetstrokecolor{textcolor}%
\pgfsetfillcolor{textcolor}%
\pgftext[x=0.418904in, y=9.448424in, left, base]{\color{textcolor}\rmfamily\fontsize{10.000000}{12.000000}\selectfont \(\displaystyle {1.00}\)}%
\end{pgfscope}%
\begin{pgfscope}%
\definecolor{textcolor}{rgb}{0.000000,0.000000,0.000000}%
\pgfsetstrokecolor{textcolor}%
\pgfsetfillcolor{textcolor}%
\pgftext[x=0.363349in,y=8.796856in,,bottom,rotate=90.000000]{\color{textcolor}\rmfamily\fontsize{16.000000}{19.200000}\selectfont TPR}%
\end{pgfscope}%
\begin{pgfscope}%
\pgfpathrectangle{\pgfqpoint{0.763041in}{8.027083in}}{\pgfqpoint{1.687305in}{1.539545in}}%
\pgfusepath{clip}%
\pgfsetrectcap%
\pgfsetroundjoin%
\pgfsetlinewidth{1.505625pt}%
\definecolor{currentstroke}{rgb}{0.000000,0.501961,0.000000}%
\pgfsetstrokecolor{currentstroke}%
\pgfsetdash{}{0pt}%
\pgfpathmoveto{\pgfqpoint{0.839736in}{8.097062in}}%
\pgfpathlineto{\pgfqpoint{0.841220in}{8.141577in}}%
\pgfpathlineto{\pgfqpoint{0.841432in}{8.142886in}}%
\pgfpathlineto{\pgfqpoint{0.842862in}{8.204421in}}%
\pgfpathlineto{\pgfqpoint{0.842968in}{8.204421in}}%
\pgfpathlineto{\pgfqpoint{0.843127in}{8.205730in}}%
\pgfpathlineto{\pgfqpoint{0.844611in}{8.224059in}}%
\pgfpathlineto{\pgfqpoint{0.844717in}{8.224059in}}%
\pgfpathlineto{\pgfqpoint{0.846253in}{8.256791in}}%
\pgfpathlineto{\pgfqpoint{0.846359in}{8.256791in}}%
\pgfpathlineto{\pgfqpoint{0.847896in}{8.294759in}}%
\pgfpathlineto{\pgfqpoint{0.848002in}{8.296068in}}%
\pgfpathlineto{\pgfqpoint{0.849432in}{8.324872in}}%
\pgfpathlineto{\pgfqpoint{0.849856in}{8.324872in}}%
\pgfpathlineto{\pgfqpoint{0.851393in}{8.351057in}}%
\pgfpathlineto{\pgfqpoint{0.851711in}{8.352366in}}%
\pgfpathlineto{\pgfqpoint{0.853194in}{8.373314in}}%
\pgfpathlineto{\pgfqpoint{0.853300in}{8.373314in}}%
\pgfpathlineto{\pgfqpoint{0.854837in}{8.409973in}}%
\pgfpathlineto{\pgfqpoint{0.855260in}{8.411282in}}%
\pgfpathlineto{\pgfqpoint{0.856426in}{8.438776in}}%
\pgfpathlineto{\pgfqpoint{0.857274in}{8.440085in}}%
\pgfpathlineto{\pgfqpoint{0.858757in}{8.453178in}}%
\pgfpathlineto{\pgfqpoint{0.859764in}{8.454487in}}%
\pgfpathlineto{\pgfqpoint{0.861248in}{8.466270in}}%
\pgfpathlineto{\pgfqpoint{0.861618in}{8.467580in}}%
\pgfpathlineto{\pgfqpoint{0.862890in}{8.480672in}}%
\pgfpathlineto{\pgfqpoint{0.863314in}{8.481981in}}%
\pgfpathlineto{\pgfqpoint{0.864638in}{8.504239in}}%
\pgfpathlineto{\pgfqpoint{0.865433in}{8.505548in}}%
\pgfpathlineto{\pgfqpoint{0.866864in}{8.523877in}}%
\pgfpathlineto{\pgfqpoint{0.867023in}{8.523877in}}%
\pgfpathlineto{\pgfqpoint{0.868506in}{8.542207in}}%
\pgfpathlineto{\pgfqpoint{0.869036in}{8.543516in}}%
\pgfpathlineto{\pgfqpoint{0.870414in}{8.557918in}}%
\pgfpathlineto{\pgfqpoint{0.870785in}{8.559227in}}%
\pgfpathlineto{\pgfqpoint{0.872268in}{8.574938in}}%
\pgfpathlineto{\pgfqpoint{0.872427in}{8.576247in}}%
\pgfpathlineto{\pgfqpoint{0.873540in}{8.588031in}}%
\pgfpathlineto{\pgfqpoint{0.874069in}{8.589340in}}%
\pgfpathlineto{\pgfqpoint{0.875447in}{8.605051in}}%
\pgfpathlineto{\pgfqpoint{0.875712in}{8.605051in}}%
\pgfpathlineto{\pgfqpoint{0.877142in}{8.628617in}}%
\pgfpathlineto{\pgfqpoint{0.878202in}{8.629926in}}%
\pgfpathlineto{\pgfqpoint{0.879262in}{8.643019in}}%
\pgfpathlineto{\pgfqpoint{0.881275in}{8.644328in}}%
\pgfpathlineto{\pgfqpoint{0.882600in}{8.653493in}}%
\pgfpathlineto{\pgfqpoint{0.882971in}{8.653493in}}%
\pgfpathlineto{\pgfqpoint{0.884295in}{8.667895in}}%
\pgfpathlineto{\pgfqpoint{0.884772in}{8.667895in}}%
\pgfpathlineto{\pgfqpoint{0.886309in}{8.678369in}}%
\pgfpathlineto{\pgfqpoint{0.886732in}{8.679678in}}%
\pgfpathlineto{\pgfqpoint{0.888269in}{8.694080in}}%
\pgfpathlineto{\pgfqpoint{0.888746in}{8.695389in}}%
\pgfpathlineto{\pgfqpoint{0.890282in}{8.708481in}}%
\pgfpathlineto{\pgfqpoint{0.890706in}{8.708481in}}%
\pgfpathlineto{\pgfqpoint{0.892137in}{8.718955in}}%
\pgfpathlineto{\pgfqpoint{0.892720in}{8.720265in}}%
\pgfpathlineto{\pgfqpoint{0.894150in}{8.728120in}}%
\pgfpathlineto{\pgfqpoint{0.894415in}{8.728120in}}%
\pgfpathlineto{\pgfqpoint{0.895210in}{8.734666in}}%
\pgfpathlineto{\pgfqpoint{0.896428in}{8.735976in}}%
\pgfpathlineto{\pgfqpoint{0.897806in}{8.747759in}}%
\pgfpathlineto{\pgfqpoint{0.898760in}{8.749068in}}%
\pgfpathlineto{\pgfqpoint{0.900296in}{8.759542in}}%
\pgfpathlineto{\pgfqpoint{0.901144in}{8.760851in}}%
\pgfpathlineto{\pgfqpoint{0.902468in}{8.764779in}}%
\pgfpathlineto{\pgfqpoint{0.902839in}{8.766088in}}%
\pgfpathlineto{\pgfqpoint{0.904217in}{8.775253in}}%
\pgfpathlineto{\pgfqpoint{0.904588in}{8.776562in}}%
\pgfpathlineto{\pgfqpoint{0.906124in}{8.785727in}}%
\pgfpathlineto{\pgfqpoint{0.906760in}{8.787036in}}%
\pgfpathlineto{\pgfqpoint{0.908191in}{8.794892in}}%
\pgfpathlineto{\pgfqpoint{0.908561in}{8.796201in}}%
\pgfpathlineto{\pgfqpoint{0.910045in}{8.805366in}}%
\pgfpathlineto{\pgfqpoint{0.911158in}{8.806675in}}%
\pgfpathlineto{\pgfqpoint{0.912535in}{8.811912in}}%
\pgfpathlineto{\pgfqpoint{0.914549in}{8.813221in}}%
\pgfpathlineto{\pgfqpoint{0.914813in}{8.815840in}}%
\pgfpathlineto{\pgfqpoint{0.916880in}{8.817149in}}%
\pgfpathlineto{\pgfqpoint{0.917939in}{8.821077in}}%
\pgfpathlineto{\pgfqpoint{0.919052in}{8.822386in}}%
\pgfpathlineto{\pgfqpoint{0.920006in}{8.828932in}}%
\pgfpathlineto{\pgfqpoint{0.920642in}{8.828932in}}%
\pgfpathlineto{\pgfqpoint{0.921807in}{8.836788in}}%
\pgfpathlineto{\pgfqpoint{0.922655in}{8.838097in}}%
\pgfpathlineto{\pgfqpoint{0.924191in}{8.845952in}}%
\pgfpathlineto{\pgfqpoint{0.925675in}{8.847262in}}%
\pgfpathlineto{\pgfqpoint{0.926788in}{8.857736in}}%
\pgfpathlineto{\pgfqpoint{0.928536in}{8.859045in}}%
\pgfpathlineto{\pgfqpoint{0.929861in}{8.866900in}}%
\pgfpathlineto{\pgfqpoint{0.930496in}{8.868210in}}%
\pgfpathlineto{\pgfqpoint{0.931397in}{8.872137in}}%
\pgfpathlineto{\pgfqpoint{0.933199in}{8.873447in}}%
\pgfpathlineto{\pgfqpoint{0.934629in}{8.877374in}}%
\pgfpathlineto{\pgfqpoint{0.935795in}{8.878684in}}%
\pgfpathlineto{\pgfqpoint{0.936695in}{8.889158in}}%
\pgfpathlineto{\pgfqpoint{0.939080in}{8.890467in}}%
\pgfpathlineto{\pgfqpoint{0.940510in}{8.895704in}}%
\pgfpathlineto{\pgfqpoint{0.941782in}{8.897013in}}%
\pgfpathlineto{\pgfqpoint{0.942842in}{8.903559in}}%
\pgfpathlineto{\pgfqpoint{0.944325in}{8.904869in}}%
\pgfpathlineto{\pgfqpoint{0.945491in}{8.910106in}}%
\pgfpathlineto{\pgfqpoint{0.946391in}{8.911415in}}%
\pgfpathlineto{\pgfqpoint{0.947610in}{8.917961in}}%
\pgfpathlineto{\pgfqpoint{0.949782in}{8.919270in}}%
\pgfpathlineto{\pgfqpoint{0.951266in}{8.924507in}}%
\pgfpathlineto{\pgfqpoint{0.954286in}{8.925817in}}%
\pgfpathlineto{\pgfqpoint{0.955293in}{8.928435in}}%
\pgfpathlineto{\pgfqpoint{0.957995in}{8.929744in}}%
\pgfpathlineto{\pgfqpoint{0.959372in}{8.938909in}}%
\pgfpathlineto{\pgfqpoint{0.960114in}{8.940218in}}%
\pgfpathlineto{\pgfqpoint{0.960114in}{8.941528in}}%
\pgfpathlineto{\pgfqpoint{0.962657in}{8.941528in}}%
\pgfpathlineto{\pgfqpoint{0.964141in}{8.949383in}}%
\pgfpathlineto{\pgfqpoint{0.964724in}{8.949383in}}%
\pgfpathlineto{\pgfqpoint{0.965995in}{8.957239in}}%
\pgfpathlineto{\pgfqpoint{0.967002in}{8.957239in}}%
\pgfpathlineto{\pgfqpoint{0.968273in}{8.966403in}}%
\pgfpathlineto{\pgfqpoint{0.969280in}{8.967713in}}%
\pgfpathlineto{\pgfqpoint{0.970234in}{8.972950in}}%
\pgfpathlineto{\pgfqpoint{0.972618in}{8.974259in}}%
\pgfpathlineto{\pgfqpoint{0.972618in}{8.975568in}}%
\pgfpathlineto{\pgfqpoint{0.974896in}{8.976877in}}%
\pgfpathlineto{\pgfqpoint{0.976221in}{8.980805in}}%
\pgfpathlineto{\pgfqpoint{0.977334in}{8.982114in}}%
\pgfpathlineto{\pgfqpoint{0.978870in}{8.988661in}}%
\pgfpathlineto{\pgfqpoint{0.979665in}{8.989970in}}%
\pgfpathlineto{\pgfqpoint{0.981148in}{8.993898in}}%
\pgfpathlineto{\pgfqpoint{0.982314in}{8.995207in}}%
\pgfpathlineto{\pgfqpoint{0.982844in}{8.999134in}}%
\pgfpathlineto{\pgfqpoint{0.984433in}{9.000444in}}%
\pgfpathlineto{\pgfqpoint{0.985864in}{9.006990in}}%
\pgfpathlineto{\pgfqpoint{0.986870in}{9.008299in}}%
\pgfpathlineto{\pgfqpoint{0.988354in}{9.013536in}}%
\pgfpathlineto{\pgfqpoint{0.990791in}{9.014845in}}%
\pgfpathlineto{\pgfqpoint{0.992169in}{9.021392in}}%
\pgfpathlineto{\pgfqpoint{0.992699in}{9.022701in}}%
\pgfpathlineto{\pgfqpoint{0.992699in}{9.024010in}}%
\pgfpathlineto{\pgfqpoint{0.995454in}{9.025319in}}%
\pgfpathlineto{\pgfqpoint{0.995666in}{9.029247in}}%
\pgfpathlineto{\pgfqpoint{0.997997in}{9.030556in}}%
\pgfpathlineto{\pgfqpoint{0.999269in}{9.038412in}}%
\pgfpathlineto{\pgfqpoint{1.000063in}{9.039721in}}%
\pgfpathlineto{\pgfqpoint{1.001017in}{9.043649in}}%
\pgfpathlineto{\pgfqpoint{1.002924in}{9.044958in}}%
\pgfpathlineto{\pgfqpoint{1.004090in}{9.047577in}}%
\pgfpathlineto{\pgfqpoint{1.006898in}{9.048886in}}%
\pgfpathlineto{\pgfqpoint{1.008170in}{9.054123in}}%
\pgfpathlineto{\pgfqpoint{1.012514in}{9.055432in}}%
\pgfpathlineto{\pgfqpoint{1.013309in}{9.060669in}}%
\pgfpathlineto{\pgfqpoint{1.016170in}{9.061978in}}%
\pgfpathlineto{\pgfqpoint{1.017548in}{9.067215in}}%
\pgfpathlineto{\pgfqpoint{1.019561in}{9.068525in}}%
\pgfpathlineto{\pgfqpoint{1.020462in}{9.072452in}}%
\pgfpathlineto{\pgfqpoint{1.022952in}{9.073762in}}%
\pgfpathlineto{\pgfqpoint{1.024436in}{9.077689in}}%
\pgfpathlineto{\pgfqpoint{1.024647in}{9.077689in}}%
\pgfpathlineto{\pgfqpoint{1.024647in}{9.080308in}}%
\pgfpathlineto{\pgfqpoint{1.027562in}{9.081617in}}%
\pgfpathlineto{\pgfqpoint{1.027562in}{9.082926in}}%
\pgfpathlineto{\pgfqpoint{1.030529in}{9.084236in}}%
\pgfpathlineto{\pgfqpoint{1.031747in}{9.088163in}}%
\pgfpathlineto{\pgfqpoint{1.035509in}{9.089473in}}%
\pgfpathlineto{\pgfqpoint{1.036622in}{9.093400in}}%
\pgfpathlineto{\pgfqpoint{1.038582in}{9.094710in}}%
\pgfpathlineto{\pgfqpoint{1.039483in}{9.097328in}}%
\pgfpathlineto{\pgfqpoint{1.042874in}{9.098637in}}%
\pgfpathlineto{\pgfqpoint{1.043456in}{9.101256in}}%
\pgfpathlineto{\pgfqpoint{1.045788in}{9.102565in}}%
\pgfpathlineto{\pgfqpoint{1.045788in}{9.103874in}}%
\pgfpathlineto{\pgfqpoint{1.050715in}{9.105184in}}%
\pgfpathlineto{\pgfqpoint{1.050874in}{9.107802in}}%
\pgfpathlineto{\pgfqpoint{1.055484in}{9.109111in}}%
\pgfpathlineto{\pgfqpoint{1.055484in}{9.110421in}}%
\pgfpathlineto{\pgfqpoint{1.061630in}{9.111730in}}%
\pgfpathlineto{\pgfqpoint{1.061630in}{9.113039in}}%
\pgfpathlineto{\pgfqpoint{1.064703in}{9.113039in}}%
\pgfpathlineto{\pgfqpoint{1.066080in}{9.118276in}}%
\pgfpathlineto{\pgfqpoint{1.067670in}{9.119585in}}%
\pgfpathlineto{\pgfqpoint{1.069100in}{9.122204in}}%
\pgfpathlineto{\pgfqpoint{1.071379in}{9.123513in}}%
\pgfpathlineto{\pgfqpoint{1.071379in}{9.124822in}}%
\pgfpathlineto{\pgfqpoint{1.074399in}{9.126132in}}%
\pgfpathlineto{\pgfqpoint{1.074769in}{9.131369in}}%
\pgfpathlineto{\pgfqpoint{1.076306in}{9.132678in}}%
\pgfpathlineto{\pgfqpoint{1.077578in}{9.136606in}}%
\pgfpathlineto{\pgfqpoint{1.079273in}{9.137915in}}%
\pgfpathlineto{\pgfqpoint{1.080386in}{9.141843in}}%
\pgfpathlineto{\pgfqpoint{1.083565in}{9.143152in}}%
\pgfpathlineto{\pgfqpoint{1.084465in}{9.145770in}}%
\pgfpathlineto{\pgfqpoint{1.086532in}{9.147080in}}%
\pgfpathlineto{\pgfqpoint{1.088015in}{9.151007in}}%
\pgfpathlineto{\pgfqpoint{1.093314in}{9.152317in}}%
\pgfpathlineto{\pgfqpoint{1.094691in}{9.157554in}}%
\pgfpathlineto{\pgfqpoint{1.096810in}{9.158863in}}%
\pgfpathlineto{\pgfqpoint{1.096810in}{9.160172in}}%
\pgfpathlineto{\pgfqpoint{1.099460in}{9.161481in}}%
\pgfpathlineto{\pgfqpoint{1.099460in}{9.162791in}}%
\pgfpathlineto{\pgfqpoint{1.105235in}{9.164100in}}%
\pgfpathlineto{\pgfqpoint{1.105235in}{9.165409in}}%
\pgfpathlineto{\pgfqpoint{1.109897in}{9.166718in}}%
\pgfpathlineto{\pgfqpoint{1.109897in}{9.168028in}}%
\pgfpathlineto{\pgfqpoint{1.115090in}{9.169337in}}%
\pgfpathlineto{\pgfqpoint{1.116361in}{9.173265in}}%
\pgfpathlineto{\pgfqpoint{1.119063in}{9.174574in}}%
\pgfpathlineto{\pgfqpoint{1.120335in}{9.177192in}}%
\pgfpathlineto{\pgfqpoint{1.122189in}{9.178502in}}%
\pgfpathlineto{\pgfqpoint{1.122613in}{9.181120in}}%
\pgfpathlineto{\pgfqpoint{1.127488in}{9.182429in}}%
\pgfpathlineto{\pgfqpoint{1.127488in}{9.183739in}}%
\pgfpathlineto{\pgfqpoint{1.131038in}{9.185048in}}%
\pgfpathlineto{\pgfqpoint{1.131038in}{9.186357in}}%
\pgfpathlineto{\pgfqpoint{1.134428in}{9.187666in}}%
\pgfpathlineto{\pgfqpoint{1.135965in}{9.191594in}}%
\pgfpathlineto{\pgfqpoint{1.138614in}{9.192903in}}%
\pgfpathlineto{\pgfqpoint{1.138826in}{9.195522in}}%
\pgfpathlineto{\pgfqpoint{1.143065in}{9.196831in}}%
\pgfpathlineto{\pgfqpoint{1.144230in}{9.202068in}}%
\pgfpathlineto{\pgfqpoint{1.151330in}{9.203377in}}%
\pgfpathlineto{\pgfqpoint{1.151330in}{9.204686in}}%
\pgfpathlineto{\pgfqpoint{1.154085in}{9.205996in}}%
\pgfpathlineto{\pgfqpoint{1.154138in}{9.208614in}}%
\pgfpathlineto{\pgfqpoint{1.158589in}{9.209923in}}%
\pgfpathlineto{\pgfqpoint{1.158589in}{9.211233in}}%
\pgfpathlineto{\pgfqpoint{1.161503in}{9.212542in}}%
\pgfpathlineto{\pgfqpoint{1.162192in}{9.215160in}}%
\pgfpathlineto{\pgfqpoint{1.165212in}{9.216470in}}%
\pgfpathlineto{\pgfqpoint{1.165371in}{9.220397in}}%
\pgfpathlineto{\pgfqpoint{1.168020in}{9.220397in}}%
\pgfpathlineto{\pgfqpoint{1.168020in}{9.223016in}}%
\pgfpathlineto{\pgfqpoint{1.180948in}{9.224325in}}%
\pgfpathlineto{\pgfqpoint{1.181848in}{9.226944in}}%
\pgfpathlineto{\pgfqpoint{1.182749in}{9.226944in}}%
\pgfpathlineto{\pgfqpoint{1.183862in}{9.230871in}}%
\pgfpathlineto{\pgfqpoint{1.186882in}{9.232181in}}%
\pgfpathlineto{\pgfqpoint{1.186882in}{9.233490in}}%
\pgfpathlineto{\pgfqpoint{1.196525in}{9.234799in}}%
\pgfpathlineto{\pgfqpoint{1.197637in}{9.240036in}}%
\pgfpathlineto{\pgfqpoint{1.200922in}{9.241345in}}%
\pgfpathlineto{\pgfqpoint{1.201187in}{9.243964in}}%
\pgfpathlineto{\pgfqpoint{1.205956in}{9.245273in}}%
\pgfpathlineto{\pgfqpoint{1.207386in}{9.250510in}}%
\pgfpathlineto{\pgfqpoint{1.211572in}{9.251819in}}%
\pgfpathlineto{\pgfqpoint{1.213055in}{9.254438in}}%
\pgfpathlineto{\pgfqpoint{1.225453in}{9.255747in}}%
\pgfpathlineto{\pgfqpoint{1.225453in}{9.257056in}}%
\pgfpathlineto{\pgfqpoint{1.228579in}{9.258366in}}%
\pgfpathlineto{\pgfqpoint{1.228579in}{9.259675in}}%
\pgfpathlineto{\pgfqpoint{1.230540in}{9.259675in}}%
\pgfpathlineto{\pgfqpoint{1.230540in}{9.262293in}}%
\pgfpathlineto{\pgfqpoint{1.239812in}{9.263603in}}%
\pgfpathlineto{\pgfqpoint{1.240289in}{9.266221in}}%
\pgfpathlineto{\pgfqpoint{1.246011in}{9.267530in}}%
\pgfpathlineto{\pgfqpoint{1.246488in}{9.270149in}}%
\pgfpathlineto{\pgfqpoint{1.249349in}{9.271458in}}%
\pgfpathlineto{\pgfqpoint{1.249349in}{9.272767in}}%
\pgfpathlineto{\pgfqpoint{1.253746in}{9.274077in}}%
\pgfpathlineto{\pgfqpoint{1.254117in}{9.278004in}}%
\pgfpathlineto{\pgfqpoint{1.259681in}{9.279314in}}%
\pgfpathlineto{\pgfqpoint{1.259892in}{9.281932in}}%
\pgfpathlineto{\pgfqpoint{1.274304in}{9.283241in}}%
\pgfpathlineto{\pgfqpoint{1.275205in}{9.285860in}}%
\pgfpathlineto{\pgfqpoint{1.278119in}{9.287169in}}%
\pgfpathlineto{\pgfqpoint{1.278119in}{9.288478in}}%
\pgfpathlineto{\pgfqpoint{1.281510in}{9.289788in}}%
\pgfpathlineto{\pgfqpoint{1.281510in}{9.291097in}}%
\pgfpathlineto{\pgfqpoint{1.288132in}{9.292406in}}%
\pgfpathlineto{\pgfqpoint{1.289192in}{9.296334in}}%
\pgfpathlineto{\pgfqpoint{1.290570in}{9.297643in}}%
\pgfpathlineto{\pgfqpoint{1.290570in}{9.298952in}}%
\pgfpathlineto{\pgfqpoint{1.299100in}{9.300262in}}%
\pgfpathlineto{\pgfqpoint{1.299100in}{9.301571in}}%
\pgfpathlineto{\pgfqpoint{1.300636in}{9.301571in}}%
\pgfpathlineto{\pgfqpoint{1.310703in}{9.302880in}}%
\pgfpathlineto{\pgfqpoint{1.311922in}{9.306808in}}%
\pgfpathlineto{\pgfqpoint{1.315896in}{9.308117in}}%
\pgfpathlineto{\pgfqpoint{1.317167in}{9.312045in}}%
\pgfpathlineto{\pgfqpoint{1.319869in}{9.313354in}}%
\pgfpathlineto{\pgfqpoint{1.319869in}{9.314663in}}%
\pgfpathlineto{\pgfqpoint{1.327499in}{9.315973in}}%
\pgfpathlineto{\pgfqpoint{1.327499in}{9.317282in}}%
\pgfpathlineto{\pgfqpoint{1.334016in}{9.318591in}}%
\pgfpathlineto{\pgfqpoint{1.334016in}{9.319900in}}%
\pgfpathlineto{\pgfqpoint{1.341328in}{9.321210in}}%
\pgfpathlineto{\pgfqpoint{1.341328in}{9.322519in}}%
\pgfpathlineto{\pgfqpoint{1.347209in}{9.323828in}}%
\pgfpathlineto{\pgfqpoint{1.347209in}{9.325137in}}%
\pgfpathlineto{\pgfqpoint{1.356481in}{9.326447in}}%
\pgfpathlineto{\pgfqpoint{1.356481in}{9.327756in}}%
\pgfpathlineto{\pgfqpoint{1.366812in}{9.329065in}}%
\pgfpathlineto{\pgfqpoint{1.366812in}{9.330374in}}%
\pgfpathlineto{\pgfqpoint{1.369515in}{9.331684in}}%
\pgfpathlineto{\pgfqpoint{1.369515in}{9.332993in}}%
\pgfpathlineto{\pgfqpoint{1.382336in}{9.334302in}}%
\pgfpathlineto{\pgfqpoint{1.382442in}{9.336921in}}%
\pgfpathlineto{\pgfqpoint{1.387423in}{9.338230in}}%
\pgfpathlineto{\pgfqpoint{1.387423in}{9.339539in}}%
\pgfpathlineto{\pgfqpoint{1.393887in}{9.340848in}}%
\pgfpathlineto{\pgfqpoint{1.393887in}{9.342158in}}%
\pgfpathlineto{\pgfqpoint{1.395264in}{9.342158in}}%
\pgfpathlineto{\pgfqpoint{1.398920in}{9.343467in}}%
\pgfpathlineto{\pgfqpoint{1.398920in}{9.344776in}}%
\pgfpathlineto{\pgfqpoint{1.403954in}{9.346085in}}%
\pgfpathlineto{\pgfqpoint{1.403954in}{9.347395in}}%
\pgfpathlineto{\pgfqpoint{1.405437in}{9.347395in}}%
\pgfpathlineto{\pgfqpoint{1.411212in}{9.348704in}}%
\pgfpathlineto{\pgfqpoint{1.411530in}{9.351322in}}%
\pgfpathlineto{\pgfqpoint{1.418471in}{9.352632in}}%
\pgfpathlineto{\pgfqpoint{1.419478in}{9.355250in}}%
\pgfpathlineto{\pgfqpoint{1.422127in}{9.356559in}}%
\pgfpathlineto{\pgfqpoint{1.423663in}{9.360487in}}%
\pgfpathlineto{\pgfqpoint{1.428909in}{9.361796in}}%
\pgfpathlineto{\pgfqpoint{1.428909in}{9.363106in}}%
\pgfpathlineto{\pgfqpoint{1.431346in}{9.364415in}}%
\pgfpathlineto{\pgfqpoint{1.431717in}{9.367033in}}%
\pgfpathlineto{\pgfqpoint{1.431823in}{9.367033in}}%
\pgfpathlineto{\pgfqpoint{1.439929in}{9.368343in}}%
\pgfpathlineto{\pgfqpoint{1.439929in}{9.369652in}}%
\pgfpathlineto{\pgfqpoint{1.441731in}{9.370961in}}%
\pgfpathlineto{\pgfqpoint{1.441731in}{9.372270in}}%
\pgfpathlineto{\pgfqpoint{1.459957in}{9.373580in}}%
\pgfpathlineto{\pgfqpoint{1.459957in}{9.374889in}}%
\pgfpathlineto{\pgfqpoint{1.466156in}{9.376198in}}%
\pgfpathlineto{\pgfqpoint{1.466156in}{9.377507in}}%
\pgfpathlineto{\pgfqpoint{1.475057in}{9.378817in}}%
\pgfpathlineto{\pgfqpoint{1.475852in}{9.381435in}}%
\pgfpathlineto{\pgfqpoint{1.482263in}{9.382744in}}%
\pgfpathlineto{\pgfqpoint{1.483587in}{9.385363in}}%
\pgfpathlineto{\pgfqpoint{1.508966in}{9.386672in}}%
\pgfpathlineto{\pgfqpoint{1.510291in}{9.389291in}}%
\pgfpathlineto{\pgfqpoint{1.516225in}{9.390600in}}%
\pgfpathlineto{\pgfqpoint{1.516225in}{9.391909in}}%
\pgfpathlineto{\pgfqpoint{1.521947in}{9.393218in}}%
\pgfpathlineto{\pgfqpoint{1.522583in}{9.395837in}}%
\pgfpathlineto{\pgfqpoint{1.532438in}{9.397146in}}%
\pgfpathlineto{\pgfqpoint{1.532438in}{9.398455in}}%
\pgfpathlineto{\pgfqpoint{1.553472in}{9.399764in}}%
\pgfpathlineto{\pgfqpoint{1.553472in}{9.401074in}}%
\pgfpathlineto{\pgfqpoint{1.582295in}{9.402383in}}%
\pgfpathlineto{\pgfqpoint{1.582295in}{9.403692in}}%
\pgfpathlineto{\pgfqpoint{1.588706in}{9.405001in}}%
\pgfpathlineto{\pgfqpoint{1.588706in}{9.406311in}}%
\pgfpathlineto{\pgfqpoint{1.602852in}{9.407620in}}%
\pgfpathlineto{\pgfqpoint{1.602852in}{9.408929in}}%
\pgfpathlineto{\pgfqpoint{1.609475in}{9.410238in}}%
\pgfpathlineto{\pgfqpoint{1.609793in}{9.412857in}}%
\pgfpathlineto{\pgfqpoint{1.622456in}{9.414166in}}%
\pgfpathlineto{\pgfqpoint{1.622456in}{9.415475in}}%
\pgfpathlineto{\pgfqpoint{1.628602in}{9.416785in}}%
\pgfpathlineto{\pgfqpoint{1.628602in}{9.418094in}}%
\pgfpathlineto{\pgfqpoint{1.641212in}{9.419403in}}%
\pgfpathlineto{\pgfqpoint{1.641212in}{9.420712in}}%
\pgfpathlineto{\pgfqpoint{1.646033in}{9.422022in}}%
\pgfpathlineto{\pgfqpoint{1.646033in}{9.423331in}}%
\pgfpathlineto{\pgfqpoint{1.658961in}{9.424640in}}%
\pgfpathlineto{\pgfqpoint{1.659862in}{9.427259in}}%
\pgfpathlineto{\pgfqpoint{1.676128in}{9.428568in}}%
\pgfpathlineto{\pgfqpoint{1.676128in}{9.429877in}}%
\pgfpathlineto{\pgfqpoint{1.695626in}{9.431186in}}%
\pgfpathlineto{\pgfqpoint{1.695626in}{9.432496in}}%
\pgfpathlineto{\pgfqpoint{1.704050in}{9.433805in}}%
\pgfpathlineto{\pgfqpoint{1.704792in}{9.436423in}}%
\pgfpathlineto{\pgfqpoint{1.732184in}{9.437733in}}%
\pgfpathlineto{\pgfqpoint{1.732184in}{9.439042in}}%
\pgfpathlineto{\pgfqpoint{1.751576in}{9.440351in}}%
\pgfpathlineto{\pgfqpoint{1.751576in}{9.441660in}}%
\pgfpathlineto{\pgfqpoint{1.764716in}{9.442970in}}%
\pgfpathlineto{\pgfqpoint{1.766199in}{9.445588in}}%
\pgfpathlineto{\pgfqpoint{1.792161in}{9.446897in}}%
\pgfpathlineto{\pgfqpoint{1.792161in}{9.448207in}}%
\pgfpathlineto{\pgfqpoint{1.815420in}{9.449516in}}%
\pgfpathlineto{\pgfqpoint{1.815420in}{9.450825in}}%
\pgfpathlineto{\pgfqpoint{1.818440in}{9.452134in}}%
\pgfpathlineto{\pgfqpoint{1.818440in}{9.453444in}}%
\pgfpathlineto{\pgfqpoint{1.826017in}{9.454753in}}%
\pgfpathlineto{\pgfqpoint{1.826017in}{9.456062in}}%
\pgfpathlineto{\pgfqpoint{1.833700in}{9.457371in}}%
\pgfpathlineto{\pgfqpoint{1.835236in}{9.459990in}}%
\pgfpathlineto{\pgfqpoint{1.852615in}{9.461299in}}%
\pgfpathlineto{\pgfqpoint{1.852615in}{9.462608in}}%
\pgfpathlineto{\pgfqpoint{1.866284in}{9.463918in}}%
\pgfpathlineto{\pgfqpoint{1.866284in}{9.465227in}}%
\pgfpathlineto{\pgfqpoint{1.867609in}{9.465227in}}%
\pgfpathlineto{\pgfqpoint{1.894100in}{9.466536in}}%
\pgfpathlineto{\pgfqpoint{1.894100in}{9.467845in}}%
\pgfpathlineto{\pgfqpoint{1.923294in}{9.469155in}}%
\pgfpathlineto{\pgfqpoint{1.923294in}{9.470464in}}%
\pgfpathlineto{\pgfqpoint{1.935480in}{9.471773in}}%
\pgfpathlineto{\pgfqpoint{1.935480in}{9.473082in}}%
\pgfpathlineto{\pgfqpoint{1.963932in}{9.474392in}}%
\pgfpathlineto{\pgfqpoint{1.963932in}{9.475701in}}%
\pgfpathlineto{\pgfqpoint{1.989258in}{9.477010in}}%
\pgfpathlineto{\pgfqpoint{1.989258in}{9.478319in}}%
\pgfpathlineto{\pgfqpoint{2.013789in}{9.479629in}}%
\pgfpathlineto{\pgfqpoint{2.013789in}{9.480938in}}%
\pgfpathlineto{\pgfqpoint{2.058931in}{9.482247in}}%
\pgfpathlineto{\pgfqpoint{2.058931in}{9.483556in}}%
\pgfpathlineto{\pgfqpoint{2.078270in}{9.484866in}}%
\pgfpathlineto{\pgfqpoint{2.078270in}{9.486175in}}%
\pgfpathlineto{\pgfqpoint{2.095701in}{9.487484in}}%
\pgfpathlineto{\pgfqpoint{2.095701in}{9.488793in}}%
\pgfpathlineto{\pgfqpoint{2.145929in}{9.490103in}}%
\pgfpathlineto{\pgfqpoint{2.146565in}{9.492721in}}%
\pgfpathlineto{\pgfqpoint{2.280135in}{9.494030in}}%
\pgfpathlineto{\pgfqpoint{2.280135in}{9.495340in}}%
\pgfpathlineto{\pgfqpoint{2.373650in}{9.496649in}}%
\pgfpathlineto{\pgfqpoint{2.373650in}{9.496649in}}%
\pgfusepath{stroke}%
\end{pgfscope}%
\begin{pgfscope}%
\pgfpathrectangle{\pgfqpoint{0.763041in}{8.027083in}}{\pgfqpoint{1.687305in}{1.539545in}}%
\pgfusepath{clip}%
\pgfsetrectcap%
\pgfsetroundjoin%
\pgfsetlinewidth{1.505625pt}%
\definecolor{currentstroke}{rgb}{0.501961,0.501961,0.501961}%
\pgfsetstrokecolor{currentstroke}%
\pgfsetdash{}{0pt}%
\pgfpathmoveto{\pgfqpoint{0.839736in}{8.097062in}}%
\pgfpathlineto{\pgfqpoint{2.373650in}{9.496649in}}%
\pgfusepath{stroke}%
\end{pgfscope}%
\begin{pgfscope}%
\pgfsetrectcap%
\pgfsetmiterjoin%
\pgfsetlinewidth{0.803000pt}%
\definecolor{currentstroke}{rgb}{0.000000,0.000000,0.000000}%
\pgfsetstrokecolor{currentstroke}%
\pgfsetdash{}{0pt}%
\pgfpathmoveto{\pgfqpoint{0.763041in}{8.027083in}}%
\pgfpathlineto{\pgfqpoint{0.763041in}{9.566628in}}%
\pgfusepath{stroke}%
\end{pgfscope}%
\begin{pgfscope}%
\pgfsetrectcap%
\pgfsetmiterjoin%
\pgfsetlinewidth{0.803000pt}%
\definecolor{currentstroke}{rgb}{0.000000,0.000000,0.000000}%
\pgfsetstrokecolor{currentstroke}%
\pgfsetdash{}{0pt}%
\pgfpathmoveto{\pgfqpoint{2.450346in}{8.027083in}}%
\pgfpathlineto{\pgfqpoint{2.450346in}{9.566628in}}%
\pgfusepath{stroke}%
\end{pgfscope}%
\begin{pgfscope}%
\pgfsetrectcap%
\pgfsetmiterjoin%
\pgfsetlinewidth{0.803000pt}%
\definecolor{currentstroke}{rgb}{0.000000,0.000000,0.000000}%
\pgfsetstrokecolor{currentstroke}%
\pgfsetdash{}{0pt}%
\pgfpathmoveto{\pgfqpoint{0.763041in}{8.027083in}}%
\pgfpathlineto{\pgfqpoint{2.450346in}{8.027083in}}%
\pgfusepath{stroke}%
\end{pgfscope}%
\begin{pgfscope}%
\pgfsetrectcap%
\pgfsetmiterjoin%
\pgfsetlinewidth{0.803000pt}%
\definecolor{currentstroke}{rgb}{0.000000,0.000000,0.000000}%
\pgfsetstrokecolor{currentstroke}%
\pgfsetdash{}{0pt}%
\pgfpathmoveto{\pgfqpoint{0.763041in}{9.566628in}}%
\pgfpathlineto{\pgfqpoint{2.450346in}{9.566628in}}%
\pgfusepath{stroke}%
\end{pgfscope}%
\begin{pgfscope}%
\definecolor{textcolor}{rgb}{0.000000,0.000000,0.000000}%
\pgfsetstrokecolor{textcolor}%
\pgfsetfillcolor{textcolor}%
\pgftext[x=1.606693in,y=9.649962in,,base]{\color{textcolor}\rmfamily\fontsize{20.000000}{24.000000}\selectfont Cardiomegaly}%
\end{pgfscope}%
\begin{pgfscope}%
\pgfsetbuttcap%
\pgfsetmiterjoin%
\definecolor{currentfill}{rgb}{1.000000,1.000000,1.000000}%
\pgfsetfillcolor{currentfill}%
\pgfsetfillopacity{0.800000}%
\pgfsetlinewidth{1.003750pt}%
\definecolor{currentstroke}{rgb}{0.800000,0.800000,0.800000}%
\pgfsetstrokecolor{currentstroke}%
\pgfsetstrokeopacity{0.800000}%
\pgfsetdash{}{0pt}%
\pgfpathmoveto{\pgfqpoint{1.241240in}{8.096527in}}%
\pgfpathlineto{\pgfqpoint{2.353124in}{8.096527in}}%
\pgfpathquadraticcurveto{\pgfqpoint{2.380902in}{8.096527in}}{\pgfqpoint{2.380902in}{8.124305in}}%
\pgfpathlineto{\pgfqpoint{2.380902in}{8.304089in}}%
\pgfpathquadraticcurveto{\pgfqpoint{2.380902in}{8.331867in}}{\pgfqpoint{2.353124in}{8.331867in}}%
\pgfpathlineto{\pgfqpoint{1.241240in}{8.331867in}}%
\pgfpathquadraticcurveto{\pgfqpoint{1.213462in}{8.331867in}}{\pgfqpoint{1.213462in}{8.304089in}}%
\pgfpathlineto{\pgfqpoint{1.213462in}{8.124305in}}%
\pgfpathquadraticcurveto{\pgfqpoint{1.213462in}{8.096527in}}{\pgfqpoint{1.241240in}{8.096527in}}%
\pgfpathclose%
\pgfusepath{stroke,fill}%
\end{pgfscope}%
\begin{pgfscope}%
\pgfsetrectcap%
\pgfsetroundjoin%
\pgfsetlinewidth{1.505625pt}%
\definecolor{currentstroke}{rgb}{0.000000,0.501961,0.000000}%
\pgfsetstrokecolor{currentstroke}%
\pgfsetdash{}{0pt}%
\pgfpathmoveto{\pgfqpoint{1.269018in}{8.227700in}}%
\pgfpathlineto{\pgfqpoint{1.546795in}{8.227700in}}%
\pgfusepath{stroke}%
\end{pgfscope}%
\begin{pgfscope}%
\definecolor{textcolor}{rgb}{0.000000,0.000000,0.000000}%
\pgfsetstrokecolor{textcolor}%
\pgfsetfillcolor{textcolor}%
\pgftext[x=1.657907in,y=8.179089in,left,base]{\color{textcolor}\rmfamily\fontsize{10.000000}{12.000000}\selectfont AUC 0.875}%
\end{pgfscope}%
\begin{pgfscope}%
\pgfsetbuttcap%
\pgfsetmiterjoin%
\definecolor{currentfill}{rgb}{1.000000,1.000000,1.000000}%
\pgfsetfillcolor{currentfill}%
\pgfsetlinewidth{0.000000pt}%
\definecolor{currentstroke}{rgb}{0.000000,0.000000,0.000000}%
\pgfsetstrokecolor{currentstroke}%
\pgfsetstrokeopacity{0.000000}%
\pgfsetdash{}{0pt}%
\pgfpathmoveto{\pgfqpoint{3.225541in}{8.027083in}}%
\pgfpathlineto{\pgfqpoint{4.912846in}{8.027083in}}%
\pgfpathlineto{\pgfqpoint{4.912846in}{9.566628in}}%
\pgfpathlineto{\pgfqpoint{3.225541in}{9.566628in}}%
\pgfpathclose%
\pgfusepath{fill}%
\end{pgfscope}%
\begin{pgfscope}%
\pgfsetbuttcap%
\pgfsetroundjoin%
\definecolor{currentfill}{rgb}{0.000000,0.000000,0.000000}%
\pgfsetfillcolor{currentfill}%
\pgfsetlinewidth{0.803000pt}%
\definecolor{currentstroke}{rgb}{0.000000,0.000000,0.000000}%
\pgfsetstrokecolor{currentstroke}%
\pgfsetdash{}{0pt}%
\pgfsys@defobject{currentmarker}{\pgfqpoint{0.000000in}{-0.048611in}}{\pgfqpoint{0.000000in}{0.000000in}}{%
\pgfpathmoveto{\pgfqpoint{0.000000in}{0.000000in}}%
\pgfpathlineto{\pgfqpoint{0.000000in}{-0.048611in}}%
\pgfusepath{stroke,fill}%
}%
\begin{pgfscope}%
\pgfsys@transformshift{3.302236in}{8.027083in}%
\pgfsys@useobject{currentmarker}{}%
\end{pgfscope}%
\end{pgfscope}%
\begin{pgfscope}%
\definecolor{textcolor}{rgb}{0.000000,0.000000,0.000000}%
\pgfsetstrokecolor{textcolor}%
\pgfsetfillcolor{textcolor}%
\pgftext[x=3.302236in,y=7.929861in,,top]{\color{textcolor}\rmfamily\fontsize{10.000000}{12.000000}\selectfont \(\displaystyle {0.0}\)}%
\end{pgfscope}%
\begin{pgfscope}%
\pgfsetbuttcap%
\pgfsetroundjoin%
\definecolor{currentfill}{rgb}{0.000000,0.000000,0.000000}%
\pgfsetfillcolor{currentfill}%
\pgfsetlinewidth{0.803000pt}%
\definecolor{currentstroke}{rgb}{0.000000,0.000000,0.000000}%
\pgfsetstrokecolor{currentstroke}%
\pgfsetdash{}{0pt}%
\pgfsys@defobject{currentmarker}{\pgfqpoint{0.000000in}{-0.048611in}}{\pgfqpoint{0.000000in}{0.000000in}}{%
\pgfpathmoveto{\pgfqpoint{0.000000in}{0.000000in}}%
\pgfpathlineto{\pgfqpoint{0.000000in}{-0.048611in}}%
\pgfusepath{stroke,fill}%
}%
\begin{pgfscope}%
\pgfsys@transformshift{4.069193in}{8.027083in}%
\pgfsys@useobject{currentmarker}{}%
\end{pgfscope}%
\end{pgfscope}%
\begin{pgfscope}%
\definecolor{textcolor}{rgb}{0.000000,0.000000,0.000000}%
\pgfsetstrokecolor{textcolor}%
\pgfsetfillcolor{textcolor}%
\pgftext[x=4.069193in,y=7.929861in,,top]{\color{textcolor}\rmfamily\fontsize{10.000000}{12.000000}\selectfont \(\displaystyle {0.5}\)}%
\end{pgfscope}%
\begin{pgfscope}%
\pgfsetbuttcap%
\pgfsetroundjoin%
\definecolor{currentfill}{rgb}{0.000000,0.000000,0.000000}%
\pgfsetfillcolor{currentfill}%
\pgfsetlinewidth{0.803000pt}%
\definecolor{currentstroke}{rgb}{0.000000,0.000000,0.000000}%
\pgfsetstrokecolor{currentstroke}%
\pgfsetdash{}{0pt}%
\pgfsys@defobject{currentmarker}{\pgfqpoint{0.000000in}{-0.048611in}}{\pgfqpoint{0.000000in}{0.000000in}}{%
\pgfpathmoveto{\pgfqpoint{0.000000in}{0.000000in}}%
\pgfpathlineto{\pgfqpoint{0.000000in}{-0.048611in}}%
\pgfusepath{stroke,fill}%
}%
\begin{pgfscope}%
\pgfsys@transformshift{4.836150in}{8.027083in}%
\pgfsys@useobject{currentmarker}{}%
\end{pgfscope}%
\end{pgfscope}%
\begin{pgfscope}%
\definecolor{textcolor}{rgb}{0.000000,0.000000,0.000000}%
\pgfsetstrokecolor{textcolor}%
\pgfsetfillcolor{textcolor}%
\pgftext[x=4.836150in,y=7.929861in,,top]{\color{textcolor}\rmfamily\fontsize{10.000000}{12.000000}\selectfont \(\displaystyle {1.0}\)}%
\end{pgfscope}%
\begin{pgfscope}%
\definecolor{textcolor}{rgb}{0.000000,0.000000,0.000000}%
\pgfsetstrokecolor{textcolor}%
\pgfsetfillcolor{textcolor}%
\pgftext[x=4.069193in,y=7.750849in,,top]{\color{textcolor}\rmfamily\fontsize{16.000000}{19.200000}\selectfont FPR}%
\end{pgfscope}%
\begin{pgfscope}%
\pgfsetbuttcap%
\pgfsetroundjoin%
\definecolor{currentfill}{rgb}{0.000000,0.000000,0.000000}%
\pgfsetfillcolor{currentfill}%
\pgfsetlinewidth{0.803000pt}%
\definecolor{currentstroke}{rgb}{0.000000,0.000000,0.000000}%
\pgfsetstrokecolor{currentstroke}%
\pgfsetdash{}{0pt}%
\pgfsys@defobject{currentmarker}{\pgfqpoint{-0.048611in}{0.000000in}}{\pgfqpoint{-0.000000in}{0.000000in}}{%
\pgfpathmoveto{\pgfqpoint{-0.000000in}{0.000000in}}%
\pgfpathlineto{\pgfqpoint{-0.048611in}{0.000000in}}%
\pgfusepath{stroke,fill}%
}%
\begin{pgfscope}%
\pgfsys@transformshift{3.225541in}{8.097062in}%
\pgfsys@useobject{currentmarker}{}%
\end{pgfscope}%
\end{pgfscope}%
\begin{pgfscope}%
\definecolor{textcolor}{rgb}{0.000000,0.000000,0.000000}%
\pgfsetstrokecolor{textcolor}%
\pgfsetfillcolor{textcolor}%
\pgftext[x=2.881404in, y=8.048837in, left, base]{\color{textcolor}\rmfamily\fontsize{10.000000}{12.000000}\selectfont \(\displaystyle {0.00}\)}%
\end{pgfscope}%
\begin{pgfscope}%
\pgfsetbuttcap%
\pgfsetroundjoin%
\definecolor{currentfill}{rgb}{0.000000,0.000000,0.000000}%
\pgfsetfillcolor{currentfill}%
\pgfsetlinewidth{0.803000pt}%
\definecolor{currentstroke}{rgb}{0.000000,0.000000,0.000000}%
\pgfsetstrokecolor{currentstroke}%
\pgfsetdash{}{0pt}%
\pgfsys@defobject{currentmarker}{\pgfqpoint{-0.048611in}{0.000000in}}{\pgfqpoint{-0.000000in}{0.000000in}}{%
\pgfpathmoveto{\pgfqpoint{-0.000000in}{0.000000in}}%
\pgfpathlineto{\pgfqpoint{-0.048611in}{0.000000in}}%
\pgfusepath{stroke,fill}%
}%
\begin{pgfscope}%
\pgfsys@transformshift{3.225541in}{8.446959in}%
\pgfsys@useobject{currentmarker}{}%
\end{pgfscope}%
\end{pgfscope}%
\begin{pgfscope}%
\definecolor{textcolor}{rgb}{0.000000,0.000000,0.000000}%
\pgfsetstrokecolor{textcolor}%
\pgfsetfillcolor{textcolor}%
\pgftext[x=2.881404in, y=8.398734in, left, base]{\color{textcolor}\rmfamily\fontsize{10.000000}{12.000000}\selectfont \(\displaystyle {0.25}\)}%
\end{pgfscope}%
\begin{pgfscope}%
\pgfsetbuttcap%
\pgfsetroundjoin%
\definecolor{currentfill}{rgb}{0.000000,0.000000,0.000000}%
\pgfsetfillcolor{currentfill}%
\pgfsetlinewidth{0.803000pt}%
\definecolor{currentstroke}{rgb}{0.000000,0.000000,0.000000}%
\pgfsetstrokecolor{currentstroke}%
\pgfsetdash{}{0pt}%
\pgfsys@defobject{currentmarker}{\pgfqpoint{-0.048611in}{0.000000in}}{\pgfqpoint{-0.000000in}{0.000000in}}{%
\pgfpathmoveto{\pgfqpoint{-0.000000in}{0.000000in}}%
\pgfpathlineto{\pgfqpoint{-0.048611in}{0.000000in}}%
\pgfusepath{stroke,fill}%
}%
\begin{pgfscope}%
\pgfsys@transformshift{3.225541in}{8.796856in}%
\pgfsys@useobject{currentmarker}{}%
\end{pgfscope}%
\end{pgfscope}%
\begin{pgfscope}%
\definecolor{textcolor}{rgb}{0.000000,0.000000,0.000000}%
\pgfsetstrokecolor{textcolor}%
\pgfsetfillcolor{textcolor}%
\pgftext[x=2.881404in, y=8.748630in, left, base]{\color{textcolor}\rmfamily\fontsize{10.000000}{12.000000}\selectfont \(\displaystyle {0.50}\)}%
\end{pgfscope}%
\begin{pgfscope}%
\pgfsetbuttcap%
\pgfsetroundjoin%
\definecolor{currentfill}{rgb}{0.000000,0.000000,0.000000}%
\pgfsetfillcolor{currentfill}%
\pgfsetlinewidth{0.803000pt}%
\definecolor{currentstroke}{rgb}{0.000000,0.000000,0.000000}%
\pgfsetstrokecolor{currentstroke}%
\pgfsetdash{}{0pt}%
\pgfsys@defobject{currentmarker}{\pgfqpoint{-0.048611in}{0.000000in}}{\pgfqpoint{-0.000000in}{0.000000in}}{%
\pgfpathmoveto{\pgfqpoint{-0.000000in}{0.000000in}}%
\pgfpathlineto{\pgfqpoint{-0.048611in}{0.000000in}}%
\pgfusepath{stroke,fill}%
}%
\begin{pgfscope}%
\pgfsys@transformshift{3.225541in}{9.146752in}%
\pgfsys@useobject{currentmarker}{}%
\end{pgfscope}%
\end{pgfscope}%
\begin{pgfscope}%
\definecolor{textcolor}{rgb}{0.000000,0.000000,0.000000}%
\pgfsetstrokecolor{textcolor}%
\pgfsetfillcolor{textcolor}%
\pgftext[x=2.881404in, y=9.098527in, left, base]{\color{textcolor}\rmfamily\fontsize{10.000000}{12.000000}\selectfont \(\displaystyle {0.75}\)}%
\end{pgfscope}%
\begin{pgfscope}%
\pgfsetbuttcap%
\pgfsetroundjoin%
\definecolor{currentfill}{rgb}{0.000000,0.000000,0.000000}%
\pgfsetfillcolor{currentfill}%
\pgfsetlinewidth{0.803000pt}%
\definecolor{currentstroke}{rgb}{0.000000,0.000000,0.000000}%
\pgfsetstrokecolor{currentstroke}%
\pgfsetdash{}{0pt}%
\pgfsys@defobject{currentmarker}{\pgfqpoint{-0.048611in}{0.000000in}}{\pgfqpoint{-0.000000in}{0.000000in}}{%
\pgfpathmoveto{\pgfqpoint{-0.000000in}{0.000000in}}%
\pgfpathlineto{\pgfqpoint{-0.048611in}{0.000000in}}%
\pgfusepath{stroke,fill}%
}%
\begin{pgfscope}%
\pgfsys@transformshift{3.225541in}{9.496649in}%
\pgfsys@useobject{currentmarker}{}%
\end{pgfscope}%
\end{pgfscope}%
\begin{pgfscope}%
\definecolor{textcolor}{rgb}{0.000000,0.000000,0.000000}%
\pgfsetstrokecolor{textcolor}%
\pgfsetfillcolor{textcolor}%
\pgftext[x=2.881404in, y=9.448424in, left, base]{\color{textcolor}\rmfamily\fontsize{10.000000}{12.000000}\selectfont \(\displaystyle {1.00}\)}%
\end{pgfscope}%
\begin{pgfscope}%
\definecolor{textcolor}{rgb}{0.000000,0.000000,0.000000}%
\pgfsetstrokecolor{textcolor}%
\pgfsetfillcolor{textcolor}%
\pgftext[x=2.825849in,y=8.796856in,,bottom,rotate=90.000000]{\color{textcolor}\rmfamily\fontsize{16.000000}{19.200000}\selectfont TPR}%
\end{pgfscope}%
\begin{pgfscope}%
\pgfpathrectangle{\pgfqpoint{3.225541in}{8.027083in}}{\pgfqpoint{1.687305in}{1.539545in}}%
\pgfusepath{clip}%
\pgfsetrectcap%
\pgfsetroundjoin%
\pgfsetlinewidth{1.505625pt}%
\definecolor{currentstroke}{rgb}{0.000000,0.501961,0.000000}%
\pgfsetstrokecolor{currentstroke}%
\pgfsetdash{}{0pt}%
\pgfpathmoveto{\pgfqpoint{3.302236in}{8.097062in}}%
\pgfpathlineto{\pgfqpoint{3.308971in}{8.289137in}}%
\pgfpathlineto{\pgfqpoint{3.309766in}{8.307064in}}%
\pgfpathlineto{\pgfqpoint{3.311251in}{8.364687in}}%
\pgfpathlineto{\pgfqpoint{3.311304in}{8.364687in}}%
\pgfpathlineto{\pgfqpoint{3.322811in}{8.606701in}}%
\pgfpathlineto{\pgfqpoint{3.323235in}{8.607982in}}%
\pgfpathlineto{\pgfqpoint{3.324773in}{8.642555in}}%
\pgfpathlineto{\pgfqpoint{3.324932in}{8.643836in}}%
\pgfpathlineto{\pgfqpoint{3.326311in}{8.665604in}}%
\pgfpathlineto{\pgfqpoint{3.326417in}{8.665604in}}%
\pgfpathlineto{\pgfqpoint{3.326523in}{8.665604in}}%
\pgfpathlineto{\pgfqpoint{3.326523in}{8.666885in}}%
\pgfpathlineto{\pgfqpoint{3.330182in}{8.721946in}}%
\pgfpathlineto{\pgfqpoint{3.330553in}{8.723227in}}%
\pgfpathlineto{\pgfqpoint{3.332038in}{8.739873in}}%
\pgfpathlineto{\pgfqpoint{3.332833in}{8.741154in}}%
\pgfpathlineto{\pgfqpoint{3.334106in}{8.753959in}}%
\pgfpathlineto{\pgfqpoint{3.334477in}{8.753959in}}%
\pgfpathlineto{\pgfqpoint{3.334477in}{8.755239in}}%
\pgfpathlineto{\pgfqpoint{3.336333in}{8.774447in}}%
\pgfpathlineto{\pgfqpoint{3.336969in}{8.775727in}}%
\pgfpathlineto{\pgfqpoint{3.338030in}{8.792374in}}%
\pgfpathlineto{\pgfqpoint{3.338878in}{8.793654in}}%
\pgfpathlineto{\pgfqpoint{3.340257in}{8.814142in}}%
\pgfpathlineto{\pgfqpoint{3.340681in}{8.814142in}}%
\pgfpathlineto{\pgfqpoint{3.342219in}{8.835911in}}%
\pgfpathlineto{\pgfqpoint{3.342484in}{8.837191in}}%
\pgfpathlineto{\pgfqpoint{3.343969in}{8.852557in}}%
\pgfpathlineto{\pgfqpoint{3.344499in}{8.853838in}}%
\pgfpathlineto{\pgfqpoint{3.345931in}{8.869204in}}%
\pgfpathlineto{\pgfqpoint{3.346196in}{8.870484in}}%
\pgfpathlineto{\pgfqpoint{3.347734in}{8.888411in}}%
\pgfpathlineto{\pgfqpoint{3.347999in}{8.889692in}}%
\pgfpathlineto{\pgfqpoint{3.349430in}{8.896094in}}%
\pgfpathlineto{\pgfqpoint{3.349749in}{8.897375in}}%
\pgfpathlineto{\pgfqpoint{3.351180in}{8.907619in}}%
\pgfpathlineto{\pgfqpoint{3.351923in}{8.908899in}}%
\pgfpathlineto{\pgfqpoint{3.353354in}{8.920424in}}%
\pgfpathlineto{\pgfqpoint{3.354203in}{8.921704in}}%
\pgfpathlineto{\pgfqpoint{3.355741in}{8.935790in}}%
\pgfpathlineto{\pgfqpoint{3.356218in}{8.937070in}}%
\pgfpathlineto{\pgfqpoint{3.357756in}{8.946034in}}%
\pgfpathlineto{\pgfqpoint{3.358551in}{8.947314in}}%
\pgfpathlineto{\pgfqpoint{3.359771in}{8.963961in}}%
\pgfpathlineto{\pgfqpoint{3.360460in}{8.965241in}}%
\pgfpathlineto{\pgfqpoint{3.361839in}{8.974205in}}%
\pgfpathlineto{\pgfqpoint{3.362846in}{8.975485in}}%
\pgfpathlineto{\pgfqpoint{3.364225in}{8.990851in}}%
\pgfpathlineto{\pgfqpoint{3.364384in}{8.990851in}}%
\pgfpathlineto{\pgfqpoint{3.366399in}{9.006217in}}%
\pgfpathlineto{\pgfqpoint{3.366876in}{9.007498in}}%
\pgfpathlineto{\pgfqpoint{3.368202in}{9.016461in}}%
\pgfpathlineto{\pgfqpoint{3.368573in}{9.017742in}}%
\pgfpathlineto{\pgfqpoint{3.369952in}{9.027986in}}%
\pgfpathlineto{\pgfqpoint{3.370217in}{9.029266in}}%
\pgfpathlineto{\pgfqpoint{3.371066in}{9.038230in}}%
\pgfpathlineto{\pgfqpoint{3.372444in}{9.039510in}}%
\pgfpathlineto{\pgfqpoint{3.373028in}{9.048474in}}%
\pgfpathlineto{\pgfqpoint{3.374937in}{9.049754in}}%
\pgfpathlineto{\pgfqpoint{3.376421in}{9.056157in}}%
\pgfpathlineto{\pgfqpoint{3.376845in}{9.057437in}}%
\pgfpathlineto{\pgfqpoint{3.378277in}{9.068962in}}%
\pgfpathlineto{\pgfqpoint{3.378807in}{9.070242in}}%
\pgfpathlineto{\pgfqpoint{3.380345in}{9.076645in}}%
\pgfpathlineto{\pgfqpoint{3.380982in}{9.077925in}}%
\pgfpathlineto{\pgfqpoint{3.381883in}{9.085608in}}%
\pgfpathlineto{\pgfqpoint{3.382785in}{9.086889in}}%
\pgfpathlineto{\pgfqpoint{3.383898in}{9.093291in}}%
\pgfpathlineto{\pgfqpoint{3.384640in}{9.094572in}}%
\pgfpathlineto{\pgfqpoint{3.385595in}{9.103535in}}%
\pgfpathlineto{\pgfqpoint{3.387133in}{9.104816in}}%
\pgfpathlineto{\pgfqpoint{3.388405in}{9.118901in}}%
\pgfpathlineto{\pgfqpoint{3.389784in}{9.120182in}}%
\pgfpathlineto{\pgfqpoint{3.391163in}{9.132987in}}%
\pgfpathlineto{\pgfqpoint{3.391534in}{9.134267in}}%
\pgfpathlineto{\pgfqpoint{3.392435in}{9.136828in}}%
\pgfpathlineto{\pgfqpoint{3.393602in}{9.138109in}}%
\pgfpathlineto{\pgfqpoint{3.395034in}{9.147072in}}%
\pgfpathlineto{\pgfqpoint{3.396412in}{9.148353in}}%
\pgfpathlineto{\pgfqpoint{3.396412in}{9.150914in}}%
\pgfpathlineto{\pgfqpoint{3.398481in}{9.152194in}}%
\pgfpathlineto{\pgfqpoint{3.399753in}{9.154755in}}%
\pgfpathlineto{\pgfqpoint{3.402139in}{9.156036in}}%
\pgfpathlineto{\pgfqpoint{3.402139in}{9.158597in}}%
\pgfpathlineto{\pgfqpoint{3.404367in}{9.159877in}}%
\pgfpathlineto{\pgfqpoint{3.405904in}{9.166280in}}%
\pgfpathlineto{\pgfqpoint{3.406541in}{9.167560in}}%
\pgfpathlineto{\pgfqpoint{3.407813in}{9.175243in}}%
\pgfpathlineto{\pgfqpoint{3.409192in}{9.176524in}}%
\pgfpathlineto{\pgfqpoint{3.410359in}{9.181646in}}%
\pgfpathlineto{\pgfqpoint{3.411525in}{9.182926in}}%
\pgfpathlineto{\pgfqpoint{3.412639in}{9.186768in}}%
\pgfpathlineto{\pgfqpoint{3.413752in}{9.188048in}}%
\pgfpathlineto{\pgfqpoint{3.415131in}{9.195731in}}%
\pgfpathlineto{\pgfqpoint{3.415237in}{9.195731in}}%
\pgfpathlineto{\pgfqpoint{3.416510in}{9.197012in}}%
\pgfpathlineto{\pgfqpoint{3.417835in}{9.204695in}}%
\pgfpathlineto{\pgfqpoint{3.419585in}{9.205975in}}%
\pgfpathlineto{\pgfqpoint{3.420699in}{9.213658in}}%
\pgfpathlineto{\pgfqpoint{3.422661in}{9.214939in}}%
\pgfpathlineto{\pgfqpoint{3.422661in}{9.216219in}}%
\pgfpathlineto{\pgfqpoint{3.428229in}{9.217500in}}%
\pgfpathlineto{\pgfqpoint{3.429607in}{9.222622in}}%
\pgfpathlineto{\pgfqpoint{3.430827in}{9.223902in}}%
\pgfpathlineto{\pgfqpoint{3.430827in}{9.225183in}}%
\pgfpathlineto{\pgfqpoint{3.435653in}{9.226463in}}%
\pgfpathlineto{\pgfqpoint{3.437190in}{9.231585in}}%
\pgfpathlineto{\pgfqpoint{3.440637in}{9.232866in}}%
\pgfpathlineto{\pgfqpoint{3.442016in}{9.239268in}}%
\pgfpathlineto{\pgfqpoint{3.444879in}{9.240549in}}%
\pgfpathlineto{\pgfqpoint{3.446258in}{9.245671in}}%
\pgfpathlineto{\pgfqpoint{3.448591in}{9.246951in}}%
\pgfpathlineto{\pgfqpoint{3.449864in}{9.250793in}}%
\pgfpathlineto{\pgfqpoint{3.452992in}{9.252073in}}%
\pgfpathlineto{\pgfqpoint{3.452992in}{9.253354in}}%
\pgfpathlineto{\pgfqpoint{3.455750in}{9.254634in}}%
\pgfpathlineto{\pgfqpoint{3.457128in}{9.258476in}}%
\pgfpathlineto{\pgfqpoint{3.459939in}{9.259756in}}%
\pgfpathlineto{\pgfqpoint{3.460522in}{9.263598in}}%
\pgfpathlineto{\pgfqpoint{3.462378in}{9.264878in}}%
\pgfpathlineto{\pgfqpoint{3.463863in}{9.267439in}}%
\pgfpathlineto{\pgfqpoint{3.465242in}{9.267439in}}%
\pgfpathlineto{\pgfqpoint{3.465242in}{9.270000in}}%
\pgfpathlineto{\pgfqpoint{3.472612in}{9.271281in}}%
\pgfpathlineto{\pgfqpoint{3.474150in}{9.280244in}}%
\pgfpathlineto{\pgfqpoint{3.478604in}{9.281525in}}%
\pgfpathlineto{\pgfqpoint{3.479930in}{9.284086in}}%
\pgfpathlineto{\pgfqpoint{3.482953in}{9.285366in}}%
\pgfpathlineto{\pgfqpoint{3.482953in}{9.286647in}}%
\pgfpathlineto{\pgfqpoint{3.485286in}{9.287927in}}%
\pgfpathlineto{\pgfqpoint{3.486240in}{9.293049in}}%
\pgfpathlineto{\pgfqpoint{3.488786in}{9.294330in}}%
\pgfpathlineto{\pgfqpoint{3.490270in}{9.298171in}}%
\pgfpathlineto{\pgfqpoint{3.494619in}{9.299452in}}%
\pgfpathlineto{\pgfqpoint{3.495520in}{9.302013in}}%
\pgfpathlineto{\pgfqpoint{3.498065in}{9.303293in}}%
\pgfpathlineto{\pgfqpoint{3.498224in}{9.305854in}}%
\pgfpathlineto{\pgfqpoint{3.500293in}{9.307135in}}%
\pgfpathlineto{\pgfqpoint{3.500293in}{9.308415in}}%
\pgfpathlineto{\pgfqpoint{3.502042in}{9.309696in}}%
\pgfpathlineto{\pgfqpoint{3.502148in}{9.312257in}}%
\pgfpathlineto{\pgfqpoint{3.504588in}{9.313537in}}%
\pgfpathlineto{\pgfqpoint{3.505966in}{9.319940in}}%
\pgfpathlineto{\pgfqpoint{3.508194in}{9.319940in}}%
\pgfpathlineto{\pgfqpoint{3.508724in}{9.323781in}}%
\pgfpathlineto{\pgfqpoint{3.513337in}{9.325062in}}%
\pgfpathlineto{\pgfqpoint{3.513337in}{9.326342in}}%
\pgfpathlineto{\pgfqpoint{3.515087in}{9.326342in}}%
\pgfpathlineto{\pgfqpoint{3.515776in}{9.330184in}}%
\pgfpathlineto{\pgfqpoint{3.516996in}{9.331464in}}%
\pgfpathlineto{\pgfqpoint{3.517049in}{9.334025in}}%
\pgfpathlineto{\pgfqpoint{3.521344in}{9.335306in}}%
\pgfpathlineto{\pgfqpoint{3.522034in}{9.337867in}}%
\pgfpathlineto{\pgfqpoint{3.525109in}{9.339147in}}%
\pgfpathlineto{\pgfqpoint{3.525109in}{9.340428in}}%
\pgfpathlineto{\pgfqpoint{3.528450in}{9.341708in}}%
\pgfpathlineto{\pgfqpoint{3.528609in}{9.345550in}}%
\pgfpathlineto{\pgfqpoint{3.530412in}{9.346830in}}%
\pgfpathlineto{\pgfqpoint{3.530412in}{9.348111in}}%
\pgfpathlineto{\pgfqpoint{3.534283in}{9.349391in}}%
\pgfpathlineto{\pgfqpoint{3.535768in}{9.351952in}}%
\pgfpathlineto{\pgfqpoint{3.539639in}{9.353233in}}%
\pgfpathlineto{\pgfqpoint{3.539639in}{9.354513in}}%
\pgfpathlineto{\pgfqpoint{3.545153in}{9.355794in}}%
\pgfpathlineto{\pgfqpoint{3.545153in}{9.357074in}}%
\pgfpathlineto{\pgfqpoint{3.549183in}{9.358355in}}%
\pgfpathlineto{\pgfqpoint{3.549183in}{9.359635in}}%
\pgfpathlineto{\pgfqpoint{3.552630in}{9.360916in}}%
\pgfpathlineto{\pgfqpoint{3.552630in}{9.362196in}}%
\pgfpathlineto{\pgfqpoint{3.560849in}{9.362196in}}%
\pgfpathlineto{\pgfqpoint{3.560849in}{9.364757in}}%
\pgfpathlineto{\pgfqpoint{3.563236in}{9.366038in}}%
\pgfpathlineto{\pgfqpoint{3.563236in}{9.367318in}}%
\pgfpathlineto{\pgfqpoint{3.568008in}{9.368599in}}%
\pgfpathlineto{\pgfqpoint{3.568008in}{9.369879in}}%
\pgfpathlineto{\pgfqpoint{3.573099in}{9.371160in}}%
\pgfpathlineto{\pgfqpoint{3.573099in}{9.372440in}}%
\pgfpathlineto{\pgfqpoint{3.582484in}{9.373721in}}%
\pgfpathlineto{\pgfqpoint{3.582484in}{9.375001in}}%
\pgfpathlineto{\pgfqpoint{3.586674in}{9.376282in}}%
\pgfpathlineto{\pgfqpoint{3.586674in}{9.377562in}}%
\pgfpathlineto{\pgfqpoint{3.590492in}{9.378843in}}%
\pgfpathlineto{\pgfqpoint{3.590492in}{9.380123in}}%
\pgfpathlineto{\pgfqpoint{3.597332in}{9.381404in}}%
\pgfpathlineto{\pgfqpoint{3.598074in}{9.383965in}}%
\pgfpathlineto{\pgfqpoint{3.614089in}{9.385245in}}%
\pgfpathlineto{\pgfqpoint{3.615096in}{9.387806in}}%
\pgfpathlineto{\pgfqpoint{3.615732in}{9.387806in}}%
\pgfpathlineto{\pgfqpoint{3.616740in}{9.391648in}}%
\pgfpathlineto{\pgfqpoint{3.621990in}{9.392928in}}%
\pgfpathlineto{\pgfqpoint{3.621990in}{9.394209in}}%
\pgfpathlineto{\pgfqpoint{3.624376in}{9.395489in}}%
\pgfpathlineto{\pgfqpoint{3.625702in}{9.399331in}}%
\pgfpathlineto{\pgfqpoint{3.630474in}{9.400611in}}%
\pgfpathlineto{\pgfqpoint{3.630474in}{9.401892in}}%
\pgfpathlineto{\pgfqpoint{3.636572in}{9.403172in}}%
\pgfpathlineto{\pgfqpoint{3.636572in}{9.404453in}}%
\pgfpathlineto{\pgfqpoint{3.641557in}{9.405733in}}%
\pgfpathlineto{\pgfqpoint{3.641557in}{9.407014in}}%
\pgfpathlineto{\pgfqpoint{3.644155in}{9.408294in}}%
\pgfpathlineto{\pgfqpoint{3.644155in}{9.410855in}}%
\pgfpathlineto{\pgfqpoint{3.652003in}{9.410855in}}%
\pgfpathlineto{\pgfqpoint{3.652003in}{9.413416in}}%
\pgfpathlineto{\pgfqpoint{3.659003in}{9.414697in}}%
\pgfpathlineto{\pgfqpoint{3.660328in}{9.417258in}}%
\pgfpathlineto{\pgfqpoint{3.664889in}{9.418538in}}%
\pgfpathlineto{\pgfqpoint{3.664889in}{9.419819in}}%
\pgfpathlineto{\pgfqpoint{3.677191in}{9.421099in}}%
\pgfpathlineto{\pgfqpoint{3.677456in}{9.423660in}}%
\pgfpathlineto{\pgfqpoint{3.699674in}{9.424941in}}%
\pgfpathlineto{\pgfqpoint{3.699674in}{9.426221in}}%
\pgfpathlineto{\pgfqpoint{3.705985in}{9.427502in}}%
\pgfpathlineto{\pgfqpoint{3.705985in}{9.428782in}}%
\pgfpathlineto{\pgfqpoint{3.717491in}{9.430063in}}%
\pgfpathlineto{\pgfqpoint{3.719029in}{9.432624in}}%
\pgfpathlineto{\pgfqpoint{3.726824in}{9.433904in}}%
\pgfpathlineto{\pgfqpoint{3.727248in}{9.436465in}}%
\pgfpathlineto{\pgfqpoint{3.736422in}{9.437746in}}%
\pgfpathlineto{\pgfqpoint{3.736422in}{9.439026in}}%
\pgfpathlineto{\pgfqpoint{3.754557in}{9.440307in}}%
\pgfpathlineto{\pgfqpoint{3.754557in}{9.441587in}}%
\pgfpathlineto{\pgfqpoint{3.778101in}{9.442868in}}%
\pgfpathlineto{\pgfqpoint{3.778101in}{9.444148in}}%
\pgfpathlineto{\pgfqpoint{3.792684in}{9.445429in}}%
\pgfpathlineto{\pgfqpoint{3.792684in}{9.446709in}}%
\pgfpathlineto{\pgfqpoint{3.814107in}{9.447990in}}%
\pgfpathlineto{\pgfqpoint{3.814107in}{9.449270in}}%
\pgfpathlineto{\pgfqpoint{3.846506in}{9.450551in}}%
\pgfpathlineto{\pgfqpoint{3.846506in}{9.451831in}}%
\pgfpathlineto{\pgfqpoint{3.857801in}{9.453112in}}%
\pgfpathlineto{\pgfqpoint{3.857801in}{9.454392in}}%
\pgfpathlineto{\pgfqpoint{3.883731in}{9.455673in}}%
\pgfpathlineto{\pgfqpoint{3.885004in}{9.458234in}}%
\pgfpathlineto{\pgfqpoint{3.885110in}{9.458234in}}%
\pgfpathlineto{\pgfqpoint{3.911464in}{9.459514in}}%
\pgfpathlineto{\pgfqpoint{3.911464in}{9.460795in}}%
\pgfpathlineto{\pgfqpoint{3.927160in}{9.462075in}}%
\pgfpathlineto{\pgfqpoint{3.927160in}{9.463356in}}%
\pgfpathlineto{\pgfqpoint{3.962476in}{9.464636in}}%
\pgfpathlineto{\pgfqpoint{3.962954in}{9.467197in}}%
\pgfpathlineto{\pgfqpoint{3.979021in}{9.468478in}}%
\pgfpathlineto{\pgfqpoint{3.979021in}{9.469758in}}%
\pgfpathlineto{\pgfqpoint{4.001610in}{9.471039in}}%
\pgfpathlineto{\pgfqpoint{4.001610in}{9.472319in}}%
\pgfpathlineto{\pgfqpoint{4.041911in}{9.473600in}}%
\pgfpathlineto{\pgfqpoint{4.041911in}{9.474880in}}%
\pgfpathlineto{\pgfqpoint{4.062592in}{9.476161in}}%
\pgfpathlineto{\pgfqpoint{4.062592in}{9.477441in}}%
\pgfpathlineto{\pgfqpoint{4.074788in}{9.478722in}}%
\pgfpathlineto{\pgfqpoint{4.074788in}{9.480002in}}%
\pgfpathlineto{\pgfqpoint{4.153904in}{9.481283in}}%
\pgfpathlineto{\pgfqpoint{4.153904in}{9.482563in}}%
\pgfpathlineto{\pgfqpoint{4.183334in}{9.483844in}}%
\pgfpathlineto{\pgfqpoint{4.183334in}{9.485124in}}%
\pgfpathlineto{\pgfqpoint{4.247815in}{9.486405in}}%
\pgfpathlineto{\pgfqpoint{4.247815in}{9.487685in}}%
\pgfpathlineto{\pgfqpoint{4.260330in}{9.488966in}}%
\pgfpathlineto{\pgfqpoint{4.260330in}{9.490246in}}%
\pgfpathlineto{\pgfqpoint{4.356892in}{9.491527in}}%
\pgfpathlineto{\pgfqpoint{4.356892in}{9.492807in}}%
\pgfpathlineto{\pgfqpoint{4.447780in}{9.494088in}}%
\pgfpathlineto{\pgfqpoint{4.447780in}{9.495368in}}%
\pgfpathlineto{\pgfqpoint{4.836150in}{9.496649in}}%
\pgfpathlineto{\pgfqpoint{4.836150in}{9.496649in}}%
\pgfusepath{stroke}%
\end{pgfscope}%
\begin{pgfscope}%
\pgfpathrectangle{\pgfqpoint{3.225541in}{8.027083in}}{\pgfqpoint{1.687305in}{1.539545in}}%
\pgfusepath{clip}%
\pgfsetrectcap%
\pgfsetroundjoin%
\pgfsetlinewidth{1.505625pt}%
\definecolor{currentstroke}{rgb}{0.501961,0.501961,0.501961}%
\pgfsetstrokecolor{currentstroke}%
\pgfsetdash{}{0pt}%
\pgfpathmoveto{\pgfqpoint{3.302236in}{8.097062in}}%
\pgfpathlineto{\pgfqpoint{4.836150in}{9.496649in}}%
\pgfusepath{stroke}%
\end{pgfscope}%
\begin{pgfscope}%
\pgfsetrectcap%
\pgfsetmiterjoin%
\pgfsetlinewidth{0.803000pt}%
\definecolor{currentstroke}{rgb}{0.000000,0.000000,0.000000}%
\pgfsetstrokecolor{currentstroke}%
\pgfsetdash{}{0pt}%
\pgfpathmoveto{\pgfqpoint{3.225541in}{8.027083in}}%
\pgfpathlineto{\pgfqpoint{3.225541in}{9.566628in}}%
\pgfusepath{stroke}%
\end{pgfscope}%
\begin{pgfscope}%
\pgfsetrectcap%
\pgfsetmiterjoin%
\pgfsetlinewidth{0.803000pt}%
\definecolor{currentstroke}{rgb}{0.000000,0.000000,0.000000}%
\pgfsetstrokecolor{currentstroke}%
\pgfsetdash{}{0pt}%
\pgfpathmoveto{\pgfqpoint{4.912846in}{8.027083in}}%
\pgfpathlineto{\pgfqpoint{4.912846in}{9.566628in}}%
\pgfusepath{stroke}%
\end{pgfscope}%
\begin{pgfscope}%
\pgfsetrectcap%
\pgfsetmiterjoin%
\pgfsetlinewidth{0.803000pt}%
\definecolor{currentstroke}{rgb}{0.000000,0.000000,0.000000}%
\pgfsetstrokecolor{currentstroke}%
\pgfsetdash{}{0pt}%
\pgfpathmoveto{\pgfqpoint{3.225541in}{8.027083in}}%
\pgfpathlineto{\pgfqpoint{4.912846in}{8.027083in}}%
\pgfusepath{stroke}%
\end{pgfscope}%
\begin{pgfscope}%
\pgfsetrectcap%
\pgfsetmiterjoin%
\pgfsetlinewidth{0.803000pt}%
\definecolor{currentstroke}{rgb}{0.000000,0.000000,0.000000}%
\pgfsetstrokecolor{currentstroke}%
\pgfsetdash{}{0pt}%
\pgfpathmoveto{\pgfqpoint{3.225541in}{9.566628in}}%
\pgfpathlineto{\pgfqpoint{4.912846in}{9.566628in}}%
\pgfusepath{stroke}%
\end{pgfscope}%
\begin{pgfscope}%
\definecolor{textcolor}{rgb}{0.000000,0.000000,0.000000}%
\pgfsetstrokecolor{textcolor}%
\pgfsetfillcolor{textcolor}%
\pgftext[x=4.069193in,y=9.649962in,,base]{\color{textcolor}\rmfamily\fontsize{20.000000}{24.000000}\selectfont Emphysema}%
\end{pgfscope}%
\begin{pgfscope}%
\pgfsetbuttcap%
\pgfsetmiterjoin%
\definecolor{currentfill}{rgb}{1.000000,1.000000,1.000000}%
\pgfsetfillcolor{currentfill}%
\pgfsetfillopacity{0.800000}%
\pgfsetlinewidth{1.003750pt}%
\definecolor{currentstroke}{rgb}{0.800000,0.800000,0.800000}%
\pgfsetstrokecolor{currentstroke}%
\pgfsetstrokeopacity{0.800000}%
\pgfsetdash{}{0pt}%
\pgfpathmoveto{\pgfqpoint{3.703740in}{8.096527in}}%
\pgfpathlineto{\pgfqpoint{4.815624in}{8.096527in}}%
\pgfpathquadraticcurveto{\pgfqpoint{4.843402in}{8.096527in}}{\pgfqpoint{4.843402in}{8.124305in}}%
\pgfpathlineto{\pgfqpoint{4.843402in}{8.304089in}}%
\pgfpathquadraticcurveto{\pgfqpoint{4.843402in}{8.331867in}}{\pgfqpoint{4.815624in}{8.331867in}}%
\pgfpathlineto{\pgfqpoint{3.703740in}{8.331867in}}%
\pgfpathquadraticcurveto{\pgfqpoint{3.675962in}{8.331867in}}{\pgfqpoint{3.675962in}{8.304089in}}%
\pgfpathlineto{\pgfqpoint{3.675962in}{8.124305in}}%
\pgfpathquadraticcurveto{\pgfqpoint{3.675962in}{8.096527in}}{\pgfqpoint{3.703740in}{8.096527in}}%
\pgfpathclose%
\pgfusepath{stroke,fill}%
\end{pgfscope}%
\begin{pgfscope}%
\pgfsetrectcap%
\pgfsetroundjoin%
\pgfsetlinewidth{1.505625pt}%
\definecolor{currentstroke}{rgb}{0.000000,0.501961,0.000000}%
\pgfsetstrokecolor{currentstroke}%
\pgfsetdash{}{0pt}%
\pgfpathmoveto{\pgfqpoint{3.731518in}{8.227700in}}%
\pgfpathlineto{\pgfqpoint{4.009295in}{8.227700in}}%
\pgfusepath{stroke}%
\end{pgfscope}%
\begin{pgfscope}%
\definecolor{textcolor}{rgb}{0.000000,0.000000,0.000000}%
\pgfsetstrokecolor{textcolor}%
\pgfsetfillcolor{textcolor}%
\pgftext[x=4.120407in,y=8.179089in,left,base]{\color{textcolor}\rmfamily\fontsize{10.000000}{12.000000}\selectfont AUC 0.938}%
\end{pgfscope}%
\begin{pgfscope}%
\pgfsetbuttcap%
\pgfsetmiterjoin%
\definecolor{currentfill}{rgb}{1.000000,1.000000,1.000000}%
\pgfsetfillcolor{currentfill}%
\pgfsetlinewidth{0.000000pt}%
\definecolor{currentstroke}{rgb}{0.000000,0.000000,0.000000}%
\pgfsetstrokecolor{currentstroke}%
\pgfsetstrokeopacity{0.000000}%
\pgfsetdash{}{0pt}%
\pgfpathmoveto{\pgfqpoint{5.688041in}{8.027083in}}%
\pgfpathlineto{\pgfqpoint{7.375346in}{8.027083in}}%
\pgfpathlineto{\pgfqpoint{7.375346in}{9.566628in}}%
\pgfpathlineto{\pgfqpoint{5.688041in}{9.566628in}}%
\pgfpathclose%
\pgfusepath{fill}%
\end{pgfscope}%
\begin{pgfscope}%
\pgfsetbuttcap%
\pgfsetroundjoin%
\definecolor{currentfill}{rgb}{0.000000,0.000000,0.000000}%
\pgfsetfillcolor{currentfill}%
\pgfsetlinewidth{0.803000pt}%
\definecolor{currentstroke}{rgb}{0.000000,0.000000,0.000000}%
\pgfsetstrokecolor{currentstroke}%
\pgfsetdash{}{0pt}%
\pgfsys@defobject{currentmarker}{\pgfqpoint{0.000000in}{-0.048611in}}{\pgfqpoint{0.000000in}{0.000000in}}{%
\pgfpathmoveto{\pgfqpoint{0.000000in}{0.000000in}}%
\pgfpathlineto{\pgfqpoint{0.000000in}{-0.048611in}}%
\pgfusepath{stroke,fill}%
}%
\begin{pgfscope}%
\pgfsys@transformshift{5.764736in}{8.027083in}%
\pgfsys@useobject{currentmarker}{}%
\end{pgfscope}%
\end{pgfscope}%
\begin{pgfscope}%
\definecolor{textcolor}{rgb}{0.000000,0.000000,0.000000}%
\pgfsetstrokecolor{textcolor}%
\pgfsetfillcolor{textcolor}%
\pgftext[x=5.764736in,y=7.929861in,,top]{\color{textcolor}\rmfamily\fontsize{10.000000}{12.000000}\selectfont \(\displaystyle {0.0}\)}%
\end{pgfscope}%
\begin{pgfscope}%
\pgfsetbuttcap%
\pgfsetroundjoin%
\definecolor{currentfill}{rgb}{0.000000,0.000000,0.000000}%
\pgfsetfillcolor{currentfill}%
\pgfsetlinewidth{0.803000pt}%
\definecolor{currentstroke}{rgb}{0.000000,0.000000,0.000000}%
\pgfsetstrokecolor{currentstroke}%
\pgfsetdash{}{0pt}%
\pgfsys@defobject{currentmarker}{\pgfqpoint{0.000000in}{-0.048611in}}{\pgfqpoint{0.000000in}{0.000000in}}{%
\pgfpathmoveto{\pgfqpoint{0.000000in}{0.000000in}}%
\pgfpathlineto{\pgfqpoint{0.000000in}{-0.048611in}}%
\pgfusepath{stroke,fill}%
}%
\begin{pgfscope}%
\pgfsys@transformshift{6.531693in}{8.027083in}%
\pgfsys@useobject{currentmarker}{}%
\end{pgfscope}%
\end{pgfscope}%
\begin{pgfscope}%
\definecolor{textcolor}{rgb}{0.000000,0.000000,0.000000}%
\pgfsetstrokecolor{textcolor}%
\pgfsetfillcolor{textcolor}%
\pgftext[x=6.531693in,y=7.929861in,,top]{\color{textcolor}\rmfamily\fontsize{10.000000}{12.000000}\selectfont \(\displaystyle {0.5}\)}%
\end{pgfscope}%
\begin{pgfscope}%
\pgfsetbuttcap%
\pgfsetroundjoin%
\definecolor{currentfill}{rgb}{0.000000,0.000000,0.000000}%
\pgfsetfillcolor{currentfill}%
\pgfsetlinewidth{0.803000pt}%
\definecolor{currentstroke}{rgb}{0.000000,0.000000,0.000000}%
\pgfsetstrokecolor{currentstroke}%
\pgfsetdash{}{0pt}%
\pgfsys@defobject{currentmarker}{\pgfqpoint{0.000000in}{-0.048611in}}{\pgfqpoint{0.000000in}{0.000000in}}{%
\pgfpathmoveto{\pgfqpoint{0.000000in}{0.000000in}}%
\pgfpathlineto{\pgfqpoint{0.000000in}{-0.048611in}}%
\pgfusepath{stroke,fill}%
}%
\begin{pgfscope}%
\pgfsys@transformshift{7.298650in}{8.027083in}%
\pgfsys@useobject{currentmarker}{}%
\end{pgfscope}%
\end{pgfscope}%
\begin{pgfscope}%
\definecolor{textcolor}{rgb}{0.000000,0.000000,0.000000}%
\pgfsetstrokecolor{textcolor}%
\pgfsetfillcolor{textcolor}%
\pgftext[x=7.298650in,y=7.929861in,,top]{\color{textcolor}\rmfamily\fontsize{10.000000}{12.000000}\selectfont \(\displaystyle {1.0}\)}%
\end{pgfscope}%
\begin{pgfscope}%
\definecolor{textcolor}{rgb}{0.000000,0.000000,0.000000}%
\pgfsetstrokecolor{textcolor}%
\pgfsetfillcolor{textcolor}%
\pgftext[x=6.531693in,y=7.750849in,,top]{\color{textcolor}\rmfamily\fontsize{16.000000}{19.200000}\selectfont FPR}%
\end{pgfscope}%
\begin{pgfscope}%
\pgfsetbuttcap%
\pgfsetroundjoin%
\definecolor{currentfill}{rgb}{0.000000,0.000000,0.000000}%
\pgfsetfillcolor{currentfill}%
\pgfsetlinewidth{0.803000pt}%
\definecolor{currentstroke}{rgb}{0.000000,0.000000,0.000000}%
\pgfsetstrokecolor{currentstroke}%
\pgfsetdash{}{0pt}%
\pgfsys@defobject{currentmarker}{\pgfqpoint{-0.048611in}{0.000000in}}{\pgfqpoint{-0.000000in}{0.000000in}}{%
\pgfpathmoveto{\pgfqpoint{-0.000000in}{0.000000in}}%
\pgfpathlineto{\pgfqpoint{-0.048611in}{0.000000in}}%
\pgfusepath{stroke,fill}%
}%
\begin{pgfscope}%
\pgfsys@transformshift{5.688041in}{8.097062in}%
\pgfsys@useobject{currentmarker}{}%
\end{pgfscope}%
\end{pgfscope}%
\begin{pgfscope}%
\definecolor{textcolor}{rgb}{0.000000,0.000000,0.000000}%
\pgfsetstrokecolor{textcolor}%
\pgfsetfillcolor{textcolor}%
\pgftext[x=5.343904in, y=8.048837in, left, base]{\color{textcolor}\rmfamily\fontsize{10.000000}{12.000000}\selectfont \(\displaystyle {0.00}\)}%
\end{pgfscope}%
\begin{pgfscope}%
\pgfsetbuttcap%
\pgfsetroundjoin%
\definecolor{currentfill}{rgb}{0.000000,0.000000,0.000000}%
\pgfsetfillcolor{currentfill}%
\pgfsetlinewidth{0.803000pt}%
\definecolor{currentstroke}{rgb}{0.000000,0.000000,0.000000}%
\pgfsetstrokecolor{currentstroke}%
\pgfsetdash{}{0pt}%
\pgfsys@defobject{currentmarker}{\pgfqpoint{-0.048611in}{0.000000in}}{\pgfqpoint{-0.000000in}{0.000000in}}{%
\pgfpathmoveto{\pgfqpoint{-0.000000in}{0.000000in}}%
\pgfpathlineto{\pgfqpoint{-0.048611in}{0.000000in}}%
\pgfusepath{stroke,fill}%
}%
\begin{pgfscope}%
\pgfsys@transformshift{5.688041in}{8.446959in}%
\pgfsys@useobject{currentmarker}{}%
\end{pgfscope}%
\end{pgfscope}%
\begin{pgfscope}%
\definecolor{textcolor}{rgb}{0.000000,0.000000,0.000000}%
\pgfsetstrokecolor{textcolor}%
\pgfsetfillcolor{textcolor}%
\pgftext[x=5.343904in, y=8.398734in, left, base]{\color{textcolor}\rmfamily\fontsize{10.000000}{12.000000}\selectfont \(\displaystyle {0.25}\)}%
\end{pgfscope}%
\begin{pgfscope}%
\pgfsetbuttcap%
\pgfsetroundjoin%
\definecolor{currentfill}{rgb}{0.000000,0.000000,0.000000}%
\pgfsetfillcolor{currentfill}%
\pgfsetlinewidth{0.803000pt}%
\definecolor{currentstroke}{rgb}{0.000000,0.000000,0.000000}%
\pgfsetstrokecolor{currentstroke}%
\pgfsetdash{}{0pt}%
\pgfsys@defobject{currentmarker}{\pgfqpoint{-0.048611in}{0.000000in}}{\pgfqpoint{-0.000000in}{0.000000in}}{%
\pgfpathmoveto{\pgfqpoint{-0.000000in}{0.000000in}}%
\pgfpathlineto{\pgfqpoint{-0.048611in}{0.000000in}}%
\pgfusepath{stroke,fill}%
}%
\begin{pgfscope}%
\pgfsys@transformshift{5.688041in}{8.796856in}%
\pgfsys@useobject{currentmarker}{}%
\end{pgfscope}%
\end{pgfscope}%
\begin{pgfscope}%
\definecolor{textcolor}{rgb}{0.000000,0.000000,0.000000}%
\pgfsetstrokecolor{textcolor}%
\pgfsetfillcolor{textcolor}%
\pgftext[x=5.343904in, y=8.748630in, left, base]{\color{textcolor}\rmfamily\fontsize{10.000000}{12.000000}\selectfont \(\displaystyle {0.50}\)}%
\end{pgfscope}%
\begin{pgfscope}%
\pgfsetbuttcap%
\pgfsetroundjoin%
\definecolor{currentfill}{rgb}{0.000000,0.000000,0.000000}%
\pgfsetfillcolor{currentfill}%
\pgfsetlinewidth{0.803000pt}%
\definecolor{currentstroke}{rgb}{0.000000,0.000000,0.000000}%
\pgfsetstrokecolor{currentstroke}%
\pgfsetdash{}{0pt}%
\pgfsys@defobject{currentmarker}{\pgfqpoint{-0.048611in}{0.000000in}}{\pgfqpoint{-0.000000in}{0.000000in}}{%
\pgfpathmoveto{\pgfqpoint{-0.000000in}{0.000000in}}%
\pgfpathlineto{\pgfqpoint{-0.048611in}{0.000000in}}%
\pgfusepath{stroke,fill}%
}%
\begin{pgfscope}%
\pgfsys@transformshift{5.688041in}{9.146752in}%
\pgfsys@useobject{currentmarker}{}%
\end{pgfscope}%
\end{pgfscope}%
\begin{pgfscope}%
\definecolor{textcolor}{rgb}{0.000000,0.000000,0.000000}%
\pgfsetstrokecolor{textcolor}%
\pgfsetfillcolor{textcolor}%
\pgftext[x=5.343904in, y=9.098527in, left, base]{\color{textcolor}\rmfamily\fontsize{10.000000}{12.000000}\selectfont \(\displaystyle {0.75}\)}%
\end{pgfscope}%
\begin{pgfscope}%
\pgfsetbuttcap%
\pgfsetroundjoin%
\definecolor{currentfill}{rgb}{0.000000,0.000000,0.000000}%
\pgfsetfillcolor{currentfill}%
\pgfsetlinewidth{0.803000pt}%
\definecolor{currentstroke}{rgb}{0.000000,0.000000,0.000000}%
\pgfsetstrokecolor{currentstroke}%
\pgfsetdash{}{0pt}%
\pgfsys@defobject{currentmarker}{\pgfqpoint{-0.048611in}{0.000000in}}{\pgfqpoint{-0.000000in}{0.000000in}}{%
\pgfpathmoveto{\pgfqpoint{-0.000000in}{0.000000in}}%
\pgfpathlineto{\pgfqpoint{-0.048611in}{0.000000in}}%
\pgfusepath{stroke,fill}%
}%
\begin{pgfscope}%
\pgfsys@transformshift{5.688041in}{9.496649in}%
\pgfsys@useobject{currentmarker}{}%
\end{pgfscope}%
\end{pgfscope}%
\begin{pgfscope}%
\definecolor{textcolor}{rgb}{0.000000,0.000000,0.000000}%
\pgfsetstrokecolor{textcolor}%
\pgfsetfillcolor{textcolor}%
\pgftext[x=5.343904in, y=9.448424in, left, base]{\color{textcolor}\rmfamily\fontsize{10.000000}{12.000000}\selectfont \(\displaystyle {1.00}\)}%
\end{pgfscope}%
\begin{pgfscope}%
\definecolor{textcolor}{rgb}{0.000000,0.000000,0.000000}%
\pgfsetstrokecolor{textcolor}%
\pgfsetfillcolor{textcolor}%
\pgftext[x=5.288349in,y=8.796856in,,bottom,rotate=90.000000]{\color{textcolor}\rmfamily\fontsize{16.000000}{19.200000}\selectfont TPR}%
\end{pgfscope}%
\begin{pgfscope}%
\pgfpathrectangle{\pgfqpoint{5.688041in}{8.027083in}}{\pgfqpoint{1.687305in}{1.539545in}}%
\pgfusepath{clip}%
\pgfsetrectcap%
\pgfsetroundjoin%
\pgfsetlinewidth{1.505625pt}%
\definecolor{currentstroke}{rgb}{0.000000,0.501961,0.000000}%
\pgfsetstrokecolor{currentstroke}%
\pgfsetdash{}{0pt}%
\pgfpathmoveto{\pgfqpoint{5.764736in}{8.097062in}}%
\pgfpathlineto{\pgfqpoint{5.770240in}{8.154152in}}%
\pgfpathlineto{\pgfqpoint{5.772296in}{8.168574in}}%
\pgfpathlineto{\pgfqpoint{5.773809in}{8.180893in}}%
\pgfpathlineto{\pgfqpoint{5.776107in}{8.209438in}}%
\pgfpathlineto{\pgfqpoint{5.780824in}{8.250001in}}%
\pgfpathlineto{\pgfqpoint{5.781006in}{8.250903in}}%
\pgfpathlineto{\pgfqpoint{5.781006in}{8.251504in}}%
\pgfpathlineto{\pgfqpoint{5.782457in}{8.267128in}}%
\pgfpathlineto{\pgfqpoint{5.783062in}{8.267729in}}%
\pgfpathlineto{\pgfqpoint{5.784514in}{8.281551in}}%
\pgfpathlineto{\pgfqpoint{5.785058in}{8.284255in}}%
\pgfpathlineto{\pgfqpoint{5.786509in}{8.300180in}}%
\pgfpathlineto{\pgfqpoint{5.787054in}{8.301682in}}%
\pgfpathlineto{\pgfqpoint{5.788566in}{8.310095in}}%
\pgfpathlineto{\pgfqpoint{5.794916in}{8.349457in}}%
\pgfpathlineto{\pgfqpoint{5.796368in}{8.361175in}}%
\pgfpathlineto{\pgfqpoint{5.796791in}{8.362677in}}%
\pgfpathlineto{\pgfqpoint{5.798303in}{8.370490in}}%
\pgfpathlineto{\pgfqpoint{5.798545in}{8.371992in}}%
\pgfpathlineto{\pgfqpoint{5.800057in}{8.379804in}}%
\pgfpathlineto{\pgfqpoint{5.800239in}{8.381006in}}%
\pgfpathlineto{\pgfqpoint{5.801690in}{8.389419in}}%
\pgfpathlineto{\pgfqpoint{5.802053in}{8.390020in}}%
\pgfpathlineto{\pgfqpoint{5.803565in}{8.402039in}}%
\pgfpathlineto{\pgfqpoint{5.803988in}{8.402940in}}%
\pgfpathlineto{\pgfqpoint{5.805500in}{8.413757in}}%
\pgfpathlineto{\pgfqpoint{5.805621in}{8.414358in}}%
\pgfpathlineto{\pgfqpoint{5.805621in}{8.414959in}}%
\pgfpathlineto{\pgfqpoint{5.807133in}{8.424574in}}%
\pgfpathlineto{\pgfqpoint{5.807557in}{8.425475in}}%
\pgfpathlineto{\pgfqpoint{5.809069in}{8.436593in}}%
\pgfpathlineto{\pgfqpoint{5.809795in}{8.438095in}}%
\pgfpathlineto{\pgfqpoint{5.811307in}{8.447109in}}%
\pgfpathlineto{\pgfqpoint{5.811911in}{8.448612in}}%
\pgfpathlineto{\pgfqpoint{5.813363in}{8.454621in}}%
\pgfpathlineto{\pgfqpoint{5.813847in}{8.456123in}}%
\pgfpathlineto{\pgfqpoint{5.815359in}{8.469344in}}%
\pgfpathlineto{\pgfqpoint{5.815722in}{8.470546in}}%
\pgfpathlineto{\pgfqpoint{5.817173in}{8.475954in}}%
\pgfpathlineto{\pgfqpoint{5.817294in}{8.475954in}}%
\pgfpathlineto{\pgfqpoint{5.818746in}{8.482565in}}%
\pgfpathlineto{\pgfqpoint{5.819351in}{8.484067in}}%
\pgfpathlineto{\pgfqpoint{5.820742in}{8.491579in}}%
\pgfpathlineto{\pgfqpoint{5.821346in}{8.493081in}}%
\pgfpathlineto{\pgfqpoint{5.822798in}{8.502095in}}%
\pgfpathlineto{\pgfqpoint{5.822979in}{8.502696in}}%
\pgfpathlineto{\pgfqpoint{5.824491in}{8.514114in}}%
\pgfpathlineto{\pgfqpoint{5.824733in}{8.514715in}}%
\pgfpathlineto{\pgfqpoint{5.826245in}{8.524029in}}%
\pgfpathlineto{\pgfqpoint{5.826669in}{8.525532in}}%
\pgfpathlineto{\pgfqpoint{5.828060in}{8.533043in}}%
\pgfpathlineto{\pgfqpoint{5.828665in}{8.534546in}}%
\pgfpathlineto{\pgfqpoint{5.830116in}{8.541757in}}%
\pgfpathlineto{\pgfqpoint{5.830600in}{8.542358in}}%
\pgfpathlineto{\pgfqpoint{5.832112in}{8.548367in}}%
\pgfpathlineto{\pgfqpoint{5.833080in}{8.549569in}}%
\pgfpathlineto{\pgfqpoint{5.834592in}{8.558583in}}%
\pgfpathlineto{\pgfqpoint{5.834773in}{8.559785in}}%
\pgfpathlineto{\pgfqpoint{5.836285in}{8.565795in}}%
\pgfpathlineto{\pgfqpoint{5.836467in}{8.566997in}}%
\pgfpathlineto{\pgfqpoint{5.837918in}{8.573607in}}%
\pgfpathlineto{\pgfqpoint{5.838462in}{8.575109in}}%
\pgfpathlineto{\pgfqpoint{5.839975in}{8.585325in}}%
\pgfpathlineto{\pgfqpoint{5.840458in}{8.586527in}}%
\pgfpathlineto{\pgfqpoint{5.841668in}{8.590734in}}%
\pgfpathlineto{\pgfqpoint{5.842333in}{8.592236in}}%
\pgfpathlineto{\pgfqpoint{5.843845in}{8.601851in}}%
\pgfpathlineto{\pgfqpoint{5.844208in}{8.603353in}}%
\pgfpathlineto{\pgfqpoint{5.845720in}{8.612668in}}%
\pgfpathlineto{\pgfqpoint{5.846567in}{8.614170in}}%
\pgfpathlineto{\pgfqpoint{5.848018in}{8.620480in}}%
\pgfpathlineto{\pgfqpoint{5.848260in}{8.621682in}}%
\pgfpathlineto{\pgfqpoint{5.849772in}{8.626489in}}%
\pgfpathlineto{\pgfqpoint{5.849954in}{8.627691in}}%
\pgfpathlineto{\pgfqpoint{5.851405in}{8.631597in}}%
\pgfpathlineto{\pgfqpoint{5.852192in}{8.633100in}}%
\pgfpathlineto{\pgfqpoint{5.853401in}{8.636705in}}%
\pgfpathlineto{\pgfqpoint{5.853885in}{8.638208in}}%
\pgfpathlineto{\pgfqpoint{5.855397in}{8.646921in}}%
\pgfpathlineto{\pgfqpoint{5.856002in}{8.648123in}}%
\pgfpathlineto{\pgfqpoint{5.857333in}{8.655635in}}%
\pgfpathlineto{\pgfqpoint{5.857756in}{8.656837in}}%
\pgfpathlineto{\pgfqpoint{5.859268in}{8.664649in}}%
\pgfpathlineto{\pgfqpoint{5.859691in}{8.665851in}}%
\pgfpathlineto{\pgfqpoint{5.861203in}{8.672161in}}%
\pgfpathlineto{\pgfqpoint{5.861748in}{8.673663in}}%
\pgfpathlineto{\pgfqpoint{5.863139in}{8.679072in}}%
\pgfpathlineto{\pgfqpoint{5.863743in}{8.680574in}}%
\pgfpathlineto{\pgfqpoint{5.865255in}{8.684480in}}%
\pgfpathlineto{\pgfqpoint{5.865679in}{8.685982in}}%
\pgfpathlineto{\pgfqpoint{5.866949in}{8.691691in}}%
\pgfpathlineto{\pgfqpoint{5.867675in}{8.693194in}}%
\pgfpathlineto{\pgfqpoint{5.869187in}{8.698001in}}%
\pgfpathlineto{\pgfqpoint{5.869429in}{8.698302in}}%
\pgfpathlineto{\pgfqpoint{5.870941in}{8.704311in}}%
\pgfpathlineto{\pgfqpoint{5.872332in}{8.705513in}}%
\pgfpathlineto{\pgfqpoint{5.873723in}{8.710621in}}%
\pgfpathlineto{\pgfqpoint{5.874207in}{8.712123in}}%
\pgfpathlineto{\pgfqpoint{5.875719in}{8.718433in}}%
\pgfpathlineto{\pgfqpoint{5.876082in}{8.719935in}}%
\pgfpathlineto{\pgfqpoint{5.877594in}{8.725344in}}%
\pgfpathlineto{\pgfqpoint{5.878561in}{8.726846in}}%
\pgfpathlineto{\pgfqpoint{5.880013in}{8.732856in}}%
\pgfpathlineto{\pgfqpoint{5.880436in}{8.734358in}}%
\pgfpathlineto{\pgfqpoint{5.881827in}{8.739165in}}%
\pgfpathlineto{\pgfqpoint{5.882493in}{8.740668in}}%
\pgfpathlineto{\pgfqpoint{5.883944in}{8.745175in}}%
\pgfpathlineto{\pgfqpoint{5.884730in}{8.746076in}}%
\pgfpathlineto{\pgfqpoint{5.886121in}{8.751184in}}%
\pgfpathlineto{\pgfqpoint{5.887331in}{8.752386in}}%
\pgfpathlineto{\pgfqpoint{5.888783in}{8.758696in}}%
\pgfpathlineto{\pgfqpoint{5.889266in}{8.760198in}}%
\pgfpathlineto{\pgfqpoint{5.890778in}{8.763203in}}%
\pgfpathlineto{\pgfqpoint{5.891444in}{8.763504in}}%
\pgfpathlineto{\pgfqpoint{5.892411in}{8.768612in}}%
\pgfpathlineto{\pgfqpoint{5.893923in}{8.769813in}}%
\pgfpathlineto{\pgfqpoint{5.895435in}{8.778227in}}%
\pgfpathlineto{\pgfqpoint{5.895738in}{8.779428in}}%
\pgfpathlineto{\pgfqpoint{5.897189in}{8.783034in}}%
\pgfpathlineto{\pgfqpoint{5.897552in}{8.784236in}}%
\pgfpathlineto{\pgfqpoint{5.898943in}{8.791748in}}%
\pgfpathlineto{\pgfqpoint{5.899609in}{8.792649in}}%
\pgfpathlineto{\pgfqpoint{5.901121in}{8.797757in}}%
\pgfpathlineto{\pgfqpoint{5.902814in}{8.798959in}}%
\pgfpathlineto{\pgfqpoint{5.903842in}{8.803466in}}%
\pgfpathlineto{\pgfqpoint{5.904689in}{8.804067in}}%
\pgfpathlineto{\pgfqpoint{5.906201in}{8.811579in}}%
\pgfpathlineto{\pgfqpoint{5.906564in}{8.813081in}}%
\pgfpathlineto{\pgfqpoint{5.908015in}{8.819391in}}%
\pgfpathlineto{\pgfqpoint{5.908620in}{8.820893in}}%
\pgfpathlineto{\pgfqpoint{5.909951in}{8.826001in}}%
\pgfpathlineto{\pgfqpoint{5.910979in}{8.827504in}}%
\pgfpathlineto{\pgfqpoint{5.912128in}{8.831710in}}%
\pgfpathlineto{\pgfqpoint{5.913156in}{8.832912in}}%
\pgfpathlineto{\pgfqpoint{5.914063in}{8.835616in}}%
\pgfpathlineto{\pgfqpoint{5.916422in}{8.837119in}}%
\pgfpathlineto{\pgfqpoint{5.917450in}{8.839522in}}%
\pgfpathlineto{\pgfqpoint{5.918600in}{8.841025in}}%
\pgfpathlineto{\pgfqpoint{5.920112in}{8.844931in}}%
\pgfpathlineto{\pgfqpoint{5.920716in}{8.846433in}}%
\pgfpathlineto{\pgfqpoint{5.922047in}{8.849137in}}%
\pgfpathlineto{\pgfqpoint{5.923378in}{8.850640in}}%
\pgfpathlineto{\pgfqpoint{5.924890in}{8.855748in}}%
\pgfpathlineto{\pgfqpoint{5.925434in}{8.856649in}}%
\pgfpathlineto{\pgfqpoint{5.926281in}{8.859353in}}%
\pgfpathlineto{\pgfqpoint{5.927188in}{8.860555in}}%
\pgfpathlineto{\pgfqpoint{5.928700in}{8.863860in}}%
\pgfpathlineto{\pgfqpoint{5.929123in}{8.865363in}}%
\pgfpathlineto{\pgfqpoint{5.930635in}{8.869569in}}%
\pgfpathlineto{\pgfqpoint{5.931542in}{8.870771in}}%
\pgfpathlineto{\pgfqpoint{5.932994in}{8.873175in}}%
\pgfpathlineto{\pgfqpoint{5.933538in}{8.874076in}}%
\pgfpathlineto{\pgfqpoint{5.934808in}{8.876780in}}%
\pgfpathlineto{\pgfqpoint{5.935534in}{8.878283in}}%
\pgfpathlineto{\pgfqpoint{5.937046in}{8.882489in}}%
\pgfpathlineto{\pgfqpoint{5.937470in}{8.883992in}}%
\pgfpathlineto{\pgfqpoint{5.938800in}{8.887597in}}%
\pgfpathlineto{\pgfqpoint{5.939526in}{8.889100in}}%
\pgfpathlineto{\pgfqpoint{5.940977in}{8.891804in}}%
\pgfpathlineto{\pgfqpoint{5.942006in}{8.893306in}}%
\pgfpathlineto{\pgfqpoint{5.943276in}{8.895710in}}%
\pgfpathlineto{\pgfqpoint{5.944304in}{8.896611in}}%
\pgfpathlineto{\pgfqpoint{5.945574in}{8.901719in}}%
\pgfpathlineto{\pgfqpoint{5.946542in}{8.903222in}}%
\pgfpathlineto{\pgfqpoint{5.947509in}{8.906226in}}%
\pgfpathlineto{\pgfqpoint{5.948900in}{8.907729in}}%
\pgfpathlineto{\pgfqpoint{5.950412in}{8.911034in}}%
\pgfpathlineto{\pgfqpoint{5.952287in}{8.912536in}}%
\pgfpathlineto{\pgfqpoint{5.953739in}{8.917644in}}%
\pgfpathlineto{\pgfqpoint{5.955190in}{8.919147in}}%
\pgfpathlineto{\pgfqpoint{5.956702in}{8.923654in}}%
\pgfpathlineto{\pgfqpoint{5.957307in}{8.925156in}}%
\pgfpathlineto{\pgfqpoint{5.958819in}{8.929964in}}%
\pgfpathlineto{\pgfqpoint{5.959424in}{8.931466in}}%
\pgfpathlineto{\pgfqpoint{5.960876in}{8.935372in}}%
\pgfpathlineto{\pgfqpoint{5.961662in}{8.936874in}}%
\pgfpathlineto{\pgfqpoint{5.963113in}{8.944687in}}%
\pgfpathlineto{\pgfqpoint{5.964686in}{8.946189in}}%
\pgfpathlineto{\pgfqpoint{5.966137in}{8.950095in}}%
\pgfpathlineto{\pgfqpoint{5.966742in}{8.951297in}}%
\pgfpathlineto{\pgfqpoint{5.968133in}{8.956405in}}%
\pgfpathlineto{\pgfqpoint{5.969282in}{8.957907in}}%
\pgfpathlineto{\pgfqpoint{5.970553in}{8.961212in}}%
\pgfpathlineto{\pgfqpoint{5.971218in}{8.962715in}}%
\pgfpathlineto{\pgfqpoint{5.972488in}{8.966320in}}%
\pgfpathlineto{\pgfqpoint{5.973879in}{8.967823in}}%
\pgfpathlineto{\pgfqpoint{5.975331in}{8.970226in}}%
\pgfpathlineto{\pgfqpoint{5.976056in}{8.971729in}}%
\pgfpathlineto{\pgfqpoint{5.977024in}{8.974133in}}%
\pgfpathlineto{\pgfqpoint{5.978899in}{8.975635in}}%
\pgfpathlineto{\pgfqpoint{5.980350in}{8.979241in}}%
\pgfpathlineto{\pgfqpoint{5.980653in}{8.980142in}}%
\pgfpathlineto{\pgfqpoint{5.981923in}{8.983748in}}%
\pgfpathlineto{\pgfqpoint{5.982891in}{8.985250in}}%
\pgfpathlineto{\pgfqpoint{5.984403in}{8.990358in}}%
\pgfpathlineto{\pgfqpoint{5.985673in}{8.991860in}}%
\pgfpathlineto{\pgfqpoint{5.987064in}{8.995766in}}%
\pgfpathlineto{\pgfqpoint{5.988939in}{8.997269in}}%
\pgfpathlineto{\pgfqpoint{5.990390in}{8.999672in}}%
\pgfpathlineto{\pgfqpoint{5.991418in}{9.000874in}}%
\pgfpathlineto{\pgfqpoint{5.992930in}{9.003879in}}%
\pgfpathlineto{\pgfqpoint{5.993717in}{9.005081in}}%
\pgfpathlineto{\pgfqpoint{5.995047in}{9.008386in}}%
\pgfpathlineto{\pgfqpoint{5.995834in}{9.009588in}}%
\pgfpathlineto{\pgfqpoint{5.997285in}{9.014696in}}%
\pgfpathlineto{\pgfqpoint{5.998313in}{9.016198in}}%
\pgfpathlineto{\pgfqpoint{5.999644in}{9.019804in}}%
\pgfpathlineto{\pgfqpoint{6.001156in}{9.021306in}}%
\pgfpathlineto{\pgfqpoint{6.002668in}{9.023410in}}%
\pgfpathlineto{\pgfqpoint{6.003877in}{9.024912in}}%
\pgfpathlineto{\pgfqpoint{6.005390in}{9.027917in}}%
\pgfpathlineto{\pgfqpoint{6.006539in}{9.029419in}}%
\pgfpathlineto{\pgfqpoint{6.007930in}{9.032123in}}%
\pgfpathlineto{\pgfqpoint{6.009200in}{9.033626in}}%
\pgfpathlineto{\pgfqpoint{6.010591in}{9.039334in}}%
\pgfpathlineto{\pgfqpoint{6.011680in}{9.040837in}}%
\pgfpathlineto{\pgfqpoint{6.013192in}{9.044142in}}%
\pgfpathlineto{\pgfqpoint{6.014522in}{9.045644in}}%
\pgfpathlineto{\pgfqpoint{6.016034in}{9.048649in}}%
\pgfpathlineto{\pgfqpoint{6.018030in}{9.050151in}}%
\pgfpathlineto{\pgfqpoint{6.019300in}{9.052856in}}%
\pgfpathlineto{\pgfqpoint{6.021054in}{9.054358in}}%
\pgfpathlineto{\pgfqpoint{6.022506in}{9.056762in}}%
\pgfpathlineto{\pgfqpoint{6.023110in}{9.057964in}}%
\pgfpathlineto{\pgfqpoint{6.024320in}{9.060367in}}%
\pgfpathlineto{\pgfqpoint{6.025590in}{9.061870in}}%
\pgfpathlineto{\pgfqpoint{6.026860in}{9.063672in}}%
\pgfpathlineto{\pgfqpoint{6.028312in}{9.065175in}}%
\pgfpathlineto{\pgfqpoint{6.029703in}{9.069081in}}%
\pgfpathlineto{\pgfqpoint{6.030731in}{9.070583in}}%
\pgfpathlineto{\pgfqpoint{6.032062in}{9.074489in}}%
\pgfpathlineto{\pgfqpoint{6.033755in}{9.075992in}}%
\pgfpathlineto{\pgfqpoint{6.035267in}{9.077494in}}%
\pgfpathlineto{\pgfqpoint{6.036053in}{9.078996in}}%
\pgfpathlineto{\pgfqpoint{6.037444in}{9.082602in}}%
\pgfpathlineto{\pgfqpoint{6.038775in}{9.083804in}}%
\pgfpathlineto{\pgfqpoint{6.040287in}{9.087410in}}%
\pgfpathlineto{\pgfqpoint{6.041618in}{9.088912in}}%
\pgfpathlineto{\pgfqpoint{6.043069in}{9.090715in}}%
\pgfpathlineto{\pgfqpoint{6.044339in}{9.092217in}}%
\pgfpathlineto{\pgfqpoint{6.045609in}{9.097325in}}%
\pgfpathlineto{\pgfqpoint{6.046456in}{9.098827in}}%
\pgfpathlineto{\pgfqpoint{6.047968in}{9.100931in}}%
\pgfpathlineto{\pgfqpoint{6.048996in}{9.102133in}}%
\pgfpathlineto{\pgfqpoint{6.050266in}{9.107541in}}%
\pgfpathlineto{\pgfqpoint{6.052202in}{9.109043in}}%
\pgfpathlineto{\pgfqpoint{6.053472in}{9.109945in}}%
\pgfpathlineto{\pgfqpoint{6.056617in}{9.111447in}}%
\pgfpathlineto{\pgfqpoint{6.057584in}{9.114452in}}%
\pgfpathlineto{\pgfqpoint{6.059036in}{9.115654in}}%
\pgfpathlineto{\pgfqpoint{6.059943in}{9.119860in}}%
\pgfpathlineto{\pgfqpoint{6.061213in}{9.121363in}}%
\pgfpathlineto{\pgfqpoint{6.062241in}{9.124067in}}%
\pgfpathlineto{\pgfqpoint{6.063512in}{9.125569in}}%
\pgfpathlineto{\pgfqpoint{6.064842in}{9.128874in}}%
\pgfpathlineto{\pgfqpoint{6.066838in}{9.130377in}}%
\pgfpathlineto{\pgfqpoint{6.067987in}{9.132480in}}%
\pgfpathlineto{\pgfqpoint{6.068894in}{9.133982in}}%
\pgfpathlineto{\pgfqpoint{6.070164in}{9.137888in}}%
\pgfpathlineto{\pgfqpoint{6.071616in}{9.139391in}}%
\pgfpathlineto{\pgfqpoint{6.072765in}{9.142395in}}%
\pgfpathlineto{\pgfqpoint{6.076092in}{9.143898in}}%
\pgfpathlineto{\pgfqpoint{6.077362in}{9.145400in}}%
\pgfpathlineto{\pgfqpoint{6.078511in}{9.146602in}}%
\pgfpathlineto{\pgfqpoint{6.079902in}{9.148104in}}%
\pgfpathlineto{\pgfqpoint{6.082623in}{9.149607in}}%
\pgfpathlineto{\pgfqpoint{6.083773in}{9.152912in}}%
\pgfpathlineto{\pgfqpoint{6.085043in}{9.154114in}}%
\pgfpathlineto{\pgfqpoint{6.085527in}{9.155015in}}%
\pgfpathlineto{\pgfqpoint{6.088309in}{9.156217in}}%
\pgfpathlineto{\pgfqpoint{6.089821in}{9.158621in}}%
\pgfpathlineto{\pgfqpoint{6.092482in}{9.160123in}}%
\pgfpathlineto{\pgfqpoint{6.093933in}{9.162226in}}%
\pgfpathlineto{\pgfqpoint{6.095506in}{9.163729in}}%
\pgfpathlineto{\pgfqpoint{6.096716in}{9.164931in}}%
\pgfpathlineto{\pgfqpoint{6.099679in}{9.166433in}}%
\pgfpathlineto{\pgfqpoint{6.101010in}{9.167935in}}%
\pgfpathlineto{\pgfqpoint{6.102582in}{9.168536in}}%
\pgfpathlineto{\pgfqpoint{6.103731in}{9.171541in}}%
\pgfpathlineto{\pgfqpoint{6.105122in}{9.173043in}}%
\pgfpathlineto{\pgfqpoint{6.105122in}{9.173344in}}%
\pgfpathlineto{\pgfqpoint{6.108449in}{9.174546in}}%
\pgfpathlineto{\pgfqpoint{6.109779in}{9.177250in}}%
\pgfpathlineto{\pgfqpoint{6.110929in}{9.178752in}}%
\pgfpathlineto{\pgfqpoint{6.111957in}{9.179954in}}%
\pgfpathlineto{\pgfqpoint{6.114134in}{9.181456in}}%
\pgfpathlineto{\pgfqpoint{6.115646in}{9.183259in}}%
\pgfpathlineto{\pgfqpoint{6.116432in}{9.184762in}}%
\pgfpathlineto{\pgfqpoint{6.117581in}{9.186564in}}%
\pgfpathlineto{\pgfqpoint{6.120122in}{9.188067in}}%
\pgfpathlineto{\pgfqpoint{6.121513in}{9.189870in}}%
\pgfpathlineto{\pgfqpoint{6.122964in}{9.191372in}}%
\pgfpathlineto{\pgfqpoint{6.124355in}{9.192874in}}%
\pgfpathlineto{\pgfqpoint{6.125867in}{9.194377in}}%
\pgfpathlineto{\pgfqpoint{6.126230in}{9.195278in}}%
\pgfpathlineto{\pgfqpoint{6.128105in}{9.196780in}}%
\pgfpathlineto{\pgfqpoint{6.128407in}{9.197682in}}%
\pgfpathlineto{\pgfqpoint{6.131915in}{9.198884in}}%
\pgfpathlineto{\pgfqpoint{6.133246in}{9.200987in}}%
\pgfpathlineto{\pgfqpoint{6.135907in}{9.202489in}}%
\pgfpathlineto{\pgfqpoint{6.137419in}{9.205194in}}%
\pgfpathlineto{\pgfqpoint{6.138931in}{9.206395in}}%
\pgfpathlineto{\pgfqpoint{6.140383in}{9.209701in}}%
\pgfpathlineto{\pgfqpoint{6.142499in}{9.211203in}}%
\pgfpathlineto{\pgfqpoint{6.143891in}{9.212705in}}%
\pgfpathlineto{\pgfqpoint{6.146794in}{9.214208in}}%
\pgfpathlineto{\pgfqpoint{6.148245in}{9.216010in}}%
\pgfpathlineto{\pgfqpoint{6.150241in}{9.217513in}}%
\pgfpathlineto{\pgfqpoint{6.151753in}{9.221118in}}%
\pgfpathlineto{\pgfqpoint{6.153688in}{9.222621in}}%
\pgfpathlineto{\pgfqpoint{6.155140in}{9.224123in}}%
\pgfpathlineto{\pgfqpoint{6.157438in}{9.225625in}}%
\pgfpathlineto{\pgfqpoint{6.157983in}{9.226827in}}%
\pgfpathlineto{\pgfqpoint{6.162277in}{9.228330in}}%
\pgfpathlineto{\pgfqpoint{6.163728in}{9.230133in}}%
\pgfpathlineto{\pgfqpoint{6.165119in}{9.231034in}}%
\pgfpathlineto{\pgfqpoint{6.166571in}{9.233438in}}%
\pgfpathlineto{\pgfqpoint{6.169172in}{9.234940in}}%
\pgfpathlineto{\pgfqpoint{6.170381in}{9.236442in}}%
\pgfpathlineto{\pgfqpoint{6.172498in}{9.237945in}}%
\pgfpathlineto{\pgfqpoint{6.174010in}{9.239748in}}%
\pgfpathlineto{\pgfqpoint{6.177276in}{9.241250in}}%
\pgfpathlineto{\pgfqpoint{6.178425in}{9.242151in}}%
\pgfpathlineto{\pgfqpoint{6.180119in}{9.243654in}}%
\pgfpathlineto{\pgfqpoint{6.180784in}{9.244555in}}%
\pgfpathlineto{\pgfqpoint{6.183747in}{9.246057in}}%
\pgfpathlineto{\pgfqpoint{6.185259in}{9.247860in}}%
\pgfpathlineto{\pgfqpoint{6.186469in}{9.249363in}}%
\pgfpathlineto{\pgfqpoint{6.187255in}{9.250564in}}%
\pgfpathlineto{\pgfqpoint{6.189251in}{9.251766in}}%
\pgfpathlineto{\pgfqpoint{6.190763in}{9.254471in}}%
\pgfpathlineto{\pgfqpoint{6.195844in}{9.255973in}}%
\pgfpathlineto{\pgfqpoint{6.197356in}{9.258076in}}%
\pgfpathlineto{\pgfqpoint{6.198747in}{9.259579in}}%
\pgfpathlineto{\pgfqpoint{6.199412in}{9.261081in}}%
\pgfpathlineto{\pgfqpoint{6.203404in}{9.262583in}}%
\pgfpathlineto{\pgfqpoint{6.204855in}{9.264386in}}%
\pgfpathlineto{\pgfqpoint{6.208303in}{9.265888in}}%
\pgfpathlineto{\pgfqpoint{6.209815in}{9.267992in}}%
\pgfpathlineto{\pgfqpoint{6.213141in}{9.269494in}}%
\pgfpathlineto{\pgfqpoint{6.214653in}{9.271297in}}%
\pgfpathlineto{\pgfqpoint{6.218645in}{9.272799in}}%
\pgfpathlineto{\pgfqpoint{6.219975in}{9.274001in}}%
\pgfpathlineto{\pgfqpoint{6.221427in}{9.275503in}}%
\pgfpathlineto{\pgfqpoint{6.222758in}{9.278208in}}%
\pgfpathlineto{\pgfqpoint{6.223604in}{9.279710in}}%
\pgfpathlineto{\pgfqpoint{6.225116in}{9.281813in}}%
\pgfpathlineto{\pgfqpoint{6.228382in}{9.283316in}}%
\pgfpathlineto{\pgfqpoint{6.229834in}{9.284818in}}%
\pgfpathlineto{\pgfqpoint{6.232253in}{9.286320in}}%
\pgfpathlineto{\pgfqpoint{6.233523in}{9.289625in}}%
\pgfpathlineto{\pgfqpoint{6.235700in}{9.291128in}}%
\pgfpathlineto{\pgfqpoint{6.237091in}{9.292630in}}%
\pgfpathlineto{\pgfqpoint{6.240962in}{9.294132in}}%
\pgfpathlineto{\pgfqpoint{6.241507in}{9.295334in}}%
\pgfpathlineto{\pgfqpoint{6.245377in}{9.296837in}}%
\pgfpathlineto{\pgfqpoint{6.246587in}{9.297738in}}%
\pgfpathlineto{\pgfqpoint{6.249974in}{9.298940in}}%
\pgfpathlineto{\pgfqpoint{6.251365in}{9.301043in}}%
\pgfpathlineto{\pgfqpoint{6.253784in}{9.302546in}}%
\pgfpathlineto{\pgfqpoint{6.255296in}{9.304649in}}%
\pgfpathlineto{\pgfqpoint{6.257715in}{9.305550in}}%
\pgfpathlineto{\pgfqpoint{6.258502in}{9.307654in}}%
\pgfpathlineto{\pgfqpoint{6.261465in}{9.309156in}}%
\pgfpathlineto{\pgfqpoint{6.262856in}{9.310358in}}%
\pgfpathlineto{\pgfqpoint{6.266183in}{9.311860in}}%
\pgfpathlineto{\pgfqpoint{6.267695in}{9.313663in}}%
\pgfpathlineto{\pgfqpoint{6.270961in}{9.315165in}}%
\pgfpathlineto{\pgfqpoint{6.272352in}{9.316067in}}%
\pgfpathlineto{\pgfqpoint{6.279549in}{9.317569in}}%
\pgfpathlineto{\pgfqpoint{6.280698in}{9.318771in}}%
\pgfpathlineto{\pgfqpoint{6.283178in}{9.320273in}}%
\pgfpathlineto{\pgfqpoint{6.284085in}{9.321475in}}%
\pgfpathlineto{\pgfqpoint{6.291222in}{9.322978in}}%
\pgfpathlineto{\pgfqpoint{6.292492in}{9.324179in}}%
\pgfpathlineto{\pgfqpoint{6.295758in}{9.325682in}}%
\pgfpathlineto{\pgfqpoint{6.296786in}{9.326283in}}%
\pgfpathlineto{\pgfqpoint{6.300294in}{9.327785in}}%
\pgfpathlineto{\pgfqpoint{6.301745in}{9.328987in}}%
\pgfpathlineto{\pgfqpoint{6.305011in}{9.330489in}}%
\pgfpathlineto{\pgfqpoint{6.306523in}{9.331391in}}%
\pgfpathlineto{\pgfqpoint{6.308459in}{9.332893in}}%
\pgfpathlineto{\pgfqpoint{6.309910in}{9.334696in}}%
\pgfpathlineto{\pgfqpoint{6.316140in}{9.336198in}}%
\pgfpathlineto{\pgfqpoint{6.317229in}{9.337400in}}%
\pgfpathlineto{\pgfqpoint{6.321160in}{9.338902in}}%
\pgfpathlineto{\pgfqpoint{6.322672in}{9.340405in}}%
\pgfpathlineto{\pgfqpoint{6.326664in}{9.341907in}}%
\pgfpathlineto{\pgfqpoint{6.327934in}{9.342809in}}%
\pgfpathlineto{\pgfqpoint{6.330837in}{9.344311in}}%
\pgfpathlineto{\pgfqpoint{6.330837in}{9.344611in}}%
\pgfpathlineto{\pgfqpoint{6.337913in}{9.346114in}}%
\pgfpathlineto{\pgfqpoint{6.338397in}{9.346715in}}%
\pgfpathlineto{\pgfqpoint{6.342570in}{9.348217in}}%
\pgfpathlineto{\pgfqpoint{6.343840in}{9.349118in}}%
\pgfpathlineto{\pgfqpoint{6.346259in}{9.350621in}}%
\pgfpathlineto{\pgfqpoint{6.347167in}{9.351522in}}%
\pgfpathlineto{\pgfqpoint{6.349525in}{9.353024in}}%
\pgfpathlineto{\pgfqpoint{6.350614in}{9.354226in}}%
\pgfpathlineto{\pgfqpoint{6.353698in}{9.355729in}}%
\pgfpathlineto{\pgfqpoint{6.354182in}{9.356330in}}%
\pgfpathlineto{\pgfqpoint{6.359444in}{9.357832in}}%
\pgfpathlineto{\pgfqpoint{6.360835in}{9.358733in}}%
\pgfpathlineto{\pgfqpoint{6.363678in}{9.360236in}}%
\pgfpathlineto{\pgfqpoint{6.365008in}{9.361438in}}%
\pgfpathlineto{\pgfqpoint{6.369907in}{9.362940in}}%
\pgfpathlineto{\pgfqpoint{6.371238in}{9.364142in}}%
\pgfpathlineto{\pgfqpoint{6.374564in}{9.365644in}}%
\pgfpathlineto{\pgfqpoint{6.375955in}{9.367147in}}%
\pgfpathlineto{\pgfqpoint{6.377407in}{9.368649in}}%
\pgfpathlineto{\pgfqpoint{6.378858in}{9.370752in}}%
\pgfpathlineto{\pgfqpoint{6.383455in}{9.372255in}}%
\pgfpathlineto{\pgfqpoint{6.383455in}{9.372555in}}%
\pgfpathlineto{\pgfqpoint{6.388596in}{9.374057in}}%
\pgfpathlineto{\pgfqpoint{6.389443in}{9.374658in}}%
\pgfpathlineto{\pgfqpoint{6.394765in}{9.376161in}}%
\pgfpathlineto{\pgfqpoint{6.394765in}{9.376461in}}%
\pgfpathlineto{\pgfqpoint{6.402083in}{9.377963in}}%
\pgfpathlineto{\pgfqpoint{6.402385in}{9.378564in}}%
\pgfpathlineto{\pgfqpoint{6.406438in}{9.380067in}}%
\pgfpathlineto{\pgfqpoint{6.407829in}{9.380668in}}%
\pgfpathlineto{\pgfqpoint{6.411579in}{9.382170in}}%
\pgfpathlineto{\pgfqpoint{6.412728in}{9.383071in}}%
\pgfpathlineto{\pgfqpoint{6.419441in}{9.384574in}}%
\pgfpathlineto{\pgfqpoint{6.420288in}{9.385475in}}%
\pgfpathlineto{\pgfqpoint{6.425066in}{9.386978in}}%
\pgfpathlineto{\pgfqpoint{6.425852in}{9.387578in}}%
\pgfpathlineto{\pgfqpoint{6.431477in}{9.389081in}}%
\pgfpathlineto{\pgfqpoint{6.432747in}{9.389982in}}%
\pgfpathlineto{\pgfqpoint{6.438916in}{9.391485in}}%
\pgfpathlineto{\pgfqpoint{6.438916in}{9.391785in}}%
\pgfpathlineto{\pgfqpoint{6.443875in}{9.393287in}}%
\pgfpathlineto{\pgfqpoint{6.444843in}{9.393888in}}%
\pgfpathlineto{\pgfqpoint{6.445387in}{9.393888in}}%
\pgfpathlineto{\pgfqpoint{6.452645in}{9.395391in}}%
\pgfpathlineto{\pgfqpoint{6.453976in}{9.395992in}}%
\pgfpathlineto{\pgfqpoint{6.467765in}{9.397494in}}%
\pgfpathlineto{\pgfqpoint{6.469156in}{9.398395in}}%
\pgfpathlineto{\pgfqpoint{6.475749in}{9.399898in}}%
\pgfpathlineto{\pgfqpoint{6.476595in}{9.400499in}}%
\pgfpathlineto{\pgfqpoint{6.486817in}{9.402001in}}%
\pgfpathlineto{\pgfqpoint{6.486998in}{9.402602in}}%
\pgfpathlineto{\pgfqpoint{6.492925in}{9.404104in}}%
\pgfpathlineto{\pgfqpoint{6.493107in}{9.404705in}}%
\pgfpathlineto{\pgfqpoint{6.497764in}{9.406208in}}%
\pgfpathlineto{\pgfqpoint{6.499094in}{9.407409in}}%
\pgfpathlineto{\pgfqpoint{6.502723in}{9.408912in}}%
\pgfpathlineto{\pgfqpoint{6.504054in}{9.410114in}}%
\pgfpathlineto{\pgfqpoint{6.510344in}{9.411616in}}%
\pgfpathlineto{\pgfqpoint{6.510344in}{9.411916in}}%
\pgfpathlineto{\pgfqpoint{6.517299in}{9.413419in}}%
\pgfpathlineto{\pgfqpoint{6.518327in}{9.414320in}}%
\pgfpathlineto{\pgfqpoint{6.525887in}{9.415823in}}%
\pgfpathlineto{\pgfqpoint{6.526734in}{9.417625in}}%
\pgfpathlineto{\pgfqpoint{6.534415in}{9.419128in}}%
\pgfpathlineto{\pgfqpoint{6.535746in}{9.420931in}}%
\pgfpathlineto{\pgfqpoint{6.542217in}{9.422433in}}%
\pgfpathlineto{\pgfqpoint{6.542217in}{9.422733in}}%
\pgfpathlineto{\pgfqpoint{6.547177in}{9.424236in}}%
\pgfpathlineto{\pgfqpoint{6.547177in}{9.424536in}}%
\pgfpathlineto{\pgfqpoint{6.561027in}{9.426039in}}%
\pgfpathlineto{\pgfqpoint{6.562297in}{9.426940in}}%
\pgfpathlineto{\pgfqpoint{6.569554in}{9.428442in}}%
\pgfpathlineto{\pgfqpoint{6.570583in}{9.429043in}}%
\pgfpathlineto{\pgfqpoint{6.580380in}{9.430546in}}%
\pgfpathlineto{\pgfqpoint{6.581832in}{9.432048in}}%
\pgfpathlineto{\pgfqpoint{6.587396in}{9.433550in}}%
\pgfpathlineto{\pgfqpoint{6.588424in}{9.434452in}}%
\pgfpathlineto{\pgfqpoint{6.593384in}{9.435954in}}%
\pgfpathlineto{\pgfqpoint{6.593384in}{9.436555in}}%
\pgfpathlineto{\pgfqpoint{6.600944in}{9.438057in}}%
\pgfpathlineto{\pgfqpoint{6.600944in}{9.438358in}}%
\pgfpathlineto{\pgfqpoint{6.607597in}{9.439860in}}%
\pgfpathlineto{\pgfqpoint{6.608988in}{9.440762in}}%
\pgfpathlineto{\pgfqpoint{6.623140in}{9.441963in}}%
\pgfpathlineto{\pgfqpoint{6.623140in}{9.442564in}}%
\pgfpathlineto{\pgfqpoint{6.633846in}{9.443766in}}%
\pgfpathlineto{\pgfqpoint{6.633846in}{9.444367in}}%
\pgfpathlineto{\pgfqpoint{6.642010in}{9.445870in}}%
\pgfpathlineto{\pgfqpoint{6.642857in}{9.446470in}}%
\pgfpathlineto{\pgfqpoint{6.648119in}{9.447973in}}%
\pgfpathlineto{\pgfqpoint{6.648119in}{9.448273in}}%
\pgfpathlineto{\pgfqpoint{6.665416in}{9.449776in}}%
\pgfpathlineto{\pgfqpoint{6.666082in}{9.450377in}}%
\pgfpathlineto{\pgfqpoint{6.682230in}{9.451879in}}%
\pgfpathlineto{\pgfqpoint{6.682533in}{9.452480in}}%
\pgfpathlineto{\pgfqpoint{6.694266in}{9.453982in}}%
\pgfpathlineto{\pgfqpoint{6.695778in}{9.455184in}}%
\pgfpathlineto{\pgfqpoint{6.706060in}{9.456686in}}%
\pgfpathlineto{\pgfqpoint{6.706604in}{9.457287in}}%
\pgfpathlineto{\pgfqpoint{6.719910in}{9.458790in}}%
\pgfpathlineto{\pgfqpoint{6.719910in}{9.459090in}}%
\pgfpathlineto{\pgfqpoint{6.726925in}{9.460593in}}%
\pgfpathlineto{\pgfqpoint{6.726925in}{9.460893in}}%
\pgfpathlineto{\pgfqpoint{6.743860in}{9.462395in}}%
\pgfpathlineto{\pgfqpoint{6.743860in}{9.462696in}}%
\pgfpathlineto{\pgfqpoint{6.759646in}{9.464198in}}%
\pgfpathlineto{\pgfqpoint{6.759706in}{9.464799in}}%
\pgfpathlineto{\pgfqpoint{6.778818in}{9.466301in}}%
\pgfpathlineto{\pgfqpoint{6.778818in}{9.466602in}}%
\pgfpathlineto{\pgfqpoint{6.795813in}{9.468104in}}%
\pgfpathlineto{\pgfqpoint{6.796236in}{9.468705in}}%
\pgfpathlineto{\pgfqpoint{6.809240in}{9.470208in}}%
\pgfpathlineto{\pgfqpoint{6.809240in}{9.470508in}}%
\pgfpathlineto{\pgfqpoint{6.828594in}{9.472010in}}%
\pgfpathlineto{\pgfqpoint{6.828715in}{9.472611in}}%
\pgfpathlineto{\pgfqpoint{6.854117in}{9.474114in}}%
\pgfpathlineto{\pgfqpoint{6.854117in}{9.474414in}}%
\pgfpathlineto{\pgfqpoint{6.863370in}{9.475916in}}%
\pgfpathlineto{\pgfqpoint{6.864761in}{9.477118in}}%
\pgfpathlineto{\pgfqpoint{6.881998in}{9.478621in}}%
\pgfpathlineto{\pgfqpoint{6.881998in}{9.478921in}}%
\pgfpathlineto{\pgfqpoint{6.901715in}{9.480424in}}%
\pgfpathlineto{\pgfqpoint{6.902199in}{9.481024in}}%
\pgfpathlineto{\pgfqpoint{6.920706in}{9.482527in}}%
\pgfpathlineto{\pgfqpoint{6.921250in}{9.483128in}}%
\pgfpathlineto{\pgfqpoint{6.942600in}{9.484630in}}%
\pgfpathlineto{\pgfqpoint{6.942963in}{9.485231in}}%
\pgfpathlineto{\pgfqpoint{6.964494in}{9.486733in}}%
\pgfpathlineto{\pgfqpoint{6.964494in}{9.487034in}}%
\pgfpathlineto{\pgfqpoint{6.986872in}{9.488536in}}%
\pgfpathlineto{\pgfqpoint{6.987658in}{9.489137in}}%
\pgfpathlineto{\pgfqpoint{7.012939in}{9.490639in}}%
\pgfpathlineto{\pgfqpoint{7.012939in}{9.490940in}}%
\pgfpathlineto{\pgfqpoint{7.052856in}{9.492442in}}%
\pgfpathlineto{\pgfqpoint{7.053219in}{9.493043in}}%
\pgfpathlineto{\pgfqpoint{7.092411in}{9.494546in}}%
\pgfpathlineto{\pgfqpoint{7.092411in}{9.494846in}}%
\pgfpathlineto{\pgfqpoint{7.172911in}{9.496348in}}%
\pgfpathlineto{\pgfqpoint{7.172911in}{9.496649in}}%
\pgfpathlineto{\pgfqpoint{7.173455in}{9.496649in}}%
\pgfpathlineto{\pgfqpoint{7.298650in}{9.496649in}}%
\pgfpathlineto{\pgfqpoint{7.298650in}{9.496649in}}%
\pgfusepath{stroke}%
\end{pgfscope}%
\begin{pgfscope}%
\pgfpathrectangle{\pgfqpoint{5.688041in}{8.027083in}}{\pgfqpoint{1.687305in}{1.539545in}}%
\pgfusepath{clip}%
\pgfsetrectcap%
\pgfsetroundjoin%
\pgfsetlinewidth{1.505625pt}%
\definecolor{currentstroke}{rgb}{0.501961,0.501961,0.501961}%
\pgfsetstrokecolor{currentstroke}%
\pgfsetdash{}{0pt}%
\pgfpathmoveto{\pgfqpoint{5.764736in}{8.097062in}}%
\pgfpathlineto{\pgfqpoint{7.298650in}{9.496649in}}%
\pgfusepath{stroke}%
\end{pgfscope}%
\begin{pgfscope}%
\pgfsetrectcap%
\pgfsetmiterjoin%
\pgfsetlinewidth{0.803000pt}%
\definecolor{currentstroke}{rgb}{0.000000,0.000000,0.000000}%
\pgfsetstrokecolor{currentstroke}%
\pgfsetdash{}{0pt}%
\pgfpathmoveto{\pgfqpoint{5.688041in}{8.027083in}}%
\pgfpathlineto{\pgfqpoint{5.688041in}{9.566628in}}%
\pgfusepath{stroke}%
\end{pgfscope}%
\begin{pgfscope}%
\pgfsetrectcap%
\pgfsetmiterjoin%
\pgfsetlinewidth{0.803000pt}%
\definecolor{currentstroke}{rgb}{0.000000,0.000000,0.000000}%
\pgfsetstrokecolor{currentstroke}%
\pgfsetdash{}{0pt}%
\pgfpathmoveto{\pgfqpoint{7.375346in}{8.027083in}}%
\pgfpathlineto{\pgfqpoint{7.375346in}{9.566628in}}%
\pgfusepath{stroke}%
\end{pgfscope}%
\begin{pgfscope}%
\pgfsetrectcap%
\pgfsetmiterjoin%
\pgfsetlinewidth{0.803000pt}%
\definecolor{currentstroke}{rgb}{0.000000,0.000000,0.000000}%
\pgfsetstrokecolor{currentstroke}%
\pgfsetdash{}{0pt}%
\pgfpathmoveto{\pgfqpoint{5.688041in}{8.027083in}}%
\pgfpathlineto{\pgfqpoint{7.375346in}{8.027083in}}%
\pgfusepath{stroke}%
\end{pgfscope}%
\begin{pgfscope}%
\pgfsetrectcap%
\pgfsetmiterjoin%
\pgfsetlinewidth{0.803000pt}%
\definecolor{currentstroke}{rgb}{0.000000,0.000000,0.000000}%
\pgfsetstrokecolor{currentstroke}%
\pgfsetdash{}{0pt}%
\pgfpathmoveto{\pgfqpoint{5.688041in}{9.566628in}}%
\pgfpathlineto{\pgfqpoint{7.375346in}{9.566628in}}%
\pgfusepath{stroke}%
\end{pgfscope}%
\begin{pgfscope}%
\definecolor{textcolor}{rgb}{0.000000,0.000000,0.000000}%
\pgfsetstrokecolor{textcolor}%
\pgfsetfillcolor{textcolor}%
\pgftext[x=6.531693in,y=9.649962in,,base]{\color{textcolor}\rmfamily\fontsize{20.000000}{24.000000}\selectfont Effusion}%
\end{pgfscope}%
\begin{pgfscope}%
\pgfsetbuttcap%
\pgfsetmiterjoin%
\definecolor{currentfill}{rgb}{1.000000,1.000000,1.000000}%
\pgfsetfillcolor{currentfill}%
\pgfsetfillopacity{0.800000}%
\pgfsetlinewidth{1.003750pt}%
\definecolor{currentstroke}{rgb}{0.800000,0.800000,0.800000}%
\pgfsetstrokecolor{currentstroke}%
\pgfsetstrokeopacity{0.800000}%
\pgfsetdash{}{0pt}%
\pgfpathmoveto{\pgfqpoint{6.166240in}{8.096527in}}%
\pgfpathlineto{\pgfqpoint{7.278124in}{8.096527in}}%
\pgfpathquadraticcurveto{\pgfqpoint{7.305902in}{8.096527in}}{\pgfqpoint{7.305902in}{8.124305in}}%
\pgfpathlineto{\pgfqpoint{7.305902in}{8.304089in}}%
\pgfpathquadraticcurveto{\pgfqpoint{7.305902in}{8.331867in}}{\pgfqpoint{7.278124in}{8.331867in}}%
\pgfpathlineto{\pgfqpoint{6.166240in}{8.331867in}}%
\pgfpathquadraticcurveto{\pgfqpoint{6.138462in}{8.331867in}}{\pgfqpoint{6.138462in}{8.304089in}}%
\pgfpathlineto{\pgfqpoint{6.138462in}{8.124305in}}%
\pgfpathquadraticcurveto{\pgfqpoint{6.138462in}{8.096527in}}{\pgfqpoint{6.166240in}{8.096527in}}%
\pgfpathclose%
\pgfusepath{stroke,fill}%
\end{pgfscope}%
\begin{pgfscope}%
\pgfsetrectcap%
\pgfsetroundjoin%
\pgfsetlinewidth{1.505625pt}%
\definecolor{currentstroke}{rgb}{0.000000,0.501961,0.000000}%
\pgfsetstrokecolor{currentstroke}%
\pgfsetdash{}{0pt}%
\pgfpathmoveto{\pgfqpoint{6.194018in}{8.227700in}}%
\pgfpathlineto{\pgfqpoint{6.471795in}{8.227700in}}%
\pgfusepath{stroke}%
\end{pgfscope}%
\begin{pgfscope}%
\definecolor{textcolor}{rgb}{0.000000,0.000000,0.000000}%
\pgfsetstrokecolor{textcolor}%
\pgfsetfillcolor{textcolor}%
\pgftext[x=6.582907in,y=8.179089in,left,base]{\color{textcolor}\rmfamily\fontsize{10.000000}{12.000000}\selectfont AUC 0.850}%
\end{pgfscope}%
\begin{pgfscope}%
\pgfsetbuttcap%
\pgfsetmiterjoin%
\definecolor{currentfill}{rgb}{1.000000,1.000000,1.000000}%
\pgfsetfillcolor{currentfill}%
\pgfsetlinewidth{0.000000pt}%
\definecolor{currentstroke}{rgb}{0.000000,0.000000,0.000000}%
\pgfsetstrokecolor{currentstroke}%
\pgfsetstrokeopacity{0.000000}%
\pgfsetdash{}{0pt}%
\pgfpathmoveto{\pgfqpoint{8.150541in}{8.027083in}}%
\pgfpathlineto{\pgfqpoint{9.837846in}{8.027083in}}%
\pgfpathlineto{\pgfqpoint{9.837846in}{9.566628in}}%
\pgfpathlineto{\pgfqpoint{8.150541in}{9.566628in}}%
\pgfpathclose%
\pgfusepath{fill}%
\end{pgfscope}%
\begin{pgfscope}%
\pgfsetbuttcap%
\pgfsetroundjoin%
\definecolor{currentfill}{rgb}{0.000000,0.000000,0.000000}%
\pgfsetfillcolor{currentfill}%
\pgfsetlinewidth{0.803000pt}%
\definecolor{currentstroke}{rgb}{0.000000,0.000000,0.000000}%
\pgfsetstrokecolor{currentstroke}%
\pgfsetdash{}{0pt}%
\pgfsys@defobject{currentmarker}{\pgfqpoint{0.000000in}{-0.048611in}}{\pgfqpoint{0.000000in}{0.000000in}}{%
\pgfpathmoveto{\pgfqpoint{0.000000in}{0.000000in}}%
\pgfpathlineto{\pgfqpoint{0.000000in}{-0.048611in}}%
\pgfusepath{stroke,fill}%
}%
\begin{pgfscope}%
\pgfsys@transformshift{8.227236in}{8.027083in}%
\pgfsys@useobject{currentmarker}{}%
\end{pgfscope}%
\end{pgfscope}%
\begin{pgfscope}%
\definecolor{textcolor}{rgb}{0.000000,0.000000,0.000000}%
\pgfsetstrokecolor{textcolor}%
\pgfsetfillcolor{textcolor}%
\pgftext[x=8.227236in,y=7.929861in,,top]{\color{textcolor}\rmfamily\fontsize{10.000000}{12.000000}\selectfont \(\displaystyle {0.0}\)}%
\end{pgfscope}%
\begin{pgfscope}%
\pgfsetbuttcap%
\pgfsetroundjoin%
\definecolor{currentfill}{rgb}{0.000000,0.000000,0.000000}%
\pgfsetfillcolor{currentfill}%
\pgfsetlinewidth{0.803000pt}%
\definecolor{currentstroke}{rgb}{0.000000,0.000000,0.000000}%
\pgfsetstrokecolor{currentstroke}%
\pgfsetdash{}{0pt}%
\pgfsys@defobject{currentmarker}{\pgfqpoint{0.000000in}{-0.048611in}}{\pgfqpoint{0.000000in}{0.000000in}}{%
\pgfpathmoveto{\pgfqpoint{0.000000in}{0.000000in}}%
\pgfpathlineto{\pgfqpoint{0.000000in}{-0.048611in}}%
\pgfusepath{stroke,fill}%
}%
\begin{pgfscope}%
\pgfsys@transformshift{8.994193in}{8.027083in}%
\pgfsys@useobject{currentmarker}{}%
\end{pgfscope}%
\end{pgfscope}%
\begin{pgfscope}%
\definecolor{textcolor}{rgb}{0.000000,0.000000,0.000000}%
\pgfsetstrokecolor{textcolor}%
\pgfsetfillcolor{textcolor}%
\pgftext[x=8.994193in,y=7.929861in,,top]{\color{textcolor}\rmfamily\fontsize{10.000000}{12.000000}\selectfont \(\displaystyle {0.5}\)}%
\end{pgfscope}%
\begin{pgfscope}%
\pgfsetbuttcap%
\pgfsetroundjoin%
\definecolor{currentfill}{rgb}{0.000000,0.000000,0.000000}%
\pgfsetfillcolor{currentfill}%
\pgfsetlinewidth{0.803000pt}%
\definecolor{currentstroke}{rgb}{0.000000,0.000000,0.000000}%
\pgfsetstrokecolor{currentstroke}%
\pgfsetdash{}{0pt}%
\pgfsys@defobject{currentmarker}{\pgfqpoint{0.000000in}{-0.048611in}}{\pgfqpoint{0.000000in}{0.000000in}}{%
\pgfpathmoveto{\pgfqpoint{0.000000in}{0.000000in}}%
\pgfpathlineto{\pgfqpoint{0.000000in}{-0.048611in}}%
\pgfusepath{stroke,fill}%
}%
\begin{pgfscope}%
\pgfsys@transformshift{9.761150in}{8.027083in}%
\pgfsys@useobject{currentmarker}{}%
\end{pgfscope}%
\end{pgfscope}%
\begin{pgfscope}%
\definecolor{textcolor}{rgb}{0.000000,0.000000,0.000000}%
\pgfsetstrokecolor{textcolor}%
\pgfsetfillcolor{textcolor}%
\pgftext[x=9.761150in,y=7.929861in,,top]{\color{textcolor}\rmfamily\fontsize{10.000000}{12.000000}\selectfont \(\displaystyle {1.0}\)}%
\end{pgfscope}%
\begin{pgfscope}%
\definecolor{textcolor}{rgb}{0.000000,0.000000,0.000000}%
\pgfsetstrokecolor{textcolor}%
\pgfsetfillcolor{textcolor}%
\pgftext[x=8.994193in,y=7.750849in,,top]{\color{textcolor}\rmfamily\fontsize{16.000000}{19.200000}\selectfont FPR}%
\end{pgfscope}%
\begin{pgfscope}%
\pgfsetbuttcap%
\pgfsetroundjoin%
\definecolor{currentfill}{rgb}{0.000000,0.000000,0.000000}%
\pgfsetfillcolor{currentfill}%
\pgfsetlinewidth{0.803000pt}%
\definecolor{currentstroke}{rgb}{0.000000,0.000000,0.000000}%
\pgfsetstrokecolor{currentstroke}%
\pgfsetdash{}{0pt}%
\pgfsys@defobject{currentmarker}{\pgfqpoint{-0.048611in}{0.000000in}}{\pgfqpoint{-0.000000in}{0.000000in}}{%
\pgfpathmoveto{\pgfqpoint{-0.000000in}{0.000000in}}%
\pgfpathlineto{\pgfqpoint{-0.048611in}{0.000000in}}%
\pgfusepath{stroke,fill}%
}%
\begin{pgfscope}%
\pgfsys@transformshift{8.150541in}{8.097062in}%
\pgfsys@useobject{currentmarker}{}%
\end{pgfscope}%
\end{pgfscope}%
\begin{pgfscope}%
\definecolor{textcolor}{rgb}{0.000000,0.000000,0.000000}%
\pgfsetstrokecolor{textcolor}%
\pgfsetfillcolor{textcolor}%
\pgftext[x=7.806404in, y=8.048837in, left, base]{\color{textcolor}\rmfamily\fontsize{10.000000}{12.000000}\selectfont \(\displaystyle {0.00}\)}%
\end{pgfscope}%
\begin{pgfscope}%
\pgfsetbuttcap%
\pgfsetroundjoin%
\definecolor{currentfill}{rgb}{0.000000,0.000000,0.000000}%
\pgfsetfillcolor{currentfill}%
\pgfsetlinewidth{0.803000pt}%
\definecolor{currentstroke}{rgb}{0.000000,0.000000,0.000000}%
\pgfsetstrokecolor{currentstroke}%
\pgfsetdash{}{0pt}%
\pgfsys@defobject{currentmarker}{\pgfqpoint{-0.048611in}{0.000000in}}{\pgfqpoint{-0.000000in}{0.000000in}}{%
\pgfpathmoveto{\pgfqpoint{-0.000000in}{0.000000in}}%
\pgfpathlineto{\pgfqpoint{-0.048611in}{0.000000in}}%
\pgfusepath{stroke,fill}%
}%
\begin{pgfscope}%
\pgfsys@transformshift{8.150541in}{8.446959in}%
\pgfsys@useobject{currentmarker}{}%
\end{pgfscope}%
\end{pgfscope}%
\begin{pgfscope}%
\definecolor{textcolor}{rgb}{0.000000,0.000000,0.000000}%
\pgfsetstrokecolor{textcolor}%
\pgfsetfillcolor{textcolor}%
\pgftext[x=7.806404in, y=8.398734in, left, base]{\color{textcolor}\rmfamily\fontsize{10.000000}{12.000000}\selectfont \(\displaystyle {0.25}\)}%
\end{pgfscope}%
\begin{pgfscope}%
\pgfsetbuttcap%
\pgfsetroundjoin%
\definecolor{currentfill}{rgb}{0.000000,0.000000,0.000000}%
\pgfsetfillcolor{currentfill}%
\pgfsetlinewidth{0.803000pt}%
\definecolor{currentstroke}{rgb}{0.000000,0.000000,0.000000}%
\pgfsetstrokecolor{currentstroke}%
\pgfsetdash{}{0pt}%
\pgfsys@defobject{currentmarker}{\pgfqpoint{-0.048611in}{0.000000in}}{\pgfqpoint{-0.000000in}{0.000000in}}{%
\pgfpathmoveto{\pgfqpoint{-0.000000in}{0.000000in}}%
\pgfpathlineto{\pgfqpoint{-0.048611in}{0.000000in}}%
\pgfusepath{stroke,fill}%
}%
\begin{pgfscope}%
\pgfsys@transformshift{8.150541in}{8.796856in}%
\pgfsys@useobject{currentmarker}{}%
\end{pgfscope}%
\end{pgfscope}%
\begin{pgfscope}%
\definecolor{textcolor}{rgb}{0.000000,0.000000,0.000000}%
\pgfsetstrokecolor{textcolor}%
\pgfsetfillcolor{textcolor}%
\pgftext[x=7.806404in, y=8.748630in, left, base]{\color{textcolor}\rmfamily\fontsize{10.000000}{12.000000}\selectfont \(\displaystyle {0.50}\)}%
\end{pgfscope}%
\begin{pgfscope}%
\pgfsetbuttcap%
\pgfsetroundjoin%
\definecolor{currentfill}{rgb}{0.000000,0.000000,0.000000}%
\pgfsetfillcolor{currentfill}%
\pgfsetlinewidth{0.803000pt}%
\definecolor{currentstroke}{rgb}{0.000000,0.000000,0.000000}%
\pgfsetstrokecolor{currentstroke}%
\pgfsetdash{}{0pt}%
\pgfsys@defobject{currentmarker}{\pgfqpoint{-0.048611in}{0.000000in}}{\pgfqpoint{-0.000000in}{0.000000in}}{%
\pgfpathmoveto{\pgfqpoint{-0.000000in}{0.000000in}}%
\pgfpathlineto{\pgfqpoint{-0.048611in}{0.000000in}}%
\pgfusepath{stroke,fill}%
}%
\begin{pgfscope}%
\pgfsys@transformshift{8.150541in}{9.146752in}%
\pgfsys@useobject{currentmarker}{}%
\end{pgfscope}%
\end{pgfscope}%
\begin{pgfscope}%
\definecolor{textcolor}{rgb}{0.000000,0.000000,0.000000}%
\pgfsetstrokecolor{textcolor}%
\pgfsetfillcolor{textcolor}%
\pgftext[x=7.806404in, y=9.098527in, left, base]{\color{textcolor}\rmfamily\fontsize{10.000000}{12.000000}\selectfont \(\displaystyle {0.75}\)}%
\end{pgfscope}%
\begin{pgfscope}%
\pgfsetbuttcap%
\pgfsetroundjoin%
\definecolor{currentfill}{rgb}{0.000000,0.000000,0.000000}%
\pgfsetfillcolor{currentfill}%
\pgfsetlinewidth{0.803000pt}%
\definecolor{currentstroke}{rgb}{0.000000,0.000000,0.000000}%
\pgfsetstrokecolor{currentstroke}%
\pgfsetdash{}{0pt}%
\pgfsys@defobject{currentmarker}{\pgfqpoint{-0.048611in}{0.000000in}}{\pgfqpoint{-0.000000in}{0.000000in}}{%
\pgfpathmoveto{\pgfqpoint{-0.000000in}{0.000000in}}%
\pgfpathlineto{\pgfqpoint{-0.048611in}{0.000000in}}%
\pgfusepath{stroke,fill}%
}%
\begin{pgfscope}%
\pgfsys@transformshift{8.150541in}{9.496649in}%
\pgfsys@useobject{currentmarker}{}%
\end{pgfscope}%
\end{pgfscope}%
\begin{pgfscope}%
\definecolor{textcolor}{rgb}{0.000000,0.000000,0.000000}%
\pgfsetstrokecolor{textcolor}%
\pgfsetfillcolor{textcolor}%
\pgftext[x=7.806404in, y=9.448424in, left, base]{\color{textcolor}\rmfamily\fontsize{10.000000}{12.000000}\selectfont \(\displaystyle {1.00}\)}%
\end{pgfscope}%
\begin{pgfscope}%
\definecolor{textcolor}{rgb}{0.000000,0.000000,0.000000}%
\pgfsetstrokecolor{textcolor}%
\pgfsetfillcolor{textcolor}%
\pgftext[x=7.750849in,y=8.796856in,,bottom,rotate=90.000000]{\color{textcolor}\rmfamily\fontsize{16.000000}{19.200000}\selectfont TPR}%
\end{pgfscope}%
\begin{pgfscope}%
\pgfpathrectangle{\pgfqpoint{8.150541in}{8.027083in}}{\pgfqpoint{1.687305in}{1.539545in}}%
\pgfusepath{clip}%
\pgfsetrectcap%
\pgfsetroundjoin%
\pgfsetlinewidth{1.505625pt}%
\definecolor{currentstroke}{rgb}{0.000000,0.501961,0.000000}%
\pgfsetstrokecolor{currentstroke}%
\pgfsetdash{}{0pt}%
\pgfpathmoveto{\pgfqpoint{8.227236in}{8.097062in}}%
\pgfpathlineto{\pgfqpoint{8.228159in}{8.178434in}}%
\pgfpathlineto{\pgfqpoint{8.228313in}{8.178434in}}%
\pgfpathlineto{\pgfqpoint{8.229747in}{8.178434in}}%
\pgfpathlineto{\pgfqpoint{8.231182in}{8.243531in}}%
\pgfpathlineto{\pgfqpoint{8.231285in}{8.243531in}}%
\pgfpathlineto{\pgfqpoint{8.232924in}{8.243531in}}%
\pgfpathlineto{\pgfqpoint{8.234359in}{8.308628in}}%
\pgfpathlineto{\pgfqpoint{8.235435in}{8.308628in}}%
\pgfpathlineto{\pgfqpoint{8.236973in}{8.373725in}}%
\pgfpathlineto{\pgfqpoint{8.239330in}{8.373725in}}%
\pgfpathlineto{\pgfqpoint{8.239330in}{8.389999in}}%
\pgfpathlineto{\pgfqpoint{8.241943in}{8.389999in}}%
\pgfpathlineto{\pgfqpoint{8.242302in}{8.422548in}}%
\pgfpathlineto{\pgfqpoint{8.243788in}{8.422548in}}%
\pgfpathlineto{\pgfqpoint{8.245018in}{8.455096in}}%
\pgfpathlineto{\pgfqpoint{8.250296in}{8.455096in}}%
\pgfpathlineto{\pgfqpoint{8.250296in}{8.471370in}}%
\pgfpathlineto{\pgfqpoint{8.253627in}{8.471370in}}%
\pgfpathlineto{\pgfqpoint{8.255113in}{8.536467in}}%
\pgfpathlineto{\pgfqpoint{8.257982in}{8.536467in}}%
\pgfpathlineto{\pgfqpoint{8.257982in}{8.552742in}}%
\pgfpathlineto{\pgfqpoint{8.266745in}{8.552742in}}%
\pgfpathlineto{\pgfqpoint{8.266745in}{8.569016in}}%
\pgfpathlineto{\pgfqpoint{8.271357in}{8.569016in}}%
\pgfpathlineto{\pgfqpoint{8.271767in}{8.617839in}}%
\pgfpathlineto{\pgfqpoint{8.277557in}{8.617839in}}%
\pgfpathlineto{\pgfqpoint{8.277557in}{8.634113in}}%
\pgfpathlineto{\pgfqpoint{8.279248in}{8.634113in}}%
\pgfpathlineto{\pgfqpoint{8.279248in}{8.650387in}}%
\pgfpathlineto{\pgfqpoint{8.282887in}{8.650387in}}%
\pgfpathlineto{\pgfqpoint{8.282887in}{8.666662in}}%
\pgfpathlineto{\pgfqpoint{8.283706in}{8.666662in}}%
\pgfpathlineto{\pgfqpoint{8.291854in}{8.666662in}}%
\pgfpathlineto{\pgfqpoint{8.291854in}{8.682936in}}%
\pgfpathlineto{\pgfqpoint{8.303640in}{8.682936in}}%
\pgfpathlineto{\pgfqpoint{8.303640in}{8.699210in}}%
\pgfpathlineto{\pgfqpoint{8.307534in}{8.699210in}}%
\pgfpathlineto{\pgfqpoint{8.307534in}{8.715484in}}%
\pgfpathlineto{\pgfqpoint{8.321063in}{8.715484in}}%
\pgfpathlineto{\pgfqpoint{8.321063in}{8.731759in}}%
\pgfpathlineto{\pgfqpoint{8.324342in}{8.731759in}}%
\pgfpathlineto{\pgfqpoint{8.324342in}{8.748033in}}%
\pgfpathlineto{\pgfqpoint{8.344430in}{8.748033in}}%
\pgfpathlineto{\pgfqpoint{8.344788in}{8.780581in}}%
\pgfpathlineto{\pgfqpoint{8.348222in}{8.780581in}}%
\pgfpathlineto{\pgfqpoint{8.348222in}{8.796856in}}%
\pgfpathlineto{\pgfqpoint{8.360520in}{8.796856in}}%
\pgfpathlineto{\pgfqpoint{8.360520in}{8.829404in}}%
\pgfpathlineto{\pgfqpoint{8.369590in}{8.829404in}}%
\pgfpathlineto{\pgfqpoint{8.369590in}{8.845678in}}%
\pgfpathlineto{\pgfqpoint{8.375073in}{8.845678in}}%
\pgfpathlineto{\pgfqpoint{8.375073in}{8.861953in}}%
\pgfpathlineto{\pgfqpoint{8.401720in}{8.861953in}}%
\pgfpathlineto{\pgfqpoint{8.401720in}{8.878227in}}%
\pgfpathlineto{\pgfqpoint{8.403462in}{8.878227in}}%
\pgfpathlineto{\pgfqpoint{8.403462in}{8.894501in}}%
\pgfpathlineto{\pgfqpoint{8.412993in}{8.894501in}}%
\pgfpathlineto{\pgfqpoint{8.412993in}{8.910775in}}%
\pgfpathlineto{\pgfqpoint{8.424267in}{8.910775in}}%
\pgfpathlineto{\pgfqpoint{8.424267in}{8.927050in}}%
\pgfpathlineto{\pgfqpoint{8.437897in}{8.927050in}}%
\pgfpathlineto{\pgfqpoint{8.437897in}{8.943324in}}%
\pgfpathlineto{\pgfqpoint{8.448146in}{8.943324in}}%
\pgfpathlineto{\pgfqpoint{8.448146in}{8.959598in}}%
\pgfpathlineto{\pgfqpoint{8.450349in}{8.959598in}}%
\pgfpathlineto{\pgfqpoint{8.450349in}{8.975873in}}%
\pgfpathlineto{\pgfqpoint{8.451835in}{8.975873in}}%
\pgfpathlineto{\pgfqpoint{8.458651in}{8.975873in}}%
\pgfpathlineto{\pgfqpoint{8.458651in}{8.992147in}}%
\pgfpathlineto{\pgfqpoint{8.471052in}{8.992147in}}%
\pgfpathlineto{\pgfqpoint{8.471052in}{9.008421in}}%
\pgfpathlineto{\pgfqpoint{8.489704in}{9.008421in}}%
\pgfpathlineto{\pgfqpoint{8.489704in}{9.024695in}}%
\pgfpathlineto{\pgfqpoint{8.507024in}{9.024695in}}%
\pgfpathlineto{\pgfqpoint{8.507024in}{9.040970in}}%
\pgfpathlineto{\pgfqpoint{8.513840in}{9.040970in}}%
\pgfpathlineto{\pgfqpoint{8.513840in}{9.057244in}}%
\pgfpathlineto{\pgfqpoint{8.544791in}{9.057244in}}%
\pgfpathlineto{\pgfqpoint{8.544791in}{9.073518in}}%
\pgfpathlineto{\pgfqpoint{8.550017in}{9.073518in}}%
\pgfpathlineto{\pgfqpoint{8.551196in}{9.106067in}}%
\pgfpathlineto{\pgfqpoint{8.570873in}{9.106067in}}%
\pgfpathlineto{\pgfqpoint{8.570873in}{9.122341in}}%
\pgfpathlineto{\pgfqpoint{8.580456in}{9.122341in}}%
\pgfpathlineto{\pgfqpoint{8.580456in}{9.138615in}}%
\pgfpathlineto{\pgfqpoint{8.594958in}{9.138615in}}%
\pgfpathlineto{\pgfqpoint{8.594958in}{9.154889in}}%
\pgfpathlineto{\pgfqpoint{8.598391in}{9.154889in}}%
\pgfpathlineto{\pgfqpoint{8.598391in}{9.171164in}}%
\pgfpathlineto{\pgfqpoint{8.601056in}{9.171164in}}%
\pgfpathlineto{\pgfqpoint{8.601056in}{9.187438in}}%
\pgfpathlineto{\pgfqpoint{8.610792in}{9.187438in}}%
\pgfpathlineto{\pgfqpoint{8.610792in}{9.203712in}}%
\pgfpathlineto{\pgfqpoint{8.616890in}{9.203712in}}%
\pgfpathlineto{\pgfqpoint{8.616890in}{9.219986in}}%
\pgfpathlineto{\pgfqpoint{8.623756in}{9.219986in}}%
\pgfpathlineto{\pgfqpoint{8.623756in}{9.236261in}}%
\pgfpathlineto{\pgfqpoint{8.634312in}{9.236261in}}%
\pgfpathlineto{\pgfqpoint{8.634312in}{9.252535in}}%
\pgfpathlineto{\pgfqpoint{8.648558in}{9.252535in}}%
\pgfpathlineto{\pgfqpoint{8.648558in}{9.268809in}}%
\pgfpathlineto{\pgfqpoint{8.672335in}{9.268809in}}%
\pgfpathlineto{\pgfqpoint{8.672335in}{9.285083in}}%
\pgfpathlineto{\pgfqpoint{8.674333in}{9.285083in}}%
\pgfpathlineto{\pgfqpoint{8.674333in}{9.301358in}}%
\pgfpathlineto{\pgfqpoint{8.681917in}{9.301358in}}%
\pgfpathlineto{\pgfqpoint{8.681917in}{9.317632in}}%
\pgfpathlineto{\pgfqpoint{8.762574in}{9.317632in}}%
\pgfpathlineto{\pgfqpoint{8.763035in}{9.350181in}}%
\pgfpathlineto{\pgfqpoint{8.767289in}{9.350181in}}%
\pgfpathlineto{\pgfqpoint{8.767289in}{9.366455in}}%
\pgfpathlineto{\pgfqpoint{8.775129in}{9.366455in}}%
\pgfpathlineto{\pgfqpoint{8.775129in}{9.382729in}}%
\pgfpathlineto{\pgfqpoint{8.814022in}{9.382729in}}%
\pgfpathlineto{\pgfqpoint{8.814022in}{9.399003in}}%
\pgfpathlineto{\pgfqpoint{8.895294in}{9.399003in}}%
\pgfpathlineto{\pgfqpoint{8.895294in}{9.415278in}}%
\pgfpathlineto{\pgfqpoint{9.070392in}{9.415278in}}%
\pgfpathlineto{\pgfqpoint{9.070392in}{9.431552in}}%
\pgfpathlineto{\pgfqpoint{9.089813in}{9.431552in}}%
\pgfpathlineto{\pgfqpoint{9.089813in}{9.447826in}}%
\pgfpathlineto{\pgfqpoint{9.090121in}{9.447826in}}%
\pgfpathlineto{\pgfqpoint{9.165243in}{9.447826in}}%
\pgfpathlineto{\pgfqpoint{9.165243in}{9.464100in}}%
\pgfpathlineto{\pgfqpoint{9.202600in}{9.464100in}}%
\pgfpathlineto{\pgfqpoint{9.202600in}{9.480375in}}%
\pgfpathlineto{\pgfqpoint{9.533784in}{9.480375in}}%
\pgfpathlineto{\pgfqpoint{9.533784in}{9.496649in}}%
\pgfpathlineto{\pgfqpoint{9.761150in}{9.496649in}}%
\pgfpathlineto{\pgfqpoint{9.761150in}{9.496649in}}%
\pgfusepath{stroke}%
\end{pgfscope}%
\begin{pgfscope}%
\pgfpathrectangle{\pgfqpoint{8.150541in}{8.027083in}}{\pgfqpoint{1.687305in}{1.539545in}}%
\pgfusepath{clip}%
\pgfsetrectcap%
\pgfsetroundjoin%
\pgfsetlinewidth{1.505625pt}%
\definecolor{currentstroke}{rgb}{0.501961,0.501961,0.501961}%
\pgfsetstrokecolor{currentstroke}%
\pgfsetdash{}{0pt}%
\pgfpathmoveto{\pgfqpoint{8.227236in}{8.097062in}}%
\pgfpathlineto{\pgfqpoint{9.761150in}{9.496649in}}%
\pgfusepath{stroke}%
\end{pgfscope}%
\begin{pgfscope}%
\pgfsetrectcap%
\pgfsetmiterjoin%
\pgfsetlinewidth{0.803000pt}%
\definecolor{currentstroke}{rgb}{0.000000,0.000000,0.000000}%
\pgfsetstrokecolor{currentstroke}%
\pgfsetdash{}{0pt}%
\pgfpathmoveto{\pgfqpoint{8.150541in}{8.027083in}}%
\pgfpathlineto{\pgfqpoint{8.150541in}{9.566628in}}%
\pgfusepath{stroke}%
\end{pgfscope}%
\begin{pgfscope}%
\pgfsetrectcap%
\pgfsetmiterjoin%
\pgfsetlinewidth{0.803000pt}%
\definecolor{currentstroke}{rgb}{0.000000,0.000000,0.000000}%
\pgfsetstrokecolor{currentstroke}%
\pgfsetdash{}{0pt}%
\pgfpathmoveto{\pgfqpoint{9.837846in}{8.027083in}}%
\pgfpathlineto{\pgfqpoint{9.837846in}{9.566628in}}%
\pgfusepath{stroke}%
\end{pgfscope}%
\begin{pgfscope}%
\pgfsetrectcap%
\pgfsetmiterjoin%
\pgfsetlinewidth{0.803000pt}%
\definecolor{currentstroke}{rgb}{0.000000,0.000000,0.000000}%
\pgfsetstrokecolor{currentstroke}%
\pgfsetdash{}{0pt}%
\pgfpathmoveto{\pgfqpoint{8.150541in}{8.027083in}}%
\pgfpathlineto{\pgfqpoint{9.837846in}{8.027083in}}%
\pgfusepath{stroke}%
\end{pgfscope}%
\begin{pgfscope}%
\pgfsetrectcap%
\pgfsetmiterjoin%
\pgfsetlinewidth{0.803000pt}%
\definecolor{currentstroke}{rgb}{0.000000,0.000000,0.000000}%
\pgfsetstrokecolor{currentstroke}%
\pgfsetdash{}{0pt}%
\pgfpathmoveto{\pgfqpoint{8.150541in}{9.566628in}}%
\pgfpathlineto{\pgfqpoint{9.837846in}{9.566628in}}%
\pgfusepath{stroke}%
\end{pgfscope}%
\begin{pgfscope}%
\definecolor{textcolor}{rgb}{0.000000,0.000000,0.000000}%
\pgfsetstrokecolor{textcolor}%
\pgfsetfillcolor{textcolor}%
\pgftext[x=8.994193in,y=9.649962in,,base]{\color{textcolor}\rmfamily\fontsize{20.000000}{24.000000}\selectfont Hernia}%
\end{pgfscope}%
\begin{pgfscope}%
\pgfsetbuttcap%
\pgfsetmiterjoin%
\definecolor{currentfill}{rgb}{1.000000,1.000000,1.000000}%
\pgfsetfillcolor{currentfill}%
\pgfsetfillopacity{0.800000}%
\pgfsetlinewidth{1.003750pt}%
\definecolor{currentstroke}{rgb}{0.800000,0.800000,0.800000}%
\pgfsetstrokecolor{currentstroke}%
\pgfsetstrokeopacity{0.800000}%
\pgfsetdash{}{0pt}%
\pgfpathmoveto{\pgfqpoint{8.628740in}{8.096527in}}%
\pgfpathlineto{\pgfqpoint{9.740624in}{8.096527in}}%
\pgfpathquadraticcurveto{\pgfqpoint{9.768402in}{8.096527in}}{\pgfqpoint{9.768402in}{8.124305in}}%
\pgfpathlineto{\pgfqpoint{9.768402in}{8.304089in}}%
\pgfpathquadraticcurveto{\pgfqpoint{9.768402in}{8.331867in}}{\pgfqpoint{9.740624in}{8.331867in}}%
\pgfpathlineto{\pgfqpoint{8.628740in}{8.331867in}}%
\pgfpathquadraticcurveto{\pgfqpoint{8.600962in}{8.331867in}}{\pgfqpoint{8.600962in}{8.304089in}}%
\pgfpathlineto{\pgfqpoint{8.600962in}{8.124305in}}%
\pgfpathquadraticcurveto{\pgfqpoint{8.600962in}{8.096527in}}{\pgfqpoint{8.628740in}{8.096527in}}%
\pgfpathclose%
\pgfusepath{stroke,fill}%
\end{pgfscope}%
\begin{pgfscope}%
\pgfsetrectcap%
\pgfsetroundjoin%
\pgfsetlinewidth{1.505625pt}%
\definecolor{currentstroke}{rgb}{0.000000,0.501961,0.000000}%
\pgfsetstrokecolor{currentstroke}%
\pgfsetdash{}{0pt}%
\pgfpathmoveto{\pgfqpoint{8.656518in}{8.227700in}}%
\pgfpathlineto{\pgfqpoint{8.934295in}{8.227700in}}%
\pgfusepath{stroke}%
\end{pgfscope}%
\begin{pgfscope}%
\definecolor{textcolor}{rgb}{0.000000,0.000000,0.000000}%
\pgfsetstrokecolor{textcolor}%
\pgfsetfillcolor{textcolor}%
\pgftext[x=9.045407in,y=8.179089in,left,base]{\color{textcolor}\rmfamily\fontsize{10.000000}{12.000000}\selectfont AUC 0.855}%
\end{pgfscope}%
\begin{pgfscope}%
\pgfsetbuttcap%
\pgfsetmiterjoin%
\definecolor{currentfill}{rgb}{1.000000,1.000000,1.000000}%
\pgfsetfillcolor{currentfill}%
\pgfsetlinewidth{0.000000pt}%
\definecolor{currentstroke}{rgb}{0.000000,0.000000,0.000000}%
\pgfsetstrokecolor{currentstroke}%
\pgfsetstrokeopacity{0.000000}%
\pgfsetdash{}{0pt}%
\pgfpathmoveto{\pgfqpoint{0.763041in}{5.564583in}}%
\pgfpathlineto{\pgfqpoint{2.450346in}{5.564583in}}%
\pgfpathlineto{\pgfqpoint{2.450346in}{7.104128in}}%
\pgfpathlineto{\pgfqpoint{0.763041in}{7.104128in}}%
\pgfpathclose%
\pgfusepath{fill}%
\end{pgfscope}%
\begin{pgfscope}%
\pgfsetbuttcap%
\pgfsetroundjoin%
\definecolor{currentfill}{rgb}{0.000000,0.000000,0.000000}%
\pgfsetfillcolor{currentfill}%
\pgfsetlinewidth{0.803000pt}%
\definecolor{currentstroke}{rgb}{0.000000,0.000000,0.000000}%
\pgfsetstrokecolor{currentstroke}%
\pgfsetdash{}{0pt}%
\pgfsys@defobject{currentmarker}{\pgfqpoint{0.000000in}{-0.048611in}}{\pgfqpoint{0.000000in}{0.000000in}}{%
\pgfpathmoveto{\pgfqpoint{0.000000in}{0.000000in}}%
\pgfpathlineto{\pgfqpoint{0.000000in}{-0.048611in}}%
\pgfusepath{stroke,fill}%
}%
\begin{pgfscope}%
\pgfsys@transformshift{0.839736in}{5.564583in}%
\pgfsys@useobject{currentmarker}{}%
\end{pgfscope}%
\end{pgfscope}%
\begin{pgfscope}%
\definecolor{textcolor}{rgb}{0.000000,0.000000,0.000000}%
\pgfsetstrokecolor{textcolor}%
\pgfsetfillcolor{textcolor}%
\pgftext[x=0.839736in,y=5.467361in,,top]{\color{textcolor}\rmfamily\fontsize{10.000000}{12.000000}\selectfont \(\displaystyle {0.0}\)}%
\end{pgfscope}%
\begin{pgfscope}%
\pgfsetbuttcap%
\pgfsetroundjoin%
\definecolor{currentfill}{rgb}{0.000000,0.000000,0.000000}%
\pgfsetfillcolor{currentfill}%
\pgfsetlinewidth{0.803000pt}%
\definecolor{currentstroke}{rgb}{0.000000,0.000000,0.000000}%
\pgfsetstrokecolor{currentstroke}%
\pgfsetdash{}{0pt}%
\pgfsys@defobject{currentmarker}{\pgfqpoint{0.000000in}{-0.048611in}}{\pgfqpoint{0.000000in}{0.000000in}}{%
\pgfpathmoveto{\pgfqpoint{0.000000in}{0.000000in}}%
\pgfpathlineto{\pgfqpoint{0.000000in}{-0.048611in}}%
\pgfusepath{stroke,fill}%
}%
\begin{pgfscope}%
\pgfsys@transformshift{1.606693in}{5.564583in}%
\pgfsys@useobject{currentmarker}{}%
\end{pgfscope}%
\end{pgfscope}%
\begin{pgfscope}%
\definecolor{textcolor}{rgb}{0.000000,0.000000,0.000000}%
\pgfsetstrokecolor{textcolor}%
\pgfsetfillcolor{textcolor}%
\pgftext[x=1.606693in,y=5.467361in,,top]{\color{textcolor}\rmfamily\fontsize{10.000000}{12.000000}\selectfont \(\displaystyle {0.5}\)}%
\end{pgfscope}%
\begin{pgfscope}%
\pgfsetbuttcap%
\pgfsetroundjoin%
\definecolor{currentfill}{rgb}{0.000000,0.000000,0.000000}%
\pgfsetfillcolor{currentfill}%
\pgfsetlinewidth{0.803000pt}%
\definecolor{currentstroke}{rgb}{0.000000,0.000000,0.000000}%
\pgfsetstrokecolor{currentstroke}%
\pgfsetdash{}{0pt}%
\pgfsys@defobject{currentmarker}{\pgfqpoint{0.000000in}{-0.048611in}}{\pgfqpoint{0.000000in}{0.000000in}}{%
\pgfpathmoveto{\pgfqpoint{0.000000in}{0.000000in}}%
\pgfpathlineto{\pgfqpoint{0.000000in}{-0.048611in}}%
\pgfusepath{stroke,fill}%
}%
\begin{pgfscope}%
\pgfsys@transformshift{2.373650in}{5.564583in}%
\pgfsys@useobject{currentmarker}{}%
\end{pgfscope}%
\end{pgfscope}%
\begin{pgfscope}%
\definecolor{textcolor}{rgb}{0.000000,0.000000,0.000000}%
\pgfsetstrokecolor{textcolor}%
\pgfsetfillcolor{textcolor}%
\pgftext[x=2.373650in,y=5.467361in,,top]{\color{textcolor}\rmfamily\fontsize{10.000000}{12.000000}\selectfont \(\displaystyle {1.0}\)}%
\end{pgfscope}%
\begin{pgfscope}%
\definecolor{textcolor}{rgb}{0.000000,0.000000,0.000000}%
\pgfsetstrokecolor{textcolor}%
\pgfsetfillcolor{textcolor}%
\pgftext[x=1.606693in,y=5.288349in,,top]{\color{textcolor}\rmfamily\fontsize{16.000000}{19.200000}\selectfont FPR}%
\end{pgfscope}%
\begin{pgfscope}%
\pgfsetbuttcap%
\pgfsetroundjoin%
\definecolor{currentfill}{rgb}{0.000000,0.000000,0.000000}%
\pgfsetfillcolor{currentfill}%
\pgfsetlinewidth{0.803000pt}%
\definecolor{currentstroke}{rgb}{0.000000,0.000000,0.000000}%
\pgfsetstrokecolor{currentstroke}%
\pgfsetdash{}{0pt}%
\pgfsys@defobject{currentmarker}{\pgfqpoint{-0.048611in}{0.000000in}}{\pgfqpoint{-0.000000in}{0.000000in}}{%
\pgfpathmoveto{\pgfqpoint{-0.000000in}{0.000000in}}%
\pgfpathlineto{\pgfqpoint{-0.048611in}{0.000000in}}%
\pgfusepath{stroke,fill}%
}%
\begin{pgfscope}%
\pgfsys@transformshift{0.763041in}{5.634562in}%
\pgfsys@useobject{currentmarker}{}%
\end{pgfscope}%
\end{pgfscope}%
\begin{pgfscope}%
\definecolor{textcolor}{rgb}{0.000000,0.000000,0.000000}%
\pgfsetstrokecolor{textcolor}%
\pgfsetfillcolor{textcolor}%
\pgftext[x=0.418904in, y=5.586337in, left, base]{\color{textcolor}\rmfamily\fontsize{10.000000}{12.000000}\selectfont \(\displaystyle {0.00}\)}%
\end{pgfscope}%
\begin{pgfscope}%
\pgfsetbuttcap%
\pgfsetroundjoin%
\definecolor{currentfill}{rgb}{0.000000,0.000000,0.000000}%
\pgfsetfillcolor{currentfill}%
\pgfsetlinewidth{0.803000pt}%
\definecolor{currentstroke}{rgb}{0.000000,0.000000,0.000000}%
\pgfsetstrokecolor{currentstroke}%
\pgfsetdash{}{0pt}%
\pgfsys@defobject{currentmarker}{\pgfqpoint{-0.048611in}{0.000000in}}{\pgfqpoint{-0.000000in}{0.000000in}}{%
\pgfpathmoveto{\pgfqpoint{-0.000000in}{0.000000in}}%
\pgfpathlineto{\pgfqpoint{-0.048611in}{0.000000in}}%
\pgfusepath{stroke,fill}%
}%
\begin{pgfscope}%
\pgfsys@transformshift{0.763041in}{5.984459in}%
\pgfsys@useobject{currentmarker}{}%
\end{pgfscope}%
\end{pgfscope}%
\begin{pgfscope}%
\definecolor{textcolor}{rgb}{0.000000,0.000000,0.000000}%
\pgfsetstrokecolor{textcolor}%
\pgfsetfillcolor{textcolor}%
\pgftext[x=0.418904in, y=5.936234in, left, base]{\color{textcolor}\rmfamily\fontsize{10.000000}{12.000000}\selectfont \(\displaystyle {0.25}\)}%
\end{pgfscope}%
\begin{pgfscope}%
\pgfsetbuttcap%
\pgfsetroundjoin%
\definecolor{currentfill}{rgb}{0.000000,0.000000,0.000000}%
\pgfsetfillcolor{currentfill}%
\pgfsetlinewidth{0.803000pt}%
\definecolor{currentstroke}{rgb}{0.000000,0.000000,0.000000}%
\pgfsetstrokecolor{currentstroke}%
\pgfsetdash{}{0pt}%
\pgfsys@defobject{currentmarker}{\pgfqpoint{-0.048611in}{0.000000in}}{\pgfqpoint{-0.000000in}{0.000000in}}{%
\pgfpathmoveto{\pgfqpoint{-0.000000in}{0.000000in}}%
\pgfpathlineto{\pgfqpoint{-0.048611in}{0.000000in}}%
\pgfusepath{stroke,fill}%
}%
\begin{pgfscope}%
\pgfsys@transformshift{0.763041in}{6.334356in}%
\pgfsys@useobject{currentmarker}{}%
\end{pgfscope}%
\end{pgfscope}%
\begin{pgfscope}%
\definecolor{textcolor}{rgb}{0.000000,0.000000,0.000000}%
\pgfsetstrokecolor{textcolor}%
\pgfsetfillcolor{textcolor}%
\pgftext[x=0.418904in, y=6.286130in, left, base]{\color{textcolor}\rmfamily\fontsize{10.000000}{12.000000}\selectfont \(\displaystyle {0.50}\)}%
\end{pgfscope}%
\begin{pgfscope}%
\pgfsetbuttcap%
\pgfsetroundjoin%
\definecolor{currentfill}{rgb}{0.000000,0.000000,0.000000}%
\pgfsetfillcolor{currentfill}%
\pgfsetlinewidth{0.803000pt}%
\definecolor{currentstroke}{rgb}{0.000000,0.000000,0.000000}%
\pgfsetstrokecolor{currentstroke}%
\pgfsetdash{}{0pt}%
\pgfsys@defobject{currentmarker}{\pgfqpoint{-0.048611in}{0.000000in}}{\pgfqpoint{-0.000000in}{0.000000in}}{%
\pgfpathmoveto{\pgfqpoint{-0.000000in}{0.000000in}}%
\pgfpathlineto{\pgfqpoint{-0.048611in}{0.000000in}}%
\pgfusepath{stroke,fill}%
}%
\begin{pgfscope}%
\pgfsys@transformshift{0.763041in}{6.684252in}%
\pgfsys@useobject{currentmarker}{}%
\end{pgfscope}%
\end{pgfscope}%
\begin{pgfscope}%
\definecolor{textcolor}{rgb}{0.000000,0.000000,0.000000}%
\pgfsetstrokecolor{textcolor}%
\pgfsetfillcolor{textcolor}%
\pgftext[x=0.418904in, y=6.636027in, left, base]{\color{textcolor}\rmfamily\fontsize{10.000000}{12.000000}\selectfont \(\displaystyle {0.75}\)}%
\end{pgfscope}%
\begin{pgfscope}%
\pgfsetbuttcap%
\pgfsetroundjoin%
\definecolor{currentfill}{rgb}{0.000000,0.000000,0.000000}%
\pgfsetfillcolor{currentfill}%
\pgfsetlinewidth{0.803000pt}%
\definecolor{currentstroke}{rgb}{0.000000,0.000000,0.000000}%
\pgfsetstrokecolor{currentstroke}%
\pgfsetdash{}{0pt}%
\pgfsys@defobject{currentmarker}{\pgfqpoint{-0.048611in}{0.000000in}}{\pgfqpoint{-0.000000in}{0.000000in}}{%
\pgfpathmoveto{\pgfqpoint{-0.000000in}{0.000000in}}%
\pgfpathlineto{\pgfqpoint{-0.048611in}{0.000000in}}%
\pgfusepath{stroke,fill}%
}%
\begin{pgfscope}%
\pgfsys@transformshift{0.763041in}{7.034149in}%
\pgfsys@useobject{currentmarker}{}%
\end{pgfscope}%
\end{pgfscope}%
\begin{pgfscope}%
\definecolor{textcolor}{rgb}{0.000000,0.000000,0.000000}%
\pgfsetstrokecolor{textcolor}%
\pgfsetfillcolor{textcolor}%
\pgftext[x=0.418904in, y=6.985924in, left, base]{\color{textcolor}\rmfamily\fontsize{10.000000}{12.000000}\selectfont \(\displaystyle {1.00}\)}%
\end{pgfscope}%
\begin{pgfscope}%
\definecolor{textcolor}{rgb}{0.000000,0.000000,0.000000}%
\pgfsetstrokecolor{textcolor}%
\pgfsetfillcolor{textcolor}%
\pgftext[x=0.363349in,y=6.334356in,,bottom,rotate=90.000000]{\color{textcolor}\rmfamily\fontsize{16.000000}{19.200000}\selectfont TPR}%
\end{pgfscope}%
\begin{pgfscope}%
\pgfpathrectangle{\pgfqpoint{0.763041in}{5.564583in}}{\pgfqpoint{1.687305in}{1.539545in}}%
\pgfusepath{clip}%
\pgfsetrectcap%
\pgfsetroundjoin%
\pgfsetlinewidth{1.505625pt}%
\definecolor{currentstroke}{rgb}{0.000000,0.501961,0.000000}%
\pgfsetstrokecolor{currentstroke}%
\pgfsetdash{}{0pt}%
\pgfpathmoveto{\pgfqpoint{0.839736in}{5.634562in}}%
\pgfpathlineto{\pgfqpoint{0.857188in}{5.688833in}}%
\pgfpathlineto{\pgfqpoint{0.858663in}{5.692268in}}%
\pgfpathlineto{\pgfqpoint{0.858727in}{5.692268in}}%
\pgfpathlineto{\pgfqpoint{0.864309in}{5.706923in}}%
\pgfpathlineto{\pgfqpoint{0.864694in}{5.708068in}}%
\pgfpathlineto{\pgfqpoint{0.866170in}{5.714022in}}%
\pgfpathlineto{\pgfqpoint{0.866812in}{5.715396in}}%
\pgfpathlineto{\pgfqpoint{0.868287in}{5.720205in}}%
\pgfpathlineto{\pgfqpoint{0.868351in}{5.720205in}}%
\pgfpathlineto{\pgfqpoint{0.871046in}{5.728906in}}%
\pgfpathlineto{\pgfqpoint{0.871559in}{5.729822in}}%
\pgfpathlineto{\pgfqpoint{0.871559in}{5.730051in}}%
\pgfpathlineto{\pgfqpoint{0.873099in}{5.737608in}}%
\pgfpathlineto{\pgfqpoint{0.873677in}{5.738753in}}%
\pgfpathlineto{\pgfqpoint{0.875216in}{5.743104in}}%
\pgfpathlineto{\pgfqpoint{0.875601in}{5.744249in}}%
\pgfpathlineto{\pgfqpoint{0.875601in}{5.744478in}}%
\pgfpathlineto{\pgfqpoint{0.880157in}{5.755469in}}%
\pgfpathlineto{\pgfqpoint{0.881311in}{5.756843in}}%
\pgfpathlineto{\pgfqpoint{0.882787in}{5.762797in}}%
\pgfpathlineto{\pgfqpoint{0.883172in}{5.763713in}}%
\pgfpathlineto{\pgfqpoint{0.884712in}{5.768750in}}%
\pgfpathlineto{\pgfqpoint{0.885289in}{5.769666in}}%
\pgfpathlineto{\pgfqpoint{0.886829in}{5.774246in}}%
\pgfpathlineto{\pgfqpoint{0.887599in}{5.775620in}}%
\pgfpathlineto{\pgfqpoint{0.889011in}{5.781116in}}%
\pgfpathlineto{\pgfqpoint{0.889588in}{5.782261in}}%
\pgfpathlineto{\pgfqpoint{0.891128in}{5.788902in}}%
\pgfpathlineto{\pgfqpoint{0.891834in}{5.790047in}}%
\pgfpathlineto{\pgfqpoint{0.893117in}{5.793939in}}%
\pgfpathlineto{\pgfqpoint{0.894207in}{5.795313in}}%
\pgfpathlineto{\pgfqpoint{0.895747in}{5.802412in}}%
\pgfpathlineto{\pgfqpoint{0.896325in}{5.803786in}}%
\pgfpathlineto{\pgfqpoint{0.897864in}{5.810198in}}%
\pgfpathlineto{\pgfqpoint{0.898185in}{5.811572in}}%
\pgfpathlineto{\pgfqpoint{0.899340in}{5.816609in}}%
\pgfpathlineto{\pgfqpoint{0.900495in}{5.817983in}}%
\pgfpathlineto{\pgfqpoint{0.902035in}{5.822334in}}%
\pgfpathlineto{\pgfqpoint{0.902741in}{5.823708in}}%
\pgfpathlineto{\pgfqpoint{0.904280in}{5.828059in}}%
\pgfpathlineto{\pgfqpoint{0.905050in}{5.828517in}}%
\pgfpathlineto{\pgfqpoint{0.906526in}{5.835845in}}%
\pgfpathlineto{\pgfqpoint{0.907296in}{5.837218in}}%
\pgfpathlineto{\pgfqpoint{0.908836in}{5.844775in}}%
\pgfpathlineto{\pgfqpoint{0.909477in}{5.846149in}}%
\pgfpathlineto{\pgfqpoint{0.910889in}{5.850729in}}%
\pgfpathlineto{\pgfqpoint{0.911338in}{5.851645in}}%
\pgfpathlineto{\pgfqpoint{0.912814in}{5.855538in}}%
\pgfpathlineto{\pgfqpoint{0.913455in}{5.856912in}}%
\pgfpathlineto{\pgfqpoint{0.914610in}{5.863094in}}%
\pgfpathlineto{\pgfqpoint{0.915508in}{5.864468in}}%
\pgfpathlineto{\pgfqpoint{0.916984in}{5.868819in}}%
\pgfpathlineto{\pgfqpoint{0.917625in}{5.870193in}}%
\pgfpathlineto{\pgfqpoint{0.918973in}{5.875231in}}%
\pgfpathlineto{\pgfqpoint{0.919743in}{5.876376in}}%
\pgfpathlineto{\pgfqpoint{0.921283in}{5.880040in}}%
\pgfpathlineto{\pgfqpoint{0.922245in}{5.881414in}}%
\pgfpathlineto{\pgfqpoint{0.923464in}{5.885077in}}%
\pgfpathlineto{\pgfqpoint{0.923721in}{5.885077in}}%
\pgfpathlineto{\pgfqpoint{0.924362in}{5.885993in}}%
\pgfpathlineto{\pgfqpoint{0.925838in}{5.890344in}}%
\pgfpathlineto{\pgfqpoint{0.926864in}{5.891718in}}%
\pgfpathlineto{\pgfqpoint{0.928340in}{5.898359in}}%
\pgfpathlineto{\pgfqpoint{0.928982in}{5.899733in}}%
\pgfpathlineto{\pgfqpoint{0.930521in}{5.902023in}}%
\pgfpathlineto{\pgfqpoint{0.931099in}{5.903168in}}%
\pgfpathlineto{\pgfqpoint{0.932575in}{5.908663in}}%
\pgfpathlineto{\pgfqpoint{0.933088in}{5.910037in}}%
\pgfpathlineto{\pgfqpoint{0.934628in}{5.913701in}}%
\pgfpathlineto{\pgfqpoint{0.935077in}{5.915075in}}%
\pgfpathlineto{\pgfqpoint{0.936488in}{5.919426in}}%
\pgfpathlineto{\pgfqpoint{0.937579in}{5.920571in}}%
\pgfpathlineto{\pgfqpoint{0.939119in}{5.925838in}}%
\pgfpathlineto{\pgfqpoint{0.939568in}{5.927212in}}%
\pgfpathlineto{\pgfqpoint{0.941108in}{5.929959in}}%
\pgfpathlineto{\pgfqpoint{0.941557in}{5.931333in}}%
\pgfpathlineto{\pgfqpoint{0.943097in}{5.937516in}}%
\pgfpathlineto{\pgfqpoint{0.943482in}{5.938661in}}%
\pgfpathlineto{\pgfqpoint{0.944765in}{5.942783in}}%
\pgfpathlineto{\pgfqpoint{0.945214in}{5.943012in}}%
\pgfpathlineto{\pgfqpoint{0.946689in}{5.949424in}}%
\pgfpathlineto{\pgfqpoint{0.947652in}{5.950568in}}%
\pgfpathlineto{\pgfqpoint{0.949192in}{5.957667in}}%
\pgfpathlineto{\pgfqpoint{0.949577in}{5.958812in}}%
\pgfpathlineto{\pgfqpoint{0.951052in}{5.963621in}}%
\pgfpathlineto{\pgfqpoint{0.952079in}{5.964995in}}%
\pgfpathlineto{\pgfqpoint{0.953619in}{5.971407in}}%
\pgfpathlineto{\pgfqpoint{0.954260in}{5.972552in}}%
\pgfpathlineto{\pgfqpoint{0.955672in}{5.975528in}}%
\pgfpathlineto{\pgfqpoint{0.956442in}{5.975986in}}%
\pgfpathlineto{\pgfqpoint{0.957917in}{5.980337in}}%
\pgfpathlineto{\pgfqpoint{0.958944in}{5.981482in}}%
\pgfpathlineto{\pgfqpoint{0.960035in}{5.986749in}}%
\pgfpathlineto{\pgfqpoint{0.960869in}{5.987894in}}%
\pgfpathlineto{\pgfqpoint{0.962408in}{5.991558in}}%
\pgfpathlineto{\pgfqpoint{0.962793in}{5.992016in}}%
\pgfpathlineto{\pgfqpoint{0.964333in}{5.996366in}}%
\pgfpathlineto{\pgfqpoint{0.964911in}{5.997282in}}%
\pgfpathlineto{\pgfqpoint{0.966194in}{6.002549in}}%
\pgfpathlineto{\pgfqpoint{0.967926in}{6.003923in}}%
\pgfpathlineto{\pgfqpoint{0.969466in}{6.006900in}}%
\pgfpathlineto{\pgfqpoint{0.970172in}{6.008045in}}%
\pgfpathlineto{\pgfqpoint{0.971647in}{6.011709in}}%
\pgfpathlineto{\pgfqpoint{0.972866in}{6.012854in}}%
\pgfpathlineto{\pgfqpoint{0.974085in}{6.016289in}}%
\pgfpathlineto{\pgfqpoint{0.974984in}{6.017663in}}%
\pgfpathlineto{\pgfqpoint{0.976459in}{6.019036in}}%
\pgfpathlineto{\pgfqpoint{0.976973in}{6.019952in}}%
\pgfpathlineto{\pgfqpoint{0.978512in}{6.025448in}}%
\pgfpathlineto{\pgfqpoint{0.979282in}{6.026364in}}%
\pgfpathlineto{\pgfqpoint{0.980565in}{6.032318in}}%
\pgfpathlineto{\pgfqpoint{0.981592in}{6.033692in}}%
\pgfpathlineto{\pgfqpoint{0.982747in}{6.036898in}}%
\pgfpathlineto{\pgfqpoint{0.983966in}{6.038043in}}%
\pgfpathlineto{\pgfqpoint{0.985313in}{6.040562in}}%
\pgfpathlineto{\pgfqpoint{0.987110in}{6.041935in}}%
\pgfpathlineto{\pgfqpoint{0.988650in}{6.045828in}}%
\pgfpathlineto{\pgfqpoint{0.989355in}{6.046973in}}%
\pgfpathlineto{\pgfqpoint{0.990831in}{6.051553in}}%
\pgfpathlineto{\pgfqpoint{0.991665in}{6.052698in}}%
\pgfpathlineto{\pgfqpoint{0.993205in}{6.057736in}}%
\pgfpathlineto{\pgfqpoint{0.994488in}{6.059110in}}%
\pgfpathlineto{\pgfqpoint{0.996028in}{6.065292in}}%
\pgfpathlineto{\pgfqpoint{0.996734in}{6.066666in}}%
\pgfpathlineto{\pgfqpoint{0.998209in}{6.069414in}}%
\pgfpathlineto{\pgfqpoint{0.999364in}{6.070788in}}%
\pgfpathlineto{\pgfqpoint{1.000776in}{6.074910in}}%
\pgfpathlineto{\pgfqpoint{1.001674in}{6.076284in}}%
\pgfpathlineto{\pgfqpoint{1.003214in}{6.081551in}}%
\pgfpathlineto{\pgfqpoint{1.003791in}{6.082925in}}%
\pgfpathlineto{\pgfqpoint{1.005203in}{6.087046in}}%
\pgfpathlineto{\pgfqpoint{1.005972in}{6.088420in}}%
\pgfpathlineto{\pgfqpoint{1.007512in}{6.092084in}}%
\pgfpathlineto{\pgfqpoint{1.008539in}{6.093458in}}%
\pgfpathlineto{\pgfqpoint{1.009437in}{6.095061in}}%
\pgfpathlineto{\pgfqpoint{1.010977in}{6.096435in}}%
\pgfpathlineto{\pgfqpoint{1.012517in}{6.101015in}}%
\pgfpathlineto{\pgfqpoint{1.013287in}{6.102389in}}%
\pgfpathlineto{\pgfqpoint{1.014826in}{6.106282in}}%
\pgfpathlineto{\pgfqpoint{1.015660in}{6.107656in}}%
\pgfpathlineto{\pgfqpoint{1.017008in}{6.109716in}}%
\pgfpathlineto{\pgfqpoint{1.017714in}{6.110861in}}%
\pgfpathlineto{\pgfqpoint{1.019125in}{6.113151in}}%
\pgfpathlineto{\pgfqpoint{1.020087in}{6.114296in}}%
\pgfpathlineto{\pgfqpoint{1.021242in}{6.119105in}}%
\pgfpathlineto{\pgfqpoint{1.022141in}{6.120021in}}%
\pgfpathlineto{\pgfqpoint{1.023680in}{6.124372in}}%
\pgfpathlineto{\pgfqpoint{1.023873in}{6.125746in}}%
\pgfpathlineto{\pgfqpoint{1.025028in}{6.128265in}}%
\pgfpathlineto{\pgfqpoint{1.026503in}{6.129639in}}%
\pgfpathlineto{\pgfqpoint{1.027915in}{6.131929in}}%
\pgfpathlineto{\pgfqpoint{1.028556in}{6.133302in}}%
\pgfpathlineto{\pgfqpoint{1.030096in}{6.135821in}}%
\pgfpathlineto{\pgfqpoint{1.031829in}{6.136966in}}%
\pgfpathlineto{\pgfqpoint{1.033368in}{6.141088in}}%
\pgfpathlineto{\pgfqpoint{1.034202in}{6.142462in}}%
\pgfpathlineto{\pgfqpoint{1.035486in}{6.145897in}}%
\pgfpathlineto{\pgfqpoint{1.036576in}{6.147271in}}%
\pgfpathlineto{\pgfqpoint{1.037795in}{6.149103in}}%
\pgfpathlineto{\pgfqpoint{1.038886in}{6.150477in}}%
\pgfpathlineto{\pgfqpoint{1.040426in}{6.154828in}}%
\pgfpathlineto{\pgfqpoint{1.041517in}{6.156201in}}%
\pgfpathlineto{\pgfqpoint{1.043056in}{6.160094in}}%
\pgfpathlineto{\pgfqpoint{1.044083in}{6.161468in}}%
\pgfpathlineto{\pgfqpoint{1.045238in}{6.163071in}}%
\pgfpathlineto{\pgfqpoint{1.046649in}{6.164445in}}%
\pgfpathlineto{\pgfqpoint{1.047612in}{6.167422in}}%
\pgfpathlineto{\pgfqpoint{1.049087in}{6.168338in}}%
\pgfpathlineto{\pgfqpoint{1.050563in}{6.174292in}}%
\pgfpathlineto{\pgfqpoint{1.051461in}{6.175437in}}%
\pgfpathlineto{\pgfqpoint{1.052937in}{6.180245in}}%
\pgfpathlineto{\pgfqpoint{1.053771in}{6.181390in}}%
\pgfpathlineto{\pgfqpoint{1.055054in}{6.184825in}}%
\pgfpathlineto{\pgfqpoint{1.056722in}{6.186199in}}%
\pgfpathlineto{\pgfqpoint{1.058262in}{6.190321in}}%
\pgfpathlineto{\pgfqpoint{1.058904in}{6.191466in}}%
\pgfpathlineto{\pgfqpoint{1.060315in}{6.195359in}}%
\pgfpathlineto{\pgfqpoint{1.060828in}{6.196275in}}%
\pgfpathlineto{\pgfqpoint{1.062176in}{6.200854in}}%
\pgfpathlineto{\pgfqpoint{1.063010in}{6.201770in}}%
\pgfpathlineto{\pgfqpoint{1.064550in}{6.207037in}}%
\pgfpathlineto{\pgfqpoint{1.065127in}{6.208411in}}%
\pgfpathlineto{\pgfqpoint{1.066539in}{6.211388in}}%
\pgfpathlineto{\pgfqpoint{1.067629in}{6.212304in}}%
\pgfpathlineto{\pgfqpoint{1.069105in}{6.216655in}}%
\pgfpathlineto{\pgfqpoint{1.069490in}{6.218029in}}%
\pgfpathlineto{\pgfqpoint{1.071030in}{6.221464in}}%
\pgfpathlineto{\pgfqpoint{1.072056in}{6.222838in}}%
\pgfpathlineto{\pgfqpoint{1.073596in}{6.224898in}}%
\pgfpathlineto{\pgfqpoint{1.074173in}{6.226043in}}%
\pgfpathlineto{\pgfqpoint{1.075713in}{6.228791in}}%
\pgfpathlineto{\pgfqpoint{1.077253in}{6.230165in}}%
\pgfpathlineto{\pgfqpoint{1.078793in}{6.233829in}}%
\pgfpathlineto{\pgfqpoint{1.080012in}{6.235203in}}%
\pgfpathlineto{\pgfqpoint{1.081552in}{6.238180in}}%
\pgfpathlineto{\pgfqpoint{1.082578in}{6.239325in}}%
\pgfpathlineto{\pgfqpoint{1.083990in}{6.243676in}}%
\pgfpathlineto{\pgfqpoint{1.084888in}{6.245050in}}%
\pgfpathlineto{\pgfqpoint{1.085081in}{6.246652in}}%
\pgfpathlineto{\pgfqpoint{1.087069in}{6.248026in}}%
\pgfpathlineto{\pgfqpoint{1.088609in}{6.251690in}}%
\pgfpathlineto{\pgfqpoint{1.089636in}{6.253064in}}%
\pgfpathlineto{\pgfqpoint{1.090919in}{6.255812in}}%
\pgfpathlineto{\pgfqpoint{1.092138in}{6.257186in}}%
\pgfpathlineto{\pgfqpoint{1.093614in}{6.258789in}}%
\pgfpathlineto{\pgfqpoint{1.094384in}{6.259934in}}%
\pgfpathlineto{\pgfqpoint{1.095923in}{6.265888in}}%
\pgfpathlineto{\pgfqpoint{1.097078in}{6.267262in}}%
\pgfpathlineto{\pgfqpoint{1.098618in}{6.270009in}}%
\pgfpathlineto{\pgfqpoint{1.099837in}{6.271383in}}%
\pgfpathlineto{\pgfqpoint{1.101249in}{6.274360in}}%
\pgfpathlineto{\pgfqpoint{1.102788in}{6.275734in}}%
\pgfpathlineto{\pgfqpoint{1.104328in}{6.279169in}}%
\pgfpathlineto{\pgfqpoint{1.107087in}{6.284436in}}%
\pgfpathlineto{\pgfqpoint{1.107665in}{6.285810in}}%
\pgfpathlineto{\pgfqpoint{1.109204in}{6.288100in}}%
\pgfpathlineto{\pgfqpoint{1.109974in}{6.289474in}}%
\pgfpathlineto{\pgfqpoint{1.111386in}{6.292679in}}%
\pgfpathlineto{\pgfqpoint{1.113182in}{6.294053in}}%
\pgfpathlineto{\pgfqpoint{1.114401in}{6.296343in}}%
\pgfpathlineto{\pgfqpoint{1.115556in}{6.297259in}}%
\pgfpathlineto{\pgfqpoint{1.117032in}{6.299549in}}%
\pgfpathlineto{\pgfqpoint{1.117866in}{6.300465in}}%
\pgfpathlineto{\pgfqpoint{1.119277in}{6.305732in}}%
\pgfpathlineto{\pgfqpoint{1.119983in}{6.306648in}}%
\pgfpathlineto{\pgfqpoint{1.121523in}{6.310770in}}%
\pgfpathlineto{\pgfqpoint{1.122229in}{6.311915in}}%
\pgfpathlineto{\pgfqpoint{1.123704in}{6.315120in}}%
\pgfpathlineto{\pgfqpoint{1.125565in}{6.316494in}}%
\pgfpathlineto{\pgfqpoint{1.127041in}{6.319013in}}%
\pgfpathlineto{\pgfqpoint{1.127746in}{6.320387in}}%
\pgfpathlineto{\pgfqpoint{1.129094in}{6.324051in}}%
\pgfpathlineto{\pgfqpoint{1.130120in}{6.325425in}}%
\pgfpathlineto{\pgfqpoint{1.131275in}{6.328631in}}%
\pgfpathlineto{\pgfqpoint{1.132751in}{6.330005in}}%
\pgfpathlineto{\pgfqpoint{1.134290in}{6.332066in}}%
\pgfpathlineto{\pgfqpoint{1.135381in}{6.333211in}}%
\pgfpathlineto{\pgfqpoint{1.136279in}{6.334356in}}%
\pgfpathlineto{\pgfqpoint{1.138140in}{6.335501in}}%
\pgfpathlineto{\pgfqpoint{1.139231in}{6.337790in}}%
\pgfpathlineto{\pgfqpoint{1.140065in}{6.338935in}}%
\pgfpathlineto{\pgfqpoint{1.141540in}{6.340767in}}%
\pgfpathlineto{\pgfqpoint{1.142375in}{6.342141in}}%
\pgfpathlineto{\pgfqpoint{1.143465in}{6.346034in}}%
\pgfpathlineto{\pgfqpoint{1.145262in}{6.347408in}}%
\pgfpathlineto{\pgfqpoint{1.146737in}{6.350385in}}%
\pgfpathlineto{\pgfqpoint{1.148405in}{6.351759in}}%
\pgfpathlineto{\pgfqpoint{1.149945in}{6.354507in}}%
\pgfpathlineto{\pgfqpoint{1.152832in}{6.355881in}}%
\pgfpathlineto{\pgfqpoint{1.153987in}{6.358400in}}%
\pgfpathlineto{\pgfqpoint{1.155206in}{6.359774in}}%
\pgfpathlineto{\pgfqpoint{1.156682in}{6.363437in}}%
\pgfpathlineto{\pgfqpoint{1.158735in}{6.364811in}}%
\pgfpathlineto{\pgfqpoint{1.160018in}{6.367559in}}%
\pgfpathlineto{\pgfqpoint{1.161045in}{6.368933in}}%
\pgfpathlineto{\pgfqpoint{1.162520in}{6.370994in}}%
\pgfpathlineto{\pgfqpoint{1.163675in}{6.372139in}}%
\pgfpathlineto{\pgfqpoint{1.165151in}{6.374887in}}%
\pgfpathlineto{\pgfqpoint{1.167653in}{6.376032in}}%
\pgfpathlineto{\pgfqpoint{1.168936in}{6.378780in}}%
\pgfpathlineto{\pgfqpoint{1.169899in}{6.380154in}}%
\pgfpathlineto{\pgfqpoint{1.171310in}{6.383588in}}%
\pgfpathlineto{\pgfqpoint{1.172208in}{6.384962in}}%
\pgfpathlineto{\pgfqpoint{1.173748in}{6.387481in}}%
\pgfpathlineto{\pgfqpoint{1.174839in}{6.388855in}}%
\pgfpathlineto{\pgfqpoint{1.175737in}{6.390000in}}%
\pgfpathlineto{\pgfqpoint{1.177470in}{6.391374in}}%
\pgfpathlineto{\pgfqpoint{1.179009in}{6.394580in}}%
\pgfpathlineto{\pgfqpoint{1.180870in}{6.395954in}}%
\pgfpathlineto{\pgfqpoint{1.182153in}{6.399389in}}%
\pgfpathlineto{\pgfqpoint{1.183308in}{6.400763in}}%
\pgfpathlineto{\pgfqpoint{1.184848in}{6.403511in}}%
\pgfpathlineto{\pgfqpoint{1.186708in}{6.404656in}}%
\pgfpathlineto{\pgfqpoint{1.188248in}{6.406716in}}%
\pgfpathlineto{\pgfqpoint{1.189082in}{6.408090in}}%
\pgfpathlineto{\pgfqpoint{1.190622in}{6.410838in}}%
\pgfpathlineto{\pgfqpoint{1.192419in}{6.412212in}}%
\pgfpathlineto{\pgfqpoint{1.193766in}{6.414960in}}%
\pgfpathlineto{\pgfqpoint{1.195242in}{6.416105in}}%
\pgfpathlineto{\pgfqpoint{1.196781in}{6.418395in}}%
\pgfpathlineto{\pgfqpoint{1.198000in}{6.419769in}}%
\pgfpathlineto{\pgfqpoint{1.199476in}{6.422059in}}%
\pgfpathlineto{\pgfqpoint{1.201273in}{6.423433in}}%
\pgfpathlineto{\pgfqpoint{1.202684in}{6.426410in}}%
\pgfpathlineto{\pgfqpoint{1.203646in}{6.427784in}}%
\pgfpathlineto{\pgfqpoint{1.205058in}{6.430073in}}%
\pgfpathlineto{\pgfqpoint{1.206534in}{6.431447in}}%
\pgfpathlineto{\pgfqpoint{1.208073in}{6.432821in}}%
\pgfpathlineto{\pgfqpoint{1.209100in}{6.434195in}}%
\pgfpathlineto{\pgfqpoint{1.210576in}{6.438088in}}%
\pgfpathlineto{\pgfqpoint{1.211153in}{6.439233in}}%
\pgfpathlineto{\pgfqpoint{1.211987in}{6.440378in}}%
\pgfpathlineto{\pgfqpoint{1.213463in}{6.441752in}}%
\pgfpathlineto{\pgfqpoint{1.214874in}{6.443584in}}%
\pgfpathlineto{\pgfqpoint{1.216093in}{6.444958in}}%
\pgfpathlineto{\pgfqpoint{1.217569in}{6.448393in}}%
\pgfpathlineto{\pgfqpoint{1.218467in}{6.449538in}}%
\pgfpathlineto{\pgfqpoint{1.219301in}{6.451140in}}%
\pgfpathlineto{\pgfqpoint{1.221354in}{6.452514in}}%
\pgfpathlineto{\pgfqpoint{1.222766in}{6.454575in}}%
\pgfpathlineto{\pgfqpoint{1.223921in}{6.455491in}}%
\pgfpathlineto{\pgfqpoint{1.225460in}{6.458926in}}%
\pgfpathlineto{\pgfqpoint{1.226936in}{6.460300in}}%
\pgfpathlineto{\pgfqpoint{1.228027in}{6.462819in}}%
\pgfpathlineto{\pgfqpoint{1.229310in}{6.464193in}}%
\pgfpathlineto{\pgfqpoint{1.230465in}{6.465796in}}%
\pgfpathlineto{\pgfqpoint{1.232646in}{6.467170in}}%
\pgfpathlineto{\pgfqpoint{1.234186in}{6.470147in}}%
\pgfpathlineto{\pgfqpoint{1.236688in}{6.471292in}}%
\pgfpathlineto{\pgfqpoint{1.238100in}{6.472895in}}%
\pgfpathlineto{\pgfqpoint{1.239126in}{6.474039in}}%
\pgfpathlineto{\pgfqpoint{1.240345in}{6.476100in}}%
\pgfpathlineto{\pgfqpoint{1.242912in}{6.477474in}}%
\pgfpathlineto{\pgfqpoint{1.244323in}{6.479077in}}%
\pgfpathlineto{\pgfqpoint{1.246312in}{6.480222in}}%
\pgfpathlineto{\pgfqpoint{1.247852in}{6.482054in}}%
\pgfpathlineto{\pgfqpoint{1.248879in}{6.483199in}}%
\pgfpathlineto{\pgfqpoint{1.249905in}{6.485031in}}%
\pgfpathlineto{\pgfqpoint{1.251766in}{6.486405in}}%
\pgfpathlineto{\pgfqpoint{1.253177in}{6.488237in}}%
\pgfpathlineto{\pgfqpoint{1.254460in}{6.489611in}}%
\pgfpathlineto{\pgfqpoint{1.255744in}{6.491901in}}%
\pgfpathlineto{\pgfqpoint{1.256898in}{6.493275in}}%
\pgfpathlineto{\pgfqpoint{1.258310in}{6.496252in}}%
\pgfpathlineto{\pgfqpoint{1.259144in}{6.497396in}}%
\pgfpathlineto{\pgfqpoint{1.259978in}{6.499457in}}%
\pgfpathlineto{\pgfqpoint{1.262224in}{6.500602in}}%
\pgfpathlineto{\pgfqpoint{1.263635in}{6.502892in}}%
\pgfpathlineto{\pgfqpoint{1.264597in}{6.504266in}}%
\pgfpathlineto{\pgfqpoint{1.265432in}{6.505869in}}%
\pgfpathlineto{\pgfqpoint{1.267228in}{6.507243in}}%
\pgfpathlineto{\pgfqpoint{1.268768in}{6.509533in}}%
\pgfpathlineto{\pgfqpoint{1.270179in}{6.510907in}}%
\pgfpathlineto{\pgfqpoint{1.271719in}{6.512510in}}%
\pgfpathlineto{\pgfqpoint{1.273131in}{6.513884in}}%
\pgfpathlineto{\pgfqpoint{1.274350in}{6.515716in}}%
\pgfpathlineto{\pgfqpoint{1.276018in}{6.516861in}}%
\pgfpathlineto{\pgfqpoint{1.277558in}{6.519151in}}%
\pgfpathlineto{\pgfqpoint{1.278328in}{6.520524in}}%
\pgfpathlineto{\pgfqpoint{1.279803in}{6.523272in}}%
\pgfpathlineto{\pgfqpoint{1.282434in}{6.524646in}}%
\pgfpathlineto{\pgfqpoint{1.283845in}{6.526936in}}%
\pgfpathlineto{\pgfqpoint{1.285385in}{6.528081in}}%
\pgfpathlineto{\pgfqpoint{1.286925in}{6.530371in}}%
\pgfpathlineto{\pgfqpoint{1.288144in}{6.531516in}}%
\pgfpathlineto{\pgfqpoint{1.289620in}{6.533577in}}%
\pgfpathlineto{\pgfqpoint{1.291673in}{6.534951in}}%
\pgfpathlineto{\pgfqpoint{1.293148in}{6.537699in}}%
\pgfpathlineto{\pgfqpoint{1.294560in}{6.539073in}}%
\pgfpathlineto{\pgfqpoint{1.296100in}{6.542278in}}%
\pgfpathlineto{\pgfqpoint{1.297062in}{6.543652in}}%
\pgfpathlineto{\pgfqpoint{1.298538in}{6.545026in}}%
\pgfpathlineto{\pgfqpoint{1.299757in}{6.546400in}}%
\pgfpathlineto{\pgfqpoint{1.300911in}{6.548461in}}%
\pgfpathlineto{\pgfqpoint{1.304119in}{6.549835in}}%
\pgfpathlineto{\pgfqpoint{1.305595in}{6.550980in}}%
\pgfpathlineto{\pgfqpoint{1.307327in}{6.551896in}}%
\pgfpathlineto{\pgfqpoint{1.308675in}{6.554415in}}%
\pgfpathlineto{\pgfqpoint{1.310535in}{6.555789in}}%
\pgfpathlineto{\pgfqpoint{1.311819in}{6.557621in}}%
\pgfpathlineto{\pgfqpoint{1.313743in}{6.558995in}}%
\pgfpathlineto{\pgfqpoint{1.315155in}{6.561056in}}%
\pgfpathlineto{\pgfqpoint{1.315732in}{6.561285in}}%
\pgfpathlineto{\pgfqpoint{1.317272in}{6.564491in}}%
\pgfpathlineto{\pgfqpoint{1.317785in}{6.565635in}}%
\pgfpathlineto{\pgfqpoint{1.319069in}{6.567009in}}%
\pgfpathlineto{\pgfqpoint{1.321057in}{6.568383in}}%
\pgfpathlineto{\pgfqpoint{1.322533in}{6.570673in}}%
\pgfpathlineto{\pgfqpoint{1.324458in}{6.572047in}}%
\pgfpathlineto{\pgfqpoint{1.325677in}{6.572963in}}%
\pgfpathlineto{\pgfqpoint{1.326383in}{6.574108in}}%
\pgfpathlineto{\pgfqpoint{1.327730in}{6.576169in}}%
\pgfpathlineto{\pgfqpoint{1.328307in}{6.577314in}}%
\pgfpathlineto{\pgfqpoint{1.329847in}{6.578688in}}%
\pgfpathlineto{\pgfqpoint{1.332414in}{6.580062in}}%
\pgfpathlineto{\pgfqpoint{1.333825in}{6.582123in}}%
\pgfpathlineto{\pgfqpoint{1.335172in}{6.583268in}}%
\pgfpathlineto{\pgfqpoint{1.336520in}{6.584871in}}%
\pgfpathlineto{\pgfqpoint{1.338765in}{6.586245in}}%
\pgfpathlineto{\pgfqpoint{1.340177in}{6.587161in}}%
\pgfpathlineto{\pgfqpoint{1.341139in}{6.588534in}}%
\pgfpathlineto{\pgfqpoint{1.342230in}{6.589679in}}%
\pgfpathlineto{\pgfqpoint{1.346785in}{6.591053in}}%
\pgfpathlineto{\pgfqpoint{1.348261in}{6.593343in}}%
\pgfpathlineto{\pgfqpoint{1.349416in}{6.594488in}}%
\pgfpathlineto{\pgfqpoint{1.350827in}{6.595633in}}%
\pgfpathlineto{\pgfqpoint{1.352880in}{6.596778in}}%
\pgfpathlineto{\pgfqpoint{1.354099in}{6.598381in}}%
\pgfpathlineto{\pgfqpoint{1.355318in}{6.599526in}}%
\pgfpathlineto{\pgfqpoint{1.355318in}{6.599755in}}%
\pgfpathlineto{\pgfqpoint{1.357179in}{6.601587in}}%
\pgfpathlineto{\pgfqpoint{1.360002in}{6.602961in}}%
\pgfpathlineto{\pgfqpoint{1.361093in}{6.604793in}}%
\pgfpathlineto{\pgfqpoint{1.364429in}{6.606167in}}%
\pgfpathlineto{\pgfqpoint{1.365712in}{6.607312in}}%
\pgfpathlineto{\pgfqpoint{1.368022in}{6.608686in}}%
\pgfpathlineto{\pgfqpoint{1.368920in}{6.610060in}}%
\pgfpathlineto{\pgfqpoint{1.370524in}{6.611204in}}%
\pgfpathlineto{\pgfqpoint{1.371807in}{6.613265in}}%
\pgfpathlineto{\pgfqpoint{1.374245in}{6.614639in}}%
\pgfpathlineto{\pgfqpoint{1.375785in}{6.616929in}}%
\pgfpathlineto{\pgfqpoint{1.377068in}{6.618303in}}%
\pgfpathlineto{\pgfqpoint{1.378544in}{6.620135in}}%
\pgfpathlineto{\pgfqpoint{1.380405in}{6.621509in}}%
\pgfpathlineto{\pgfqpoint{1.381880in}{6.623570in}}%
\pgfpathlineto{\pgfqpoint{1.383228in}{6.624944in}}%
\pgfpathlineto{\pgfqpoint{1.384703in}{6.627463in}}%
\pgfpathlineto{\pgfqpoint{1.387590in}{6.628837in}}%
\pgfpathlineto{\pgfqpoint{1.388809in}{6.630898in}}%
\pgfpathlineto{\pgfqpoint{1.390734in}{6.632272in}}%
\pgfpathlineto{\pgfqpoint{1.392081in}{6.633187in}}%
\pgfpathlineto{\pgfqpoint{1.394776in}{6.634561in}}%
\pgfpathlineto{\pgfqpoint{1.395867in}{6.635477in}}%
\pgfpathlineto{\pgfqpoint{1.398369in}{6.636851in}}%
\pgfpathlineto{\pgfqpoint{1.399909in}{6.639141in}}%
\pgfpathlineto{\pgfqpoint{1.402026in}{6.640515in}}%
\pgfpathlineto{\pgfqpoint{1.403309in}{6.642805in}}%
\pgfpathlineto{\pgfqpoint{1.405362in}{6.644179in}}%
\pgfpathlineto{\pgfqpoint{1.406453in}{6.645324in}}%
\pgfpathlineto{\pgfqpoint{1.409533in}{6.646698in}}%
\pgfpathlineto{\pgfqpoint{1.410752in}{6.647843in}}%
\pgfpathlineto{\pgfqpoint{1.415692in}{6.649217in}}%
\pgfpathlineto{\pgfqpoint{1.417168in}{6.650133in}}%
\pgfpathlineto{\pgfqpoint{1.418964in}{6.651507in}}%
\pgfpathlineto{\pgfqpoint{1.420311in}{6.654026in}}%
\pgfpathlineto{\pgfqpoint{1.421787in}{6.654942in}}%
\pgfpathlineto{\pgfqpoint{1.423327in}{6.656773in}}%
\pgfpathlineto{\pgfqpoint{1.425444in}{6.658147in}}%
\pgfpathlineto{\pgfqpoint{1.426599in}{6.659063in}}%
\pgfpathlineto{\pgfqpoint{1.428139in}{6.660437in}}%
\pgfpathlineto{\pgfqpoint{1.429486in}{6.662040in}}%
\pgfpathlineto{\pgfqpoint{1.432181in}{6.663414in}}%
\pgfpathlineto{\pgfqpoint{1.433657in}{6.665246in}}%
\pgfpathlineto{\pgfqpoint{1.434683in}{6.666620in}}%
\pgfpathlineto{\pgfqpoint{1.436030in}{6.667536in}}%
\pgfpathlineto{\pgfqpoint{1.436223in}{6.667536in}}%
\pgfpathlineto{\pgfqpoint{1.438276in}{6.669597in}}%
\pgfpathlineto{\pgfqpoint{1.441227in}{6.670971in}}%
\pgfpathlineto{\pgfqpoint{1.442639in}{6.673261in}}%
\pgfpathlineto{\pgfqpoint{1.445333in}{6.674635in}}%
\pgfpathlineto{\pgfqpoint{1.446873in}{6.676467in}}%
\pgfpathlineto{\pgfqpoint{1.447964in}{6.677383in}}%
\pgfpathlineto{\pgfqpoint{1.449504in}{6.679214in}}%
\pgfpathlineto{\pgfqpoint{1.451108in}{6.680588in}}%
\pgfpathlineto{\pgfqpoint{1.452391in}{6.682649in}}%
\pgfpathlineto{\pgfqpoint{1.454123in}{6.684023in}}%
\pgfpathlineto{\pgfqpoint{1.455535in}{6.686084in}}%
\pgfpathlineto{\pgfqpoint{1.457973in}{6.687458in}}%
\pgfpathlineto{\pgfqpoint{1.459513in}{6.688374in}}%
\pgfpathlineto{\pgfqpoint{1.460539in}{6.689748in}}%
\pgfpathlineto{\pgfqpoint{1.462079in}{6.692725in}}%
\pgfpathlineto{\pgfqpoint{1.463298in}{6.694099in}}%
\pgfpathlineto{\pgfqpoint{1.464838in}{6.696618in}}%
\pgfpathlineto{\pgfqpoint{1.466827in}{6.697992in}}%
\pgfpathlineto{\pgfqpoint{1.467212in}{6.699366in}}%
\pgfpathlineto{\pgfqpoint{1.469008in}{6.700740in}}%
\pgfpathlineto{\pgfqpoint{1.470548in}{6.702800in}}%
\pgfpathlineto{\pgfqpoint{1.472409in}{6.703945in}}%
\pgfpathlineto{\pgfqpoint{1.473884in}{6.705319in}}%
\pgfpathlineto{\pgfqpoint{1.475039in}{6.706693in}}%
\pgfpathlineto{\pgfqpoint{1.476386in}{6.707838in}}%
\pgfpathlineto{\pgfqpoint{1.479209in}{6.709212in}}%
\pgfpathlineto{\pgfqpoint{1.480557in}{6.710815in}}%
\pgfpathlineto{\pgfqpoint{1.483059in}{6.712189in}}%
\pgfpathlineto{\pgfqpoint{1.484406in}{6.712876in}}%
\pgfpathlineto{\pgfqpoint{1.489026in}{6.714250in}}%
\pgfpathlineto{\pgfqpoint{1.490437in}{6.717227in}}%
\pgfpathlineto{\pgfqpoint{1.493068in}{6.718372in}}%
\pgfpathlineto{\pgfqpoint{1.494415in}{6.720204in}}%
\pgfpathlineto{\pgfqpoint{1.495827in}{6.721120in}}%
\pgfpathlineto{\pgfqpoint{1.497302in}{6.722265in}}%
\pgfpathlineto{\pgfqpoint{1.498906in}{6.723639in}}%
\pgfpathlineto{\pgfqpoint{1.499997in}{6.724554in}}%
\pgfpathlineto{\pgfqpoint{1.503269in}{6.725928in}}%
\pgfpathlineto{\pgfqpoint{1.504616in}{6.727989in}}%
\pgfpathlineto{\pgfqpoint{1.506670in}{6.729363in}}%
\pgfpathlineto{\pgfqpoint{1.507696in}{6.730737in}}%
\pgfpathlineto{\pgfqpoint{1.510968in}{6.732111in}}%
\pgfpathlineto{\pgfqpoint{1.512123in}{6.733714in}}%
\pgfpathlineto{\pgfqpoint{1.514818in}{6.735088in}}%
\pgfpathlineto{\pgfqpoint{1.516358in}{6.736004in}}%
\pgfpathlineto{\pgfqpoint{1.519052in}{6.737378in}}%
\pgfpathlineto{\pgfqpoint{1.520528in}{6.740584in}}%
\pgfpathlineto{\pgfqpoint{1.523864in}{6.741729in}}%
\pgfpathlineto{\pgfqpoint{1.524891in}{6.743103in}}%
\pgfpathlineto{\pgfqpoint{1.527906in}{6.744477in}}%
\pgfpathlineto{\pgfqpoint{1.527906in}{6.745393in}}%
\pgfpathlineto{\pgfqpoint{1.532654in}{6.746767in}}%
\pgfpathlineto{\pgfqpoint{1.534194in}{6.748140in}}%
\pgfpathlineto{\pgfqpoint{1.536118in}{6.749514in}}%
\pgfpathlineto{\pgfqpoint{1.537530in}{6.751346in}}%
\pgfpathlineto{\pgfqpoint{1.539904in}{6.752720in}}%
\pgfpathlineto{\pgfqpoint{1.540417in}{6.753407in}}%
\pgfpathlineto{\pgfqpoint{1.541893in}{6.754781in}}%
\pgfpathlineto{\pgfqpoint{1.543176in}{6.757071in}}%
\pgfpathlineto{\pgfqpoint{1.544780in}{6.758445in}}%
\pgfpathlineto{\pgfqpoint{1.546191in}{6.759361in}}%
\pgfpathlineto{\pgfqpoint{1.548565in}{6.760506in}}%
\pgfpathlineto{\pgfqpoint{1.550041in}{6.762338in}}%
\pgfpathlineto{\pgfqpoint{1.551902in}{6.763712in}}%
\pgfpathlineto{\pgfqpoint{1.552030in}{6.764170in}}%
\pgfpathlineto{\pgfqpoint{1.553955in}{6.765315in}}%
\pgfpathlineto{\pgfqpoint{1.555302in}{6.767834in}}%
\pgfpathlineto{\pgfqpoint{1.557997in}{6.769208in}}%
\pgfpathlineto{\pgfqpoint{1.558510in}{6.770123in}}%
\pgfpathlineto{\pgfqpoint{1.561141in}{6.771497in}}%
\pgfpathlineto{\pgfqpoint{1.561718in}{6.772184in}}%
\pgfpathlineto{\pgfqpoint{1.567300in}{6.773558in}}%
\pgfpathlineto{\pgfqpoint{1.568775in}{6.774474in}}%
\pgfpathlineto{\pgfqpoint{1.571727in}{6.775848in}}%
\pgfpathlineto{\pgfqpoint{1.572753in}{6.776993in}}%
\pgfpathlineto{\pgfqpoint{1.575576in}{6.778367in}}%
\pgfpathlineto{\pgfqpoint{1.576090in}{6.778825in}}%
\pgfpathlineto{\pgfqpoint{1.579105in}{6.780199in}}%
\pgfpathlineto{\pgfqpoint{1.580517in}{6.781802in}}%
\pgfpathlineto{\pgfqpoint{1.583340in}{6.783176in}}%
\pgfpathlineto{\pgfqpoint{1.584815in}{6.785237in}}%
\pgfpathlineto{\pgfqpoint{1.586291in}{6.786382in}}%
\pgfpathlineto{\pgfqpoint{1.587702in}{6.788443in}}%
\pgfpathlineto{\pgfqpoint{1.590140in}{6.789588in}}%
\pgfpathlineto{\pgfqpoint{1.591359in}{6.791191in}}%
\pgfpathlineto{\pgfqpoint{1.593990in}{6.792335in}}%
\pgfpathlineto{\pgfqpoint{1.595273in}{6.793938in}}%
\pgfpathlineto{\pgfqpoint{1.598545in}{6.795083in}}%
\pgfpathlineto{\pgfqpoint{1.600085in}{6.798289in}}%
\pgfpathlineto{\pgfqpoint{1.603165in}{6.799663in}}%
\pgfpathlineto{\pgfqpoint{1.604576in}{6.801724in}}%
\pgfpathlineto{\pgfqpoint{1.607784in}{6.802869in}}%
\pgfpathlineto{\pgfqpoint{1.608939in}{6.804243in}}%
\pgfpathlineto{\pgfqpoint{1.613045in}{6.805617in}}%
\pgfpathlineto{\pgfqpoint{1.614264in}{6.807449in}}%
\pgfpathlineto{\pgfqpoint{1.618242in}{6.808823in}}%
\pgfpathlineto{\pgfqpoint{1.619140in}{6.809968in}}%
\pgfpathlineto{\pgfqpoint{1.622733in}{6.811113in}}%
\pgfpathlineto{\pgfqpoint{1.624016in}{6.812716in}}%
\pgfpathlineto{\pgfqpoint{1.627481in}{6.814090in}}%
\pgfpathlineto{\pgfqpoint{1.628058in}{6.814777in}}%
\pgfpathlineto{\pgfqpoint{1.630817in}{6.816150in}}%
\pgfpathlineto{\pgfqpoint{1.632357in}{6.817982in}}%
\pgfpathlineto{\pgfqpoint{1.635373in}{6.819356in}}%
\pgfpathlineto{\pgfqpoint{1.636527in}{6.820730in}}%
\pgfpathlineto{\pgfqpoint{1.640056in}{6.822104in}}%
\pgfpathlineto{\pgfqpoint{1.641596in}{6.823249in}}%
\pgfpathlineto{\pgfqpoint{1.644611in}{6.824623in}}%
\pgfpathlineto{\pgfqpoint{1.645830in}{6.825310in}}%
\pgfpathlineto{\pgfqpoint{1.650514in}{6.826684in}}%
\pgfpathlineto{\pgfqpoint{1.652054in}{6.827829in}}%
\pgfpathlineto{\pgfqpoint{1.654107in}{6.829203in}}%
\pgfpathlineto{\pgfqpoint{1.655326in}{6.830577in}}%
\pgfpathlineto{\pgfqpoint{1.658598in}{6.831951in}}%
\pgfpathlineto{\pgfqpoint{1.660010in}{6.832867in}}%
\pgfpathlineto{\pgfqpoint{1.661870in}{6.834241in}}%
\pgfpathlineto{\pgfqpoint{1.663346in}{6.834928in}}%
\pgfpathlineto{\pgfqpoint{1.665527in}{6.836073in}}%
\pgfpathlineto{\pgfqpoint{1.666169in}{6.837218in}}%
\pgfpathlineto{\pgfqpoint{1.668864in}{6.838591in}}%
\pgfpathlineto{\pgfqpoint{1.670275in}{6.839507in}}%
\pgfpathlineto{\pgfqpoint{1.672328in}{6.840881in}}%
\pgfpathlineto{\pgfqpoint{1.672328in}{6.841110in}}%
\pgfpathlineto{\pgfqpoint{1.676049in}{6.842484in}}%
\pgfpathlineto{\pgfqpoint{1.676049in}{6.842713in}}%
\pgfpathlineto{\pgfqpoint{1.680284in}{6.843858in}}%
\pgfpathlineto{\pgfqpoint{1.681824in}{6.846606in}}%
\pgfpathlineto{\pgfqpoint{1.686892in}{6.847980in}}%
\pgfpathlineto{\pgfqpoint{1.688240in}{6.849354in}}%
\pgfpathlineto{\pgfqpoint{1.689523in}{6.850041in}}%
\pgfpathlineto{\pgfqpoint{1.690613in}{6.851644in}}%
\pgfpathlineto{\pgfqpoint{1.694014in}{6.852560in}}%
\pgfpathlineto{\pgfqpoint{1.695169in}{6.854850in}}%
\pgfpathlineto{\pgfqpoint{1.697543in}{6.856224in}}%
\pgfpathlineto{\pgfqpoint{1.699018in}{6.858285in}}%
\pgfpathlineto{\pgfqpoint{1.700237in}{6.859659in}}%
\pgfpathlineto{\pgfqpoint{1.701777in}{6.860346in}}%
\pgfpathlineto{\pgfqpoint{1.704793in}{6.861719in}}%
\pgfpathlineto{\pgfqpoint{1.705883in}{6.862406in}}%
\pgfpathlineto{\pgfqpoint{1.708385in}{6.863780in}}%
\pgfpathlineto{\pgfqpoint{1.709605in}{6.864696in}}%
\pgfpathlineto{\pgfqpoint{1.713262in}{6.866070in}}%
\pgfpathlineto{\pgfqpoint{1.714609in}{6.868131in}}%
\pgfpathlineto{\pgfqpoint{1.717175in}{6.869276in}}%
\pgfpathlineto{\pgfqpoint{1.718651in}{6.870421in}}%
\pgfpathlineto{\pgfqpoint{1.720961in}{6.871795in}}%
\pgfpathlineto{\pgfqpoint{1.721474in}{6.872253in}}%
\pgfpathlineto{\pgfqpoint{1.724040in}{6.873627in}}%
\pgfpathlineto{\pgfqpoint{1.725323in}{6.875230in}}%
\pgfpathlineto{\pgfqpoint{1.728403in}{6.876604in}}%
\pgfpathlineto{\pgfqpoint{1.729301in}{6.877291in}}%
\pgfpathlineto{\pgfqpoint{1.733023in}{6.878665in}}%
\pgfpathlineto{\pgfqpoint{1.734434in}{6.881184in}}%
\pgfpathlineto{\pgfqpoint{1.736744in}{6.882558in}}%
\pgfpathlineto{\pgfqpoint{1.737514in}{6.883245in}}%
\pgfpathlineto{\pgfqpoint{1.741107in}{6.884618in}}%
\pgfpathlineto{\pgfqpoint{1.741107in}{6.884847in}}%
\pgfpathlineto{\pgfqpoint{1.745919in}{6.886221in}}%
\pgfpathlineto{\pgfqpoint{1.747330in}{6.887137in}}%
\pgfpathlineto{\pgfqpoint{1.750987in}{6.888511in}}%
\pgfpathlineto{\pgfqpoint{1.752334in}{6.889656in}}%
\pgfpathlineto{\pgfqpoint{1.755991in}{6.891030in}}%
\pgfpathlineto{\pgfqpoint{1.756761in}{6.891717in}}%
\pgfpathlineto{\pgfqpoint{1.759649in}{6.892633in}}%
\pgfpathlineto{\pgfqpoint{1.760418in}{6.894007in}}%
\pgfpathlineto{\pgfqpoint{1.764781in}{6.895381in}}%
\pgfpathlineto{\pgfqpoint{1.766321in}{6.896755in}}%
\pgfpathlineto{\pgfqpoint{1.770106in}{6.898129in}}%
\pgfpathlineto{\pgfqpoint{1.771133in}{6.899045in}}%
\pgfpathlineto{\pgfqpoint{1.773443in}{6.900190in}}%
\pgfpathlineto{\pgfqpoint{1.774726in}{6.901335in}}%
\pgfpathlineto{\pgfqpoint{1.781206in}{6.902709in}}%
\pgfpathlineto{\pgfqpoint{1.781591in}{6.903396in}}%
\pgfpathlineto{\pgfqpoint{1.787750in}{6.904770in}}%
\pgfpathlineto{\pgfqpoint{1.789162in}{6.906143in}}%
\pgfpathlineto{\pgfqpoint{1.790509in}{6.907288in}}%
\pgfpathlineto{\pgfqpoint{1.791985in}{6.908433in}}%
\pgfpathlineto{\pgfqpoint{1.795385in}{6.909578in}}%
\pgfpathlineto{\pgfqpoint{1.796540in}{6.911410in}}%
\pgfpathlineto{\pgfqpoint{1.797502in}{6.912784in}}%
\pgfpathlineto{\pgfqpoint{1.797502in}{6.913013in}}%
\pgfpathlineto{\pgfqpoint{1.801480in}{6.914387in}}%
\pgfpathlineto{\pgfqpoint{1.803020in}{6.915074in}}%
\pgfpathlineto{\pgfqpoint{1.806549in}{6.916448in}}%
\pgfpathlineto{\pgfqpoint{1.806741in}{6.916906in}}%
\pgfpathlineto{\pgfqpoint{1.811232in}{6.918280in}}%
\pgfpathlineto{\pgfqpoint{1.812772in}{6.919196in}}%
\pgfpathlineto{\pgfqpoint{1.814889in}{6.920570in}}%
\pgfpathlineto{\pgfqpoint{1.816365in}{6.922402in}}%
\pgfpathlineto{\pgfqpoint{1.822075in}{6.923776in}}%
\pgfpathlineto{\pgfqpoint{1.822653in}{6.924692in}}%
\pgfpathlineto{\pgfqpoint{1.826759in}{6.926066in}}%
\pgfpathlineto{\pgfqpoint{1.828042in}{6.926524in}}%
\pgfpathlineto{\pgfqpoint{1.833752in}{6.927898in}}%
\pgfpathlineto{\pgfqpoint{1.834971in}{6.929271in}}%
\pgfpathlineto{\pgfqpoint{1.837538in}{6.930645in}}%
\pgfpathlineto{\pgfqpoint{1.838243in}{6.931332in}}%
\pgfpathlineto{\pgfqpoint{1.845878in}{6.932706in}}%
\pgfpathlineto{\pgfqpoint{1.847033in}{6.933393in}}%
\pgfpathlineto{\pgfqpoint{1.851396in}{6.934538in}}%
\pgfpathlineto{\pgfqpoint{1.852936in}{6.935225in}}%
\pgfpathlineto{\pgfqpoint{1.857042in}{6.936599in}}%
\pgfpathlineto{\pgfqpoint{1.858582in}{6.937515in}}%
\pgfpathlineto{\pgfqpoint{1.862175in}{6.938889in}}%
\pgfpathlineto{\pgfqpoint{1.863522in}{6.940034in}}%
\pgfpathlineto{\pgfqpoint{1.866024in}{6.941408in}}%
\pgfpathlineto{\pgfqpoint{1.867564in}{6.942553in}}%
\pgfpathlineto{\pgfqpoint{1.870259in}{6.943927in}}%
\pgfpathlineto{\pgfqpoint{1.870836in}{6.944614in}}%
\pgfpathlineto{\pgfqpoint{1.876097in}{6.945988in}}%
\pgfpathlineto{\pgfqpoint{1.877316in}{6.947133in}}%
\pgfpathlineto{\pgfqpoint{1.880203in}{6.948507in}}%
\pgfpathlineto{\pgfqpoint{1.881679in}{6.950110in}}%
\pgfpathlineto{\pgfqpoint{1.885400in}{6.951484in}}%
\pgfpathlineto{\pgfqpoint{1.885400in}{6.951712in}}%
\pgfpathlineto{\pgfqpoint{1.890212in}{6.953086in}}%
\pgfpathlineto{\pgfqpoint{1.891367in}{6.954002in}}%
\pgfpathlineto{\pgfqpoint{1.895217in}{6.955376in}}%
\pgfpathlineto{\pgfqpoint{1.896500in}{6.956063in}}%
\pgfpathlineto{\pgfqpoint{1.903172in}{6.957437in}}%
\pgfpathlineto{\pgfqpoint{1.904071in}{6.958353in}}%
\pgfpathlineto{\pgfqpoint{1.909460in}{6.959727in}}%
\pgfpathlineto{\pgfqpoint{1.910615in}{6.960872in}}%
\pgfpathlineto{\pgfqpoint{1.914015in}{6.962246in}}%
\pgfpathlineto{\pgfqpoint{1.915491in}{6.962933in}}%
\pgfpathlineto{\pgfqpoint{1.919148in}{6.964307in}}%
\pgfpathlineto{\pgfqpoint{1.919469in}{6.965452in}}%
\pgfpathlineto{\pgfqpoint{1.924537in}{6.966826in}}%
\pgfpathlineto{\pgfqpoint{1.925692in}{6.967742in}}%
\pgfpathlineto{\pgfqpoint{1.930825in}{6.969116in}}%
\pgfpathlineto{\pgfqpoint{1.930825in}{6.969345in}}%
\pgfpathlineto{\pgfqpoint{1.940577in}{6.970719in}}%
\pgfpathlineto{\pgfqpoint{1.941411in}{6.971177in}}%
\pgfpathlineto{\pgfqpoint{1.945389in}{6.972551in}}%
\pgfpathlineto{\pgfqpoint{1.945966in}{6.973238in}}%
\pgfpathlineto{\pgfqpoint{1.950650in}{6.974611in}}%
\pgfpathlineto{\pgfqpoint{1.951227in}{6.975527in}}%
\pgfpathlineto{\pgfqpoint{1.955783in}{6.976901in}}%
\pgfpathlineto{\pgfqpoint{1.957002in}{6.977817in}}%
\pgfpathlineto{\pgfqpoint{1.961044in}{6.979191in}}%
\pgfpathlineto{\pgfqpoint{1.962391in}{6.979878in}}%
\pgfpathlineto{\pgfqpoint{1.966112in}{6.981252in}}%
\pgfpathlineto{\pgfqpoint{1.966112in}{6.981481in}}%
\pgfpathlineto{\pgfqpoint{1.975030in}{6.982855in}}%
\pgfpathlineto{\pgfqpoint{1.976249in}{6.983542in}}%
\pgfpathlineto{\pgfqpoint{1.983499in}{6.984916in}}%
\pgfpathlineto{\pgfqpoint{1.984654in}{6.985603in}}%
\pgfpathlineto{\pgfqpoint{1.989145in}{6.986977in}}%
\pgfpathlineto{\pgfqpoint{1.990621in}{6.987664in}}%
\pgfpathlineto{\pgfqpoint{1.990685in}{6.987664in}}%
\pgfpathlineto{\pgfqpoint{1.994791in}{6.988809in}}%
\pgfpathlineto{\pgfqpoint{1.994920in}{6.989954in}}%
\pgfpathlineto{\pgfqpoint{2.000309in}{6.991099in}}%
\pgfpathlineto{\pgfqpoint{2.001721in}{6.992244in}}%
\pgfpathlineto{\pgfqpoint{2.005634in}{6.993618in}}%
\pgfpathlineto{\pgfqpoint{2.006340in}{6.994076in}}%
\pgfpathlineto{\pgfqpoint{2.012178in}{6.995450in}}%
\pgfpathlineto{\pgfqpoint{2.012371in}{6.996137in}}%
\pgfpathlineto{\pgfqpoint{2.013590in}{6.996137in}}%
\pgfpathlineto{\pgfqpoint{2.018659in}{6.997510in}}%
\pgfpathlineto{\pgfqpoint{2.019300in}{6.998197in}}%
\pgfpathlineto{\pgfqpoint{2.026871in}{6.999571in}}%
\pgfpathlineto{\pgfqpoint{2.027192in}{7.000029in}}%
\pgfpathlineto{\pgfqpoint{2.033543in}{7.001403in}}%
\pgfpathlineto{\pgfqpoint{2.033543in}{7.001632in}}%
\pgfpathlineto{\pgfqpoint{2.034057in}{7.001632in}}%
\pgfpathlineto{\pgfqpoint{2.043167in}{7.003006in}}%
\pgfpathlineto{\pgfqpoint{2.044579in}{7.003922in}}%
\pgfpathlineto{\pgfqpoint{2.051316in}{7.005296in}}%
\pgfpathlineto{\pgfqpoint{2.051380in}{7.005754in}}%
\pgfpathlineto{\pgfqpoint{2.057796in}{7.007128in}}%
\pgfpathlineto{\pgfqpoint{2.059207in}{7.008044in}}%
\pgfpathlineto{\pgfqpoint{2.069023in}{7.009418in}}%
\pgfpathlineto{\pgfqpoint{2.070178in}{7.010563in}}%
\pgfpathlineto{\pgfqpoint{2.074798in}{7.011937in}}%
\pgfpathlineto{\pgfqpoint{2.074798in}{7.012166in}}%
\pgfpathlineto{\pgfqpoint{2.079545in}{7.013311in}}%
\pgfpathlineto{\pgfqpoint{2.080187in}{7.014227in}}%
\pgfpathlineto{\pgfqpoint{2.083395in}{7.015601in}}%
\pgfpathlineto{\pgfqpoint{2.083908in}{7.016288in}}%
\pgfpathlineto{\pgfqpoint{2.091351in}{7.017662in}}%
\pgfpathlineto{\pgfqpoint{2.091351in}{7.017891in}}%
\pgfpathlineto{\pgfqpoint{2.096933in}{7.019265in}}%
\pgfpathlineto{\pgfqpoint{2.097318in}{7.019723in}}%
\pgfpathlineto{\pgfqpoint{2.106556in}{7.020867in}}%
\pgfpathlineto{\pgfqpoint{2.107968in}{7.021783in}}%
\pgfpathlineto{\pgfqpoint{2.117913in}{7.023157in}}%
\pgfpathlineto{\pgfqpoint{2.118490in}{7.024073in}}%
\pgfpathlineto{\pgfqpoint{2.125291in}{7.025447in}}%
\pgfpathlineto{\pgfqpoint{2.125419in}{7.025905in}}%
\pgfpathlineto{\pgfqpoint{2.138764in}{7.027279in}}%
\pgfpathlineto{\pgfqpoint{2.138764in}{7.027737in}}%
\pgfpathlineto{\pgfqpoint{2.146399in}{7.029111in}}%
\pgfpathlineto{\pgfqpoint{2.146399in}{7.029340in}}%
\pgfpathlineto{\pgfqpoint{2.160065in}{7.030714in}}%
\pgfpathlineto{\pgfqpoint{2.160065in}{7.030943in}}%
\pgfpathlineto{\pgfqpoint{2.176554in}{7.032317in}}%
\pgfpathlineto{\pgfqpoint{2.176554in}{7.032546in}}%
\pgfpathlineto{\pgfqpoint{2.255598in}{7.033920in}}%
\pgfpathlineto{\pgfqpoint{2.255598in}{7.034149in}}%
\pgfpathlineto{\pgfqpoint{2.373650in}{7.034149in}}%
\pgfpathlineto{\pgfqpoint{2.373650in}{7.034149in}}%
\pgfusepath{stroke}%
\end{pgfscope}%
\begin{pgfscope}%
\pgfpathrectangle{\pgfqpoint{0.763041in}{5.564583in}}{\pgfqpoint{1.687305in}{1.539545in}}%
\pgfusepath{clip}%
\pgfsetrectcap%
\pgfsetroundjoin%
\pgfsetlinewidth{1.505625pt}%
\definecolor{currentstroke}{rgb}{0.501961,0.501961,0.501961}%
\pgfsetstrokecolor{currentstroke}%
\pgfsetdash{}{0pt}%
\pgfpathmoveto{\pgfqpoint{0.839736in}{5.634562in}}%
\pgfpathlineto{\pgfqpoint{2.373650in}{7.034149in}}%
\pgfusepath{stroke}%
\end{pgfscope}%
\begin{pgfscope}%
\pgfsetrectcap%
\pgfsetmiterjoin%
\pgfsetlinewidth{0.803000pt}%
\definecolor{currentstroke}{rgb}{0.000000,0.000000,0.000000}%
\pgfsetstrokecolor{currentstroke}%
\pgfsetdash{}{0pt}%
\pgfpathmoveto{\pgfqpoint{0.763041in}{5.564583in}}%
\pgfpathlineto{\pgfqpoint{0.763041in}{7.104128in}}%
\pgfusepath{stroke}%
\end{pgfscope}%
\begin{pgfscope}%
\pgfsetrectcap%
\pgfsetmiterjoin%
\pgfsetlinewidth{0.803000pt}%
\definecolor{currentstroke}{rgb}{0.000000,0.000000,0.000000}%
\pgfsetstrokecolor{currentstroke}%
\pgfsetdash{}{0pt}%
\pgfpathmoveto{\pgfqpoint{2.450346in}{5.564583in}}%
\pgfpathlineto{\pgfqpoint{2.450346in}{7.104128in}}%
\pgfusepath{stroke}%
\end{pgfscope}%
\begin{pgfscope}%
\pgfsetrectcap%
\pgfsetmiterjoin%
\pgfsetlinewidth{0.803000pt}%
\definecolor{currentstroke}{rgb}{0.000000,0.000000,0.000000}%
\pgfsetstrokecolor{currentstroke}%
\pgfsetdash{}{0pt}%
\pgfpathmoveto{\pgfqpoint{0.763041in}{5.564583in}}%
\pgfpathlineto{\pgfqpoint{2.450346in}{5.564583in}}%
\pgfusepath{stroke}%
\end{pgfscope}%
\begin{pgfscope}%
\pgfsetrectcap%
\pgfsetmiterjoin%
\pgfsetlinewidth{0.803000pt}%
\definecolor{currentstroke}{rgb}{0.000000,0.000000,0.000000}%
\pgfsetstrokecolor{currentstroke}%
\pgfsetdash{}{0pt}%
\pgfpathmoveto{\pgfqpoint{0.763041in}{7.104128in}}%
\pgfpathlineto{\pgfqpoint{2.450346in}{7.104128in}}%
\pgfusepath{stroke}%
\end{pgfscope}%
\begin{pgfscope}%
\definecolor{textcolor}{rgb}{0.000000,0.000000,0.000000}%
\pgfsetstrokecolor{textcolor}%
\pgfsetfillcolor{textcolor}%
\pgftext[x=1.606693in,y=7.187462in,,base]{\color{textcolor}\rmfamily\fontsize{20.000000}{24.000000}\selectfont Infiltration}%
\end{pgfscope}%
\begin{pgfscope}%
\pgfsetbuttcap%
\pgfsetmiterjoin%
\definecolor{currentfill}{rgb}{1.000000,1.000000,1.000000}%
\pgfsetfillcolor{currentfill}%
\pgfsetfillopacity{0.800000}%
\pgfsetlinewidth{1.003750pt}%
\definecolor{currentstroke}{rgb}{0.800000,0.800000,0.800000}%
\pgfsetstrokecolor{currentstroke}%
\pgfsetstrokeopacity{0.800000}%
\pgfsetdash{}{0pt}%
\pgfpathmoveto{\pgfqpoint{1.241240in}{5.634027in}}%
\pgfpathlineto{\pgfqpoint{2.353124in}{5.634027in}}%
\pgfpathquadraticcurveto{\pgfqpoint{2.380902in}{5.634027in}}{\pgfqpoint{2.380902in}{5.661805in}}%
\pgfpathlineto{\pgfqpoint{2.380902in}{5.841589in}}%
\pgfpathquadraticcurveto{\pgfqpoint{2.380902in}{5.869367in}}{\pgfqpoint{2.353124in}{5.869367in}}%
\pgfpathlineto{\pgfqpoint{1.241240in}{5.869367in}}%
\pgfpathquadraticcurveto{\pgfqpoint{1.213462in}{5.869367in}}{\pgfqpoint{1.213462in}{5.841589in}}%
\pgfpathlineto{\pgfqpoint{1.213462in}{5.661805in}}%
\pgfpathquadraticcurveto{\pgfqpoint{1.213462in}{5.634027in}}{\pgfqpoint{1.241240in}{5.634027in}}%
\pgfpathclose%
\pgfusepath{stroke,fill}%
\end{pgfscope}%
\begin{pgfscope}%
\pgfsetrectcap%
\pgfsetroundjoin%
\pgfsetlinewidth{1.505625pt}%
\definecolor{currentstroke}{rgb}{0.000000,0.501961,0.000000}%
\pgfsetstrokecolor{currentstroke}%
\pgfsetdash{}{0pt}%
\pgfpathmoveto{\pgfqpoint{1.269018in}{5.765200in}}%
\pgfpathlineto{\pgfqpoint{1.546795in}{5.765200in}}%
\pgfusepath{stroke}%
\end{pgfscope}%
\begin{pgfscope}%
\definecolor{textcolor}{rgb}{0.000000,0.000000,0.000000}%
\pgfsetstrokecolor{textcolor}%
\pgfsetfillcolor{textcolor}%
\pgftext[x=1.657907in,y=5.716589in,left,base]{\color{textcolor}\rmfamily\fontsize{10.000000}{12.000000}\selectfont AUC 0.740}%
\end{pgfscope}%
\begin{pgfscope}%
\pgfsetbuttcap%
\pgfsetmiterjoin%
\definecolor{currentfill}{rgb}{1.000000,1.000000,1.000000}%
\pgfsetfillcolor{currentfill}%
\pgfsetlinewidth{0.000000pt}%
\definecolor{currentstroke}{rgb}{0.000000,0.000000,0.000000}%
\pgfsetstrokecolor{currentstroke}%
\pgfsetstrokeopacity{0.000000}%
\pgfsetdash{}{0pt}%
\pgfpathmoveto{\pgfqpoint{3.225541in}{5.564583in}}%
\pgfpathlineto{\pgfqpoint{4.912846in}{5.564583in}}%
\pgfpathlineto{\pgfqpoint{4.912846in}{7.104128in}}%
\pgfpathlineto{\pgfqpoint{3.225541in}{7.104128in}}%
\pgfpathclose%
\pgfusepath{fill}%
\end{pgfscope}%
\begin{pgfscope}%
\pgfsetbuttcap%
\pgfsetroundjoin%
\definecolor{currentfill}{rgb}{0.000000,0.000000,0.000000}%
\pgfsetfillcolor{currentfill}%
\pgfsetlinewidth{0.803000pt}%
\definecolor{currentstroke}{rgb}{0.000000,0.000000,0.000000}%
\pgfsetstrokecolor{currentstroke}%
\pgfsetdash{}{0pt}%
\pgfsys@defobject{currentmarker}{\pgfqpoint{0.000000in}{-0.048611in}}{\pgfqpoint{0.000000in}{0.000000in}}{%
\pgfpathmoveto{\pgfqpoint{0.000000in}{0.000000in}}%
\pgfpathlineto{\pgfqpoint{0.000000in}{-0.048611in}}%
\pgfusepath{stroke,fill}%
}%
\begin{pgfscope}%
\pgfsys@transformshift{3.302236in}{5.564583in}%
\pgfsys@useobject{currentmarker}{}%
\end{pgfscope}%
\end{pgfscope}%
\begin{pgfscope}%
\definecolor{textcolor}{rgb}{0.000000,0.000000,0.000000}%
\pgfsetstrokecolor{textcolor}%
\pgfsetfillcolor{textcolor}%
\pgftext[x=3.302236in,y=5.467361in,,top]{\color{textcolor}\rmfamily\fontsize{10.000000}{12.000000}\selectfont \(\displaystyle {0.0}\)}%
\end{pgfscope}%
\begin{pgfscope}%
\pgfsetbuttcap%
\pgfsetroundjoin%
\definecolor{currentfill}{rgb}{0.000000,0.000000,0.000000}%
\pgfsetfillcolor{currentfill}%
\pgfsetlinewidth{0.803000pt}%
\definecolor{currentstroke}{rgb}{0.000000,0.000000,0.000000}%
\pgfsetstrokecolor{currentstroke}%
\pgfsetdash{}{0pt}%
\pgfsys@defobject{currentmarker}{\pgfqpoint{0.000000in}{-0.048611in}}{\pgfqpoint{0.000000in}{0.000000in}}{%
\pgfpathmoveto{\pgfqpoint{0.000000in}{0.000000in}}%
\pgfpathlineto{\pgfqpoint{0.000000in}{-0.048611in}}%
\pgfusepath{stroke,fill}%
}%
\begin{pgfscope}%
\pgfsys@transformshift{4.069193in}{5.564583in}%
\pgfsys@useobject{currentmarker}{}%
\end{pgfscope}%
\end{pgfscope}%
\begin{pgfscope}%
\definecolor{textcolor}{rgb}{0.000000,0.000000,0.000000}%
\pgfsetstrokecolor{textcolor}%
\pgfsetfillcolor{textcolor}%
\pgftext[x=4.069193in,y=5.467361in,,top]{\color{textcolor}\rmfamily\fontsize{10.000000}{12.000000}\selectfont \(\displaystyle {0.5}\)}%
\end{pgfscope}%
\begin{pgfscope}%
\pgfsetbuttcap%
\pgfsetroundjoin%
\definecolor{currentfill}{rgb}{0.000000,0.000000,0.000000}%
\pgfsetfillcolor{currentfill}%
\pgfsetlinewidth{0.803000pt}%
\definecolor{currentstroke}{rgb}{0.000000,0.000000,0.000000}%
\pgfsetstrokecolor{currentstroke}%
\pgfsetdash{}{0pt}%
\pgfsys@defobject{currentmarker}{\pgfqpoint{0.000000in}{-0.048611in}}{\pgfqpoint{0.000000in}{0.000000in}}{%
\pgfpathmoveto{\pgfqpoint{0.000000in}{0.000000in}}%
\pgfpathlineto{\pgfqpoint{0.000000in}{-0.048611in}}%
\pgfusepath{stroke,fill}%
}%
\begin{pgfscope}%
\pgfsys@transformshift{4.836150in}{5.564583in}%
\pgfsys@useobject{currentmarker}{}%
\end{pgfscope}%
\end{pgfscope}%
\begin{pgfscope}%
\definecolor{textcolor}{rgb}{0.000000,0.000000,0.000000}%
\pgfsetstrokecolor{textcolor}%
\pgfsetfillcolor{textcolor}%
\pgftext[x=4.836150in,y=5.467361in,,top]{\color{textcolor}\rmfamily\fontsize{10.000000}{12.000000}\selectfont \(\displaystyle {1.0}\)}%
\end{pgfscope}%
\begin{pgfscope}%
\definecolor{textcolor}{rgb}{0.000000,0.000000,0.000000}%
\pgfsetstrokecolor{textcolor}%
\pgfsetfillcolor{textcolor}%
\pgftext[x=4.069193in,y=5.288349in,,top]{\color{textcolor}\rmfamily\fontsize{16.000000}{19.200000}\selectfont FPR}%
\end{pgfscope}%
\begin{pgfscope}%
\pgfsetbuttcap%
\pgfsetroundjoin%
\definecolor{currentfill}{rgb}{0.000000,0.000000,0.000000}%
\pgfsetfillcolor{currentfill}%
\pgfsetlinewidth{0.803000pt}%
\definecolor{currentstroke}{rgb}{0.000000,0.000000,0.000000}%
\pgfsetstrokecolor{currentstroke}%
\pgfsetdash{}{0pt}%
\pgfsys@defobject{currentmarker}{\pgfqpoint{-0.048611in}{0.000000in}}{\pgfqpoint{-0.000000in}{0.000000in}}{%
\pgfpathmoveto{\pgfqpoint{-0.000000in}{0.000000in}}%
\pgfpathlineto{\pgfqpoint{-0.048611in}{0.000000in}}%
\pgfusepath{stroke,fill}%
}%
\begin{pgfscope}%
\pgfsys@transformshift{3.225541in}{5.634562in}%
\pgfsys@useobject{currentmarker}{}%
\end{pgfscope}%
\end{pgfscope}%
\begin{pgfscope}%
\definecolor{textcolor}{rgb}{0.000000,0.000000,0.000000}%
\pgfsetstrokecolor{textcolor}%
\pgfsetfillcolor{textcolor}%
\pgftext[x=2.881404in, y=5.586337in, left, base]{\color{textcolor}\rmfamily\fontsize{10.000000}{12.000000}\selectfont \(\displaystyle {0.00}\)}%
\end{pgfscope}%
\begin{pgfscope}%
\pgfsetbuttcap%
\pgfsetroundjoin%
\definecolor{currentfill}{rgb}{0.000000,0.000000,0.000000}%
\pgfsetfillcolor{currentfill}%
\pgfsetlinewidth{0.803000pt}%
\definecolor{currentstroke}{rgb}{0.000000,0.000000,0.000000}%
\pgfsetstrokecolor{currentstroke}%
\pgfsetdash{}{0pt}%
\pgfsys@defobject{currentmarker}{\pgfqpoint{-0.048611in}{0.000000in}}{\pgfqpoint{-0.000000in}{0.000000in}}{%
\pgfpathmoveto{\pgfqpoint{-0.000000in}{0.000000in}}%
\pgfpathlineto{\pgfqpoint{-0.048611in}{0.000000in}}%
\pgfusepath{stroke,fill}%
}%
\begin{pgfscope}%
\pgfsys@transformshift{3.225541in}{5.984459in}%
\pgfsys@useobject{currentmarker}{}%
\end{pgfscope}%
\end{pgfscope}%
\begin{pgfscope}%
\definecolor{textcolor}{rgb}{0.000000,0.000000,0.000000}%
\pgfsetstrokecolor{textcolor}%
\pgfsetfillcolor{textcolor}%
\pgftext[x=2.881404in, y=5.936234in, left, base]{\color{textcolor}\rmfamily\fontsize{10.000000}{12.000000}\selectfont \(\displaystyle {0.25}\)}%
\end{pgfscope}%
\begin{pgfscope}%
\pgfsetbuttcap%
\pgfsetroundjoin%
\definecolor{currentfill}{rgb}{0.000000,0.000000,0.000000}%
\pgfsetfillcolor{currentfill}%
\pgfsetlinewidth{0.803000pt}%
\definecolor{currentstroke}{rgb}{0.000000,0.000000,0.000000}%
\pgfsetstrokecolor{currentstroke}%
\pgfsetdash{}{0pt}%
\pgfsys@defobject{currentmarker}{\pgfqpoint{-0.048611in}{0.000000in}}{\pgfqpoint{-0.000000in}{0.000000in}}{%
\pgfpathmoveto{\pgfqpoint{-0.000000in}{0.000000in}}%
\pgfpathlineto{\pgfqpoint{-0.048611in}{0.000000in}}%
\pgfusepath{stroke,fill}%
}%
\begin{pgfscope}%
\pgfsys@transformshift{3.225541in}{6.334356in}%
\pgfsys@useobject{currentmarker}{}%
\end{pgfscope}%
\end{pgfscope}%
\begin{pgfscope}%
\definecolor{textcolor}{rgb}{0.000000,0.000000,0.000000}%
\pgfsetstrokecolor{textcolor}%
\pgfsetfillcolor{textcolor}%
\pgftext[x=2.881404in, y=6.286130in, left, base]{\color{textcolor}\rmfamily\fontsize{10.000000}{12.000000}\selectfont \(\displaystyle {0.50}\)}%
\end{pgfscope}%
\begin{pgfscope}%
\pgfsetbuttcap%
\pgfsetroundjoin%
\definecolor{currentfill}{rgb}{0.000000,0.000000,0.000000}%
\pgfsetfillcolor{currentfill}%
\pgfsetlinewidth{0.803000pt}%
\definecolor{currentstroke}{rgb}{0.000000,0.000000,0.000000}%
\pgfsetstrokecolor{currentstroke}%
\pgfsetdash{}{0pt}%
\pgfsys@defobject{currentmarker}{\pgfqpoint{-0.048611in}{0.000000in}}{\pgfqpoint{-0.000000in}{0.000000in}}{%
\pgfpathmoveto{\pgfqpoint{-0.000000in}{0.000000in}}%
\pgfpathlineto{\pgfqpoint{-0.048611in}{0.000000in}}%
\pgfusepath{stroke,fill}%
}%
\begin{pgfscope}%
\pgfsys@transformshift{3.225541in}{6.684252in}%
\pgfsys@useobject{currentmarker}{}%
\end{pgfscope}%
\end{pgfscope}%
\begin{pgfscope}%
\definecolor{textcolor}{rgb}{0.000000,0.000000,0.000000}%
\pgfsetstrokecolor{textcolor}%
\pgfsetfillcolor{textcolor}%
\pgftext[x=2.881404in, y=6.636027in, left, base]{\color{textcolor}\rmfamily\fontsize{10.000000}{12.000000}\selectfont \(\displaystyle {0.75}\)}%
\end{pgfscope}%
\begin{pgfscope}%
\pgfsetbuttcap%
\pgfsetroundjoin%
\definecolor{currentfill}{rgb}{0.000000,0.000000,0.000000}%
\pgfsetfillcolor{currentfill}%
\pgfsetlinewidth{0.803000pt}%
\definecolor{currentstroke}{rgb}{0.000000,0.000000,0.000000}%
\pgfsetstrokecolor{currentstroke}%
\pgfsetdash{}{0pt}%
\pgfsys@defobject{currentmarker}{\pgfqpoint{-0.048611in}{0.000000in}}{\pgfqpoint{-0.000000in}{0.000000in}}{%
\pgfpathmoveto{\pgfqpoint{-0.000000in}{0.000000in}}%
\pgfpathlineto{\pgfqpoint{-0.048611in}{0.000000in}}%
\pgfusepath{stroke,fill}%
}%
\begin{pgfscope}%
\pgfsys@transformshift{3.225541in}{7.034149in}%
\pgfsys@useobject{currentmarker}{}%
\end{pgfscope}%
\end{pgfscope}%
\begin{pgfscope}%
\definecolor{textcolor}{rgb}{0.000000,0.000000,0.000000}%
\pgfsetstrokecolor{textcolor}%
\pgfsetfillcolor{textcolor}%
\pgftext[x=2.881404in, y=6.985924in, left, base]{\color{textcolor}\rmfamily\fontsize{10.000000}{12.000000}\selectfont \(\displaystyle {1.00}\)}%
\end{pgfscope}%
\begin{pgfscope}%
\definecolor{textcolor}{rgb}{0.000000,0.000000,0.000000}%
\pgfsetstrokecolor{textcolor}%
\pgfsetfillcolor{textcolor}%
\pgftext[x=2.825849in,y=6.334356in,,bottom,rotate=90.000000]{\color{textcolor}\rmfamily\fontsize{16.000000}{19.200000}\selectfont TPR}%
\end{pgfscope}%
\begin{pgfscope}%
\pgfpathrectangle{\pgfqpoint{3.225541in}{5.564583in}}{\pgfqpoint{1.687305in}{1.539545in}}%
\pgfusepath{clip}%
\pgfsetrectcap%
\pgfsetroundjoin%
\pgfsetlinewidth{1.505625pt}%
\definecolor{currentstroke}{rgb}{0.000000,0.501961,0.000000}%
\pgfsetstrokecolor{currentstroke}%
\pgfsetdash{}{0pt}%
\pgfpathmoveto{\pgfqpoint{3.302236in}{5.634562in}}%
\pgfpathlineto{\pgfqpoint{3.310375in}{5.757066in}}%
\pgfpathlineto{\pgfqpoint{3.310592in}{5.757066in}}%
\pgfpathlineto{\pgfqpoint{3.312111in}{5.779485in}}%
\pgfpathlineto{\pgfqpoint{3.312382in}{5.780286in}}%
\pgfpathlineto{\pgfqpoint{3.313901in}{5.801104in}}%
\pgfpathlineto{\pgfqpoint{3.313956in}{5.801104in}}%
\pgfpathlineto{\pgfqpoint{3.315475in}{5.813914in}}%
\pgfpathlineto{\pgfqpoint{3.315583in}{5.813914in}}%
\pgfpathlineto{\pgfqpoint{3.316940in}{5.834732in}}%
\pgfpathlineto{\pgfqpoint{3.317428in}{5.835533in}}%
\pgfpathlineto{\pgfqpoint{3.318947in}{5.854749in}}%
\pgfpathlineto{\pgfqpoint{3.319273in}{5.854749in}}%
\pgfpathlineto{\pgfqpoint{3.320738in}{5.865158in}}%
\pgfpathlineto{\pgfqpoint{3.320846in}{5.865158in}}%
\pgfpathlineto{\pgfqpoint{3.322094in}{5.874766in}}%
\pgfpathlineto{\pgfqpoint{3.322636in}{5.875567in}}%
\pgfpathlineto{\pgfqpoint{3.324047in}{5.886776in}}%
\pgfpathlineto{\pgfqpoint{3.324210in}{5.886776in}}%
\pgfpathlineto{\pgfqpoint{3.325729in}{5.914800in}}%
\pgfpathlineto{\pgfqpoint{3.326163in}{5.915601in}}%
\pgfpathlineto{\pgfqpoint{3.327628in}{5.928411in}}%
\pgfpathlineto{\pgfqpoint{3.327791in}{5.929212in}}%
\pgfpathlineto{\pgfqpoint{3.329256in}{5.943624in}}%
\pgfpathlineto{\pgfqpoint{3.329364in}{5.943624in}}%
\pgfpathlineto{\pgfqpoint{3.330612in}{5.950830in}}%
\pgfpathlineto{\pgfqpoint{3.331426in}{5.951631in}}%
\pgfpathlineto{\pgfqpoint{3.332945in}{5.960439in}}%
\pgfpathlineto{\pgfqpoint{3.333108in}{5.961239in}}%
\pgfpathlineto{\pgfqpoint{3.334464in}{5.975652in}}%
\pgfpathlineto{\pgfqpoint{3.334898in}{5.976452in}}%
\pgfpathlineto{\pgfqpoint{3.336309in}{5.988462in}}%
\pgfpathlineto{\pgfqpoint{3.336634in}{5.989263in}}%
\pgfpathlineto{\pgfqpoint{3.337611in}{6.000473in}}%
\pgfpathlineto{\pgfqpoint{3.338425in}{6.001273in}}%
\pgfpathlineto{\pgfqpoint{3.339944in}{6.014084in}}%
\pgfpathlineto{\pgfqpoint{3.340270in}{6.014885in}}%
\pgfpathlineto{\pgfqpoint{3.341680in}{6.025294in}}%
\pgfpathlineto{\pgfqpoint{3.342223in}{6.025294in}}%
\pgfpathlineto{\pgfqpoint{3.343742in}{6.035702in}}%
\pgfpathlineto{\pgfqpoint{3.343850in}{6.035702in}}%
\pgfpathlineto{\pgfqpoint{3.345098in}{6.048513in}}%
\pgfpathlineto{\pgfqpoint{3.345695in}{6.049314in}}%
\pgfpathlineto{\pgfqpoint{3.347214in}{6.060523in}}%
\pgfpathlineto{\pgfqpoint{3.347269in}{6.060523in}}%
\pgfpathlineto{\pgfqpoint{3.348733in}{6.067730in}}%
\pgfpathlineto{\pgfqpoint{3.349059in}{6.068530in}}%
\pgfpathlineto{\pgfqpoint{3.350578in}{6.075736in}}%
\pgfpathlineto{\pgfqpoint{3.350849in}{6.076537in}}%
\pgfpathlineto{\pgfqpoint{3.352260in}{6.082142in}}%
\pgfpathlineto{\pgfqpoint{3.352586in}{6.082942in}}%
\pgfpathlineto{\pgfqpoint{3.354050in}{6.094152in}}%
\pgfpathlineto{\pgfqpoint{3.354485in}{6.094953in}}%
\pgfpathlineto{\pgfqpoint{3.355136in}{6.102959in}}%
\pgfpathlineto{\pgfqpoint{3.355515in}{6.102959in}}%
\pgfpathlineto{\pgfqpoint{3.356166in}{6.102959in}}%
\pgfpathlineto{\pgfqpoint{3.357631in}{6.119774in}}%
\pgfpathlineto{\pgfqpoint{3.358065in}{6.120574in}}%
\pgfpathlineto{\pgfqpoint{3.359585in}{6.130983in}}%
\pgfpathlineto{\pgfqpoint{3.359964in}{6.130983in}}%
\pgfpathlineto{\pgfqpoint{3.361484in}{6.141392in}}%
\pgfpathlineto{\pgfqpoint{3.361538in}{6.141392in}}%
\pgfpathlineto{\pgfqpoint{3.362623in}{6.148598in}}%
\pgfpathlineto{\pgfqpoint{3.363328in}{6.149399in}}%
\pgfpathlineto{\pgfqpoint{3.364847in}{6.158206in}}%
\pgfpathlineto{\pgfqpoint{3.365336in}{6.158206in}}%
\pgfpathlineto{\pgfqpoint{3.366855in}{6.162210in}}%
\pgfpathlineto{\pgfqpoint{3.367072in}{6.163010in}}%
\pgfpathlineto{\pgfqpoint{3.368374in}{6.171017in}}%
\pgfpathlineto{\pgfqpoint{3.368808in}{6.171818in}}%
\pgfpathlineto{\pgfqpoint{3.370327in}{6.179825in}}%
\pgfpathlineto{\pgfqpoint{3.370490in}{6.180625in}}%
\pgfpathlineto{\pgfqpoint{3.371901in}{6.191835in}}%
\pgfpathlineto{\pgfqpoint{3.372497in}{6.192635in}}%
\pgfpathlineto{\pgfqpoint{3.373854in}{6.199842in}}%
\pgfpathlineto{\pgfqpoint{3.374342in}{6.200642in}}%
\pgfpathlineto{\pgfqpoint{3.375319in}{6.204646in}}%
\pgfpathlineto{\pgfqpoint{3.376567in}{6.205446in}}%
\pgfpathlineto{\pgfqpoint{3.377435in}{6.211852in}}%
\pgfpathlineto{\pgfqpoint{3.378737in}{6.211852in}}%
\pgfpathlineto{\pgfqpoint{3.379605in}{6.217457in}}%
\pgfpathlineto{\pgfqpoint{3.381504in}{6.218257in}}%
\pgfpathlineto{\pgfqpoint{3.382697in}{6.220659in}}%
\pgfpathlineto{\pgfqpoint{3.383077in}{6.220659in}}%
\pgfpathlineto{\pgfqpoint{3.384542in}{6.227865in}}%
\pgfpathlineto{\pgfqpoint{3.385030in}{6.228666in}}%
\pgfpathlineto{\pgfqpoint{3.386170in}{6.231068in}}%
\pgfpathlineto{\pgfqpoint{3.387146in}{6.231869in}}%
\pgfpathlineto{\pgfqpoint{3.388666in}{6.236673in}}%
\pgfpathlineto{\pgfqpoint{3.389642in}{6.237473in}}%
\pgfpathlineto{\pgfqpoint{3.390944in}{6.246281in}}%
\pgfpathlineto{\pgfqpoint{3.391758in}{6.247082in}}%
\pgfpathlineto{\pgfqpoint{3.392518in}{6.251085in}}%
\pgfpathlineto{\pgfqpoint{3.393549in}{6.251886in}}%
\pgfpathlineto{\pgfqpoint{3.395068in}{6.258291in}}%
\pgfpathlineto{\pgfqpoint{3.395230in}{6.259092in}}%
\pgfpathlineto{\pgfqpoint{3.395936in}{6.262295in}}%
\pgfpathlineto{\pgfqpoint{3.396967in}{6.263095in}}%
\pgfpathlineto{\pgfqpoint{3.397943in}{6.264697in}}%
\pgfpathlineto{\pgfqpoint{3.398594in}{6.265497in}}%
\pgfpathlineto{\pgfqpoint{3.399951in}{6.269501in}}%
\pgfpathlineto{\pgfqpoint{3.400982in}{6.270301in}}%
\pgfpathlineto{\pgfqpoint{3.402501in}{6.280710in}}%
\pgfpathlineto{\pgfqpoint{3.402989in}{6.280710in}}%
\pgfpathlineto{\pgfqpoint{3.404508in}{6.287116in}}%
\pgfpathlineto{\pgfqpoint{3.404834in}{6.287916in}}%
\pgfpathlineto{\pgfqpoint{3.405810in}{6.293521in}}%
\pgfpathlineto{\pgfqpoint{3.407221in}{6.294322in}}%
\pgfpathlineto{\pgfqpoint{3.408415in}{6.302328in}}%
\pgfpathlineto{\pgfqpoint{3.409662in}{6.303129in}}%
\pgfpathlineto{\pgfqpoint{3.410910in}{6.309535in}}%
\pgfpathlineto{\pgfqpoint{3.412104in}{6.310335in}}%
\pgfpathlineto{\pgfqpoint{3.413298in}{6.315139in}}%
\pgfpathlineto{\pgfqpoint{3.414057in}{6.315940in}}%
\pgfpathlineto{\pgfqpoint{3.415414in}{6.323146in}}%
\pgfpathlineto{\pgfqpoint{3.416065in}{6.323947in}}%
\pgfpathlineto{\pgfqpoint{3.417258in}{6.326349in}}%
\pgfpathlineto{\pgfqpoint{3.418398in}{6.327150in}}%
\pgfpathlineto{\pgfqpoint{3.419537in}{6.330352in}}%
\pgfpathlineto{\pgfqpoint{3.420514in}{6.331153in}}%
\pgfpathlineto{\pgfqpoint{3.421056in}{6.333555in}}%
\pgfpathlineto{\pgfqpoint{3.423389in}{6.334356in}}%
\pgfpathlineto{\pgfqpoint{3.424908in}{6.338359in}}%
\pgfpathlineto{\pgfqpoint{3.425125in}{6.339160in}}%
\pgfpathlineto{\pgfqpoint{3.426536in}{6.340761in}}%
\pgfpathlineto{\pgfqpoint{3.427567in}{6.341562in}}%
\pgfpathlineto{\pgfqpoint{3.428543in}{6.345565in}}%
\pgfpathlineto{\pgfqpoint{3.429194in}{6.345565in}}%
\pgfpathlineto{\pgfqpoint{3.430497in}{6.351170in}}%
\pgfpathlineto{\pgfqpoint{3.431039in}{6.351971in}}%
\pgfpathlineto{\pgfqpoint{3.432504in}{6.359177in}}%
\pgfpathlineto{\pgfqpoint{3.433426in}{6.359977in}}%
\pgfpathlineto{\pgfqpoint{3.434512in}{6.362379in}}%
\pgfpathlineto{\pgfqpoint{3.436193in}{6.363180in}}%
\pgfpathlineto{\pgfqpoint{3.437062in}{6.365582in}}%
\pgfpathlineto{\pgfqpoint{3.438364in}{6.366383in}}%
\pgfpathlineto{\pgfqpoint{3.439883in}{6.368785in}}%
\pgfpathlineto{\pgfqpoint{3.440046in}{6.369585in}}%
\pgfpathlineto{\pgfqpoint{3.441131in}{6.371988in}}%
\pgfpathlineto{\pgfqpoint{3.442379in}{6.372788in}}%
\pgfpathlineto{\pgfqpoint{3.443464in}{6.375991in}}%
\pgfpathlineto{\pgfqpoint{3.444332in}{6.375991in}}%
\pgfpathlineto{\pgfqpoint{3.445363in}{6.380795in}}%
\pgfpathlineto{\pgfqpoint{3.445580in}{6.380795in}}%
\pgfpathlineto{\pgfqpoint{3.447479in}{6.381596in}}%
\pgfpathlineto{\pgfqpoint{3.448889in}{6.385599in}}%
\pgfpathlineto{\pgfqpoint{3.449866in}{6.386400in}}%
\pgfpathlineto{\pgfqpoint{3.451059in}{6.388001in}}%
\pgfpathlineto{\pgfqpoint{3.452524in}{6.388802in}}%
\pgfpathlineto{\pgfqpoint{3.452524in}{6.389602in}}%
\pgfpathlineto{\pgfqpoint{3.455617in}{6.390403in}}%
\pgfpathlineto{\pgfqpoint{3.456702in}{6.392805in}}%
\pgfpathlineto{\pgfqpoint{3.457733in}{6.393606in}}%
\pgfpathlineto{\pgfqpoint{3.458384in}{6.395207in}}%
\pgfpathlineto{\pgfqpoint{3.460120in}{6.396008in}}%
\pgfpathlineto{\pgfqpoint{3.460283in}{6.397609in}}%
\pgfpathlineto{\pgfqpoint{3.462128in}{6.398410in}}%
\pgfpathlineto{\pgfqpoint{3.463321in}{6.404815in}}%
\pgfpathlineto{\pgfqpoint{3.465491in}{6.405616in}}%
\pgfpathlineto{\pgfqpoint{3.466631in}{6.412021in}}%
\pgfpathlineto{\pgfqpoint{3.467173in}{6.412822in}}%
\pgfpathlineto{\pgfqpoint{3.467173in}{6.413623in}}%
\pgfpathlineto{\pgfqpoint{3.469506in}{6.414424in}}%
\pgfpathlineto{\pgfqpoint{3.470537in}{6.418427in}}%
\pgfpathlineto{\pgfqpoint{3.471351in}{6.419228in}}%
\pgfpathlineto{\pgfqpoint{3.471351in}{6.420028in}}%
\pgfpathlineto{\pgfqpoint{3.472979in}{6.420028in}}%
\pgfpathlineto{\pgfqpoint{3.474389in}{6.426434in}}%
\pgfpathlineto{\pgfqpoint{3.474932in}{6.427234in}}%
\pgfpathlineto{\pgfqpoint{3.475637in}{6.429636in}}%
\pgfpathlineto{\pgfqpoint{3.477645in}{6.430437in}}%
\pgfpathlineto{\pgfqpoint{3.479164in}{6.434440in}}%
\pgfpathlineto{\pgfqpoint{3.479598in}{6.435241in}}%
\pgfpathlineto{\pgfqpoint{3.480737in}{6.440045in}}%
\pgfpathlineto{\pgfqpoint{3.481388in}{6.440846in}}%
\pgfpathlineto{\pgfqpoint{3.482908in}{6.448853in}}%
\pgfpathlineto{\pgfqpoint{3.484427in}{6.449653in}}%
\pgfpathlineto{\pgfqpoint{3.485729in}{6.454457in}}%
\pgfpathlineto{\pgfqpoint{3.487519in}{6.455258in}}%
\pgfpathlineto{\pgfqpoint{3.488008in}{6.456859in}}%
\pgfpathlineto{\pgfqpoint{3.490232in}{6.457660in}}%
\pgfpathlineto{\pgfqpoint{3.491426in}{6.460863in}}%
\pgfpathlineto{\pgfqpoint{3.492674in}{6.461664in}}%
\pgfpathlineto{\pgfqpoint{3.492674in}{6.463265in}}%
\pgfpathlineto{\pgfqpoint{3.494681in}{6.464066in}}%
\pgfpathlineto{\pgfqpoint{3.496146in}{6.466468in}}%
\pgfpathlineto{\pgfqpoint{3.497122in}{6.467268in}}%
\pgfpathlineto{\pgfqpoint{3.497122in}{6.468069in}}%
\pgfpathlineto{\pgfqpoint{3.502277in}{6.468870in}}%
\pgfpathlineto{\pgfqpoint{3.503742in}{6.471272in}}%
\pgfpathlineto{\pgfqpoint{3.504284in}{6.472072in}}%
\pgfpathlineto{\pgfqpoint{3.504990in}{6.475275in}}%
\pgfpathlineto{\pgfqpoint{3.507051in}{6.476076in}}%
\pgfpathlineto{\pgfqpoint{3.508408in}{6.478478in}}%
\pgfpathlineto{\pgfqpoint{3.509764in}{6.479278in}}%
\pgfpathlineto{\pgfqpoint{3.511175in}{6.481681in}}%
\pgfpathlineto{\pgfqpoint{3.511717in}{6.481681in}}%
\pgfpathlineto{\pgfqpoint{3.513128in}{6.487285in}}%
\pgfpathlineto{\pgfqpoint{3.513670in}{6.487285in}}%
\pgfpathlineto{\pgfqpoint{3.515135in}{6.492089in}}%
\pgfpathlineto{\pgfqpoint{3.515569in}{6.492089in}}%
\pgfpathlineto{\pgfqpoint{3.515569in}{6.493691in}}%
\pgfpathlineto{\pgfqpoint{3.517523in}{6.494491in}}%
\pgfpathlineto{\pgfqpoint{3.518336in}{6.500897in}}%
\pgfpathlineto{\pgfqpoint{3.519693in}{6.500897in}}%
\pgfpathlineto{\pgfqpoint{3.520507in}{6.506502in}}%
\pgfpathlineto{\pgfqpoint{3.521809in}{6.507302in}}%
\pgfpathlineto{\pgfqpoint{3.521809in}{6.508103in}}%
\pgfpathlineto{\pgfqpoint{3.524901in}{6.508904in}}%
\pgfpathlineto{\pgfqpoint{3.525552in}{6.511306in}}%
\pgfpathlineto{\pgfqpoint{3.528862in}{6.512106in}}%
\pgfpathlineto{\pgfqpoint{3.529350in}{6.513708in}}%
\pgfpathlineto{\pgfqpoint{3.530707in}{6.514508in}}%
\pgfpathlineto{\pgfqpoint{3.531955in}{6.518512in}}%
\pgfpathlineto{\pgfqpoint{3.533474in}{6.519312in}}%
\pgfpathlineto{\pgfqpoint{3.533582in}{6.520914in}}%
\pgfpathlineto{\pgfqpoint{3.535752in}{6.521714in}}%
\pgfpathlineto{\pgfqpoint{3.537272in}{6.526519in}}%
\pgfpathlineto{\pgfqpoint{3.538194in}{6.527319in}}%
\pgfpathlineto{\pgfqpoint{3.539713in}{6.529721in}}%
\pgfpathlineto{\pgfqpoint{3.541721in}{6.530522in}}%
\pgfpathlineto{\pgfqpoint{3.543240in}{6.536127in}}%
\pgfpathlineto{\pgfqpoint{3.544216in}{6.536927in}}%
\pgfpathlineto{\pgfqpoint{3.545247in}{6.540130in}}%
\pgfpathlineto{\pgfqpoint{3.545898in}{6.540130in}}%
\pgfpathlineto{\pgfqpoint{3.547038in}{6.543333in}}%
\pgfpathlineto{\pgfqpoint{3.551812in}{6.544133in}}%
\pgfpathlineto{\pgfqpoint{3.553006in}{6.545735in}}%
\pgfpathlineto{\pgfqpoint{3.555556in}{6.546535in}}%
\pgfpathlineto{\pgfqpoint{3.557075in}{6.548938in}}%
\pgfpathlineto{\pgfqpoint{3.557889in}{6.549738in}}%
\pgfpathlineto{\pgfqpoint{3.559299in}{6.552941in}}%
\pgfpathlineto{\pgfqpoint{3.561307in}{6.553742in}}%
\pgfpathlineto{\pgfqpoint{3.562338in}{6.555343in}}%
\pgfpathlineto{\pgfqpoint{3.563586in}{6.556144in}}%
\pgfpathlineto{\pgfqpoint{3.564671in}{6.557745in}}%
\pgfpathlineto{\pgfqpoint{3.565810in}{6.558546in}}%
\pgfpathlineto{\pgfqpoint{3.567112in}{6.560147in}}%
\pgfpathlineto{\pgfqpoint{3.568306in}{6.560948in}}%
\pgfpathlineto{\pgfqpoint{3.568848in}{6.563350in}}%
\pgfpathlineto{\pgfqpoint{3.570747in}{6.564150in}}%
\pgfpathlineto{\pgfqpoint{3.571181in}{6.565752in}}%
\pgfpathlineto{\pgfqpoint{3.573297in}{6.565752in}}%
\pgfpathlineto{\pgfqpoint{3.573569in}{6.568955in}}%
\pgfpathlineto{\pgfqpoint{3.576064in}{6.569755in}}%
\pgfpathlineto{\pgfqpoint{3.576878in}{6.572157in}}%
\pgfpathlineto{\pgfqpoint{3.579374in}{6.572958in}}%
\pgfpathlineto{\pgfqpoint{3.580405in}{6.574559in}}%
\pgfpathlineto{\pgfqpoint{3.581815in}{6.575360in}}%
\pgfpathlineto{\pgfqpoint{3.582738in}{6.577762in}}%
\pgfpathlineto{\pgfqpoint{3.583823in}{6.577762in}}%
\pgfpathlineto{\pgfqpoint{3.585342in}{6.581765in}}%
\pgfpathlineto{\pgfqpoint{3.587838in}{6.582566in}}%
\pgfpathlineto{\pgfqpoint{3.589140in}{6.587370in}}%
\pgfpathlineto{\pgfqpoint{3.589900in}{6.588171in}}%
\pgfpathlineto{\pgfqpoint{3.590768in}{6.590573in}}%
\pgfpathlineto{\pgfqpoint{3.593101in}{6.591374in}}%
\pgfpathlineto{\pgfqpoint{3.594403in}{6.594576in}}%
\pgfpathlineto{\pgfqpoint{3.598472in}{6.595377in}}%
\pgfpathlineto{\pgfqpoint{3.599991in}{6.598580in}}%
\pgfpathlineto{\pgfqpoint{3.602812in}{6.599380in}}%
\pgfpathlineto{\pgfqpoint{3.604332in}{6.601782in}}%
\pgfpathlineto{\pgfqpoint{3.604820in}{6.602583in}}%
\pgfpathlineto{\pgfqpoint{3.606176in}{6.605786in}}%
\pgfpathlineto{\pgfqpoint{3.607695in}{6.606586in}}%
\pgfpathlineto{\pgfqpoint{3.609215in}{6.610590in}}%
\pgfpathlineto{\pgfqpoint{3.611710in}{6.611390in}}%
\pgfpathlineto{\pgfqpoint{3.613067in}{6.615394in}}%
\pgfpathlineto{\pgfqpoint{3.613501in}{6.616195in}}%
\pgfpathlineto{\pgfqpoint{3.614369in}{6.619397in}}%
\pgfpathlineto{\pgfqpoint{3.615454in}{6.619397in}}%
\pgfpathlineto{\pgfqpoint{3.616376in}{6.621799in}}%
\pgfpathlineto{\pgfqpoint{3.618329in}{6.622600in}}%
\pgfpathlineto{\pgfqpoint{3.618546in}{6.624201in}}%
\pgfpathlineto{\pgfqpoint{3.620391in}{6.625002in}}%
\pgfpathlineto{\pgfqpoint{3.621585in}{6.629005in}}%
\pgfpathlineto{\pgfqpoint{3.622561in}{6.629806in}}%
\pgfpathlineto{\pgfqpoint{3.623918in}{6.633809in}}%
\pgfpathlineto{\pgfqpoint{3.625383in}{6.634610in}}%
\pgfpathlineto{\pgfqpoint{3.625383in}{6.635411in}}%
\pgfpathlineto{\pgfqpoint{3.627661in}{6.636212in}}%
\pgfpathlineto{\pgfqpoint{3.628421in}{6.638614in}}%
\pgfpathlineto{\pgfqpoint{3.630211in}{6.639414in}}%
\pgfpathlineto{\pgfqpoint{3.630537in}{6.641816in}}%
\pgfpathlineto{\pgfqpoint{3.634281in}{6.642617in}}%
\pgfpathlineto{\pgfqpoint{3.635746in}{6.644218in}}%
\pgfpathlineto{\pgfqpoint{3.639815in}{6.645019in}}%
\pgfpathlineto{\pgfqpoint{3.641117in}{6.648222in}}%
\pgfpathlineto{\pgfqpoint{3.643558in}{6.649022in}}%
\pgfpathlineto{\pgfqpoint{3.644860in}{6.652225in}}%
\pgfpathlineto{\pgfqpoint{3.647139in}{6.653026in}}%
\pgfpathlineto{\pgfqpoint{3.648224in}{6.657830in}}%
\pgfpathlineto{\pgfqpoint{3.649201in}{6.658631in}}%
\pgfpathlineto{\pgfqpoint{3.649201in}{6.659431in}}%
\pgfpathlineto{\pgfqpoint{3.651914in}{6.660232in}}%
\pgfpathlineto{\pgfqpoint{3.652402in}{6.661833in}}%
\pgfpathlineto{\pgfqpoint{3.655332in}{6.662634in}}%
\pgfpathlineto{\pgfqpoint{3.656525in}{6.666637in}}%
\pgfpathlineto{\pgfqpoint{3.659672in}{6.667438in}}%
\pgfpathlineto{\pgfqpoint{3.661083in}{6.673843in}}%
\pgfpathlineto{\pgfqpoint{3.663036in}{6.674644in}}%
\pgfpathlineto{\pgfqpoint{3.663958in}{6.677046in}}%
\pgfpathlineto{\pgfqpoint{3.665098in}{6.677847in}}%
\pgfpathlineto{\pgfqpoint{3.666454in}{6.680249in}}%
\pgfpathlineto{\pgfqpoint{3.671012in}{6.681050in}}%
\pgfpathlineto{\pgfqpoint{3.672477in}{6.684252in}}%
\pgfpathlineto{\pgfqpoint{3.673996in}{6.685053in}}%
\pgfpathlineto{\pgfqpoint{3.673996in}{6.685854in}}%
\pgfpathlineto{\pgfqpoint{3.677360in}{6.686654in}}%
\pgfpathlineto{\pgfqpoint{3.677739in}{6.689056in}}%
\pgfpathlineto{\pgfqpoint{3.680181in}{6.689857in}}%
\pgfpathlineto{\pgfqpoint{3.680181in}{6.690658in}}%
\pgfpathlineto{\pgfqpoint{3.685172in}{6.691458in}}%
\pgfpathlineto{\pgfqpoint{3.685986in}{6.693060in}}%
\pgfpathlineto{\pgfqpoint{3.689567in}{6.693860in}}%
\pgfpathlineto{\pgfqpoint{3.690869in}{6.695462in}}%
\pgfpathlineto{\pgfqpoint{3.694016in}{6.696262in}}%
\pgfpathlineto{\pgfqpoint{3.694396in}{6.698664in}}%
\pgfpathlineto{\pgfqpoint{3.695969in}{6.699465in}}%
\pgfpathlineto{\pgfqpoint{3.696078in}{6.701067in}}%
\pgfpathlineto{\pgfqpoint{3.698139in}{6.701867in}}%
\pgfpathlineto{\pgfqpoint{3.698139in}{6.702668in}}%
\pgfpathlineto{\pgfqpoint{3.700093in}{6.702668in}}%
\pgfpathlineto{\pgfqpoint{3.700093in}{6.704269in}}%
\pgfpathlineto{\pgfqpoint{3.703131in}{6.705070in}}%
\pgfpathlineto{\pgfqpoint{3.704216in}{6.707472in}}%
\pgfpathlineto{\pgfqpoint{3.706332in}{6.708273in}}%
\pgfpathlineto{\pgfqpoint{3.706929in}{6.711475in}}%
\pgfpathlineto{\pgfqpoint{3.711269in}{6.712276in}}%
\pgfpathlineto{\pgfqpoint{3.712517in}{6.713877in}}%
\pgfpathlineto{\pgfqpoint{3.713114in}{6.714678in}}%
\pgfpathlineto{\pgfqpoint{3.713711in}{6.716279in}}%
\pgfpathlineto{\pgfqpoint{3.716152in}{6.717080in}}%
\pgfpathlineto{\pgfqpoint{3.716152in}{6.717881in}}%
\pgfpathlineto{\pgfqpoint{3.718811in}{6.718681in}}%
\pgfpathlineto{\pgfqpoint{3.719950in}{6.720283in}}%
\pgfpathlineto{\pgfqpoint{3.723314in}{6.721083in}}%
\pgfpathlineto{\pgfqpoint{3.723314in}{6.721884in}}%
\pgfpathlineto{\pgfqpoint{3.726515in}{6.721884in}}%
\pgfpathlineto{\pgfqpoint{3.726515in}{6.723486in}}%
\pgfpathlineto{\pgfqpoint{3.728902in}{6.724286in}}%
\pgfpathlineto{\pgfqpoint{3.728902in}{6.725087in}}%
\pgfpathlineto{\pgfqpoint{3.733080in}{6.725888in}}%
\pgfpathlineto{\pgfqpoint{3.733460in}{6.727489in}}%
\pgfpathlineto{\pgfqpoint{3.737312in}{6.727489in}}%
\pgfpathlineto{\pgfqpoint{3.737312in}{6.729090in}}%
\pgfpathlineto{\pgfqpoint{3.739482in}{6.729891in}}%
\pgfpathlineto{\pgfqpoint{3.739916in}{6.732293in}}%
\pgfpathlineto{\pgfqpoint{3.742792in}{6.733094in}}%
\pgfpathlineto{\pgfqpoint{3.743389in}{6.734695in}}%
\pgfpathlineto{\pgfqpoint{3.745505in}{6.735496in}}%
\pgfpathlineto{\pgfqpoint{3.746318in}{6.737898in}}%
\pgfpathlineto{\pgfqpoint{3.747783in}{6.738698in}}%
\pgfpathlineto{\pgfqpoint{3.748597in}{6.741100in}}%
\pgfpathlineto{\pgfqpoint{3.749845in}{6.741901in}}%
\pgfpathlineto{\pgfqpoint{3.750116in}{6.743502in}}%
\pgfpathlineto{\pgfqpoint{3.756627in}{6.744303in}}%
\pgfpathlineto{\pgfqpoint{3.756627in}{6.745104in}}%
\pgfpathlineto{\pgfqpoint{3.760099in}{6.745905in}}%
\pgfpathlineto{\pgfqpoint{3.760913in}{6.749107in}}%
\pgfpathlineto{\pgfqpoint{3.768021in}{6.749908in}}%
\pgfpathlineto{\pgfqpoint{3.768021in}{6.750709in}}%
\pgfpathlineto{\pgfqpoint{3.772470in}{6.751509in}}%
\pgfpathlineto{\pgfqpoint{3.772470in}{6.752310in}}%
\pgfpathlineto{\pgfqpoint{3.774314in}{6.753111in}}%
\pgfpathlineto{\pgfqpoint{3.774694in}{6.754712in}}%
\pgfpathlineto{\pgfqpoint{3.777949in}{6.755513in}}%
\pgfpathlineto{\pgfqpoint{3.778709in}{6.758715in}}%
\pgfpathlineto{\pgfqpoint{3.781530in}{6.759516in}}%
\pgfpathlineto{\pgfqpoint{3.782507in}{6.762719in}}%
\pgfpathlineto{\pgfqpoint{3.788963in}{6.763519in}}%
\pgfpathlineto{\pgfqpoint{3.790320in}{6.765921in}}%
\pgfpathlineto{\pgfqpoint{3.793467in}{6.766722in}}%
\pgfpathlineto{\pgfqpoint{3.794063in}{6.768324in}}%
\pgfpathlineto{\pgfqpoint{3.797915in}{6.769124in}}%
\pgfpathlineto{\pgfqpoint{3.798946in}{6.771526in}}%
\pgfpathlineto{\pgfqpoint{3.802147in}{6.772327in}}%
\pgfpathlineto{\pgfqpoint{3.802147in}{6.773128in}}%
\pgfpathlineto{\pgfqpoint{3.809201in}{6.773928in}}%
\pgfpathlineto{\pgfqpoint{3.809797in}{6.776330in}}%
\pgfpathlineto{\pgfqpoint{3.811154in}{6.777131in}}%
\pgfpathlineto{\pgfqpoint{3.811154in}{6.777932in}}%
\pgfpathlineto{\pgfqpoint{3.815494in}{6.778732in}}%
\pgfpathlineto{\pgfqpoint{3.816308in}{6.780334in}}%
\pgfpathlineto{\pgfqpoint{3.819184in}{6.780334in}}%
\pgfpathlineto{\pgfqpoint{3.819618in}{6.783536in}}%
\pgfpathlineto{\pgfqpoint{3.823253in}{6.784337in}}%
\pgfpathlineto{\pgfqpoint{3.824392in}{6.785938in}}%
\pgfpathlineto{\pgfqpoint{3.826237in}{6.786739in}}%
\pgfpathlineto{\pgfqpoint{3.827702in}{6.788340in}}%
\pgfpathlineto{\pgfqpoint{3.830794in}{6.789141in}}%
\pgfpathlineto{\pgfqpoint{3.831879in}{6.790743in}}%
\pgfpathlineto{\pgfqpoint{3.833561in}{6.791543in}}%
\pgfpathlineto{\pgfqpoint{3.834321in}{6.793145in}}%
\pgfpathlineto{\pgfqpoint{3.837685in}{6.793945in}}%
\pgfpathlineto{\pgfqpoint{3.838499in}{6.795547in}}%
\pgfpathlineto{\pgfqpoint{3.840560in}{6.796347in}}%
\pgfpathlineto{\pgfqpoint{3.841428in}{6.797949in}}%
\pgfpathlineto{\pgfqpoint{3.843653in}{6.798749in}}%
\pgfpathlineto{\pgfqpoint{3.844413in}{6.800351in}}%
\pgfpathlineto{\pgfqpoint{3.847017in}{6.801151in}}%
\pgfpathlineto{\pgfqpoint{3.848319in}{6.804354in}}%
\pgfpathlineto{\pgfqpoint{3.850760in}{6.805155in}}%
\pgfpathlineto{\pgfqpoint{3.850760in}{6.805955in}}%
\pgfpathlineto{\pgfqpoint{3.854287in}{6.806756in}}%
\pgfpathlineto{\pgfqpoint{3.854287in}{6.807557in}}%
\pgfpathlineto{\pgfqpoint{3.858790in}{6.807557in}}%
\pgfpathlineto{\pgfqpoint{3.859116in}{6.809959in}}%
\pgfpathlineto{\pgfqpoint{3.865084in}{6.810759in}}%
\pgfpathlineto{\pgfqpoint{3.866440in}{6.812361in}}%
\pgfpathlineto{\pgfqpoint{3.869750in}{6.813162in}}%
\pgfpathlineto{\pgfqpoint{3.869750in}{6.813962in}}%
\pgfpathlineto{\pgfqpoint{3.872300in}{6.814763in}}%
\pgfpathlineto{\pgfqpoint{3.873819in}{6.817165in}}%
\pgfpathlineto{\pgfqpoint{3.874579in}{6.817966in}}%
\pgfpathlineto{\pgfqpoint{3.874579in}{6.818766in}}%
\pgfpathlineto{\pgfqpoint{3.880330in}{6.819567in}}%
\pgfpathlineto{\pgfqpoint{3.881089in}{6.821969in}}%
\pgfpathlineto{\pgfqpoint{3.883748in}{6.822770in}}%
\pgfpathlineto{\pgfqpoint{3.883748in}{6.823570in}}%
\pgfpathlineto{\pgfqpoint{3.891778in}{6.824371in}}%
\pgfpathlineto{\pgfqpoint{3.892971in}{6.827574in}}%
\pgfpathlineto{\pgfqpoint{3.893243in}{6.827574in}}%
\pgfpathlineto{\pgfqpoint{3.896715in}{6.828374in}}%
\pgfpathlineto{\pgfqpoint{3.896715in}{6.829175in}}%
\pgfpathlineto{\pgfqpoint{3.902140in}{6.829976in}}%
\pgfpathlineto{\pgfqpoint{3.903497in}{6.832378in}}%
\pgfpathlineto{\pgfqpoint{3.905667in}{6.833178in}}%
\pgfpathlineto{\pgfqpoint{3.906210in}{6.834780in}}%
\pgfpathlineto{\pgfqpoint{3.910713in}{6.835581in}}%
\pgfpathlineto{\pgfqpoint{3.911147in}{6.838783in}}%
\pgfpathlineto{\pgfqpoint{3.917224in}{6.839584in}}%
\pgfpathlineto{\pgfqpoint{3.917224in}{6.840385in}}%
\pgfpathlineto{\pgfqpoint{3.923680in}{6.841185in}}%
\pgfpathlineto{\pgfqpoint{3.923843in}{6.842787in}}%
\pgfpathlineto{\pgfqpoint{3.925959in}{6.843587in}}%
\pgfpathlineto{\pgfqpoint{3.925959in}{6.844388in}}%
\pgfpathlineto{\pgfqpoint{3.930896in}{6.845189in}}%
\pgfpathlineto{\pgfqpoint{3.932307in}{6.847591in}}%
\pgfpathlineto{\pgfqpoint{3.935074in}{6.848391in}}%
\pgfpathlineto{\pgfqpoint{3.936376in}{6.849993in}}%
\pgfpathlineto{\pgfqpoint{3.939685in}{6.850793in}}%
\pgfpathlineto{\pgfqpoint{3.940608in}{6.853195in}}%
\pgfpathlineto{\pgfqpoint{3.941096in}{6.853195in}}%
\pgfpathlineto{\pgfqpoint{3.944406in}{6.853996in}}%
\pgfpathlineto{\pgfqpoint{3.945653in}{6.855598in}}%
\pgfpathlineto{\pgfqpoint{3.947118in}{6.856398in}}%
\pgfpathlineto{\pgfqpoint{3.948258in}{6.859601in}}%
\pgfpathlineto{\pgfqpoint{3.955040in}{6.860402in}}%
\pgfpathlineto{\pgfqpoint{3.955040in}{6.861202in}}%
\pgfpathlineto{\pgfqpoint{3.960737in}{6.862003in}}%
\pgfpathlineto{\pgfqpoint{3.960737in}{6.862804in}}%
\pgfpathlineto{\pgfqpoint{3.962635in}{6.863604in}}%
\pgfpathlineto{\pgfqpoint{3.963829in}{6.865206in}}%
\pgfpathlineto{\pgfqpoint{3.967410in}{6.866006in}}%
\pgfpathlineto{\pgfqpoint{3.968115in}{6.867608in}}%
\pgfpathlineto{\pgfqpoint{3.971371in}{6.868408in}}%
\pgfpathlineto{\pgfqpoint{3.971371in}{6.869209in}}%
\pgfpathlineto{\pgfqpoint{3.975168in}{6.870010in}}%
\pgfpathlineto{\pgfqpoint{3.975168in}{6.870810in}}%
\pgfpathlineto{\pgfqpoint{3.976850in}{6.870810in}}%
\pgfpathlineto{\pgfqpoint{3.977556in}{6.873212in}}%
\pgfpathlineto{\pgfqpoint{3.979075in}{6.874013in}}%
\pgfpathlineto{\pgfqpoint{3.980214in}{6.875614in}}%
\pgfpathlineto{\pgfqpoint{3.981191in}{6.876415in}}%
\pgfpathlineto{\pgfqpoint{3.981788in}{6.878017in}}%
\pgfpathlineto{\pgfqpoint{4.002459in}{6.878817in}}%
\pgfpathlineto{\pgfqpoint{4.002459in}{6.879618in}}%
\pgfpathlineto{\pgfqpoint{4.010163in}{6.880419in}}%
\pgfpathlineto{\pgfqpoint{4.010489in}{6.882020in}}%
\pgfpathlineto{\pgfqpoint{4.016620in}{6.882821in}}%
\pgfpathlineto{\pgfqpoint{4.016620in}{6.884422in}}%
\pgfpathlineto{\pgfqpoint{4.017000in}{6.884422in}}%
\pgfpathlineto{\pgfqpoint{4.020201in}{6.885223in}}%
\pgfpathlineto{\pgfqpoint{4.021286in}{6.886824in}}%
\pgfpathlineto{\pgfqpoint{4.025789in}{6.887625in}}%
\pgfpathlineto{\pgfqpoint{4.025789in}{6.888425in}}%
\pgfpathlineto{\pgfqpoint{4.031649in}{6.889226in}}%
\pgfpathlineto{\pgfqpoint{4.032137in}{6.890827in}}%
\pgfpathlineto{\pgfqpoint{4.038539in}{6.891628in}}%
\pgfpathlineto{\pgfqpoint{4.039027in}{6.893229in}}%
\pgfpathlineto{\pgfqpoint{4.046623in}{6.894030in}}%
\pgfpathlineto{\pgfqpoint{4.046623in}{6.894831in}}%
\pgfpathlineto{\pgfqpoint{4.056823in}{6.895631in}}%
\pgfpathlineto{\pgfqpoint{4.057094in}{6.897233in}}%
\pgfpathlineto{\pgfqpoint{4.066589in}{6.898033in}}%
\pgfpathlineto{\pgfqpoint{4.067294in}{6.899635in}}%
\pgfpathlineto{\pgfqpoint{4.070496in}{6.900436in}}%
\pgfpathlineto{\pgfqpoint{4.070496in}{6.901236in}}%
\pgfpathlineto{\pgfqpoint{4.073263in}{6.902037in}}%
\pgfpathlineto{\pgfqpoint{4.073263in}{6.902838in}}%
\pgfpathlineto{\pgfqpoint{4.079990in}{6.903638in}}%
\pgfpathlineto{\pgfqpoint{4.079990in}{6.904439in}}%
\pgfpathlineto{\pgfqpoint{4.083517in}{6.905240in}}%
\pgfpathlineto{\pgfqpoint{4.083517in}{6.906040in}}%
\pgfpathlineto{\pgfqpoint{4.089051in}{6.906841in}}%
\pgfpathlineto{\pgfqpoint{4.089051in}{6.907642in}}%
\pgfpathlineto{\pgfqpoint{4.096484in}{6.908442in}}%
\pgfpathlineto{\pgfqpoint{4.097244in}{6.910044in}}%
\pgfpathlineto{\pgfqpoint{4.100228in}{6.910844in}}%
\pgfpathlineto{\pgfqpoint{4.100228in}{6.912446in}}%
\pgfpathlineto{\pgfqpoint{4.103808in}{6.913246in}}%
\pgfpathlineto{\pgfqpoint{4.103808in}{6.914047in}}%
\pgfpathlineto{\pgfqpoint{4.107769in}{6.914848in}}%
\pgfpathlineto{\pgfqpoint{4.109071in}{6.918050in}}%
\pgfpathlineto{\pgfqpoint{4.112923in}{6.918851in}}%
\pgfpathlineto{\pgfqpoint{4.113466in}{6.920452in}}%
\pgfpathlineto{\pgfqpoint{4.113683in}{6.920452in}}%
\pgfpathlineto{\pgfqpoint{4.125402in}{6.921253in}}%
\pgfpathlineto{\pgfqpoint{4.125402in}{6.922054in}}%
\pgfpathlineto{\pgfqpoint{4.128169in}{6.922855in}}%
\pgfpathlineto{\pgfqpoint{4.129254in}{6.924456in}}%
\pgfpathlineto{\pgfqpoint{4.135711in}{6.925257in}}%
\pgfpathlineto{\pgfqpoint{4.135711in}{6.926057in}}%
\pgfpathlineto{\pgfqpoint{4.141353in}{6.926858in}}%
\pgfpathlineto{\pgfqpoint{4.141353in}{6.927659in}}%
\pgfpathlineto{\pgfqpoint{4.158824in}{6.928459in}}%
\pgfpathlineto{\pgfqpoint{4.158824in}{6.929260in}}%
\pgfpathlineto{\pgfqpoint{4.164141in}{6.930061in}}%
\pgfpathlineto{\pgfqpoint{4.164195in}{6.931662in}}%
\pgfpathlineto{\pgfqpoint{4.174721in}{6.932463in}}%
\pgfpathlineto{\pgfqpoint{4.175480in}{6.934064in}}%
\pgfpathlineto{\pgfqpoint{4.179495in}{6.934865in}}%
\pgfpathlineto{\pgfqpoint{4.179495in}{6.935665in}}%
\pgfpathlineto{\pgfqpoint{4.181611in}{6.936466in}}%
\pgfpathlineto{\pgfqpoint{4.182479in}{6.938868in}}%
\pgfpathlineto{\pgfqpoint{4.186440in}{6.939669in}}%
\pgfpathlineto{\pgfqpoint{4.187579in}{6.942071in}}%
\pgfpathlineto{\pgfqpoint{4.191323in}{6.942871in}}%
\pgfpathlineto{\pgfqpoint{4.192082in}{6.945274in}}%
\pgfpathlineto{\pgfqpoint{4.196640in}{6.946074in}}%
\pgfpathlineto{\pgfqpoint{4.196640in}{6.946875in}}%
\pgfpathlineto{\pgfqpoint{4.207437in}{6.947676in}}%
\pgfpathlineto{\pgfqpoint{4.207437in}{6.948476in}}%
\pgfpathlineto{\pgfqpoint{4.214381in}{6.949277in}}%
\pgfpathlineto{\pgfqpoint{4.214381in}{6.950078in}}%
\pgfpathlineto{\pgfqpoint{4.227240in}{6.950878in}}%
\pgfpathlineto{\pgfqpoint{4.227240in}{6.951679in}}%
\pgfpathlineto{\pgfqpoint{4.240804in}{6.952480in}}%
\pgfpathlineto{\pgfqpoint{4.241672in}{6.954882in}}%
\pgfpathlineto{\pgfqpoint{4.254096in}{6.955682in}}%
\pgfpathlineto{\pgfqpoint{4.254096in}{6.956483in}}%
\pgfpathlineto{\pgfqpoint{4.258654in}{6.957284in}}%
\pgfpathlineto{\pgfqpoint{4.259088in}{6.958885in}}%
\pgfpathlineto{\pgfqpoint{4.266521in}{6.959686in}}%
\pgfpathlineto{\pgfqpoint{4.266521in}{6.960486in}}%
\pgfpathlineto{\pgfqpoint{4.269180in}{6.961287in}}%
\pgfpathlineto{\pgfqpoint{4.269180in}{6.962088in}}%
\pgfpathlineto{\pgfqpoint{4.272001in}{6.962888in}}%
\pgfpathlineto{\pgfqpoint{4.272001in}{6.963689in}}%
\pgfpathlineto{\pgfqpoint{4.277643in}{6.964490in}}%
\pgfpathlineto{\pgfqpoint{4.277643in}{6.965290in}}%
\pgfpathlineto{\pgfqpoint{4.289254in}{6.966091in}}%
\pgfpathlineto{\pgfqpoint{4.289254in}{6.966892in}}%
\pgfpathlineto{\pgfqpoint{4.302167in}{6.967693in}}%
\pgfpathlineto{\pgfqpoint{4.302167in}{6.968493in}}%
\pgfpathlineto{\pgfqpoint{4.304988in}{6.969294in}}%
\pgfpathlineto{\pgfqpoint{4.305205in}{6.970895in}}%
\pgfpathlineto{\pgfqpoint{4.317467in}{6.971696in}}%
\pgfpathlineto{\pgfqpoint{4.317684in}{6.973297in}}%
\pgfpathlineto{\pgfqpoint{4.326419in}{6.974098in}}%
\pgfpathlineto{\pgfqpoint{4.326745in}{6.975699in}}%
\pgfpathlineto{\pgfqpoint{4.344107in}{6.976500in}}%
\pgfpathlineto{\pgfqpoint{4.344866in}{6.978101in}}%
\pgfpathlineto{\pgfqpoint{4.358321in}{6.978902in}}%
\pgfpathlineto{\pgfqpoint{4.358321in}{6.979703in}}%
\pgfpathlineto{\pgfqpoint{4.370149in}{6.980503in}}%
\pgfpathlineto{\pgfqpoint{4.370746in}{6.982105in}}%
\pgfpathlineto{\pgfqpoint{4.382682in}{6.982905in}}%
\pgfpathlineto{\pgfqpoint{4.383659in}{6.984507in}}%
\pgfpathlineto{\pgfqpoint{4.401238in}{6.985307in}}%
\pgfpathlineto{\pgfqpoint{4.401238in}{6.986108in}}%
\pgfpathlineto{\pgfqpoint{4.405036in}{6.986909in}}%
\pgfpathlineto{\pgfqpoint{4.405036in}{6.987710in}}%
\pgfpathlineto{\pgfqpoint{4.418816in}{6.988510in}}%
\pgfpathlineto{\pgfqpoint{4.419033in}{6.990112in}}%
\pgfpathlineto{\pgfqpoint{4.434768in}{6.990912in}}%
\pgfpathlineto{\pgfqpoint{4.434768in}{6.991713in}}%
\pgfpathlineto{\pgfqpoint{4.444479in}{6.992514in}}%
\pgfpathlineto{\pgfqpoint{4.445347in}{6.994115in}}%
\pgfpathlineto{\pgfqpoint{4.445781in}{6.994115in}}%
\pgfpathlineto{\pgfqpoint{4.447952in}{6.994916in}}%
\pgfpathlineto{\pgfqpoint{4.447952in}{6.995716in}}%
\pgfpathlineto{\pgfqpoint{4.464120in}{6.996517in}}%
\pgfpathlineto{\pgfqpoint{4.464120in}{6.997318in}}%
\pgfpathlineto{\pgfqpoint{4.469871in}{6.998118in}}%
\pgfpathlineto{\pgfqpoint{4.470793in}{6.999720in}}%
\pgfpathlineto{\pgfqpoint{4.476761in}{7.000520in}}%
\pgfpathlineto{\pgfqpoint{4.477467in}{7.002122in}}%
\pgfpathlineto{\pgfqpoint{4.487884in}{7.002922in}}%
\pgfpathlineto{\pgfqpoint{4.487884in}{7.003723in}}%
\pgfpathlineto{\pgfqpoint{4.492333in}{7.004524in}}%
\pgfpathlineto{\pgfqpoint{4.492333in}{7.005324in}}%
\pgfpathlineto{\pgfqpoint{4.496945in}{7.006125in}}%
\pgfpathlineto{\pgfqpoint{4.496945in}{7.006926in}}%
\pgfpathlineto{\pgfqpoint{4.506385in}{7.007726in}}%
\pgfpathlineto{\pgfqpoint{4.506385in}{7.008527in}}%
\pgfpathlineto{\pgfqpoint{4.520871in}{7.009328in}}%
\pgfpathlineto{\pgfqpoint{4.520871in}{7.010129in}}%
\pgfpathlineto{\pgfqpoint{4.529010in}{7.010929in}}%
\pgfpathlineto{\pgfqpoint{4.529010in}{7.011730in}}%
\pgfpathlineto{\pgfqpoint{4.534761in}{7.012531in}}%
\pgfpathlineto{\pgfqpoint{4.535140in}{7.014132in}}%
\pgfpathlineto{\pgfqpoint{4.539589in}{7.014933in}}%
\pgfpathlineto{\pgfqpoint{4.540783in}{7.017335in}}%
\pgfpathlineto{\pgfqpoint{4.551092in}{7.018135in}}%
\pgfpathlineto{\pgfqpoint{4.551092in}{7.018936in}}%
\pgfpathlineto{\pgfqpoint{4.561943in}{7.019737in}}%
\pgfpathlineto{\pgfqpoint{4.561943in}{7.020537in}}%
\pgfpathlineto{\pgfqpoint{4.569267in}{7.021338in}}%
\pgfpathlineto{\pgfqpoint{4.569267in}{7.022139in}}%
\pgfpathlineto{\pgfqpoint{4.574693in}{7.022939in}}%
\pgfpathlineto{\pgfqpoint{4.574693in}{7.023740in}}%
\pgfpathlineto{\pgfqpoint{4.581637in}{7.024541in}}%
\pgfpathlineto{\pgfqpoint{4.582506in}{7.026142in}}%
\pgfpathlineto{\pgfqpoint{4.586846in}{7.026943in}}%
\pgfpathlineto{\pgfqpoint{4.586846in}{7.027743in}}%
\pgfpathlineto{\pgfqpoint{4.599813in}{7.028544in}}%
\pgfpathlineto{\pgfqpoint{4.599813in}{7.029345in}}%
\pgfpathlineto{\pgfqpoint{4.622546in}{7.030145in}}%
\pgfpathlineto{\pgfqpoint{4.622546in}{7.030946in}}%
\pgfpathlineto{\pgfqpoint{4.623957in}{7.030946in}}%
\pgfpathlineto{\pgfqpoint{4.703333in}{7.031747in}}%
\pgfpathlineto{\pgfqpoint{4.703333in}{7.032548in}}%
\pgfpathlineto{\pgfqpoint{4.737080in}{7.033348in}}%
\pgfpathlineto{\pgfqpoint{4.737080in}{7.034149in}}%
\pgfpathlineto{\pgfqpoint{4.836150in}{7.034149in}}%
\pgfpathlineto{\pgfqpoint{4.836150in}{7.034149in}}%
\pgfusepath{stroke}%
\end{pgfscope}%
\begin{pgfscope}%
\pgfpathrectangle{\pgfqpoint{3.225541in}{5.564583in}}{\pgfqpoint{1.687305in}{1.539545in}}%
\pgfusepath{clip}%
\pgfsetrectcap%
\pgfsetroundjoin%
\pgfsetlinewidth{1.505625pt}%
\definecolor{currentstroke}{rgb}{0.501961,0.501961,0.501961}%
\pgfsetstrokecolor{currentstroke}%
\pgfsetdash{}{0pt}%
\pgfpathmoveto{\pgfqpoint{3.302236in}{5.634562in}}%
\pgfpathlineto{\pgfqpoint{4.836150in}{7.034149in}}%
\pgfusepath{stroke}%
\end{pgfscope}%
\begin{pgfscope}%
\pgfsetrectcap%
\pgfsetmiterjoin%
\pgfsetlinewidth{0.803000pt}%
\definecolor{currentstroke}{rgb}{0.000000,0.000000,0.000000}%
\pgfsetstrokecolor{currentstroke}%
\pgfsetdash{}{0pt}%
\pgfpathmoveto{\pgfqpoint{3.225541in}{5.564583in}}%
\pgfpathlineto{\pgfqpoint{3.225541in}{7.104128in}}%
\pgfusepath{stroke}%
\end{pgfscope}%
\begin{pgfscope}%
\pgfsetrectcap%
\pgfsetmiterjoin%
\pgfsetlinewidth{0.803000pt}%
\definecolor{currentstroke}{rgb}{0.000000,0.000000,0.000000}%
\pgfsetstrokecolor{currentstroke}%
\pgfsetdash{}{0pt}%
\pgfpathmoveto{\pgfqpoint{4.912846in}{5.564583in}}%
\pgfpathlineto{\pgfqpoint{4.912846in}{7.104128in}}%
\pgfusepath{stroke}%
\end{pgfscope}%
\begin{pgfscope}%
\pgfsetrectcap%
\pgfsetmiterjoin%
\pgfsetlinewidth{0.803000pt}%
\definecolor{currentstroke}{rgb}{0.000000,0.000000,0.000000}%
\pgfsetstrokecolor{currentstroke}%
\pgfsetdash{}{0pt}%
\pgfpathmoveto{\pgfqpoint{3.225541in}{5.564583in}}%
\pgfpathlineto{\pgfqpoint{4.912846in}{5.564583in}}%
\pgfusepath{stroke}%
\end{pgfscope}%
\begin{pgfscope}%
\pgfsetrectcap%
\pgfsetmiterjoin%
\pgfsetlinewidth{0.803000pt}%
\definecolor{currentstroke}{rgb}{0.000000,0.000000,0.000000}%
\pgfsetstrokecolor{currentstroke}%
\pgfsetdash{}{0pt}%
\pgfpathmoveto{\pgfqpoint{3.225541in}{7.104128in}}%
\pgfpathlineto{\pgfqpoint{4.912846in}{7.104128in}}%
\pgfusepath{stroke}%
\end{pgfscope}%
\begin{pgfscope}%
\definecolor{textcolor}{rgb}{0.000000,0.000000,0.000000}%
\pgfsetstrokecolor{textcolor}%
\pgfsetfillcolor{textcolor}%
\pgftext[x=4.069193in,y=7.187462in,,base]{\color{textcolor}\rmfamily\fontsize{20.000000}{24.000000}\selectfont Mass}%
\end{pgfscope}%
\begin{pgfscope}%
\pgfsetbuttcap%
\pgfsetmiterjoin%
\definecolor{currentfill}{rgb}{1.000000,1.000000,1.000000}%
\pgfsetfillcolor{currentfill}%
\pgfsetfillopacity{0.800000}%
\pgfsetlinewidth{1.003750pt}%
\definecolor{currentstroke}{rgb}{0.800000,0.800000,0.800000}%
\pgfsetstrokecolor{currentstroke}%
\pgfsetstrokeopacity{0.800000}%
\pgfsetdash{}{0pt}%
\pgfpathmoveto{\pgfqpoint{3.703740in}{5.634027in}}%
\pgfpathlineto{\pgfqpoint{4.815624in}{5.634027in}}%
\pgfpathquadraticcurveto{\pgfqpoint{4.843402in}{5.634027in}}{\pgfqpoint{4.843402in}{5.661805in}}%
\pgfpathlineto{\pgfqpoint{4.843402in}{5.841589in}}%
\pgfpathquadraticcurveto{\pgfqpoint{4.843402in}{5.869367in}}{\pgfqpoint{4.815624in}{5.869367in}}%
\pgfpathlineto{\pgfqpoint{3.703740in}{5.869367in}}%
\pgfpathquadraticcurveto{\pgfqpoint{3.675962in}{5.869367in}}{\pgfqpoint{3.675962in}{5.841589in}}%
\pgfpathlineto{\pgfqpoint{3.675962in}{5.661805in}}%
\pgfpathquadraticcurveto{\pgfqpoint{3.675962in}{5.634027in}}{\pgfqpoint{3.703740in}{5.634027in}}%
\pgfpathclose%
\pgfusepath{stroke,fill}%
\end{pgfscope}%
\begin{pgfscope}%
\pgfsetrectcap%
\pgfsetroundjoin%
\pgfsetlinewidth{1.505625pt}%
\definecolor{currentstroke}{rgb}{0.000000,0.501961,0.000000}%
\pgfsetstrokecolor{currentstroke}%
\pgfsetdash{}{0pt}%
\pgfpathmoveto{\pgfqpoint{3.731518in}{5.765200in}}%
\pgfpathlineto{\pgfqpoint{4.009295in}{5.765200in}}%
\pgfusepath{stroke}%
\end{pgfscope}%
\begin{pgfscope}%
\definecolor{textcolor}{rgb}{0.000000,0.000000,0.000000}%
\pgfsetstrokecolor{textcolor}%
\pgfsetfillcolor{textcolor}%
\pgftext[x=4.120407in,y=5.716589in,left,base]{\color{textcolor}\rmfamily\fontsize{10.000000}{12.000000}\selectfont AUC 0.831}%
\end{pgfscope}%
\begin{pgfscope}%
\pgfsetbuttcap%
\pgfsetmiterjoin%
\definecolor{currentfill}{rgb}{1.000000,1.000000,1.000000}%
\pgfsetfillcolor{currentfill}%
\pgfsetlinewidth{0.000000pt}%
\definecolor{currentstroke}{rgb}{0.000000,0.000000,0.000000}%
\pgfsetstrokecolor{currentstroke}%
\pgfsetstrokeopacity{0.000000}%
\pgfsetdash{}{0pt}%
\pgfpathmoveto{\pgfqpoint{5.688041in}{5.564583in}}%
\pgfpathlineto{\pgfqpoint{7.375346in}{5.564583in}}%
\pgfpathlineto{\pgfqpoint{7.375346in}{7.104128in}}%
\pgfpathlineto{\pgfqpoint{5.688041in}{7.104128in}}%
\pgfpathclose%
\pgfusepath{fill}%
\end{pgfscope}%
\begin{pgfscope}%
\pgfsetbuttcap%
\pgfsetroundjoin%
\definecolor{currentfill}{rgb}{0.000000,0.000000,0.000000}%
\pgfsetfillcolor{currentfill}%
\pgfsetlinewidth{0.803000pt}%
\definecolor{currentstroke}{rgb}{0.000000,0.000000,0.000000}%
\pgfsetstrokecolor{currentstroke}%
\pgfsetdash{}{0pt}%
\pgfsys@defobject{currentmarker}{\pgfqpoint{0.000000in}{-0.048611in}}{\pgfqpoint{0.000000in}{0.000000in}}{%
\pgfpathmoveto{\pgfqpoint{0.000000in}{0.000000in}}%
\pgfpathlineto{\pgfqpoint{0.000000in}{-0.048611in}}%
\pgfusepath{stroke,fill}%
}%
\begin{pgfscope}%
\pgfsys@transformshift{5.764736in}{5.564583in}%
\pgfsys@useobject{currentmarker}{}%
\end{pgfscope}%
\end{pgfscope}%
\begin{pgfscope}%
\definecolor{textcolor}{rgb}{0.000000,0.000000,0.000000}%
\pgfsetstrokecolor{textcolor}%
\pgfsetfillcolor{textcolor}%
\pgftext[x=5.764736in,y=5.467361in,,top]{\color{textcolor}\rmfamily\fontsize{10.000000}{12.000000}\selectfont \(\displaystyle {0.0}\)}%
\end{pgfscope}%
\begin{pgfscope}%
\pgfsetbuttcap%
\pgfsetroundjoin%
\definecolor{currentfill}{rgb}{0.000000,0.000000,0.000000}%
\pgfsetfillcolor{currentfill}%
\pgfsetlinewidth{0.803000pt}%
\definecolor{currentstroke}{rgb}{0.000000,0.000000,0.000000}%
\pgfsetstrokecolor{currentstroke}%
\pgfsetdash{}{0pt}%
\pgfsys@defobject{currentmarker}{\pgfqpoint{0.000000in}{-0.048611in}}{\pgfqpoint{0.000000in}{0.000000in}}{%
\pgfpathmoveto{\pgfqpoint{0.000000in}{0.000000in}}%
\pgfpathlineto{\pgfqpoint{0.000000in}{-0.048611in}}%
\pgfusepath{stroke,fill}%
}%
\begin{pgfscope}%
\pgfsys@transformshift{6.531693in}{5.564583in}%
\pgfsys@useobject{currentmarker}{}%
\end{pgfscope}%
\end{pgfscope}%
\begin{pgfscope}%
\definecolor{textcolor}{rgb}{0.000000,0.000000,0.000000}%
\pgfsetstrokecolor{textcolor}%
\pgfsetfillcolor{textcolor}%
\pgftext[x=6.531693in,y=5.467361in,,top]{\color{textcolor}\rmfamily\fontsize{10.000000}{12.000000}\selectfont \(\displaystyle {0.5}\)}%
\end{pgfscope}%
\begin{pgfscope}%
\pgfsetbuttcap%
\pgfsetroundjoin%
\definecolor{currentfill}{rgb}{0.000000,0.000000,0.000000}%
\pgfsetfillcolor{currentfill}%
\pgfsetlinewidth{0.803000pt}%
\definecolor{currentstroke}{rgb}{0.000000,0.000000,0.000000}%
\pgfsetstrokecolor{currentstroke}%
\pgfsetdash{}{0pt}%
\pgfsys@defobject{currentmarker}{\pgfqpoint{0.000000in}{-0.048611in}}{\pgfqpoint{0.000000in}{0.000000in}}{%
\pgfpathmoveto{\pgfqpoint{0.000000in}{0.000000in}}%
\pgfpathlineto{\pgfqpoint{0.000000in}{-0.048611in}}%
\pgfusepath{stroke,fill}%
}%
\begin{pgfscope}%
\pgfsys@transformshift{7.298650in}{5.564583in}%
\pgfsys@useobject{currentmarker}{}%
\end{pgfscope}%
\end{pgfscope}%
\begin{pgfscope}%
\definecolor{textcolor}{rgb}{0.000000,0.000000,0.000000}%
\pgfsetstrokecolor{textcolor}%
\pgfsetfillcolor{textcolor}%
\pgftext[x=7.298650in,y=5.467361in,,top]{\color{textcolor}\rmfamily\fontsize{10.000000}{12.000000}\selectfont \(\displaystyle {1.0}\)}%
\end{pgfscope}%
\begin{pgfscope}%
\definecolor{textcolor}{rgb}{0.000000,0.000000,0.000000}%
\pgfsetstrokecolor{textcolor}%
\pgfsetfillcolor{textcolor}%
\pgftext[x=6.531693in,y=5.288349in,,top]{\color{textcolor}\rmfamily\fontsize{16.000000}{19.200000}\selectfont FPR}%
\end{pgfscope}%
\begin{pgfscope}%
\pgfsetbuttcap%
\pgfsetroundjoin%
\definecolor{currentfill}{rgb}{0.000000,0.000000,0.000000}%
\pgfsetfillcolor{currentfill}%
\pgfsetlinewidth{0.803000pt}%
\definecolor{currentstroke}{rgb}{0.000000,0.000000,0.000000}%
\pgfsetstrokecolor{currentstroke}%
\pgfsetdash{}{0pt}%
\pgfsys@defobject{currentmarker}{\pgfqpoint{-0.048611in}{0.000000in}}{\pgfqpoint{-0.000000in}{0.000000in}}{%
\pgfpathmoveto{\pgfqpoint{-0.000000in}{0.000000in}}%
\pgfpathlineto{\pgfqpoint{-0.048611in}{0.000000in}}%
\pgfusepath{stroke,fill}%
}%
\begin{pgfscope}%
\pgfsys@transformshift{5.688041in}{5.634562in}%
\pgfsys@useobject{currentmarker}{}%
\end{pgfscope}%
\end{pgfscope}%
\begin{pgfscope}%
\definecolor{textcolor}{rgb}{0.000000,0.000000,0.000000}%
\pgfsetstrokecolor{textcolor}%
\pgfsetfillcolor{textcolor}%
\pgftext[x=5.343904in, y=5.586337in, left, base]{\color{textcolor}\rmfamily\fontsize{10.000000}{12.000000}\selectfont \(\displaystyle {0.00}\)}%
\end{pgfscope}%
\begin{pgfscope}%
\pgfsetbuttcap%
\pgfsetroundjoin%
\definecolor{currentfill}{rgb}{0.000000,0.000000,0.000000}%
\pgfsetfillcolor{currentfill}%
\pgfsetlinewidth{0.803000pt}%
\definecolor{currentstroke}{rgb}{0.000000,0.000000,0.000000}%
\pgfsetstrokecolor{currentstroke}%
\pgfsetdash{}{0pt}%
\pgfsys@defobject{currentmarker}{\pgfqpoint{-0.048611in}{0.000000in}}{\pgfqpoint{-0.000000in}{0.000000in}}{%
\pgfpathmoveto{\pgfqpoint{-0.000000in}{0.000000in}}%
\pgfpathlineto{\pgfqpoint{-0.048611in}{0.000000in}}%
\pgfusepath{stroke,fill}%
}%
\begin{pgfscope}%
\pgfsys@transformshift{5.688041in}{5.984459in}%
\pgfsys@useobject{currentmarker}{}%
\end{pgfscope}%
\end{pgfscope}%
\begin{pgfscope}%
\definecolor{textcolor}{rgb}{0.000000,0.000000,0.000000}%
\pgfsetstrokecolor{textcolor}%
\pgfsetfillcolor{textcolor}%
\pgftext[x=5.343904in, y=5.936234in, left, base]{\color{textcolor}\rmfamily\fontsize{10.000000}{12.000000}\selectfont \(\displaystyle {0.25}\)}%
\end{pgfscope}%
\begin{pgfscope}%
\pgfsetbuttcap%
\pgfsetroundjoin%
\definecolor{currentfill}{rgb}{0.000000,0.000000,0.000000}%
\pgfsetfillcolor{currentfill}%
\pgfsetlinewidth{0.803000pt}%
\definecolor{currentstroke}{rgb}{0.000000,0.000000,0.000000}%
\pgfsetstrokecolor{currentstroke}%
\pgfsetdash{}{0pt}%
\pgfsys@defobject{currentmarker}{\pgfqpoint{-0.048611in}{0.000000in}}{\pgfqpoint{-0.000000in}{0.000000in}}{%
\pgfpathmoveto{\pgfqpoint{-0.000000in}{0.000000in}}%
\pgfpathlineto{\pgfqpoint{-0.048611in}{0.000000in}}%
\pgfusepath{stroke,fill}%
}%
\begin{pgfscope}%
\pgfsys@transformshift{5.688041in}{6.334356in}%
\pgfsys@useobject{currentmarker}{}%
\end{pgfscope}%
\end{pgfscope}%
\begin{pgfscope}%
\definecolor{textcolor}{rgb}{0.000000,0.000000,0.000000}%
\pgfsetstrokecolor{textcolor}%
\pgfsetfillcolor{textcolor}%
\pgftext[x=5.343904in, y=6.286130in, left, base]{\color{textcolor}\rmfamily\fontsize{10.000000}{12.000000}\selectfont \(\displaystyle {0.50}\)}%
\end{pgfscope}%
\begin{pgfscope}%
\pgfsetbuttcap%
\pgfsetroundjoin%
\definecolor{currentfill}{rgb}{0.000000,0.000000,0.000000}%
\pgfsetfillcolor{currentfill}%
\pgfsetlinewidth{0.803000pt}%
\definecolor{currentstroke}{rgb}{0.000000,0.000000,0.000000}%
\pgfsetstrokecolor{currentstroke}%
\pgfsetdash{}{0pt}%
\pgfsys@defobject{currentmarker}{\pgfqpoint{-0.048611in}{0.000000in}}{\pgfqpoint{-0.000000in}{0.000000in}}{%
\pgfpathmoveto{\pgfqpoint{-0.000000in}{0.000000in}}%
\pgfpathlineto{\pgfqpoint{-0.048611in}{0.000000in}}%
\pgfusepath{stroke,fill}%
}%
\begin{pgfscope}%
\pgfsys@transformshift{5.688041in}{6.684252in}%
\pgfsys@useobject{currentmarker}{}%
\end{pgfscope}%
\end{pgfscope}%
\begin{pgfscope}%
\definecolor{textcolor}{rgb}{0.000000,0.000000,0.000000}%
\pgfsetstrokecolor{textcolor}%
\pgfsetfillcolor{textcolor}%
\pgftext[x=5.343904in, y=6.636027in, left, base]{\color{textcolor}\rmfamily\fontsize{10.000000}{12.000000}\selectfont \(\displaystyle {0.75}\)}%
\end{pgfscope}%
\begin{pgfscope}%
\pgfsetbuttcap%
\pgfsetroundjoin%
\definecolor{currentfill}{rgb}{0.000000,0.000000,0.000000}%
\pgfsetfillcolor{currentfill}%
\pgfsetlinewidth{0.803000pt}%
\definecolor{currentstroke}{rgb}{0.000000,0.000000,0.000000}%
\pgfsetstrokecolor{currentstroke}%
\pgfsetdash{}{0pt}%
\pgfsys@defobject{currentmarker}{\pgfqpoint{-0.048611in}{0.000000in}}{\pgfqpoint{-0.000000in}{0.000000in}}{%
\pgfpathmoveto{\pgfqpoint{-0.000000in}{0.000000in}}%
\pgfpathlineto{\pgfqpoint{-0.048611in}{0.000000in}}%
\pgfusepath{stroke,fill}%
}%
\begin{pgfscope}%
\pgfsys@transformshift{5.688041in}{7.034149in}%
\pgfsys@useobject{currentmarker}{}%
\end{pgfscope}%
\end{pgfscope}%
\begin{pgfscope}%
\definecolor{textcolor}{rgb}{0.000000,0.000000,0.000000}%
\pgfsetstrokecolor{textcolor}%
\pgfsetfillcolor{textcolor}%
\pgftext[x=5.343904in, y=6.985924in, left, base]{\color{textcolor}\rmfamily\fontsize{10.000000}{12.000000}\selectfont \(\displaystyle {1.00}\)}%
\end{pgfscope}%
\begin{pgfscope}%
\definecolor{textcolor}{rgb}{0.000000,0.000000,0.000000}%
\pgfsetstrokecolor{textcolor}%
\pgfsetfillcolor{textcolor}%
\pgftext[x=5.288349in,y=6.334356in,,bottom,rotate=90.000000]{\color{textcolor}\rmfamily\fontsize{16.000000}{19.200000}\selectfont TPR}%
\end{pgfscope}%
\begin{pgfscope}%
\pgfpathrectangle{\pgfqpoint{5.688041in}{5.564583in}}{\pgfqpoint{1.687305in}{1.539545in}}%
\pgfusepath{clip}%
\pgfsetrectcap%
\pgfsetroundjoin%
\pgfsetlinewidth{1.505625pt}%
\definecolor{currentstroke}{rgb}{0.000000,0.501961,0.000000}%
\pgfsetstrokecolor{currentstroke}%
\pgfsetdash{}{0pt}%
\pgfpathmoveto{\pgfqpoint{5.764736in}{5.634562in}}%
\pgfpathlineto{\pgfqpoint{5.768247in}{5.703550in}}%
\pgfpathlineto{\pgfqpoint{5.768572in}{5.703550in}}%
\pgfpathlineto{\pgfqpoint{5.769976in}{5.718210in}}%
\pgfpathlineto{\pgfqpoint{5.770408in}{5.719072in}}%
\pgfpathlineto{\pgfqpoint{5.773919in}{5.752704in}}%
\pgfpathlineto{\pgfqpoint{5.774189in}{5.753566in}}%
\pgfpathlineto{\pgfqpoint{5.775540in}{5.766501in}}%
\pgfpathlineto{\pgfqpoint{5.775702in}{5.766501in}}%
\pgfpathlineto{\pgfqpoint{5.775864in}{5.767364in}}%
\pgfpathlineto{\pgfqpoint{5.777322in}{5.780299in}}%
\pgfpathlineto{\pgfqpoint{5.777538in}{5.781161in}}%
\pgfpathlineto{\pgfqpoint{5.778727in}{5.796683in}}%
\pgfpathlineto{\pgfqpoint{5.779051in}{5.796683in}}%
\pgfpathlineto{\pgfqpoint{5.779159in}{5.796683in}}%
\pgfpathlineto{\pgfqpoint{5.780563in}{5.814793in}}%
\pgfpathlineto{\pgfqpoint{5.780833in}{5.815655in}}%
\pgfpathlineto{\pgfqpoint{5.782130in}{5.821691in}}%
\pgfpathlineto{\pgfqpoint{5.782454in}{5.821691in}}%
\pgfpathlineto{\pgfqpoint{5.783642in}{5.838938in}}%
\pgfpathlineto{\pgfqpoint{5.784507in}{5.839801in}}%
\pgfpathlineto{\pgfqpoint{5.785965in}{5.856185in}}%
\pgfpathlineto{\pgfqpoint{5.786937in}{5.857047in}}%
\pgfpathlineto{\pgfqpoint{5.788234in}{5.864809in}}%
\pgfpathlineto{\pgfqpoint{5.788558in}{5.864809in}}%
\pgfpathlineto{\pgfqpoint{5.789962in}{5.876881in}}%
\pgfpathlineto{\pgfqpoint{5.790232in}{5.876881in}}%
\pgfpathlineto{\pgfqpoint{5.791745in}{5.886367in}}%
\pgfpathlineto{\pgfqpoint{5.791853in}{5.887230in}}%
\pgfpathlineto{\pgfqpoint{5.793257in}{5.901889in}}%
\pgfpathlineto{\pgfqpoint{5.793797in}{5.902752in}}%
\pgfpathlineto{\pgfqpoint{5.795148in}{5.913962in}}%
\pgfpathlineto{\pgfqpoint{5.795634in}{5.914825in}}%
\pgfpathlineto{\pgfqpoint{5.796714in}{5.927760in}}%
\pgfpathlineto{\pgfqpoint{5.797633in}{5.928622in}}%
\pgfpathlineto{\pgfqpoint{5.798767in}{5.938108in}}%
\pgfpathlineto{\pgfqpoint{5.799469in}{5.938970in}}%
\pgfpathlineto{\pgfqpoint{5.800820in}{5.946731in}}%
\pgfpathlineto{\pgfqpoint{5.801198in}{5.947594in}}%
\pgfpathlineto{\pgfqpoint{5.802548in}{5.953630in}}%
\pgfpathlineto{\pgfqpoint{5.803088in}{5.954493in}}%
\pgfpathlineto{\pgfqpoint{5.804439in}{5.964841in}}%
\pgfpathlineto{\pgfqpoint{5.804655in}{5.964841in}}%
\pgfpathlineto{\pgfqpoint{5.806167in}{5.977776in}}%
\pgfpathlineto{\pgfqpoint{5.806869in}{5.978638in}}%
\pgfpathlineto{\pgfqpoint{5.808328in}{5.992436in}}%
\pgfpathlineto{\pgfqpoint{5.809084in}{5.993298in}}%
\pgfpathlineto{\pgfqpoint{5.809624in}{5.997610in}}%
\pgfpathlineto{\pgfqpoint{5.811083in}{5.998472in}}%
\pgfpathlineto{\pgfqpoint{5.812541in}{6.005371in}}%
\pgfpathlineto{\pgfqpoint{5.812973in}{6.006233in}}%
\pgfpathlineto{\pgfqpoint{5.814486in}{6.013994in}}%
\pgfpathlineto{\pgfqpoint{5.814540in}{6.013994in}}%
\pgfpathlineto{\pgfqpoint{5.815782in}{6.026067in}}%
\pgfpathlineto{\pgfqpoint{5.816809in}{6.026930in}}%
\pgfpathlineto{\pgfqpoint{5.817403in}{6.031241in}}%
\pgfpathlineto{\pgfqpoint{5.818915in}{6.032104in}}%
\pgfpathlineto{\pgfqpoint{5.820428in}{6.039865in}}%
\pgfpathlineto{\pgfqpoint{5.820860in}{6.040727in}}%
\pgfpathlineto{\pgfqpoint{5.822372in}{6.050213in}}%
\pgfpathlineto{\pgfqpoint{5.822534in}{6.051075in}}%
\pgfpathlineto{\pgfqpoint{5.823831in}{6.055387in}}%
\pgfpathlineto{\pgfqpoint{5.824263in}{6.056249in}}%
\pgfpathlineto{\pgfqpoint{5.825559in}{6.062286in}}%
\pgfpathlineto{\pgfqpoint{5.826207in}{6.063148in}}%
\pgfpathlineto{\pgfqpoint{5.827720in}{6.072634in}}%
\pgfpathlineto{\pgfqpoint{5.828152in}{6.073496in}}%
\pgfpathlineto{\pgfqpoint{5.829665in}{6.084707in}}%
\pgfpathlineto{\pgfqpoint{5.829989in}{6.085569in}}%
\pgfpathlineto{\pgfqpoint{5.831447in}{6.090743in}}%
\pgfpathlineto{\pgfqpoint{5.831771in}{6.090743in}}%
\pgfpathlineto{\pgfqpoint{5.833230in}{6.098504in}}%
\pgfpathlineto{\pgfqpoint{5.834040in}{6.099367in}}%
\pgfpathlineto{\pgfqpoint{5.835066in}{6.103678in}}%
\pgfpathlineto{\pgfqpoint{5.836417in}{6.104541in}}%
\pgfpathlineto{\pgfqpoint{5.837443in}{6.109715in}}%
\pgfpathlineto{\pgfqpoint{5.838199in}{6.110577in}}%
\pgfpathlineto{\pgfqpoint{5.839388in}{6.113164in}}%
\pgfpathlineto{\pgfqpoint{5.840522in}{6.114026in}}%
\pgfpathlineto{\pgfqpoint{5.841764in}{6.119200in}}%
\pgfpathlineto{\pgfqpoint{5.842521in}{6.120063in}}%
\pgfpathlineto{\pgfqpoint{5.843871in}{6.127824in}}%
\pgfpathlineto{\pgfqpoint{5.844681in}{6.127824in}}%
\pgfpathlineto{\pgfqpoint{5.846194in}{6.137310in}}%
\pgfpathlineto{\pgfqpoint{5.846950in}{6.138172in}}%
\pgfpathlineto{\pgfqpoint{5.847976in}{6.143346in}}%
\pgfpathlineto{\pgfqpoint{5.848516in}{6.143346in}}%
\pgfpathlineto{\pgfqpoint{5.849543in}{6.147658in}}%
\pgfpathlineto{\pgfqpoint{5.850245in}{6.148520in}}%
\pgfpathlineto{\pgfqpoint{5.851595in}{6.152832in}}%
\pgfpathlineto{\pgfqpoint{5.852244in}{6.153694in}}%
\pgfpathlineto{\pgfqpoint{5.853756in}{6.162318in}}%
\pgfpathlineto{\pgfqpoint{5.854188in}{6.163180in}}%
\pgfpathlineto{\pgfqpoint{5.855485in}{6.169216in}}%
\pgfpathlineto{\pgfqpoint{5.856511in}{6.170079in}}%
\pgfpathlineto{\pgfqpoint{5.857861in}{6.175253in}}%
\pgfpathlineto{\pgfqpoint{5.858347in}{6.175253in}}%
\pgfpathlineto{\pgfqpoint{5.859536in}{6.180427in}}%
\pgfpathlineto{\pgfqpoint{5.860454in}{6.181289in}}%
\pgfpathlineto{\pgfqpoint{5.861480in}{6.187326in}}%
\pgfpathlineto{\pgfqpoint{5.863317in}{6.188188in}}%
\pgfpathlineto{\pgfqpoint{5.864073in}{6.191637in}}%
\pgfpathlineto{\pgfqpoint{5.866450in}{6.192500in}}%
\pgfpathlineto{\pgfqpoint{5.867638in}{6.195949in}}%
\pgfpathlineto{\pgfqpoint{5.868287in}{6.196812in}}%
\pgfpathlineto{\pgfqpoint{5.869043in}{6.200261in}}%
\pgfpathlineto{\pgfqpoint{5.869961in}{6.201123in}}%
\pgfpathlineto{\pgfqpoint{5.871257in}{6.204573in}}%
\pgfpathlineto{\pgfqpoint{5.872014in}{6.204573in}}%
\pgfpathlineto{\pgfqpoint{5.873364in}{6.211471in}}%
\pgfpathlineto{\pgfqpoint{5.873688in}{6.212334in}}%
\pgfpathlineto{\pgfqpoint{5.875201in}{6.217508in}}%
\pgfpathlineto{\pgfqpoint{5.875795in}{6.218370in}}%
\pgfpathlineto{\pgfqpoint{5.876713in}{6.222682in}}%
\pgfpathlineto{\pgfqpoint{5.877523in}{6.223544in}}%
\pgfpathlineto{\pgfqpoint{5.878064in}{6.227856in}}%
\pgfpathlineto{\pgfqpoint{5.880386in}{6.228718in}}%
\pgfpathlineto{\pgfqpoint{5.881737in}{6.239066in}}%
\pgfpathlineto{\pgfqpoint{5.882655in}{6.239929in}}%
\pgfpathlineto{\pgfqpoint{5.883843in}{6.242516in}}%
\pgfpathlineto{\pgfqpoint{5.884870in}{6.243378in}}%
\pgfpathlineto{\pgfqpoint{5.885680in}{6.247690in}}%
\pgfpathlineto{\pgfqpoint{5.887517in}{6.248552in}}%
\pgfpathlineto{\pgfqpoint{5.888975in}{6.255451in}}%
\pgfpathlineto{\pgfqpoint{5.889461in}{6.256313in}}%
\pgfpathlineto{\pgfqpoint{5.890758in}{6.259763in}}%
\pgfpathlineto{\pgfqpoint{5.892162in}{6.260625in}}%
\pgfpathlineto{\pgfqpoint{5.893674in}{6.268386in}}%
\pgfpathlineto{\pgfqpoint{5.895619in}{6.269249in}}%
\pgfpathlineto{\pgfqpoint{5.896807in}{6.270973in}}%
\pgfpathlineto{\pgfqpoint{5.897942in}{6.271836in}}%
\pgfpathlineto{\pgfqpoint{5.899022in}{6.275285in}}%
\pgfpathlineto{\pgfqpoint{5.899778in}{6.276147in}}%
\pgfpathlineto{\pgfqpoint{5.900967in}{6.281321in}}%
\pgfpathlineto{\pgfqpoint{5.901399in}{6.281321in}}%
\pgfpathlineto{\pgfqpoint{5.902533in}{6.289945in}}%
\pgfpathlineto{\pgfqpoint{5.903181in}{6.290807in}}%
\pgfpathlineto{\pgfqpoint{5.904640in}{6.297706in}}%
\pgfpathlineto{\pgfqpoint{5.906800in}{6.298568in}}%
\pgfpathlineto{\pgfqpoint{5.907719in}{6.301155in}}%
\pgfpathlineto{\pgfqpoint{5.908961in}{6.302018in}}%
\pgfpathlineto{\pgfqpoint{5.910366in}{6.306329in}}%
\pgfpathlineto{\pgfqpoint{5.911122in}{6.307192in}}%
\pgfpathlineto{\pgfqpoint{5.911338in}{6.309779in}}%
\pgfpathlineto{\pgfqpoint{5.914741in}{6.310641in}}%
\pgfpathlineto{\pgfqpoint{5.915929in}{6.313228in}}%
\pgfpathlineto{\pgfqpoint{5.916902in}{6.314091in}}%
\pgfpathlineto{\pgfqpoint{5.918036in}{6.318402in}}%
\pgfpathlineto{\pgfqpoint{5.918252in}{6.318402in}}%
\pgfpathlineto{\pgfqpoint{5.919602in}{6.319265in}}%
\pgfpathlineto{\pgfqpoint{5.920413in}{6.321852in}}%
\pgfpathlineto{\pgfqpoint{5.922627in}{6.322714in}}%
\pgfpathlineto{\pgfqpoint{5.923546in}{6.328750in}}%
\pgfpathlineto{\pgfqpoint{5.925760in}{6.329613in}}%
\pgfpathlineto{\pgfqpoint{5.927273in}{6.335649in}}%
\pgfpathlineto{\pgfqpoint{5.928623in}{6.336511in}}%
\pgfpathlineto{\pgfqpoint{5.929325in}{6.341686in}}%
\pgfpathlineto{\pgfqpoint{5.932026in}{6.342548in}}%
\pgfpathlineto{\pgfqpoint{5.933539in}{6.350309in}}%
\pgfpathlineto{\pgfqpoint{5.936618in}{6.351171in}}%
\pgfpathlineto{\pgfqpoint{5.936618in}{6.352034in}}%
\pgfpathlineto{\pgfqpoint{5.938670in}{6.352896in}}%
\pgfpathlineto{\pgfqpoint{5.939265in}{6.354621in}}%
\pgfpathlineto{\pgfqpoint{5.940399in}{6.355483in}}%
\pgfpathlineto{\pgfqpoint{5.941857in}{6.361520in}}%
\pgfpathlineto{\pgfqpoint{5.943208in}{6.362382in}}%
\pgfpathlineto{\pgfqpoint{5.944180in}{6.364969in}}%
\pgfpathlineto{\pgfqpoint{5.945368in}{6.365831in}}%
\pgfpathlineto{\pgfqpoint{5.946881in}{6.369281in}}%
\pgfpathlineto{\pgfqpoint{5.947205in}{6.370143in}}%
\pgfpathlineto{\pgfqpoint{5.948123in}{6.374455in}}%
\pgfpathlineto{\pgfqpoint{5.951526in}{6.375317in}}%
\pgfpathlineto{\pgfqpoint{5.951526in}{6.376179in}}%
\pgfpathlineto{\pgfqpoint{5.954011in}{6.377042in}}%
\pgfpathlineto{\pgfqpoint{5.954821in}{6.378766in}}%
\pgfpathlineto{\pgfqpoint{5.956172in}{6.379629in}}%
\pgfpathlineto{\pgfqpoint{5.957630in}{6.383078in}}%
\pgfpathlineto{\pgfqpoint{5.958711in}{6.383940in}}%
\pgfpathlineto{\pgfqpoint{5.960115in}{6.388252in}}%
\pgfpathlineto{\pgfqpoint{5.961033in}{6.389115in}}%
\pgfpathlineto{\pgfqpoint{5.962384in}{6.393426in}}%
\pgfpathlineto{\pgfqpoint{5.963410in}{6.394289in}}%
\pgfpathlineto{\pgfqpoint{5.964166in}{6.399463in}}%
\pgfpathlineto{\pgfqpoint{5.965517in}{6.399463in}}%
\pgfpathlineto{\pgfqpoint{5.966489in}{6.403774in}}%
\pgfpathlineto{\pgfqpoint{5.968110in}{6.404637in}}%
\pgfpathlineto{\pgfqpoint{5.969568in}{6.408948in}}%
\pgfpathlineto{\pgfqpoint{5.970162in}{6.409811in}}%
\pgfpathlineto{\pgfqpoint{5.971026in}{6.412398in}}%
\pgfpathlineto{\pgfqpoint{5.972431in}{6.413260in}}%
\pgfpathlineto{\pgfqpoint{5.973349in}{6.414985in}}%
\pgfpathlineto{\pgfqpoint{5.974538in}{6.415847in}}%
\pgfpathlineto{\pgfqpoint{5.975726in}{6.421021in}}%
\pgfpathlineto{\pgfqpoint{5.977130in}{6.421884in}}%
\pgfpathlineto{\pgfqpoint{5.978319in}{6.425333in}}%
\pgfpathlineto{\pgfqpoint{5.979831in}{6.426195in}}%
\pgfpathlineto{\pgfqpoint{5.981074in}{6.427920in}}%
\pgfpathlineto{\pgfqpoint{5.981560in}{6.427920in}}%
\pgfpathlineto{\pgfqpoint{5.982640in}{6.431369in}}%
\pgfpathlineto{\pgfqpoint{5.983288in}{6.432232in}}%
\pgfpathlineto{\pgfqpoint{5.984693in}{6.434819in}}%
\pgfpathlineto{\pgfqpoint{5.985827in}{6.435681in}}%
\pgfpathlineto{\pgfqpoint{5.986853in}{6.441718in}}%
\pgfpathlineto{\pgfqpoint{5.988960in}{6.442580in}}%
\pgfpathlineto{\pgfqpoint{5.989986in}{6.447754in}}%
\pgfpathlineto{\pgfqpoint{5.991661in}{6.448616in}}%
\pgfpathlineto{\pgfqpoint{5.993065in}{6.450341in}}%
\pgfpathlineto{\pgfqpoint{5.994038in}{6.451203in}}%
\pgfpathlineto{\pgfqpoint{5.994470in}{6.453790in}}%
\pgfpathlineto{\pgfqpoint{5.996036in}{6.454653in}}%
\pgfpathlineto{\pgfqpoint{5.996954in}{6.457240in}}%
\pgfpathlineto{\pgfqpoint{6.000574in}{6.458102in}}%
\pgfpathlineto{\pgfqpoint{6.001762in}{6.461552in}}%
\pgfpathlineto{\pgfqpoint{6.003220in}{6.462414in}}%
\pgfpathlineto{\pgfqpoint{6.004625in}{6.465863in}}%
\pgfpathlineto{\pgfqpoint{6.006191in}{6.466726in}}%
\pgfpathlineto{\pgfqpoint{6.007704in}{6.470175in}}%
\pgfpathlineto{\pgfqpoint{6.009378in}{6.471037in}}%
\pgfpathlineto{\pgfqpoint{6.010513in}{6.474487in}}%
\pgfpathlineto{\pgfqpoint{6.012565in}{6.475349in}}%
\pgfpathlineto{\pgfqpoint{6.013646in}{6.477936in}}%
\pgfpathlineto{\pgfqpoint{6.015698in}{6.478798in}}%
\pgfpathlineto{\pgfqpoint{6.016238in}{6.480523in}}%
\pgfpathlineto{\pgfqpoint{6.019101in}{6.481386in}}%
\pgfpathlineto{\pgfqpoint{6.020560in}{6.485697in}}%
\pgfpathlineto{\pgfqpoint{6.020722in}{6.485697in}}%
\pgfpathlineto{\pgfqpoint{6.022126in}{6.490871in}}%
\pgfpathlineto{\pgfqpoint{6.024503in}{6.491734in}}%
\pgfpathlineto{\pgfqpoint{6.024503in}{6.492596in}}%
\pgfpathlineto{\pgfqpoint{6.026772in}{6.493458in}}%
\pgfpathlineto{\pgfqpoint{6.028176in}{6.497770in}}%
\pgfpathlineto{\pgfqpoint{6.028932in}{6.498632in}}%
\pgfpathlineto{\pgfqpoint{6.029094in}{6.501219in}}%
\pgfpathlineto{\pgfqpoint{6.032119in}{6.502082in}}%
\pgfpathlineto{\pgfqpoint{6.033038in}{6.504669in}}%
\pgfpathlineto{\pgfqpoint{6.034388in}{6.505531in}}%
\pgfpathlineto{\pgfqpoint{6.035630in}{6.507256in}}%
\pgfpathlineto{\pgfqpoint{6.036657in}{6.508118in}}%
\pgfpathlineto{\pgfqpoint{6.038169in}{6.511568in}}%
\pgfpathlineto{\pgfqpoint{6.040816in}{6.512430in}}%
\pgfpathlineto{\pgfqpoint{6.040816in}{6.513292in}}%
\pgfpathlineto{\pgfqpoint{6.042869in}{6.514155in}}%
\pgfpathlineto{\pgfqpoint{6.043517in}{6.515879in}}%
\pgfpathlineto{\pgfqpoint{6.045840in}{6.516742in}}%
\pgfpathlineto{\pgfqpoint{6.045840in}{6.517604in}}%
\pgfpathlineto{\pgfqpoint{6.048378in}{6.518466in}}%
\pgfpathlineto{\pgfqpoint{6.048378in}{6.519329in}}%
\pgfpathlineto{\pgfqpoint{6.050269in}{6.519329in}}%
\pgfpathlineto{\pgfqpoint{6.051349in}{6.525365in}}%
\pgfpathlineto{\pgfqpoint{6.054752in}{6.526227in}}%
\pgfpathlineto{\pgfqpoint{6.055725in}{6.528814in}}%
\pgfpathlineto{\pgfqpoint{6.056481in}{6.528814in}}%
\pgfpathlineto{\pgfqpoint{6.057615in}{6.531402in}}%
\pgfpathlineto{\pgfqpoint{6.059236in}{6.532264in}}%
\pgfpathlineto{\pgfqpoint{6.059236in}{6.533126in}}%
\pgfpathlineto{\pgfqpoint{6.062369in}{6.533989in}}%
\pgfpathlineto{\pgfqpoint{6.062369in}{6.534851in}}%
\pgfpathlineto{\pgfqpoint{6.067176in}{6.535713in}}%
\pgfpathlineto{\pgfqpoint{6.067176in}{6.536576in}}%
\pgfpathlineto{\pgfqpoint{6.068959in}{6.537438in}}%
\pgfpathlineto{\pgfqpoint{6.069823in}{6.539163in}}%
\pgfpathlineto{\pgfqpoint{6.071336in}{6.540025in}}%
\pgfpathlineto{\pgfqpoint{6.072686in}{6.541750in}}%
\pgfpathlineto{\pgfqpoint{6.074360in}{6.542612in}}%
\pgfpathlineto{\pgfqpoint{6.075333in}{6.546061in}}%
\pgfpathlineto{\pgfqpoint{6.077818in}{6.546924in}}%
\pgfpathlineto{\pgfqpoint{6.079276in}{6.549511in}}%
\pgfpathlineto{\pgfqpoint{6.084678in}{6.550373in}}%
\pgfpathlineto{\pgfqpoint{6.085596in}{6.552098in}}%
\pgfpathlineto{\pgfqpoint{6.086892in}{6.552960in}}%
\pgfpathlineto{\pgfqpoint{6.087487in}{6.554685in}}%
\pgfpathlineto{\pgfqpoint{6.089269in}{6.555547in}}%
\pgfpathlineto{\pgfqpoint{6.090566in}{6.559859in}}%
\pgfpathlineto{\pgfqpoint{6.091592in}{6.560721in}}%
\pgfpathlineto{\pgfqpoint{6.092132in}{6.562446in}}%
\pgfpathlineto{\pgfqpoint{6.095805in}{6.563308in}}%
\pgfpathlineto{\pgfqpoint{6.095805in}{6.564171in}}%
\pgfpathlineto{\pgfqpoint{6.098560in}{6.565033in}}%
\pgfpathlineto{\pgfqpoint{6.099478in}{6.567620in}}%
\pgfpathlineto{\pgfqpoint{6.100883in}{6.568482in}}%
\pgfpathlineto{\pgfqpoint{6.101207in}{6.571069in}}%
\pgfpathlineto{\pgfqpoint{6.103638in}{6.571932in}}%
\pgfpathlineto{\pgfqpoint{6.103746in}{6.573656in}}%
\pgfpathlineto{\pgfqpoint{6.106014in}{6.574519in}}%
\pgfpathlineto{\pgfqpoint{6.106500in}{6.576243in}}%
\pgfpathlineto{\pgfqpoint{6.111794in}{6.577106in}}%
\pgfpathlineto{\pgfqpoint{6.112928in}{6.580555in}}%
\pgfpathlineto{\pgfqpoint{6.114819in}{6.581418in}}%
\pgfpathlineto{\pgfqpoint{6.115953in}{6.584867in}}%
\pgfpathlineto{\pgfqpoint{6.117790in}{6.585729in}}%
\pgfpathlineto{\pgfqpoint{6.118330in}{6.589179in}}%
\pgfpathlineto{\pgfqpoint{6.120329in}{6.590041in}}%
\pgfpathlineto{\pgfqpoint{6.121787in}{6.591766in}}%
\pgfpathlineto{\pgfqpoint{6.122273in}{6.592628in}}%
\pgfpathlineto{\pgfqpoint{6.123462in}{6.596940in}}%
\pgfpathlineto{\pgfqpoint{6.126919in}{6.597802in}}%
\pgfpathlineto{\pgfqpoint{6.127945in}{6.599527in}}%
\pgfpathlineto{\pgfqpoint{6.129944in}{6.600389in}}%
\pgfpathlineto{\pgfqpoint{6.130970in}{6.602976in}}%
\pgfpathlineto{\pgfqpoint{6.135670in}{6.603839in}}%
\pgfpathlineto{\pgfqpoint{6.136804in}{6.606426in}}%
\pgfpathlineto{\pgfqpoint{6.137398in}{6.607288in}}%
\pgfpathlineto{\pgfqpoint{6.138802in}{6.613324in}}%
\pgfpathlineto{\pgfqpoint{6.140855in}{6.614187in}}%
\pgfpathlineto{\pgfqpoint{6.141611in}{6.618498in}}%
\pgfpathlineto{\pgfqpoint{6.145122in}{6.619361in}}%
\pgfpathlineto{\pgfqpoint{6.145122in}{6.620223in}}%
\pgfpathlineto{\pgfqpoint{6.149336in}{6.621085in}}%
\pgfpathlineto{\pgfqpoint{6.150686in}{6.622810in}}%
\pgfpathlineto{\pgfqpoint{6.151226in}{6.623672in}}%
\pgfpathlineto{\pgfqpoint{6.151658in}{6.626260in}}%
\pgfpathlineto{\pgfqpoint{6.153819in}{6.627122in}}%
\pgfpathlineto{\pgfqpoint{6.154575in}{6.630571in}}%
\pgfpathlineto{\pgfqpoint{6.156898in}{6.631434in}}%
\pgfpathlineto{\pgfqpoint{6.156898in}{6.633158in}}%
\pgfpathlineto{\pgfqpoint{6.163758in}{6.634021in}}%
\pgfpathlineto{\pgfqpoint{6.164893in}{6.636608in}}%
\pgfpathlineto{\pgfqpoint{6.166783in}{6.637470in}}%
\pgfpathlineto{\pgfqpoint{6.167161in}{6.640057in}}%
\pgfpathlineto{\pgfqpoint{6.171483in}{6.640919in}}%
\pgfpathlineto{\pgfqpoint{6.172185in}{6.642644in}}%
\pgfpathlineto{\pgfqpoint{6.173319in}{6.643506in}}%
\pgfpathlineto{\pgfqpoint{6.174778in}{6.647818in}}%
\pgfpathlineto{\pgfqpoint{6.177154in}{6.648681in}}%
\pgfpathlineto{\pgfqpoint{6.177154in}{6.649543in}}%
\pgfpathlineto{\pgfqpoint{6.179369in}{6.650405in}}%
\pgfpathlineto{\pgfqpoint{6.179369in}{6.651268in}}%
\pgfpathlineto{\pgfqpoint{6.182286in}{6.652130in}}%
\pgfpathlineto{\pgfqpoint{6.183420in}{6.653855in}}%
\pgfpathlineto{\pgfqpoint{6.187580in}{6.654717in}}%
\pgfpathlineto{\pgfqpoint{6.187850in}{6.656442in}}%
\pgfpathlineto{\pgfqpoint{6.189362in}{6.657304in}}%
\pgfpathlineto{\pgfqpoint{6.190605in}{6.659029in}}%
\pgfpathlineto{\pgfqpoint{6.192441in}{6.659891in}}%
\pgfpathlineto{\pgfqpoint{6.193630in}{6.662478in}}%
\pgfpathlineto{\pgfqpoint{6.196600in}{6.663340in}}%
\pgfpathlineto{\pgfqpoint{6.198005in}{6.667652in}}%
\pgfpathlineto{\pgfqpoint{6.198923in}{6.668514in}}%
\pgfpathlineto{\pgfqpoint{6.198923in}{6.669377in}}%
\pgfpathlineto{\pgfqpoint{6.202596in}{6.669377in}}%
\pgfpathlineto{\pgfqpoint{6.204001in}{6.671964in}}%
\pgfpathlineto{\pgfqpoint{6.205729in}{6.672826in}}%
\pgfpathlineto{\pgfqpoint{6.207026in}{6.675413in}}%
\pgfpathlineto{\pgfqpoint{6.208106in}{6.676276in}}%
\pgfpathlineto{\pgfqpoint{6.208592in}{6.678863in}}%
\pgfpathlineto{\pgfqpoint{6.210429in}{6.679725in}}%
\pgfpathlineto{\pgfqpoint{6.211671in}{6.684899in}}%
\pgfpathlineto{\pgfqpoint{6.215614in}{6.685761in}}%
\pgfpathlineto{\pgfqpoint{6.216695in}{6.687486in}}%
\pgfpathlineto{\pgfqpoint{6.218099in}{6.688348in}}%
\pgfpathlineto{\pgfqpoint{6.218369in}{6.690073in}}%
\pgfpathlineto{\pgfqpoint{6.224851in}{6.690935in}}%
\pgfpathlineto{\pgfqpoint{6.226364in}{6.694385in}}%
\pgfpathlineto{\pgfqpoint{6.228200in}{6.695247in}}%
\pgfpathlineto{\pgfqpoint{6.228200in}{6.696109in}}%
\pgfpathlineto{\pgfqpoint{6.231171in}{6.696109in}}%
\pgfpathlineto{\pgfqpoint{6.231873in}{6.699559in}}%
\pgfpathlineto{\pgfqpoint{6.234250in}{6.700421in}}%
\pgfpathlineto{\pgfqpoint{6.235438in}{6.703008in}}%
\pgfpathlineto{\pgfqpoint{6.239220in}{6.703871in}}%
\pgfpathlineto{\pgfqpoint{6.239220in}{6.704733in}}%
\pgfpathlineto{\pgfqpoint{6.243379in}{6.705595in}}%
\pgfpathlineto{\pgfqpoint{6.244891in}{6.708182in}}%
\pgfpathlineto{\pgfqpoint{6.246242in}{6.709045in}}%
\pgfpathlineto{\pgfqpoint{6.246890in}{6.710769in}}%
\pgfpathlineto{\pgfqpoint{6.248889in}{6.711632in}}%
\pgfpathlineto{\pgfqpoint{6.250401in}{6.714219in}}%
\pgfpathlineto{\pgfqpoint{6.251427in}{6.715081in}}%
\pgfpathlineto{\pgfqpoint{6.252022in}{6.716806in}}%
\pgfpathlineto{\pgfqpoint{6.254506in}{6.717668in}}%
\pgfpathlineto{\pgfqpoint{6.254506in}{6.718530in}}%
\pgfpathlineto{\pgfqpoint{6.254776in}{6.718530in}}%
\pgfpathlineto{\pgfqpoint{6.257801in}{6.719393in}}%
\pgfpathlineto{\pgfqpoint{6.257801in}{6.721118in}}%
\pgfpathlineto{\pgfqpoint{6.262555in}{6.721980in}}%
\pgfpathlineto{\pgfqpoint{6.264067in}{6.724567in}}%
\pgfpathlineto{\pgfqpoint{6.264770in}{6.725429in}}%
\pgfpathlineto{\pgfqpoint{6.265796in}{6.727154in}}%
\pgfpathlineto{\pgfqpoint{6.267038in}{6.728016in}}%
\pgfpathlineto{\pgfqpoint{6.267038in}{6.728879in}}%
\pgfpathlineto{\pgfqpoint{6.267362in}{6.728879in}}%
\pgfpathlineto{\pgfqpoint{6.270549in}{6.729741in}}%
\pgfpathlineto{\pgfqpoint{6.271738in}{6.732328in}}%
\pgfpathlineto{\pgfqpoint{6.272548in}{6.733190in}}%
\pgfpathlineto{\pgfqpoint{6.273358in}{6.734915in}}%
\pgfpathlineto{\pgfqpoint{6.275033in}{6.735777in}}%
\pgfpathlineto{\pgfqpoint{6.275735in}{6.737502in}}%
\pgfpathlineto{\pgfqpoint{6.280056in}{6.738364in}}%
\pgfpathlineto{\pgfqpoint{6.281029in}{6.740089in}}%
\pgfpathlineto{\pgfqpoint{6.283405in}{6.740951in}}%
\pgfpathlineto{\pgfqpoint{6.284702in}{6.742676in}}%
\pgfpathlineto{\pgfqpoint{6.287133in}{6.743538in}}%
\pgfpathlineto{\pgfqpoint{6.288213in}{6.747850in}}%
\pgfpathlineto{\pgfqpoint{6.295073in}{6.748713in}}%
\pgfpathlineto{\pgfqpoint{6.295073in}{6.749575in}}%
\pgfpathlineto{\pgfqpoint{6.298530in}{6.749575in}}%
\pgfpathlineto{\pgfqpoint{6.298530in}{6.751300in}}%
\pgfpathlineto{\pgfqpoint{6.301015in}{6.752162in}}%
\pgfpathlineto{\pgfqpoint{6.301447in}{6.753887in}}%
\pgfpathlineto{\pgfqpoint{6.302905in}{6.753887in}}%
\pgfpathlineto{\pgfqpoint{6.303932in}{6.758198in}}%
\pgfpathlineto{\pgfqpoint{6.304418in}{6.758198in}}%
\pgfpathlineto{\pgfqpoint{6.306795in}{6.759061in}}%
\pgfpathlineto{\pgfqpoint{6.306849in}{6.760785in}}%
\pgfpathlineto{\pgfqpoint{6.309117in}{6.761648in}}%
\pgfpathlineto{\pgfqpoint{6.310036in}{6.763372in}}%
\pgfpathlineto{\pgfqpoint{6.313223in}{6.764235in}}%
\pgfpathlineto{\pgfqpoint{6.314627in}{6.765959in}}%
\pgfpathlineto{\pgfqpoint{6.319056in}{6.766822in}}%
\pgfpathlineto{\pgfqpoint{6.319164in}{6.768547in}}%
\pgfpathlineto{\pgfqpoint{6.322730in}{6.769409in}}%
\pgfpathlineto{\pgfqpoint{6.322730in}{6.770271in}}%
\pgfpathlineto{\pgfqpoint{6.328509in}{6.771134in}}%
\pgfpathlineto{\pgfqpoint{6.329266in}{6.772858in}}%
\pgfpathlineto{\pgfqpoint{6.333101in}{6.773721in}}%
\pgfpathlineto{\pgfqpoint{6.333587in}{6.775445in}}%
\pgfpathlineto{\pgfqpoint{6.339367in}{6.776308in}}%
\pgfpathlineto{\pgfqpoint{6.339367in}{6.777170in}}%
\pgfpathlineto{\pgfqpoint{6.346605in}{6.778032in}}%
\pgfpathlineto{\pgfqpoint{6.346605in}{6.778895in}}%
\pgfpathlineto{\pgfqpoint{6.349036in}{6.778895in}}%
\pgfpathlineto{\pgfqpoint{6.349036in}{6.780619in}}%
\pgfpathlineto{\pgfqpoint{6.353573in}{6.781482in}}%
\pgfpathlineto{\pgfqpoint{6.354870in}{6.783206in}}%
\pgfpathlineto{\pgfqpoint{6.355356in}{6.783206in}}%
\pgfpathlineto{\pgfqpoint{6.356436in}{6.785793in}}%
\pgfpathlineto{\pgfqpoint{6.357516in}{6.786656in}}%
\pgfpathlineto{\pgfqpoint{6.357516in}{6.787518in}}%
\pgfpathlineto{\pgfqpoint{6.365565in}{6.788380in}}%
\pgfpathlineto{\pgfqpoint{6.365565in}{6.789243in}}%
\pgfpathlineto{\pgfqpoint{6.370534in}{6.790105in}}%
\pgfpathlineto{\pgfqpoint{6.370534in}{6.790967in}}%
\pgfpathlineto{\pgfqpoint{6.373343in}{6.791830in}}%
\pgfpathlineto{\pgfqpoint{6.373343in}{6.792692in}}%
\pgfpathlineto{\pgfqpoint{6.376530in}{6.793555in}}%
\pgfpathlineto{\pgfqpoint{6.376530in}{6.794417in}}%
\pgfpathlineto{\pgfqpoint{6.380690in}{6.795279in}}%
\pgfpathlineto{\pgfqpoint{6.380906in}{6.797004in}}%
\pgfpathlineto{\pgfqpoint{6.384579in}{6.797866in}}%
\pgfpathlineto{\pgfqpoint{6.384579in}{6.798729in}}%
\pgfpathlineto{\pgfqpoint{6.389332in}{6.799591in}}%
\pgfpathlineto{\pgfqpoint{6.389656in}{6.801316in}}%
\pgfpathlineto{\pgfqpoint{6.391655in}{6.802178in}}%
\pgfpathlineto{\pgfqpoint{6.392681in}{6.804765in}}%
\pgfpathlineto{\pgfqpoint{6.397057in}{6.805627in}}%
\pgfpathlineto{\pgfqpoint{6.397057in}{6.806490in}}%
\pgfpathlineto{\pgfqpoint{6.406780in}{6.807352in}}%
\pgfpathlineto{\pgfqpoint{6.407860in}{6.809077in}}%
\pgfpathlineto{\pgfqpoint{6.409859in}{6.809939in}}%
\pgfpathlineto{\pgfqpoint{6.410885in}{6.812526in}}%
\pgfpathlineto{\pgfqpoint{6.415098in}{6.813388in}}%
\pgfpathlineto{\pgfqpoint{6.415584in}{6.815113in}}%
\pgfpathlineto{\pgfqpoint{6.417961in}{6.815975in}}%
\pgfpathlineto{\pgfqpoint{6.417961in}{6.816838in}}%
\pgfpathlineto{\pgfqpoint{6.421580in}{6.817700in}}%
\pgfpathlineto{\pgfqpoint{6.421580in}{6.818563in}}%
\pgfpathlineto{\pgfqpoint{6.425523in}{6.819425in}}%
\pgfpathlineto{\pgfqpoint{6.426118in}{6.821150in}}%
\pgfpathlineto{\pgfqpoint{6.428008in}{6.822012in}}%
\pgfpathlineto{\pgfqpoint{6.428008in}{6.822874in}}%
\pgfpathlineto{\pgfqpoint{6.431735in}{6.823737in}}%
\pgfpathlineto{\pgfqpoint{6.431951in}{6.825461in}}%
\pgfpathlineto{\pgfqpoint{6.435355in}{6.826324in}}%
\pgfpathlineto{\pgfqpoint{6.435355in}{6.827186in}}%
\pgfpathlineto{\pgfqpoint{6.440756in}{6.828048in}}%
\pgfpathlineto{\pgfqpoint{6.440756in}{6.828911in}}%
\pgfpathlineto{\pgfqpoint{6.443457in}{6.829773in}}%
\pgfpathlineto{\pgfqpoint{6.444591in}{6.831498in}}%
\pgfpathlineto{\pgfqpoint{6.446536in}{6.832360in}}%
\pgfpathlineto{\pgfqpoint{6.447940in}{6.834085in}}%
\pgfpathlineto{\pgfqpoint{6.453882in}{6.834947in}}%
\pgfpathlineto{\pgfqpoint{6.455287in}{6.836672in}}%
\pgfpathlineto{\pgfqpoint{6.460256in}{6.837534in}}%
\pgfpathlineto{\pgfqpoint{6.460256in}{6.838396in}}%
\pgfpathlineto{\pgfqpoint{6.464902in}{6.839259in}}%
\pgfpathlineto{\pgfqpoint{6.465334in}{6.840984in}}%
\pgfpathlineto{\pgfqpoint{6.466738in}{6.841846in}}%
\pgfpathlineto{\pgfqpoint{6.466738in}{6.842708in}}%
\pgfpathlineto{\pgfqpoint{6.469007in}{6.843571in}}%
\pgfpathlineto{\pgfqpoint{6.469007in}{6.844433in}}%
\pgfpathlineto{\pgfqpoint{6.472356in}{6.845295in}}%
\pgfpathlineto{\pgfqpoint{6.472356in}{6.846158in}}%
\pgfpathlineto{\pgfqpoint{6.491532in}{6.847020in}}%
\pgfpathlineto{\pgfqpoint{6.491532in}{6.847882in}}%
\pgfpathlineto{\pgfqpoint{6.494881in}{6.848745in}}%
\pgfpathlineto{\pgfqpoint{6.494881in}{6.849607in}}%
\pgfpathlineto{\pgfqpoint{6.497798in}{6.849607in}}%
\pgfpathlineto{\pgfqpoint{6.498770in}{6.852194in}}%
\pgfpathlineto{\pgfqpoint{6.503416in}{6.853056in}}%
\pgfpathlineto{\pgfqpoint{6.503416in}{6.853919in}}%
\pgfpathlineto{\pgfqpoint{6.506279in}{6.854781in}}%
\pgfpathlineto{\pgfqpoint{6.506279in}{6.855643in}}%
\pgfpathlineto{\pgfqpoint{6.512707in}{6.856506in}}%
\pgfpathlineto{\pgfqpoint{6.513247in}{6.859093in}}%
\pgfpathlineto{\pgfqpoint{6.516974in}{6.859955in}}%
\pgfpathlineto{\pgfqpoint{6.518378in}{6.861680in}}%
\pgfpathlineto{\pgfqpoint{6.520539in}{6.862542in}}%
\pgfpathlineto{\pgfqpoint{6.520539in}{6.863404in}}%
\pgfpathlineto{\pgfqpoint{6.523618in}{6.864267in}}%
\pgfpathlineto{\pgfqpoint{6.524644in}{6.866854in}}%
\pgfpathlineto{\pgfqpoint{6.527939in}{6.867716in}}%
\pgfpathlineto{\pgfqpoint{6.528533in}{6.869441in}}%
\pgfpathlineto{\pgfqpoint{6.536042in}{6.870303in}}%
\pgfpathlineto{\pgfqpoint{6.536042in}{6.871166in}}%
\pgfpathlineto{\pgfqpoint{6.543010in}{6.872028in}}%
\pgfpathlineto{\pgfqpoint{6.543550in}{6.873753in}}%
\pgfpathlineto{\pgfqpoint{6.552841in}{6.874615in}}%
\pgfpathlineto{\pgfqpoint{6.552841in}{6.875477in}}%
\pgfpathlineto{\pgfqpoint{6.556352in}{6.876340in}}%
\pgfpathlineto{\pgfqpoint{6.557270in}{6.878064in}}%
\pgfpathlineto{\pgfqpoint{6.559539in}{6.878927in}}%
\pgfpathlineto{\pgfqpoint{6.559539in}{6.879789in}}%
\pgfpathlineto{\pgfqpoint{6.561754in}{6.879789in}}%
\pgfpathlineto{\pgfqpoint{6.561754in}{6.881514in}}%
\pgfpathlineto{\pgfqpoint{6.566777in}{6.882376in}}%
\pgfpathlineto{\pgfqpoint{6.567155in}{6.884101in}}%
\pgfpathlineto{\pgfqpoint{6.569910in}{6.884963in}}%
\pgfpathlineto{\pgfqpoint{6.569910in}{6.885825in}}%
\pgfpathlineto{\pgfqpoint{6.582172in}{6.886688in}}%
\pgfpathlineto{\pgfqpoint{6.582172in}{6.887550in}}%
\pgfpathlineto{\pgfqpoint{6.585251in}{6.888413in}}%
\pgfpathlineto{\pgfqpoint{6.586169in}{6.891000in}}%
\pgfpathlineto{\pgfqpoint{6.598647in}{6.891862in}}%
\pgfpathlineto{\pgfqpoint{6.599133in}{6.893587in}}%
\pgfpathlineto{\pgfqpoint{6.603455in}{6.894449in}}%
\pgfpathlineto{\pgfqpoint{6.603455in}{6.895311in}}%
\pgfpathlineto{\pgfqpoint{6.610639in}{6.896174in}}%
\pgfpathlineto{\pgfqpoint{6.610639in}{6.897036in}}%
\pgfpathlineto{\pgfqpoint{6.615284in}{6.897898in}}%
\pgfpathlineto{\pgfqpoint{6.615284in}{6.898761in}}%
\pgfpathlineto{\pgfqpoint{6.620686in}{6.899623in}}%
\pgfpathlineto{\pgfqpoint{6.621982in}{6.902210in}}%
\pgfpathlineto{\pgfqpoint{6.631381in}{6.903072in}}%
\pgfpathlineto{\pgfqpoint{6.631381in}{6.903935in}}%
\pgfpathlineto{\pgfqpoint{6.635000in}{6.904797in}}%
\pgfpathlineto{\pgfqpoint{6.635973in}{6.906522in}}%
\pgfpathlineto{\pgfqpoint{6.638458in}{6.907384in}}%
\pgfpathlineto{\pgfqpoint{6.638674in}{6.909109in}}%
\pgfpathlineto{\pgfqpoint{6.642239in}{6.909109in}}%
\pgfpathlineto{\pgfqpoint{6.643157in}{6.911696in}}%
\pgfpathlineto{\pgfqpoint{6.648829in}{6.912558in}}%
\pgfpathlineto{\pgfqpoint{6.650341in}{6.915145in}}%
\pgfpathlineto{\pgfqpoint{6.654771in}{6.916008in}}%
\pgfpathlineto{\pgfqpoint{6.654771in}{6.916870in}}%
\pgfpathlineto{\pgfqpoint{6.661955in}{6.917732in}}%
\pgfpathlineto{\pgfqpoint{6.662495in}{6.919457in}}%
\pgfpathlineto{\pgfqpoint{6.667465in}{6.920319in}}%
\pgfpathlineto{\pgfqpoint{6.668869in}{6.922044in}}%
\pgfpathlineto{\pgfqpoint{6.669301in}{6.922906in}}%
\pgfpathlineto{\pgfqpoint{6.670652in}{6.924631in}}%
\pgfpathlineto{\pgfqpoint{6.675891in}{6.925493in}}%
\pgfpathlineto{\pgfqpoint{6.676972in}{6.927218in}}%
\pgfpathlineto{\pgfqpoint{6.679078in}{6.928080in}}%
\pgfpathlineto{\pgfqpoint{6.679510in}{6.929805in}}%
\pgfpathlineto{\pgfqpoint{6.682319in}{6.930667in}}%
\pgfpathlineto{\pgfqpoint{6.683616in}{6.933254in}}%
\pgfpathlineto{\pgfqpoint{6.686911in}{6.934117in}}%
\pgfpathlineto{\pgfqpoint{6.688315in}{6.935842in}}%
\pgfpathlineto{\pgfqpoint{6.691556in}{6.936704in}}%
\pgfpathlineto{\pgfqpoint{6.691556in}{6.937566in}}%
\pgfpathlineto{\pgfqpoint{6.700253in}{6.938429in}}%
\pgfpathlineto{\pgfqpoint{6.700253in}{6.939291in}}%
\pgfpathlineto{\pgfqpoint{6.711650in}{6.940153in}}%
\pgfpathlineto{\pgfqpoint{6.711758in}{6.941878in}}%
\pgfpathlineto{\pgfqpoint{6.720455in}{6.942740in}}%
\pgfpathlineto{\pgfqpoint{6.720455in}{6.943603in}}%
\pgfpathlineto{\pgfqpoint{6.728017in}{6.944465in}}%
\pgfpathlineto{\pgfqpoint{6.728017in}{6.945327in}}%
\pgfpathlineto{\pgfqpoint{6.735742in}{6.946190in}}%
\pgfpathlineto{\pgfqpoint{6.735742in}{6.947052in}}%
\pgfpathlineto{\pgfqpoint{6.744763in}{6.947914in}}%
\pgfpathlineto{\pgfqpoint{6.744979in}{6.949639in}}%
\pgfpathlineto{\pgfqpoint{6.748544in}{6.950501in}}%
\pgfpathlineto{\pgfqpoint{6.748544in}{6.951364in}}%
\pgfpathlineto{\pgfqpoint{6.755296in}{6.952226in}}%
\pgfpathlineto{\pgfqpoint{6.755296in}{6.953088in}}%
\pgfpathlineto{\pgfqpoint{6.760157in}{6.953951in}}%
\pgfpathlineto{\pgfqpoint{6.760157in}{6.954813in}}%
\pgfpathlineto{\pgfqpoint{6.765019in}{6.955675in}}%
\pgfpathlineto{\pgfqpoint{6.765019in}{6.956538in}}%
\pgfpathlineto{\pgfqpoint{6.766153in}{6.956538in}}%
\pgfpathlineto{\pgfqpoint{6.774418in}{6.957400in}}%
\pgfpathlineto{\pgfqpoint{6.774418in}{6.958262in}}%
\pgfpathlineto{\pgfqpoint{6.780036in}{6.959125in}}%
\pgfpathlineto{\pgfqpoint{6.780036in}{6.959987in}}%
\pgfpathlineto{\pgfqpoint{6.785329in}{6.960850in}}%
\pgfpathlineto{\pgfqpoint{6.785599in}{6.962574in}}%
\pgfpathlineto{\pgfqpoint{6.792783in}{6.963437in}}%
\pgfpathlineto{\pgfqpoint{6.793432in}{6.966024in}}%
\pgfpathlineto{\pgfqpoint{6.801480in}{6.966886in}}%
\pgfpathlineto{\pgfqpoint{6.801480in}{6.967748in}}%
\pgfpathlineto{\pgfqpoint{6.810231in}{6.968611in}}%
\pgfpathlineto{\pgfqpoint{6.810231in}{6.969473in}}%
\pgfpathlineto{\pgfqpoint{6.814768in}{6.970335in}}%
\pgfpathlineto{\pgfqpoint{6.814768in}{6.972060in}}%
\pgfpathlineto{\pgfqpoint{6.816281in}{6.972060in}}%
\pgfpathlineto{\pgfqpoint{6.817037in}{6.972922in}}%
\pgfpathlineto{\pgfqpoint{6.817847in}{6.975509in}}%
\pgfpathlineto{\pgfqpoint{6.822547in}{6.976372in}}%
\pgfpathlineto{\pgfqpoint{6.822547in}{6.977234in}}%
\pgfpathlineto{\pgfqpoint{6.835025in}{6.978096in}}%
\pgfpathlineto{\pgfqpoint{6.835511in}{6.979821in}}%
\pgfpathlineto{\pgfqpoint{6.839292in}{6.980683in}}%
\pgfpathlineto{\pgfqpoint{6.839994in}{6.982408in}}%
\pgfpathlineto{\pgfqpoint{6.852040in}{6.983270in}}%
\pgfpathlineto{\pgfqpoint{6.852040in}{6.984133in}}%
\pgfpathlineto{\pgfqpoint{6.861115in}{6.984995in}}%
\pgfpathlineto{\pgfqpoint{6.861115in}{6.985858in}}%
\pgfpathlineto{\pgfqpoint{6.872188in}{6.986720in}}%
\pgfpathlineto{\pgfqpoint{6.873106in}{6.988445in}}%
\pgfpathlineto{\pgfqpoint{6.883694in}{6.989307in}}%
\pgfpathlineto{\pgfqpoint{6.885206in}{6.991032in}}%
\pgfpathlineto{\pgfqpoint{6.887745in}{6.991894in}}%
\pgfpathlineto{\pgfqpoint{6.887745in}{6.992756in}}%
\pgfpathlineto{\pgfqpoint{6.894929in}{6.993619in}}%
\pgfpathlineto{\pgfqpoint{6.894929in}{6.994481in}}%
\pgfpathlineto{\pgfqpoint{6.903950in}{6.995343in}}%
\pgfpathlineto{\pgfqpoint{6.904868in}{6.997930in}}%
\pgfpathlineto{\pgfqpoint{6.919291in}{6.998793in}}%
\pgfpathlineto{\pgfqpoint{6.920695in}{7.000517in}}%
\pgfpathlineto{\pgfqpoint{6.938467in}{7.001380in}}%
\pgfpathlineto{\pgfqpoint{6.938467in}{7.002242in}}%
\pgfpathlineto{\pgfqpoint{6.950080in}{7.003104in}}%
\pgfpathlineto{\pgfqpoint{6.950080in}{7.003967in}}%
\pgfpathlineto{\pgfqpoint{6.960938in}{7.004829in}}%
\pgfpathlineto{\pgfqpoint{6.960938in}{7.005691in}}%
\pgfpathlineto{\pgfqpoint{6.980060in}{7.006554in}}%
\pgfpathlineto{\pgfqpoint{6.980060in}{7.007416in}}%
\pgfpathlineto{\pgfqpoint{6.992321in}{7.008279in}}%
\pgfpathlineto{\pgfqpoint{6.992321in}{7.009141in}}%
\pgfpathlineto{\pgfqpoint{7.017763in}{7.010003in}}%
\pgfpathlineto{\pgfqpoint{7.017763in}{7.010866in}}%
\pgfpathlineto{\pgfqpoint{7.023813in}{7.011728in}}%
\pgfpathlineto{\pgfqpoint{7.023813in}{7.012590in}}%
\pgfpathlineto{\pgfqpoint{7.025272in}{7.012590in}}%
\pgfpathlineto{\pgfqpoint{7.036777in}{7.013453in}}%
\pgfpathlineto{\pgfqpoint{7.036777in}{7.014315in}}%
\pgfpathlineto{\pgfqpoint{7.043205in}{7.015177in}}%
\pgfpathlineto{\pgfqpoint{7.043637in}{7.016902in}}%
\pgfpathlineto{\pgfqpoint{7.057412in}{7.017764in}}%
\pgfpathlineto{\pgfqpoint{7.058384in}{7.020351in}}%
\pgfpathlineto{\pgfqpoint{7.070160in}{7.021214in}}%
\pgfpathlineto{\pgfqpoint{7.070160in}{7.022076in}}%
\pgfpathlineto{\pgfqpoint{7.076858in}{7.022938in}}%
\pgfpathlineto{\pgfqpoint{7.076858in}{7.023801in}}%
\pgfpathlineto{\pgfqpoint{7.080207in}{7.024663in}}%
\pgfpathlineto{\pgfqpoint{7.080207in}{7.025525in}}%
\pgfpathlineto{\pgfqpoint{7.080855in}{7.025525in}}%
\pgfpathlineto{\pgfqpoint{7.101489in}{7.026388in}}%
\pgfpathlineto{\pgfqpoint{7.101489in}{7.027250in}}%
\pgfpathlineto{\pgfqpoint{7.111320in}{7.028112in}}%
\pgfpathlineto{\pgfqpoint{7.111320in}{7.028975in}}%
\pgfpathlineto{\pgfqpoint{7.154318in}{7.029837in}}%
\pgfpathlineto{\pgfqpoint{7.155830in}{7.031562in}}%
\pgfpathlineto{\pgfqpoint{7.174574in}{7.032424in}}%
\pgfpathlineto{\pgfqpoint{7.174574in}{7.033287in}}%
\pgfpathlineto{\pgfqpoint{7.298650in}{7.034149in}}%
\pgfpathlineto{\pgfqpoint{7.298650in}{7.034149in}}%
\pgfusepath{stroke}%
\end{pgfscope}%
\begin{pgfscope}%
\pgfpathrectangle{\pgfqpoint{5.688041in}{5.564583in}}{\pgfqpoint{1.687305in}{1.539545in}}%
\pgfusepath{clip}%
\pgfsetrectcap%
\pgfsetroundjoin%
\pgfsetlinewidth{1.505625pt}%
\definecolor{currentstroke}{rgb}{0.501961,0.501961,0.501961}%
\pgfsetstrokecolor{currentstroke}%
\pgfsetdash{}{0pt}%
\pgfpathmoveto{\pgfqpoint{5.764736in}{5.634562in}}%
\pgfpathlineto{\pgfqpoint{7.298650in}{7.034149in}}%
\pgfusepath{stroke}%
\end{pgfscope}%
\begin{pgfscope}%
\pgfsetrectcap%
\pgfsetmiterjoin%
\pgfsetlinewidth{0.803000pt}%
\definecolor{currentstroke}{rgb}{0.000000,0.000000,0.000000}%
\pgfsetstrokecolor{currentstroke}%
\pgfsetdash{}{0pt}%
\pgfpathmoveto{\pgfqpoint{5.688041in}{5.564583in}}%
\pgfpathlineto{\pgfqpoint{5.688041in}{7.104128in}}%
\pgfusepath{stroke}%
\end{pgfscope}%
\begin{pgfscope}%
\pgfsetrectcap%
\pgfsetmiterjoin%
\pgfsetlinewidth{0.803000pt}%
\definecolor{currentstroke}{rgb}{0.000000,0.000000,0.000000}%
\pgfsetstrokecolor{currentstroke}%
\pgfsetdash{}{0pt}%
\pgfpathmoveto{\pgfqpoint{7.375346in}{5.564583in}}%
\pgfpathlineto{\pgfqpoint{7.375346in}{7.104128in}}%
\pgfusepath{stroke}%
\end{pgfscope}%
\begin{pgfscope}%
\pgfsetrectcap%
\pgfsetmiterjoin%
\pgfsetlinewidth{0.803000pt}%
\definecolor{currentstroke}{rgb}{0.000000,0.000000,0.000000}%
\pgfsetstrokecolor{currentstroke}%
\pgfsetdash{}{0pt}%
\pgfpathmoveto{\pgfqpoint{5.688041in}{5.564583in}}%
\pgfpathlineto{\pgfqpoint{7.375346in}{5.564583in}}%
\pgfusepath{stroke}%
\end{pgfscope}%
\begin{pgfscope}%
\pgfsetrectcap%
\pgfsetmiterjoin%
\pgfsetlinewidth{0.803000pt}%
\definecolor{currentstroke}{rgb}{0.000000,0.000000,0.000000}%
\pgfsetstrokecolor{currentstroke}%
\pgfsetdash{}{0pt}%
\pgfpathmoveto{\pgfqpoint{5.688041in}{7.104128in}}%
\pgfpathlineto{\pgfqpoint{7.375346in}{7.104128in}}%
\pgfusepath{stroke}%
\end{pgfscope}%
\begin{pgfscope}%
\definecolor{textcolor}{rgb}{0.000000,0.000000,0.000000}%
\pgfsetstrokecolor{textcolor}%
\pgfsetfillcolor{textcolor}%
\pgftext[x=6.531693in,y=7.187462in,,base]{\color{textcolor}\rmfamily\fontsize{20.000000}{24.000000}\selectfont Nodule}%
\end{pgfscope}%
\begin{pgfscope}%
\pgfsetbuttcap%
\pgfsetmiterjoin%
\definecolor{currentfill}{rgb}{1.000000,1.000000,1.000000}%
\pgfsetfillcolor{currentfill}%
\pgfsetfillopacity{0.800000}%
\pgfsetlinewidth{1.003750pt}%
\definecolor{currentstroke}{rgb}{0.800000,0.800000,0.800000}%
\pgfsetstrokecolor{currentstroke}%
\pgfsetstrokeopacity{0.800000}%
\pgfsetdash{}{0pt}%
\pgfpathmoveto{\pgfqpoint{6.166240in}{5.634027in}}%
\pgfpathlineto{\pgfqpoint{7.278124in}{5.634027in}}%
\pgfpathquadraticcurveto{\pgfqpoint{7.305902in}{5.634027in}}{\pgfqpoint{7.305902in}{5.661805in}}%
\pgfpathlineto{\pgfqpoint{7.305902in}{5.841589in}}%
\pgfpathquadraticcurveto{\pgfqpoint{7.305902in}{5.869367in}}{\pgfqpoint{7.278124in}{5.869367in}}%
\pgfpathlineto{\pgfqpoint{6.166240in}{5.869367in}}%
\pgfpathquadraticcurveto{\pgfqpoint{6.138462in}{5.869367in}}{\pgfqpoint{6.138462in}{5.841589in}}%
\pgfpathlineto{\pgfqpoint{6.138462in}{5.661805in}}%
\pgfpathquadraticcurveto{\pgfqpoint{6.138462in}{5.634027in}}{\pgfqpoint{6.166240in}{5.634027in}}%
\pgfpathclose%
\pgfusepath{stroke,fill}%
\end{pgfscope}%
\begin{pgfscope}%
\pgfsetrectcap%
\pgfsetroundjoin%
\pgfsetlinewidth{1.505625pt}%
\definecolor{currentstroke}{rgb}{0.000000,0.501961,0.000000}%
\pgfsetstrokecolor{currentstroke}%
\pgfsetdash{}{0pt}%
\pgfpathmoveto{\pgfqpoint{6.194018in}{5.765200in}}%
\pgfpathlineto{\pgfqpoint{6.471795in}{5.765200in}}%
\pgfusepath{stroke}%
\end{pgfscope}%
\begin{pgfscope}%
\definecolor{textcolor}{rgb}{0.000000,0.000000,0.000000}%
\pgfsetstrokecolor{textcolor}%
\pgfsetfillcolor{textcolor}%
\pgftext[x=6.582907in,y=5.716589in,left,base]{\color{textcolor}\rmfamily\fontsize{10.000000}{12.000000}\selectfont AUC 0.806}%
\end{pgfscope}%
\begin{pgfscope}%
\pgfsetbuttcap%
\pgfsetmiterjoin%
\definecolor{currentfill}{rgb}{1.000000,1.000000,1.000000}%
\pgfsetfillcolor{currentfill}%
\pgfsetlinewidth{0.000000pt}%
\definecolor{currentstroke}{rgb}{0.000000,0.000000,0.000000}%
\pgfsetstrokecolor{currentstroke}%
\pgfsetstrokeopacity{0.000000}%
\pgfsetdash{}{0pt}%
\pgfpathmoveto{\pgfqpoint{8.150541in}{5.564583in}}%
\pgfpathlineto{\pgfqpoint{9.837846in}{5.564583in}}%
\pgfpathlineto{\pgfqpoint{9.837846in}{7.104128in}}%
\pgfpathlineto{\pgfqpoint{8.150541in}{7.104128in}}%
\pgfpathclose%
\pgfusepath{fill}%
\end{pgfscope}%
\begin{pgfscope}%
\pgfsetbuttcap%
\pgfsetroundjoin%
\definecolor{currentfill}{rgb}{0.000000,0.000000,0.000000}%
\pgfsetfillcolor{currentfill}%
\pgfsetlinewidth{0.803000pt}%
\definecolor{currentstroke}{rgb}{0.000000,0.000000,0.000000}%
\pgfsetstrokecolor{currentstroke}%
\pgfsetdash{}{0pt}%
\pgfsys@defobject{currentmarker}{\pgfqpoint{0.000000in}{-0.048611in}}{\pgfqpoint{0.000000in}{0.000000in}}{%
\pgfpathmoveto{\pgfqpoint{0.000000in}{0.000000in}}%
\pgfpathlineto{\pgfqpoint{0.000000in}{-0.048611in}}%
\pgfusepath{stroke,fill}%
}%
\begin{pgfscope}%
\pgfsys@transformshift{8.227236in}{5.564583in}%
\pgfsys@useobject{currentmarker}{}%
\end{pgfscope}%
\end{pgfscope}%
\begin{pgfscope}%
\definecolor{textcolor}{rgb}{0.000000,0.000000,0.000000}%
\pgfsetstrokecolor{textcolor}%
\pgfsetfillcolor{textcolor}%
\pgftext[x=8.227236in,y=5.467361in,,top]{\color{textcolor}\rmfamily\fontsize{10.000000}{12.000000}\selectfont \(\displaystyle {0.0}\)}%
\end{pgfscope}%
\begin{pgfscope}%
\pgfsetbuttcap%
\pgfsetroundjoin%
\definecolor{currentfill}{rgb}{0.000000,0.000000,0.000000}%
\pgfsetfillcolor{currentfill}%
\pgfsetlinewidth{0.803000pt}%
\definecolor{currentstroke}{rgb}{0.000000,0.000000,0.000000}%
\pgfsetstrokecolor{currentstroke}%
\pgfsetdash{}{0pt}%
\pgfsys@defobject{currentmarker}{\pgfqpoint{0.000000in}{-0.048611in}}{\pgfqpoint{0.000000in}{0.000000in}}{%
\pgfpathmoveto{\pgfqpoint{0.000000in}{0.000000in}}%
\pgfpathlineto{\pgfqpoint{0.000000in}{-0.048611in}}%
\pgfusepath{stroke,fill}%
}%
\begin{pgfscope}%
\pgfsys@transformshift{8.994193in}{5.564583in}%
\pgfsys@useobject{currentmarker}{}%
\end{pgfscope}%
\end{pgfscope}%
\begin{pgfscope}%
\definecolor{textcolor}{rgb}{0.000000,0.000000,0.000000}%
\pgfsetstrokecolor{textcolor}%
\pgfsetfillcolor{textcolor}%
\pgftext[x=8.994193in,y=5.467361in,,top]{\color{textcolor}\rmfamily\fontsize{10.000000}{12.000000}\selectfont \(\displaystyle {0.5}\)}%
\end{pgfscope}%
\begin{pgfscope}%
\pgfsetbuttcap%
\pgfsetroundjoin%
\definecolor{currentfill}{rgb}{0.000000,0.000000,0.000000}%
\pgfsetfillcolor{currentfill}%
\pgfsetlinewidth{0.803000pt}%
\definecolor{currentstroke}{rgb}{0.000000,0.000000,0.000000}%
\pgfsetstrokecolor{currentstroke}%
\pgfsetdash{}{0pt}%
\pgfsys@defobject{currentmarker}{\pgfqpoint{0.000000in}{-0.048611in}}{\pgfqpoint{0.000000in}{0.000000in}}{%
\pgfpathmoveto{\pgfqpoint{0.000000in}{0.000000in}}%
\pgfpathlineto{\pgfqpoint{0.000000in}{-0.048611in}}%
\pgfusepath{stroke,fill}%
}%
\begin{pgfscope}%
\pgfsys@transformshift{9.761150in}{5.564583in}%
\pgfsys@useobject{currentmarker}{}%
\end{pgfscope}%
\end{pgfscope}%
\begin{pgfscope}%
\definecolor{textcolor}{rgb}{0.000000,0.000000,0.000000}%
\pgfsetstrokecolor{textcolor}%
\pgfsetfillcolor{textcolor}%
\pgftext[x=9.761150in,y=5.467361in,,top]{\color{textcolor}\rmfamily\fontsize{10.000000}{12.000000}\selectfont \(\displaystyle {1.0}\)}%
\end{pgfscope}%
\begin{pgfscope}%
\definecolor{textcolor}{rgb}{0.000000,0.000000,0.000000}%
\pgfsetstrokecolor{textcolor}%
\pgfsetfillcolor{textcolor}%
\pgftext[x=8.994193in,y=5.288349in,,top]{\color{textcolor}\rmfamily\fontsize{16.000000}{19.200000}\selectfont FPR}%
\end{pgfscope}%
\begin{pgfscope}%
\pgfsetbuttcap%
\pgfsetroundjoin%
\definecolor{currentfill}{rgb}{0.000000,0.000000,0.000000}%
\pgfsetfillcolor{currentfill}%
\pgfsetlinewidth{0.803000pt}%
\definecolor{currentstroke}{rgb}{0.000000,0.000000,0.000000}%
\pgfsetstrokecolor{currentstroke}%
\pgfsetdash{}{0pt}%
\pgfsys@defobject{currentmarker}{\pgfqpoint{-0.048611in}{0.000000in}}{\pgfqpoint{-0.000000in}{0.000000in}}{%
\pgfpathmoveto{\pgfqpoint{-0.000000in}{0.000000in}}%
\pgfpathlineto{\pgfqpoint{-0.048611in}{0.000000in}}%
\pgfusepath{stroke,fill}%
}%
\begin{pgfscope}%
\pgfsys@transformshift{8.150541in}{5.634562in}%
\pgfsys@useobject{currentmarker}{}%
\end{pgfscope}%
\end{pgfscope}%
\begin{pgfscope}%
\definecolor{textcolor}{rgb}{0.000000,0.000000,0.000000}%
\pgfsetstrokecolor{textcolor}%
\pgfsetfillcolor{textcolor}%
\pgftext[x=7.806404in, y=5.586337in, left, base]{\color{textcolor}\rmfamily\fontsize{10.000000}{12.000000}\selectfont \(\displaystyle {0.00}\)}%
\end{pgfscope}%
\begin{pgfscope}%
\pgfsetbuttcap%
\pgfsetroundjoin%
\definecolor{currentfill}{rgb}{0.000000,0.000000,0.000000}%
\pgfsetfillcolor{currentfill}%
\pgfsetlinewidth{0.803000pt}%
\definecolor{currentstroke}{rgb}{0.000000,0.000000,0.000000}%
\pgfsetstrokecolor{currentstroke}%
\pgfsetdash{}{0pt}%
\pgfsys@defobject{currentmarker}{\pgfqpoint{-0.048611in}{0.000000in}}{\pgfqpoint{-0.000000in}{0.000000in}}{%
\pgfpathmoveto{\pgfqpoint{-0.000000in}{0.000000in}}%
\pgfpathlineto{\pgfqpoint{-0.048611in}{0.000000in}}%
\pgfusepath{stroke,fill}%
}%
\begin{pgfscope}%
\pgfsys@transformshift{8.150541in}{5.984459in}%
\pgfsys@useobject{currentmarker}{}%
\end{pgfscope}%
\end{pgfscope}%
\begin{pgfscope}%
\definecolor{textcolor}{rgb}{0.000000,0.000000,0.000000}%
\pgfsetstrokecolor{textcolor}%
\pgfsetfillcolor{textcolor}%
\pgftext[x=7.806404in, y=5.936234in, left, base]{\color{textcolor}\rmfamily\fontsize{10.000000}{12.000000}\selectfont \(\displaystyle {0.25}\)}%
\end{pgfscope}%
\begin{pgfscope}%
\pgfsetbuttcap%
\pgfsetroundjoin%
\definecolor{currentfill}{rgb}{0.000000,0.000000,0.000000}%
\pgfsetfillcolor{currentfill}%
\pgfsetlinewidth{0.803000pt}%
\definecolor{currentstroke}{rgb}{0.000000,0.000000,0.000000}%
\pgfsetstrokecolor{currentstroke}%
\pgfsetdash{}{0pt}%
\pgfsys@defobject{currentmarker}{\pgfqpoint{-0.048611in}{0.000000in}}{\pgfqpoint{-0.000000in}{0.000000in}}{%
\pgfpathmoveto{\pgfqpoint{-0.000000in}{0.000000in}}%
\pgfpathlineto{\pgfqpoint{-0.048611in}{0.000000in}}%
\pgfusepath{stroke,fill}%
}%
\begin{pgfscope}%
\pgfsys@transformshift{8.150541in}{6.334356in}%
\pgfsys@useobject{currentmarker}{}%
\end{pgfscope}%
\end{pgfscope}%
\begin{pgfscope}%
\definecolor{textcolor}{rgb}{0.000000,0.000000,0.000000}%
\pgfsetstrokecolor{textcolor}%
\pgfsetfillcolor{textcolor}%
\pgftext[x=7.806404in, y=6.286130in, left, base]{\color{textcolor}\rmfamily\fontsize{10.000000}{12.000000}\selectfont \(\displaystyle {0.50}\)}%
\end{pgfscope}%
\begin{pgfscope}%
\pgfsetbuttcap%
\pgfsetroundjoin%
\definecolor{currentfill}{rgb}{0.000000,0.000000,0.000000}%
\pgfsetfillcolor{currentfill}%
\pgfsetlinewidth{0.803000pt}%
\definecolor{currentstroke}{rgb}{0.000000,0.000000,0.000000}%
\pgfsetstrokecolor{currentstroke}%
\pgfsetdash{}{0pt}%
\pgfsys@defobject{currentmarker}{\pgfqpoint{-0.048611in}{0.000000in}}{\pgfqpoint{-0.000000in}{0.000000in}}{%
\pgfpathmoveto{\pgfqpoint{-0.000000in}{0.000000in}}%
\pgfpathlineto{\pgfqpoint{-0.048611in}{0.000000in}}%
\pgfusepath{stroke,fill}%
}%
\begin{pgfscope}%
\pgfsys@transformshift{8.150541in}{6.684252in}%
\pgfsys@useobject{currentmarker}{}%
\end{pgfscope}%
\end{pgfscope}%
\begin{pgfscope}%
\definecolor{textcolor}{rgb}{0.000000,0.000000,0.000000}%
\pgfsetstrokecolor{textcolor}%
\pgfsetfillcolor{textcolor}%
\pgftext[x=7.806404in, y=6.636027in, left, base]{\color{textcolor}\rmfamily\fontsize{10.000000}{12.000000}\selectfont \(\displaystyle {0.75}\)}%
\end{pgfscope}%
\begin{pgfscope}%
\pgfsetbuttcap%
\pgfsetroundjoin%
\definecolor{currentfill}{rgb}{0.000000,0.000000,0.000000}%
\pgfsetfillcolor{currentfill}%
\pgfsetlinewidth{0.803000pt}%
\definecolor{currentstroke}{rgb}{0.000000,0.000000,0.000000}%
\pgfsetstrokecolor{currentstroke}%
\pgfsetdash{}{0pt}%
\pgfsys@defobject{currentmarker}{\pgfqpoint{-0.048611in}{0.000000in}}{\pgfqpoint{-0.000000in}{0.000000in}}{%
\pgfpathmoveto{\pgfqpoint{-0.000000in}{0.000000in}}%
\pgfpathlineto{\pgfqpoint{-0.048611in}{0.000000in}}%
\pgfusepath{stroke,fill}%
}%
\begin{pgfscope}%
\pgfsys@transformshift{8.150541in}{7.034149in}%
\pgfsys@useobject{currentmarker}{}%
\end{pgfscope}%
\end{pgfscope}%
\begin{pgfscope}%
\definecolor{textcolor}{rgb}{0.000000,0.000000,0.000000}%
\pgfsetstrokecolor{textcolor}%
\pgfsetfillcolor{textcolor}%
\pgftext[x=7.806404in, y=6.985924in, left, base]{\color{textcolor}\rmfamily\fontsize{10.000000}{12.000000}\selectfont \(\displaystyle {1.00}\)}%
\end{pgfscope}%
\begin{pgfscope}%
\definecolor{textcolor}{rgb}{0.000000,0.000000,0.000000}%
\pgfsetstrokecolor{textcolor}%
\pgfsetfillcolor{textcolor}%
\pgftext[x=7.750849in,y=6.334356in,,bottom,rotate=90.000000]{\color{textcolor}\rmfamily\fontsize{16.000000}{19.200000}\selectfont TPR}%
\end{pgfscope}%
\begin{pgfscope}%
\pgfpathrectangle{\pgfqpoint{8.150541in}{5.564583in}}{\pgfqpoint{1.687305in}{1.539545in}}%
\pgfusepath{clip}%
\pgfsetrectcap%
\pgfsetroundjoin%
\pgfsetlinewidth{1.505625pt}%
\definecolor{currentstroke}{rgb}{0.000000,0.501961,0.000000}%
\pgfsetstrokecolor{currentstroke}%
\pgfsetdash{}{0pt}%
\pgfpathmoveto{\pgfqpoint{8.227236in}{5.634562in}}%
\pgfpathlineto{\pgfqpoint{8.231768in}{5.683648in}}%
\pgfpathlineto{\pgfqpoint{8.232055in}{5.684929in}}%
\pgfpathlineto{\pgfqpoint{8.233718in}{5.696026in}}%
\pgfpathlineto{\pgfqpoint{8.235095in}{5.709685in}}%
\pgfpathlineto{\pgfqpoint{8.235152in}{5.709685in}}%
\pgfpathlineto{\pgfqpoint{8.235783in}{5.710966in}}%
\pgfpathlineto{\pgfqpoint{8.251673in}{5.818101in}}%
\pgfpathlineto{\pgfqpoint{8.252189in}{5.819381in}}%
\pgfpathlineto{\pgfqpoint{8.253623in}{5.829625in}}%
\pgfpathlineto{\pgfqpoint{8.254598in}{5.830906in}}%
\pgfpathlineto{\pgfqpoint{8.256089in}{5.839016in}}%
\pgfpathlineto{\pgfqpoint{8.256434in}{5.840296in}}%
\pgfpathlineto{\pgfqpoint{8.257868in}{5.850113in}}%
\pgfpathlineto{\pgfqpoint{8.258327in}{5.851394in}}%
\pgfpathlineto{\pgfqpoint{8.259474in}{5.859930in}}%
\pgfpathlineto{\pgfqpoint{8.260392in}{5.861211in}}%
\pgfpathlineto{\pgfqpoint{8.261883in}{5.873162in}}%
\pgfpathlineto{\pgfqpoint{8.262170in}{5.874443in}}%
\pgfpathlineto{\pgfqpoint{8.263604in}{5.879565in}}%
\pgfpathlineto{\pgfqpoint{8.264063in}{5.880418in}}%
\pgfpathlineto{\pgfqpoint{8.265497in}{5.888101in}}%
\pgfpathlineto{\pgfqpoint{8.265726in}{5.889382in}}%
\pgfpathlineto{\pgfqpoint{8.267160in}{5.900053in}}%
\pgfpathlineto{\pgfqpoint{8.267390in}{5.900480in}}%
\pgfpathlineto{\pgfqpoint{8.268824in}{5.914565in}}%
\pgfpathlineto{\pgfqpoint{8.269053in}{5.915846in}}%
\pgfpathlineto{\pgfqpoint{8.270430in}{5.923529in}}%
\pgfpathlineto{\pgfqpoint{8.271004in}{5.924809in}}%
\pgfpathlineto{\pgfqpoint{8.272208in}{5.929504in}}%
\pgfpathlineto{\pgfqpoint{8.273298in}{5.930785in}}%
\pgfpathlineto{\pgfqpoint{8.274732in}{5.937187in}}%
\pgfpathlineto{\pgfqpoint{8.275306in}{5.938468in}}%
\pgfpathlineto{\pgfqpoint{8.276740in}{5.947004in}}%
\pgfpathlineto{\pgfqpoint{8.277256in}{5.948285in}}%
\pgfpathlineto{\pgfqpoint{8.278747in}{5.953834in}}%
\pgfpathlineto{\pgfqpoint{8.279149in}{5.955114in}}%
\pgfpathlineto{\pgfqpoint{8.280124in}{5.960663in}}%
\pgfpathlineto{\pgfqpoint{8.281271in}{5.961944in}}%
\pgfpathlineto{\pgfqpoint{8.282533in}{5.964505in}}%
\pgfpathlineto{\pgfqpoint{8.283164in}{5.965358in}}%
\pgfpathlineto{\pgfqpoint{8.284598in}{5.968346in}}%
\pgfpathlineto{\pgfqpoint{8.286376in}{5.969627in}}%
\pgfpathlineto{\pgfqpoint{8.287868in}{5.974749in}}%
\pgfpathlineto{\pgfqpoint{8.288212in}{5.976029in}}%
\pgfpathlineto{\pgfqpoint{8.289589in}{5.979444in}}%
\pgfpathlineto{\pgfqpoint{8.291482in}{5.980724in}}%
\pgfpathlineto{\pgfqpoint{8.292973in}{5.987127in}}%
\pgfpathlineto{\pgfqpoint{8.294522in}{5.988407in}}%
\pgfpathlineto{\pgfqpoint{8.295956in}{5.993102in}}%
\pgfpathlineto{\pgfqpoint{8.296415in}{5.994383in}}%
\pgfpathlineto{\pgfqpoint{8.297849in}{5.998651in}}%
\pgfpathlineto{\pgfqpoint{8.298250in}{5.999505in}}%
\pgfpathlineto{\pgfqpoint{8.299627in}{6.005907in}}%
\pgfpathlineto{\pgfqpoint{8.300029in}{6.007188in}}%
\pgfpathlineto{\pgfqpoint{8.301520in}{6.012310in}}%
\pgfpathlineto{\pgfqpoint{8.301864in}{6.013590in}}%
\pgfpathlineto{\pgfqpoint{8.303356in}{6.017859in}}%
\pgfpathlineto{\pgfqpoint{8.304044in}{6.019139in}}%
\pgfpathlineto{\pgfqpoint{8.305535in}{6.027676in}}%
\pgfpathlineto{\pgfqpoint{8.305822in}{6.028530in}}%
\pgfpathlineto{\pgfqpoint{8.307141in}{6.037920in}}%
\pgfpathlineto{\pgfqpoint{8.307600in}{6.039200in}}%
\pgfpathlineto{\pgfqpoint{8.308862in}{6.041761in}}%
\pgfpathlineto{\pgfqpoint{8.309723in}{6.043042in}}%
\pgfpathlineto{\pgfqpoint{8.311042in}{6.049444in}}%
\pgfpathlineto{\pgfqpoint{8.311501in}{6.050298in}}%
\pgfpathlineto{\pgfqpoint{8.312935in}{6.057981in}}%
\pgfpathlineto{\pgfqpoint{8.313509in}{6.059262in}}%
\pgfpathlineto{\pgfqpoint{8.314771in}{6.063530in}}%
\pgfpathlineto{\pgfqpoint{8.315459in}{6.064384in}}%
\pgfpathlineto{\pgfqpoint{8.316491in}{6.068225in}}%
\pgfpathlineto{\pgfqpoint{8.317180in}{6.069506in}}%
\pgfpathlineto{\pgfqpoint{8.318614in}{6.075481in}}%
\pgfpathlineto{\pgfqpoint{8.319245in}{6.076762in}}%
\pgfpathlineto{\pgfqpoint{8.320622in}{6.081457in}}%
\pgfpathlineto{\pgfqpoint{8.321080in}{6.082737in}}%
\pgfpathlineto{\pgfqpoint{8.322285in}{6.087859in}}%
\pgfpathlineto{\pgfqpoint{8.323031in}{6.088713in}}%
\pgfpathlineto{\pgfqpoint{8.324407in}{6.093408in}}%
\pgfpathlineto{\pgfqpoint{8.325096in}{6.094689in}}%
\pgfpathlineto{\pgfqpoint{8.326587in}{6.101091in}}%
\pgfpathlineto{\pgfqpoint{8.326989in}{6.102372in}}%
\pgfpathlineto{\pgfqpoint{8.328423in}{6.110482in}}%
\pgfpathlineto{\pgfqpoint{8.329168in}{6.111762in}}%
\pgfpathlineto{\pgfqpoint{8.330603in}{6.115604in}}%
\pgfpathlineto{\pgfqpoint{8.331520in}{6.116884in}}%
\pgfpathlineto{\pgfqpoint{8.332725in}{6.120299in}}%
\pgfpathlineto{\pgfqpoint{8.333987in}{6.121579in}}%
\pgfpathlineto{\pgfqpoint{8.335478in}{6.126701in}}%
\pgfpathlineto{\pgfqpoint{8.336511in}{6.127982in}}%
\pgfpathlineto{\pgfqpoint{8.337830in}{6.133531in}}%
\pgfpathlineto{\pgfqpoint{8.338633in}{6.134384in}}%
\pgfpathlineto{\pgfqpoint{8.340067in}{6.139506in}}%
\pgfpathlineto{\pgfqpoint{8.340526in}{6.140787in}}%
\pgfpathlineto{\pgfqpoint{8.341960in}{6.144201in}}%
\pgfpathlineto{\pgfqpoint{8.342419in}{6.145055in}}%
\pgfpathlineto{\pgfqpoint{8.343337in}{6.148043in}}%
\pgfpathlineto{\pgfqpoint{8.344255in}{6.149323in}}%
\pgfpathlineto{\pgfqpoint{8.345689in}{6.152311in}}%
\pgfpathlineto{\pgfqpoint{8.346320in}{6.153592in}}%
\pgfpathlineto{\pgfqpoint{8.347754in}{6.159141in}}%
\pgfpathlineto{\pgfqpoint{8.348213in}{6.159994in}}%
\pgfpathlineto{\pgfqpoint{8.349532in}{6.164263in}}%
\pgfpathlineto{\pgfqpoint{8.350851in}{6.165543in}}%
\pgfpathlineto{\pgfqpoint{8.352056in}{6.167677in}}%
\pgfpathlineto{\pgfqpoint{8.352629in}{6.168958in}}%
\pgfpathlineto{\pgfqpoint{8.353949in}{6.174507in}}%
\pgfpathlineto{\pgfqpoint{8.354408in}{6.175360in}}%
\pgfpathlineto{\pgfqpoint{8.355842in}{6.181336in}}%
\pgfpathlineto{\pgfqpoint{8.356129in}{6.182616in}}%
\pgfpathlineto{\pgfqpoint{8.357505in}{6.184751in}}%
\pgfpathlineto{\pgfqpoint{8.358882in}{6.186031in}}%
\pgfpathlineto{\pgfqpoint{8.360373in}{6.189446in}}%
\pgfpathlineto{\pgfqpoint{8.360775in}{6.190726in}}%
\pgfpathlineto{\pgfqpoint{8.362152in}{6.195421in}}%
\pgfpathlineto{\pgfqpoint{8.363069in}{6.196702in}}%
\pgfpathlineto{\pgfqpoint{8.364503in}{6.200117in}}%
\pgfpathlineto{\pgfqpoint{8.365020in}{6.201397in}}%
\pgfpathlineto{\pgfqpoint{8.365593in}{6.204385in}}%
\pgfpathlineto{\pgfqpoint{8.367085in}{6.205665in}}%
\pgfpathlineto{\pgfqpoint{8.368519in}{6.209507in}}%
\pgfpathlineto{\pgfqpoint{8.368920in}{6.210361in}}%
\pgfpathlineto{\pgfqpoint{8.370412in}{6.213775in}}%
\pgfpathlineto{\pgfqpoint{8.371387in}{6.215056in}}%
\pgfpathlineto{\pgfqpoint{8.372878in}{6.218044in}}%
\pgfpathlineto{\pgfqpoint{8.374025in}{6.218897in}}%
\pgfpathlineto{\pgfqpoint{8.375402in}{6.222312in}}%
\pgfpathlineto{\pgfqpoint{8.376263in}{6.223592in}}%
\pgfpathlineto{\pgfqpoint{8.377697in}{6.229141in}}%
\pgfpathlineto{\pgfqpoint{8.378614in}{6.230422in}}%
\pgfpathlineto{\pgfqpoint{8.380106in}{6.235117in}}%
\pgfpathlineto{\pgfqpoint{8.380966in}{6.236397in}}%
\pgfpathlineto{\pgfqpoint{8.382458in}{6.238532in}}%
\pgfpathlineto{\pgfqpoint{8.383490in}{6.239812in}}%
\pgfpathlineto{\pgfqpoint{8.384982in}{6.243227in}}%
\pgfpathlineto{\pgfqpoint{8.386129in}{6.244507in}}%
\pgfpathlineto{\pgfqpoint{8.387563in}{6.248776in}}%
\pgfpathlineto{\pgfqpoint{8.388022in}{6.249629in}}%
\pgfpathlineto{\pgfqpoint{8.389456in}{6.252617in}}%
\pgfpathlineto{\pgfqpoint{8.389857in}{6.253898in}}%
\pgfpathlineto{\pgfqpoint{8.391119in}{6.256885in}}%
\pgfpathlineto{\pgfqpoint{8.391922in}{6.257739in}}%
\pgfpathlineto{\pgfqpoint{8.392840in}{6.263715in}}%
\pgfpathlineto{\pgfqpoint{8.393873in}{6.264995in}}%
\pgfpathlineto{\pgfqpoint{8.395364in}{6.268410in}}%
\pgfpathlineto{\pgfqpoint{8.396110in}{6.269690in}}%
\pgfpathlineto{\pgfqpoint{8.397544in}{6.273105in}}%
\pgfpathlineto{\pgfqpoint{8.398806in}{6.274386in}}%
\pgfpathlineto{\pgfqpoint{8.400240in}{6.279081in}}%
\pgfpathlineto{\pgfqpoint{8.401559in}{6.280361in}}%
\pgfpathlineto{\pgfqpoint{8.402706in}{6.283776in}}%
\pgfpathlineto{\pgfqpoint{8.403452in}{6.284203in}}%
\pgfpathlineto{\pgfqpoint{8.404599in}{6.288471in}}%
\pgfpathlineto{\pgfqpoint{8.406205in}{6.289752in}}%
\pgfpathlineto{\pgfqpoint{8.407525in}{6.292313in}}%
\pgfpathlineto{\pgfqpoint{8.408787in}{6.293593in}}%
\pgfpathlineto{\pgfqpoint{8.408787in}{6.294447in}}%
\pgfpathlineto{\pgfqpoint{8.410794in}{6.295300in}}%
\pgfpathlineto{\pgfqpoint{8.412171in}{6.300422in}}%
\pgfpathlineto{\pgfqpoint{8.413089in}{6.301703in}}%
\pgfpathlineto{\pgfqpoint{8.414408in}{6.303837in}}%
\pgfpathlineto{\pgfqpoint{8.415097in}{6.305118in}}%
\pgfpathlineto{\pgfqpoint{8.415555in}{6.307252in}}%
\pgfpathlineto{\pgfqpoint{8.417620in}{6.308532in}}%
\pgfpathlineto{\pgfqpoint{8.418882in}{6.313227in}}%
\pgfpathlineto{\pgfqpoint{8.420374in}{6.314081in}}%
\pgfpathlineto{\pgfqpoint{8.421865in}{6.317069in}}%
\pgfpathlineto{\pgfqpoint{8.423586in}{6.318349in}}%
\pgfpathlineto{\pgfqpoint{8.425020in}{6.322618in}}%
\pgfpathlineto{\pgfqpoint{8.425192in}{6.323045in}}%
\pgfpathlineto{\pgfqpoint{8.426282in}{6.327740in}}%
\pgfpathlineto{\pgfqpoint{8.428232in}{6.329020in}}%
\pgfpathlineto{\pgfqpoint{8.429724in}{6.330728in}}%
\pgfpathlineto{\pgfqpoint{8.430584in}{6.332008in}}%
\pgfpathlineto{\pgfqpoint{8.431846in}{6.336276in}}%
\pgfpathlineto{\pgfqpoint{8.433051in}{6.337557in}}%
\pgfpathlineto{\pgfqpoint{8.434198in}{6.341398in}}%
\pgfpathlineto{\pgfqpoint{8.434714in}{6.341825in}}%
\pgfpathlineto{\pgfqpoint{8.435862in}{6.344813in}}%
\pgfpathlineto{\pgfqpoint{8.436894in}{6.346094in}}%
\pgfpathlineto{\pgfqpoint{8.437984in}{6.348228in}}%
\pgfpathlineto{\pgfqpoint{8.438730in}{6.349508in}}%
\pgfpathlineto{\pgfqpoint{8.440221in}{6.352923in}}%
\pgfpathlineto{\pgfqpoint{8.441483in}{6.354203in}}%
\pgfpathlineto{\pgfqpoint{8.442974in}{6.358045in}}%
\pgfpathlineto{\pgfqpoint{8.443319in}{6.359325in}}%
\pgfpathlineto{\pgfqpoint{8.444523in}{6.363167in}}%
\pgfpathlineto{\pgfqpoint{8.446588in}{6.364447in}}%
\pgfpathlineto{\pgfqpoint{8.447621in}{6.367008in}}%
\pgfpathlineto{\pgfqpoint{8.448711in}{6.368289in}}%
\pgfpathlineto{\pgfqpoint{8.449743in}{6.370850in}}%
\pgfpathlineto{\pgfqpoint{8.453127in}{6.372130in}}%
\pgfpathlineto{\pgfqpoint{8.454275in}{6.373838in}}%
\pgfpathlineto{\pgfqpoint{8.455422in}{6.375118in}}%
\pgfpathlineto{\pgfqpoint{8.456913in}{6.379813in}}%
\pgfpathlineto{\pgfqpoint{8.457602in}{6.381094in}}%
\pgfpathlineto{\pgfqpoint{8.458347in}{6.383228in}}%
\pgfpathlineto{\pgfqpoint{8.460412in}{6.384509in}}%
\pgfpathlineto{\pgfqpoint{8.461674in}{6.388350in}}%
\pgfpathlineto{\pgfqpoint{8.462133in}{6.388777in}}%
\pgfpathlineto{\pgfqpoint{8.463281in}{6.392618in}}%
\pgfpathlineto{\pgfqpoint{8.464256in}{6.393899in}}%
\pgfpathlineto{\pgfqpoint{8.465632in}{6.399021in}}%
\pgfpathlineto{\pgfqpoint{8.466435in}{6.400301in}}%
\pgfpathlineto{\pgfqpoint{8.467697in}{6.402862in}}%
\pgfpathlineto{\pgfqpoint{8.469361in}{6.404143in}}%
\pgfpathlineto{\pgfqpoint{8.470738in}{6.407558in}}%
\pgfpathlineto{\pgfqpoint{8.473032in}{6.408838in}}%
\pgfpathlineto{\pgfqpoint{8.474065in}{6.411826in}}%
\pgfpathlineto{\pgfqpoint{8.476187in}{6.412680in}}%
\pgfpathlineto{\pgfqpoint{8.477392in}{6.416094in}}%
\pgfpathlineto{\pgfqpoint{8.477908in}{6.416521in}}%
\pgfpathlineto{\pgfqpoint{8.479399in}{6.419509in}}%
\pgfpathlineto{\pgfqpoint{8.480030in}{6.420363in}}%
\pgfpathlineto{\pgfqpoint{8.480948in}{6.426338in}}%
\pgfpathlineto{\pgfqpoint{8.482095in}{6.426765in}}%
\pgfpathlineto{\pgfqpoint{8.483472in}{6.431887in}}%
\pgfpathlineto{\pgfqpoint{8.484046in}{6.433168in}}%
\pgfpathlineto{\pgfqpoint{8.485422in}{6.437009in}}%
\pgfpathlineto{\pgfqpoint{8.487258in}{6.438290in}}%
\pgfpathlineto{\pgfqpoint{8.488577in}{6.442131in}}%
\pgfpathlineto{\pgfqpoint{8.489667in}{6.443412in}}%
\pgfpathlineto{\pgfqpoint{8.490642in}{6.448107in}}%
\pgfpathlineto{\pgfqpoint{8.491789in}{6.449387in}}%
\pgfpathlineto{\pgfqpoint{8.492994in}{6.451521in}}%
\pgfpathlineto{\pgfqpoint{8.494428in}{6.452802in}}%
\pgfpathlineto{\pgfqpoint{8.495805in}{6.454936in}}%
\pgfpathlineto{\pgfqpoint{8.496665in}{6.456217in}}%
\pgfpathlineto{\pgfqpoint{8.498099in}{6.459631in}}%
\pgfpathlineto{\pgfqpoint{8.499648in}{6.460912in}}%
\pgfpathlineto{\pgfqpoint{8.501025in}{6.462619in}}%
\pgfpathlineto{\pgfqpoint{8.501426in}{6.463900in}}%
\pgfpathlineto{\pgfqpoint{8.502631in}{6.466887in}}%
\pgfpathlineto{\pgfqpoint{8.504868in}{6.468168in}}%
\pgfpathlineto{\pgfqpoint{8.506302in}{6.471583in}}%
\pgfpathlineto{\pgfqpoint{8.508711in}{6.472009in}}%
\pgfpathlineto{\pgfqpoint{8.509457in}{6.475424in}}%
\pgfpathlineto{\pgfqpoint{8.511350in}{6.476278in}}%
\pgfpathlineto{\pgfqpoint{8.512669in}{6.480546in}}%
\pgfpathlineto{\pgfqpoint{8.514161in}{6.481827in}}%
\pgfpathlineto{\pgfqpoint{8.515308in}{6.483534in}}%
\pgfpathlineto{\pgfqpoint{8.516799in}{6.484814in}}%
\pgfpathlineto{\pgfqpoint{8.517774in}{6.486522in}}%
\pgfpathlineto{\pgfqpoint{8.519380in}{6.487802in}}%
\pgfpathlineto{\pgfqpoint{8.520470in}{6.489083in}}%
\pgfpathlineto{\pgfqpoint{8.522134in}{6.490363in}}%
\pgfpathlineto{\pgfqpoint{8.523338in}{6.491644in}}%
\pgfpathlineto{\pgfqpoint{8.524658in}{6.492924in}}%
\pgfpathlineto{\pgfqpoint{8.526149in}{6.496766in}}%
\pgfpathlineto{\pgfqpoint{8.528616in}{6.498046in}}%
\pgfpathlineto{\pgfqpoint{8.529189in}{6.499327in}}%
\pgfpathlineto{\pgfqpoint{8.531197in}{6.500607in}}%
\pgfpathlineto{\pgfqpoint{8.532574in}{6.504449in}}%
\pgfpathlineto{\pgfqpoint{8.534065in}{6.505302in}}%
\pgfpathlineto{\pgfqpoint{8.535270in}{6.509998in}}%
\pgfpathlineto{\pgfqpoint{8.538998in}{6.511278in}}%
\pgfpathlineto{\pgfqpoint{8.540203in}{6.513412in}}%
\pgfpathlineto{\pgfqpoint{8.541580in}{6.514266in}}%
\pgfpathlineto{\pgfqpoint{8.542727in}{6.517254in}}%
\pgfpathlineto{\pgfqpoint{8.544562in}{6.518534in}}%
\pgfpathlineto{\pgfqpoint{8.545652in}{6.521095in}}%
\pgfpathlineto{\pgfqpoint{8.547086in}{6.522376in}}%
\pgfpathlineto{\pgfqpoint{8.547775in}{6.524937in}}%
\pgfpathlineto{\pgfqpoint{8.549725in}{6.526217in}}%
\pgfpathlineto{\pgfqpoint{8.550872in}{6.528778in}}%
\pgfpathlineto{\pgfqpoint{8.554199in}{6.530059in}}%
\pgfpathlineto{\pgfqpoint{8.555518in}{6.530912in}}%
\pgfpathlineto{\pgfqpoint{8.558157in}{6.532193in}}%
\pgfpathlineto{\pgfqpoint{8.558501in}{6.533900in}}%
\pgfpathlineto{\pgfqpoint{8.561369in}{6.535181in}}%
\pgfpathlineto{\pgfqpoint{8.562689in}{6.537742in}}%
\pgfpathlineto{\pgfqpoint{8.563434in}{6.538595in}}%
\pgfpathlineto{\pgfqpoint{8.564467in}{6.541583in}}%
\pgfpathlineto{\pgfqpoint{8.566360in}{6.542864in}}%
\pgfpathlineto{\pgfqpoint{8.567622in}{6.544144in}}%
\pgfpathlineto{\pgfqpoint{8.568368in}{6.545425in}}%
\pgfpathlineto{\pgfqpoint{8.569515in}{6.547132in}}%
\pgfpathlineto{\pgfqpoint{8.570891in}{6.548413in}}%
\pgfpathlineto{\pgfqpoint{8.571121in}{6.549693in}}%
\pgfpathlineto{\pgfqpoint{8.573473in}{6.550974in}}%
\pgfpathlineto{\pgfqpoint{8.574964in}{6.553108in}}%
\pgfpathlineto{\pgfqpoint{8.576972in}{6.554388in}}%
\pgfpathlineto{\pgfqpoint{8.578176in}{6.556522in}}%
\pgfpathlineto{\pgfqpoint{8.582536in}{6.557803in}}%
\pgfpathlineto{\pgfqpoint{8.583568in}{6.559937in}}%
\pgfpathlineto{\pgfqpoint{8.586264in}{6.561218in}}%
\pgfpathlineto{\pgfqpoint{8.587756in}{6.565486in}}%
\pgfpathlineto{\pgfqpoint{8.589591in}{6.566766in}}%
\pgfpathlineto{\pgfqpoint{8.590968in}{6.569754in}}%
\pgfpathlineto{\pgfqpoint{8.592287in}{6.571035in}}%
\pgfpathlineto{\pgfqpoint{8.593435in}{6.573596in}}%
\pgfpathlineto{\pgfqpoint{8.595385in}{6.574876in}}%
\pgfpathlineto{\pgfqpoint{8.596704in}{6.577864in}}%
\pgfpathlineto{\pgfqpoint{8.598196in}{6.579145in}}%
\pgfpathlineto{\pgfqpoint{8.598425in}{6.580852in}}%
\pgfpathlineto{\pgfqpoint{8.600949in}{6.582132in}}%
\pgfpathlineto{\pgfqpoint{8.602268in}{6.584267in}}%
\pgfpathlineto{\pgfqpoint{8.603875in}{6.585120in}}%
\pgfpathlineto{\pgfqpoint{8.605194in}{6.587254in}}%
\pgfpathlineto{\pgfqpoint{8.605997in}{6.588108in}}%
\pgfpathlineto{\pgfqpoint{8.607488in}{6.590242in}}%
\pgfpathlineto{\pgfqpoint{8.609381in}{6.591523in}}%
\pgfpathlineto{\pgfqpoint{8.610701in}{6.593230in}}%
\pgfpathlineto{\pgfqpoint{8.612307in}{6.594511in}}%
\pgfpathlineto{\pgfqpoint{8.613741in}{6.598779in}}%
\pgfpathlineto{\pgfqpoint{8.616609in}{6.600059in}}%
\pgfpathlineto{\pgfqpoint{8.616838in}{6.600913in}}%
\pgfpathlineto{\pgfqpoint{8.620739in}{6.602194in}}%
\pgfpathlineto{\pgfqpoint{8.622116in}{6.606035in}}%
\pgfpathlineto{\pgfqpoint{8.624238in}{6.607316in}}%
\pgfpathlineto{\pgfqpoint{8.625156in}{6.609877in}}%
\pgfpathlineto{\pgfqpoint{8.626819in}{6.611157in}}%
\pgfpathlineto{\pgfqpoint{8.627049in}{6.612438in}}%
\pgfpathlineto{\pgfqpoint{8.630433in}{6.613718in}}%
\pgfpathlineto{\pgfqpoint{8.631351in}{6.614999in}}%
\pgfpathlineto{\pgfqpoint{8.631695in}{6.614999in}}%
\pgfpathlineto{\pgfqpoint{8.633760in}{6.616279in}}%
\pgfpathlineto{\pgfqpoint{8.635137in}{6.618840in}}%
\pgfpathlineto{\pgfqpoint{8.637489in}{6.620121in}}%
\pgfpathlineto{\pgfqpoint{8.638636in}{6.622255in}}%
\pgfpathlineto{\pgfqpoint{8.640471in}{6.623108in}}%
\pgfpathlineto{\pgfqpoint{8.641848in}{6.626096in}}%
\pgfpathlineto{\pgfqpoint{8.642479in}{6.627377in}}%
\pgfpathlineto{\pgfqpoint{8.643512in}{6.629938in}}%
\pgfpathlineto{\pgfqpoint{8.645634in}{6.631218in}}%
\pgfpathlineto{\pgfqpoint{8.645634in}{6.631645in}}%
\pgfpathlineto{\pgfqpoint{8.648502in}{6.632926in}}%
\pgfpathlineto{\pgfqpoint{8.649879in}{6.635913in}}%
\pgfpathlineto{\pgfqpoint{8.650854in}{6.636340in}}%
\pgfpathlineto{\pgfqpoint{8.651485in}{6.638048in}}%
\pgfpathlineto{\pgfqpoint{8.653263in}{6.639328in}}%
\pgfpathlineto{\pgfqpoint{8.654582in}{6.641889in}}%
\pgfpathlineto{\pgfqpoint{8.657795in}{6.643170in}}%
\pgfpathlineto{\pgfqpoint{8.659171in}{6.644877in}}%
\pgfpathlineto{\pgfqpoint{8.660835in}{6.646157in}}%
\pgfpathlineto{\pgfqpoint{8.662154in}{6.648718in}}%
\pgfpathlineto{\pgfqpoint{8.664965in}{6.649999in}}%
\pgfpathlineto{\pgfqpoint{8.666456in}{6.651706in}}%
\pgfpathlineto{\pgfqpoint{8.668751in}{6.652987in}}%
\pgfpathlineto{\pgfqpoint{8.670070in}{6.654267in}}%
\pgfpathlineto{\pgfqpoint{8.672709in}{6.655121in}}%
\pgfpathlineto{\pgfqpoint{8.673684in}{6.657682in}}%
\pgfpathlineto{\pgfqpoint{8.677699in}{6.658962in}}%
\pgfpathlineto{\pgfqpoint{8.679191in}{6.661950in}}%
\pgfpathlineto{\pgfqpoint{8.681829in}{6.663231in}}%
\pgfpathlineto{\pgfqpoint{8.683321in}{6.664511in}}%
\pgfpathlineto{\pgfqpoint{8.684640in}{6.665792in}}%
\pgfpathlineto{\pgfqpoint{8.685214in}{6.668780in}}%
\pgfpathlineto{\pgfqpoint{8.687910in}{6.670060in}}%
\pgfpathlineto{\pgfqpoint{8.688196in}{6.670914in}}%
\pgfpathlineto{\pgfqpoint{8.689860in}{6.671767in}}%
\pgfpathlineto{\pgfqpoint{8.691237in}{6.673048in}}%
\pgfpathlineto{\pgfqpoint{8.692671in}{6.674328in}}%
\pgfpathlineto{\pgfqpoint{8.693933in}{6.676463in}}%
\pgfpathlineto{\pgfqpoint{8.695654in}{6.677743in}}%
\pgfpathlineto{\pgfqpoint{8.696571in}{6.679877in}}%
\pgfpathlineto{\pgfqpoint{8.699153in}{6.681158in}}%
\pgfpathlineto{\pgfqpoint{8.700472in}{6.683292in}}%
\pgfpathlineto{\pgfqpoint{8.701906in}{6.684146in}}%
\pgfpathlineto{\pgfqpoint{8.702537in}{6.685426in}}%
\pgfpathlineto{\pgfqpoint{8.704545in}{6.686707in}}%
\pgfpathlineto{\pgfqpoint{8.705405in}{6.687987in}}%
\pgfpathlineto{\pgfqpoint{8.706954in}{6.689268in}}%
\pgfpathlineto{\pgfqpoint{8.708330in}{6.690548in}}%
\pgfpathlineto{\pgfqpoint{8.709879in}{6.691829in}}%
\pgfpathlineto{\pgfqpoint{8.710625in}{6.692682in}}%
\pgfpathlineto{\pgfqpoint{8.717738in}{6.693963in}}%
\pgfpathlineto{\pgfqpoint{8.719000in}{6.695243in}}%
\pgfpathlineto{\pgfqpoint{8.720950in}{6.696524in}}%
\pgfpathlineto{\pgfqpoint{8.722442in}{6.697377in}}%
\pgfpathlineto{\pgfqpoint{8.725539in}{6.698658in}}%
\pgfpathlineto{\pgfqpoint{8.727030in}{6.701219in}}%
\pgfpathlineto{\pgfqpoint{8.728178in}{6.702499in}}%
\pgfpathlineto{\pgfqpoint{8.729325in}{6.705060in}}%
\pgfpathlineto{\pgfqpoint{8.732709in}{6.706341in}}%
\pgfpathlineto{\pgfqpoint{8.733799in}{6.707195in}}%
\pgfpathlineto{\pgfqpoint{8.735864in}{6.708475in}}%
\pgfpathlineto{\pgfqpoint{8.737356in}{6.710609in}}%
\pgfpathlineto{\pgfqpoint{8.740338in}{6.711890in}}%
\pgfpathlineto{\pgfqpoint{8.741486in}{6.713170in}}%
\pgfpathlineto{\pgfqpoint{8.746648in}{6.714451in}}%
\pgfpathlineto{\pgfqpoint{8.748140in}{6.716158in}}%
\pgfpathlineto{\pgfqpoint{8.750434in}{6.717439in}}%
\pgfpathlineto{\pgfqpoint{8.751926in}{6.720000in}}%
\pgfpathlineto{\pgfqpoint{8.753991in}{6.721280in}}%
\pgfpathlineto{\pgfqpoint{8.753991in}{6.721707in}}%
\pgfpathlineto{\pgfqpoint{8.761046in}{6.722987in}}%
\pgfpathlineto{\pgfqpoint{8.761734in}{6.724268in}}%
\pgfpathlineto{\pgfqpoint{8.766151in}{6.725548in}}%
\pgfpathlineto{\pgfqpoint{8.767012in}{6.726829in}}%
\pgfpathlineto{\pgfqpoint{8.769937in}{6.728109in}}%
\pgfpathlineto{\pgfqpoint{8.771429in}{6.729390in}}%
\pgfpathlineto{\pgfqpoint{8.772920in}{6.730670in}}%
\pgfpathlineto{\pgfqpoint{8.773952in}{6.732378in}}%
\pgfpathlineto{\pgfqpoint{8.775673in}{6.733231in}}%
\pgfpathlineto{\pgfqpoint{8.776648in}{6.735366in}}%
\pgfpathlineto{\pgfqpoint{8.779861in}{6.736646in}}%
\pgfpathlineto{\pgfqpoint{8.779861in}{6.737073in}}%
\pgfpathlineto{\pgfqpoint{8.781811in}{6.738353in}}%
\pgfpathlineto{\pgfqpoint{8.783188in}{6.740488in}}%
\pgfpathlineto{\pgfqpoint{8.787260in}{6.741768in}}%
\pgfpathlineto{\pgfqpoint{8.787777in}{6.743902in}}%
\pgfpathlineto{\pgfqpoint{8.791964in}{6.745183in}}%
\pgfpathlineto{\pgfqpoint{8.792366in}{6.746890in}}%
\pgfpathlineto{\pgfqpoint{8.795979in}{6.748171in}}%
\pgfpathlineto{\pgfqpoint{8.797241in}{6.749451in}}%
\pgfpathlineto{\pgfqpoint{8.799306in}{6.750732in}}%
\pgfpathlineto{\pgfqpoint{8.799306in}{6.751158in}}%
\pgfpathlineto{\pgfqpoint{8.803436in}{6.752439in}}%
\pgfpathlineto{\pgfqpoint{8.804469in}{6.754573in}}%
\pgfpathlineto{\pgfqpoint{8.808140in}{6.755854in}}%
\pgfpathlineto{\pgfqpoint{8.808886in}{6.757134in}}%
\pgfpathlineto{\pgfqpoint{8.811525in}{6.758415in}}%
\pgfpathlineto{\pgfqpoint{8.812844in}{6.760549in}}%
\pgfpathlineto{\pgfqpoint{8.814048in}{6.761402in}}%
\pgfpathlineto{\pgfqpoint{8.815482in}{6.764390in}}%
\pgfpathlineto{\pgfqpoint{8.816974in}{6.765671in}}%
\pgfpathlineto{\pgfqpoint{8.818465in}{6.766951in}}%
\pgfpathlineto{\pgfqpoint{8.819670in}{6.768232in}}%
\pgfpathlineto{\pgfqpoint{8.820989in}{6.769512in}}%
\pgfpathlineto{\pgfqpoint{8.822538in}{6.770793in}}%
\pgfpathlineto{\pgfqpoint{8.823685in}{6.772073in}}%
\pgfpathlineto{\pgfqpoint{8.826094in}{6.773354in}}%
\pgfpathlineto{\pgfqpoint{8.826840in}{6.774207in}}%
\pgfpathlineto{\pgfqpoint{8.828733in}{6.775488in}}%
\pgfpathlineto{\pgfqpoint{8.828733in}{6.775915in}}%
\pgfpathlineto{\pgfqpoint{8.834584in}{6.777195in}}%
\pgfpathlineto{\pgfqpoint{8.836018in}{6.778476in}}%
\pgfpathlineto{\pgfqpoint{8.836993in}{6.779756in}}%
\pgfpathlineto{\pgfqpoint{8.838083in}{6.781464in}}%
\pgfpathlineto{\pgfqpoint{8.842500in}{6.782744in}}%
\pgfpathlineto{\pgfqpoint{8.843934in}{6.785305in}}%
\pgfpathlineto{\pgfqpoint{8.845311in}{6.786586in}}%
\pgfpathlineto{\pgfqpoint{8.846515in}{6.788720in}}%
\pgfpathlineto{\pgfqpoint{8.849555in}{6.790000in}}%
\pgfpathlineto{\pgfqpoint{8.849785in}{6.790854in}}%
\pgfpathlineto{\pgfqpoint{8.852309in}{6.792134in}}%
\pgfpathlineto{\pgfqpoint{8.853513in}{6.794269in}}%
\pgfpathlineto{\pgfqpoint{8.858733in}{6.795549in}}%
\pgfpathlineto{\pgfqpoint{8.858733in}{6.795976in}}%
\pgfpathlineto{\pgfqpoint{8.865445in}{6.797256in}}%
\pgfpathlineto{\pgfqpoint{8.866764in}{6.798110in}}%
\pgfpathlineto{\pgfqpoint{8.868542in}{6.799391in}}%
\pgfpathlineto{\pgfqpoint{8.869976in}{6.800671in}}%
\pgfpathlineto{\pgfqpoint{8.870837in}{6.801952in}}%
\pgfpathlineto{\pgfqpoint{8.870837in}{6.802378in}}%
\pgfpathlineto{\pgfqpoint{8.875770in}{6.803659in}}%
\pgfpathlineto{\pgfqpoint{8.876401in}{6.804513in}}%
\pgfpathlineto{\pgfqpoint{8.880301in}{6.805793in}}%
\pgfpathlineto{\pgfqpoint{8.881047in}{6.807500in}}%
\pgfpathlineto{\pgfqpoint{8.883743in}{6.808781in}}%
\pgfpathlineto{\pgfqpoint{8.884718in}{6.810061in}}%
\pgfpathlineto{\pgfqpoint{8.887931in}{6.811342in}}%
\pgfpathlineto{\pgfqpoint{8.888791in}{6.813049in}}%
\pgfpathlineto{\pgfqpoint{8.891258in}{6.814330in}}%
\pgfpathlineto{\pgfqpoint{8.891315in}{6.815183in}}%
\pgfpathlineto{\pgfqpoint{8.893954in}{6.816464in}}%
\pgfpathlineto{\pgfqpoint{8.894527in}{6.817744in}}%
\pgfpathlineto{\pgfqpoint{8.898542in}{6.819025in}}%
\pgfpathlineto{\pgfqpoint{8.899804in}{6.820732in}}%
\pgfpathlineto{\pgfqpoint{8.903304in}{6.822013in}}%
\pgfpathlineto{\pgfqpoint{8.904565in}{6.822866in}}%
\pgfpathlineto{\pgfqpoint{8.907319in}{6.824147in}}%
\pgfpathlineto{\pgfqpoint{8.908753in}{6.825854in}}%
\pgfpathlineto{\pgfqpoint{8.912481in}{6.827135in}}%
\pgfpathlineto{\pgfqpoint{8.913629in}{6.828415in}}%
\pgfpathlineto{\pgfqpoint{8.916898in}{6.829696in}}%
\pgfpathlineto{\pgfqpoint{8.918218in}{6.831403in}}%
\pgfpathlineto{\pgfqpoint{8.919881in}{6.832684in}}%
\pgfpathlineto{\pgfqpoint{8.921373in}{6.833964in}}%
\pgfpathlineto{\pgfqpoint{8.925330in}{6.835245in}}%
\pgfpathlineto{\pgfqpoint{8.925789in}{6.836098in}}%
\pgfpathlineto{\pgfqpoint{8.930952in}{6.837379in}}%
\pgfpathlineto{\pgfqpoint{8.930952in}{6.837806in}}%
\pgfpathlineto{\pgfqpoint{8.933992in}{6.839086in}}%
\pgfpathlineto{\pgfqpoint{8.935311in}{6.841220in}}%
\pgfpathlineto{\pgfqpoint{8.937549in}{6.842501in}}%
\pgfpathlineto{\pgfqpoint{8.938409in}{6.843781in}}%
\pgfpathlineto{\pgfqpoint{8.942195in}{6.845062in}}%
\pgfpathlineto{\pgfqpoint{8.943342in}{6.846342in}}%
\pgfpathlineto{\pgfqpoint{8.946612in}{6.847623in}}%
\pgfpathlineto{\pgfqpoint{8.947874in}{6.849330in}}%
\pgfpathlineto{\pgfqpoint{8.952118in}{6.850611in}}%
\pgfpathlineto{\pgfqpoint{8.952118in}{6.851037in}}%
\pgfpathlineto{\pgfqpoint{8.958543in}{6.851891in}}%
\pgfpathlineto{\pgfqpoint{8.959231in}{6.854879in}}%
\pgfpathlineto{\pgfqpoint{8.962386in}{6.856159in}}%
\pgfpathlineto{\pgfqpoint{8.963706in}{6.857013in}}%
\pgfpathlineto{\pgfqpoint{8.970302in}{6.858294in}}%
\pgfpathlineto{\pgfqpoint{8.970302in}{6.858720in}}%
\pgfpathlineto{\pgfqpoint{8.974031in}{6.860001in}}%
\pgfpathlineto{\pgfqpoint{8.974662in}{6.860855in}}%
\pgfpathlineto{\pgfqpoint{8.979996in}{6.862135in}}%
\pgfpathlineto{\pgfqpoint{8.980570in}{6.862989in}}%
\pgfpathlineto{\pgfqpoint{8.983438in}{6.863842in}}%
\pgfpathlineto{\pgfqpoint{8.984815in}{6.865123in}}%
\pgfpathlineto{\pgfqpoint{8.987396in}{6.866403in}}%
\pgfpathlineto{\pgfqpoint{8.988429in}{6.867257in}}%
\pgfpathlineto{\pgfqpoint{8.991182in}{6.868538in}}%
\pgfpathlineto{\pgfqpoint{8.992329in}{6.869818in}}%
\pgfpathlineto{\pgfqpoint{8.996459in}{6.871099in}}%
\pgfpathlineto{\pgfqpoint{8.996459in}{6.871525in}}%
\pgfpathlineto{\pgfqpoint{8.998811in}{6.872806in}}%
\pgfpathlineto{\pgfqpoint{8.998926in}{6.873660in}}%
\pgfpathlineto{\pgfqpoint{9.003457in}{6.874940in}}%
\pgfpathlineto{\pgfqpoint{9.003457in}{6.875367in}}%
\pgfpathlineto{\pgfqpoint{9.006440in}{6.876647in}}%
\pgfpathlineto{\pgfqpoint{9.007645in}{6.879208in}}%
\pgfpathlineto{\pgfqpoint{9.011316in}{6.880489in}}%
\pgfpathlineto{\pgfqpoint{9.012521in}{6.882196in}}%
\pgfpathlineto{\pgfqpoint{9.018027in}{6.883477in}}%
\pgfpathlineto{\pgfqpoint{9.018773in}{6.884757in}}%
\pgfpathlineto{\pgfqpoint{9.021871in}{6.886038in}}%
\pgfpathlineto{\pgfqpoint{9.022215in}{6.886891in}}%
\pgfpathlineto{\pgfqpoint{9.027148in}{6.888172in}}%
\pgfpathlineto{\pgfqpoint{9.028180in}{6.889026in}}%
\pgfpathlineto{\pgfqpoint{9.035064in}{6.890306in}}%
\pgfpathlineto{\pgfqpoint{9.035121in}{6.891160in}}%
\pgfpathlineto{\pgfqpoint{9.039882in}{6.892440in}}%
\pgfpathlineto{\pgfqpoint{9.041201in}{6.893721in}}%
\pgfpathlineto{\pgfqpoint{9.045217in}{6.895001in}}%
\pgfpathlineto{\pgfqpoint{9.046651in}{6.896282in}}%
\pgfpathlineto{\pgfqpoint{9.053133in}{6.897562in}}%
\pgfpathlineto{\pgfqpoint{9.053592in}{6.898843in}}%
\pgfpathlineto{\pgfqpoint{9.057148in}{6.900123in}}%
\pgfpathlineto{\pgfqpoint{9.057836in}{6.900977in}}%
\pgfpathlineto{\pgfqpoint{9.060934in}{6.902257in}}%
\pgfpathlineto{\pgfqpoint{9.061966in}{6.903111in}}%
\pgfpathlineto{\pgfqpoint{9.066842in}{6.904392in}}%
\pgfpathlineto{\pgfqpoint{9.067531in}{6.905245in}}%
\pgfpathlineto{\pgfqpoint{9.071775in}{6.906526in}}%
\pgfpathlineto{\pgfqpoint{9.073095in}{6.908233in}}%
\pgfpathlineto{\pgfqpoint{9.080437in}{6.909514in}}%
\pgfpathlineto{\pgfqpoint{9.080437in}{6.909940in}}%
\pgfpathlineto{\pgfqpoint{9.090303in}{6.911221in}}%
\pgfpathlineto{\pgfqpoint{9.091565in}{6.912928in}}%
\pgfpathlineto{\pgfqpoint{9.096670in}{6.914209in}}%
\pgfpathlineto{\pgfqpoint{9.096670in}{6.914636in}}%
\pgfpathlineto{\pgfqpoint{9.101776in}{6.915916in}}%
\pgfpathlineto{\pgfqpoint{9.103210in}{6.917197in}}%
\pgfpathlineto{\pgfqpoint{9.106365in}{6.918477in}}%
\pgfpathlineto{\pgfqpoint{9.106823in}{6.919331in}}%
\pgfpathlineto{\pgfqpoint{9.116862in}{6.920611in}}%
\pgfpathlineto{\pgfqpoint{9.116862in}{6.921038in}}%
\pgfpathlineto{\pgfqpoint{9.121795in}{6.921892in}}%
\pgfpathlineto{\pgfqpoint{9.122598in}{6.923172in}}%
\pgfpathlineto{\pgfqpoint{9.131202in}{6.924453in}}%
\pgfpathlineto{\pgfqpoint{9.132235in}{6.926587in}}%
\pgfpathlineto{\pgfqpoint{9.135160in}{6.927867in}}%
\pgfpathlineto{\pgfqpoint{9.136365in}{6.928721in}}%
\pgfpathlineto{\pgfqpoint{9.140782in}{6.930002in}}%
\pgfpathlineto{\pgfqpoint{9.141470in}{6.931709in}}%
\pgfpathlineto{\pgfqpoint{9.146002in}{6.932989in}}%
\pgfpathlineto{\pgfqpoint{9.146002in}{6.933416in}}%
\pgfpathlineto{\pgfqpoint{9.153344in}{6.934697in}}%
\pgfpathlineto{\pgfqpoint{9.153516in}{6.935550in}}%
\pgfpathlineto{\pgfqpoint{9.157417in}{6.936831in}}%
\pgfpathlineto{\pgfqpoint{9.157417in}{6.937258in}}%
\pgfpathlineto{\pgfqpoint{9.161604in}{6.938538in}}%
\pgfpathlineto{\pgfqpoint{9.162292in}{6.939392in}}%
\pgfpathlineto{\pgfqpoint{9.163038in}{6.939392in}}%
\pgfpathlineto{\pgfqpoint{9.167742in}{6.940672in}}%
\pgfpathlineto{\pgfqpoint{9.169061in}{6.941953in}}%
\pgfpathlineto{\pgfqpoint{9.172101in}{6.943233in}}%
\pgfpathlineto{\pgfqpoint{9.172618in}{6.944087in}}%
\pgfpathlineto{\pgfqpoint{9.180935in}{6.945368in}}%
\pgfpathlineto{\pgfqpoint{9.181164in}{6.946648in}}%
\pgfpathlineto{\pgfqpoint{9.183746in}{6.947929in}}%
\pgfpathlineto{\pgfqpoint{9.184664in}{6.949636in}}%
\pgfpathlineto{\pgfqpoint{9.191662in}{6.950916in}}%
\pgfpathlineto{\pgfqpoint{9.192809in}{6.952624in}}%
\pgfpathlineto{\pgfqpoint{9.196595in}{6.953904in}}%
\pgfpathlineto{\pgfqpoint{9.197627in}{6.954758in}}%
\pgfpathlineto{\pgfqpoint{9.200037in}{6.956038in}}%
\pgfpathlineto{\pgfqpoint{9.201471in}{6.956892in}}%
\pgfpathlineto{\pgfqpoint{9.208698in}{6.958173in}}%
\pgfpathlineto{\pgfqpoint{9.208698in}{6.958599in}}%
\pgfpathlineto{\pgfqpoint{9.216729in}{6.959880in}}%
\pgfpathlineto{\pgfqpoint{9.217819in}{6.961160in}}%
\pgfpathlineto{\pgfqpoint{9.218048in}{6.961160in}}%
\pgfpathlineto{\pgfqpoint{9.223211in}{6.962441in}}%
\pgfpathlineto{\pgfqpoint{9.224071in}{6.964148in}}%
\pgfpathlineto{\pgfqpoint{9.234396in}{6.965429in}}%
\pgfpathlineto{\pgfqpoint{9.234511in}{6.966282in}}%
\pgfpathlineto{\pgfqpoint{9.242829in}{6.967563in}}%
\pgfpathlineto{\pgfqpoint{9.242829in}{6.967990in}}%
\pgfpathlineto{\pgfqpoint{9.247245in}{6.969270in}}%
\pgfpathlineto{\pgfqpoint{9.248106in}{6.970124in}}%
\pgfpathlineto{\pgfqpoint{9.254645in}{6.971404in}}%
\pgfpathlineto{\pgfqpoint{9.256136in}{6.973112in}}%
\pgfpathlineto{\pgfqpoint{9.260267in}{6.974392in}}%
\pgfpathlineto{\pgfqpoint{9.261528in}{6.975673in}}%
\pgfpathlineto{\pgfqpoint{9.274550in}{6.976953in}}%
\pgfpathlineto{\pgfqpoint{9.274664in}{6.977807in}}%
\pgfpathlineto{\pgfqpoint{9.284129in}{6.979087in}}%
\pgfpathlineto{\pgfqpoint{9.285448in}{6.979941in}}%
\pgfpathlineto{\pgfqpoint{9.289980in}{6.981222in}}%
\pgfpathlineto{\pgfqpoint{9.289980in}{6.982075in}}%
\pgfpathlineto{\pgfqpoint{9.293479in}{6.983356in}}%
\pgfpathlineto{\pgfqpoint{9.293479in}{6.983783in}}%
\pgfpathlineto{\pgfqpoint{9.305525in}{6.985063in}}%
\pgfpathlineto{\pgfqpoint{9.305525in}{6.985490in}}%
\pgfpathlineto{\pgfqpoint{9.313384in}{6.986770in}}%
\pgfpathlineto{\pgfqpoint{9.313498in}{6.988051in}}%
\pgfpathlineto{\pgfqpoint{9.317170in}{6.989331in}}%
\pgfpathlineto{\pgfqpoint{9.317571in}{6.990185in}}%
\pgfpathlineto{\pgfqpoint{9.322160in}{6.991466in}}%
\pgfpathlineto{\pgfqpoint{9.322160in}{6.991892in}}%
\pgfpathlineto{\pgfqpoint{9.329560in}{6.993173in}}%
\pgfpathlineto{\pgfqpoint{9.329560in}{6.993600in}}%
\pgfpathlineto{\pgfqpoint{9.342179in}{6.994880in}}%
\pgfpathlineto{\pgfqpoint{9.342179in}{6.995307in}}%
\pgfpathlineto{\pgfqpoint{9.349522in}{6.996588in}}%
\pgfpathlineto{\pgfqpoint{9.351013in}{6.997868in}}%
\pgfpathlineto{\pgfqpoint{9.354684in}{6.999149in}}%
\pgfpathlineto{\pgfqpoint{9.355774in}{7.000002in}}%
\pgfpathlineto{\pgfqpoint{9.364321in}{7.001283in}}%
\pgfpathlineto{\pgfqpoint{9.364780in}{7.002136in}}%
\pgfpathlineto{\pgfqpoint{9.371319in}{7.003417in}}%
\pgfpathlineto{\pgfqpoint{9.371319in}{7.003844in}}%
\pgfpathlineto{\pgfqpoint{9.386405in}{7.005124in}}%
\pgfpathlineto{\pgfqpoint{9.386405in}{7.005551in}}%
\pgfpathlineto{\pgfqpoint{9.396329in}{7.006405in}}%
\pgfpathlineto{\pgfqpoint{9.396903in}{7.007685in}}%
\pgfpathlineto{\pgfqpoint{9.409293in}{7.008966in}}%
\pgfpathlineto{\pgfqpoint{9.409293in}{7.009393in}}%
\pgfpathlineto{\pgfqpoint{9.420134in}{7.010246in}}%
\pgfpathlineto{\pgfqpoint{9.420191in}{7.011527in}}%
\pgfpathlineto{\pgfqpoint{9.420364in}{7.011527in}}%
\pgfpathlineto{\pgfqpoint{9.428452in}{7.012807in}}%
\pgfpathlineto{\pgfqpoint{9.429656in}{7.013661in}}%
\pgfpathlineto{\pgfqpoint{9.449331in}{7.014941in}}%
\pgfpathlineto{\pgfqpoint{9.450306in}{7.015795in}}%
\pgfpathlineto{\pgfqpoint{9.462295in}{7.017076in}}%
\pgfpathlineto{\pgfqpoint{9.463098in}{7.017929in}}%
\pgfpathlineto{\pgfqpoint{9.482028in}{7.019210in}}%
\pgfpathlineto{\pgfqpoint{9.482028in}{7.019637in}}%
\pgfpathlineto{\pgfqpoint{9.495336in}{7.020917in}}%
\pgfpathlineto{\pgfqpoint{9.495336in}{7.021344in}}%
\pgfpathlineto{\pgfqpoint{9.496311in}{7.021344in}}%
\pgfpathlineto{\pgfqpoint{9.521607in}{7.022624in}}%
\pgfpathlineto{\pgfqpoint{9.521607in}{7.023051in}}%
\pgfpathlineto{\pgfqpoint{9.527286in}{7.024332in}}%
\pgfpathlineto{\pgfqpoint{9.527286in}{7.024759in}}%
\pgfpathlineto{\pgfqpoint{9.539562in}{7.026039in}}%
\pgfpathlineto{\pgfqpoint{9.539562in}{7.026466in}}%
\pgfpathlineto{\pgfqpoint{9.554304in}{7.027746in}}%
\pgfpathlineto{\pgfqpoint{9.554304in}{7.028173in}}%
\pgfpathlineto{\pgfqpoint{9.591187in}{7.029454in}}%
\pgfpathlineto{\pgfqpoint{9.591933in}{7.030307in}}%
\pgfpathlineto{\pgfqpoint{9.599849in}{7.031588in}}%
\pgfpathlineto{\pgfqpoint{9.599849in}{7.032015in}}%
\pgfpathlineto{\pgfqpoint{9.709984in}{7.033295in}}%
\pgfpathlineto{\pgfqpoint{9.709984in}{7.033722in}}%
\pgfpathlineto{\pgfqpoint{9.761150in}{7.034149in}}%
\pgfpathlineto{\pgfqpoint{9.761150in}{7.034149in}}%
\pgfusepath{stroke}%
\end{pgfscope}%
\begin{pgfscope}%
\pgfpathrectangle{\pgfqpoint{8.150541in}{5.564583in}}{\pgfqpoint{1.687305in}{1.539545in}}%
\pgfusepath{clip}%
\pgfsetrectcap%
\pgfsetroundjoin%
\pgfsetlinewidth{1.505625pt}%
\definecolor{currentstroke}{rgb}{0.501961,0.501961,0.501961}%
\pgfsetstrokecolor{currentstroke}%
\pgfsetdash{}{0pt}%
\pgfpathmoveto{\pgfqpoint{8.227236in}{5.634562in}}%
\pgfpathlineto{\pgfqpoint{9.761150in}{7.034149in}}%
\pgfusepath{stroke}%
\end{pgfscope}%
\begin{pgfscope}%
\pgfsetrectcap%
\pgfsetmiterjoin%
\pgfsetlinewidth{0.803000pt}%
\definecolor{currentstroke}{rgb}{0.000000,0.000000,0.000000}%
\pgfsetstrokecolor{currentstroke}%
\pgfsetdash{}{0pt}%
\pgfpathmoveto{\pgfqpoint{8.150541in}{5.564583in}}%
\pgfpathlineto{\pgfqpoint{8.150541in}{7.104128in}}%
\pgfusepath{stroke}%
\end{pgfscope}%
\begin{pgfscope}%
\pgfsetrectcap%
\pgfsetmiterjoin%
\pgfsetlinewidth{0.803000pt}%
\definecolor{currentstroke}{rgb}{0.000000,0.000000,0.000000}%
\pgfsetstrokecolor{currentstroke}%
\pgfsetdash{}{0pt}%
\pgfpathmoveto{\pgfqpoint{9.837846in}{5.564583in}}%
\pgfpathlineto{\pgfqpoint{9.837846in}{7.104128in}}%
\pgfusepath{stroke}%
\end{pgfscope}%
\begin{pgfscope}%
\pgfsetrectcap%
\pgfsetmiterjoin%
\pgfsetlinewidth{0.803000pt}%
\definecolor{currentstroke}{rgb}{0.000000,0.000000,0.000000}%
\pgfsetstrokecolor{currentstroke}%
\pgfsetdash{}{0pt}%
\pgfpathmoveto{\pgfqpoint{8.150541in}{5.564583in}}%
\pgfpathlineto{\pgfqpoint{9.837846in}{5.564583in}}%
\pgfusepath{stroke}%
\end{pgfscope}%
\begin{pgfscope}%
\pgfsetrectcap%
\pgfsetmiterjoin%
\pgfsetlinewidth{0.803000pt}%
\definecolor{currentstroke}{rgb}{0.000000,0.000000,0.000000}%
\pgfsetstrokecolor{currentstroke}%
\pgfsetdash{}{0pt}%
\pgfpathmoveto{\pgfqpoint{8.150541in}{7.104128in}}%
\pgfpathlineto{\pgfqpoint{9.837846in}{7.104128in}}%
\pgfusepath{stroke}%
\end{pgfscope}%
\begin{pgfscope}%
\definecolor{textcolor}{rgb}{0.000000,0.000000,0.000000}%
\pgfsetstrokecolor{textcolor}%
\pgfsetfillcolor{textcolor}%
\pgftext[x=8.994193in,y=7.187462in,,base]{\color{textcolor}\rmfamily\fontsize{20.000000}{24.000000}\selectfont Atelectasis}%
\end{pgfscope}%
\begin{pgfscope}%
\pgfsetbuttcap%
\pgfsetmiterjoin%
\definecolor{currentfill}{rgb}{1.000000,1.000000,1.000000}%
\pgfsetfillcolor{currentfill}%
\pgfsetfillopacity{0.800000}%
\pgfsetlinewidth{1.003750pt}%
\definecolor{currentstroke}{rgb}{0.800000,0.800000,0.800000}%
\pgfsetstrokecolor{currentstroke}%
\pgfsetstrokeopacity{0.800000}%
\pgfsetdash{}{0pt}%
\pgfpathmoveto{\pgfqpoint{8.628740in}{5.634027in}}%
\pgfpathlineto{\pgfqpoint{9.740624in}{5.634027in}}%
\pgfpathquadraticcurveto{\pgfqpoint{9.768402in}{5.634027in}}{\pgfqpoint{9.768402in}{5.661805in}}%
\pgfpathlineto{\pgfqpoint{9.768402in}{5.841589in}}%
\pgfpathquadraticcurveto{\pgfqpoint{9.768402in}{5.869367in}}{\pgfqpoint{9.740624in}{5.869367in}}%
\pgfpathlineto{\pgfqpoint{8.628740in}{5.869367in}}%
\pgfpathquadraticcurveto{\pgfqpoint{8.600962in}{5.869367in}}{\pgfqpoint{8.600962in}{5.841589in}}%
\pgfpathlineto{\pgfqpoint{8.600962in}{5.661805in}}%
\pgfpathquadraticcurveto{\pgfqpoint{8.600962in}{5.634027in}}{\pgfqpoint{8.628740in}{5.634027in}}%
\pgfpathclose%
\pgfusepath{stroke,fill}%
\end{pgfscope}%
\begin{pgfscope}%
\pgfsetrectcap%
\pgfsetroundjoin%
\pgfsetlinewidth{1.505625pt}%
\definecolor{currentstroke}{rgb}{0.000000,0.501961,0.000000}%
\pgfsetstrokecolor{currentstroke}%
\pgfsetdash{}{0pt}%
\pgfpathmoveto{\pgfqpoint{8.656518in}{5.765200in}}%
\pgfpathlineto{\pgfqpoint{8.934295in}{5.765200in}}%
\pgfusepath{stroke}%
\end{pgfscope}%
\begin{pgfscope}%
\definecolor{textcolor}{rgb}{0.000000,0.000000,0.000000}%
\pgfsetstrokecolor{textcolor}%
\pgfsetfillcolor{textcolor}%
\pgftext[x=9.045407in,y=5.716589in,left,base]{\color{textcolor}\rmfamily\fontsize{10.000000}{12.000000}\selectfont AUC 0.794}%
\end{pgfscope}%
\begin{pgfscope}%
\pgfsetbuttcap%
\pgfsetmiterjoin%
\definecolor{currentfill}{rgb}{1.000000,1.000000,1.000000}%
\pgfsetfillcolor{currentfill}%
\pgfsetlinewidth{0.000000pt}%
\definecolor{currentstroke}{rgb}{0.000000,0.000000,0.000000}%
\pgfsetstrokecolor{currentstroke}%
\pgfsetstrokeopacity{0.000000}%
\pgfsetdash{}{0pt}%
\pgfpathmoveto{\pgfqpoint{0.763041in}{3.102083in}}%
\pgfpathlineto{\pgfqpoint{2.450346in}{3.102083in}}%
\pgfpathlineto{\pgfqpoint{2.450346in}{4.641628in}}%
\pgfpathlineto{\pgfqpoint{0.763041in}{4.641628in}}%
\pgfpathclose%
\pgfusepath{fill}%
\end{pgfscope}%
\begin{pgfscope}%
\pgfsetbuttcap%
\pgfsetroundjoin%
\definecolor{currentfill}{rgb}{0.000000,0.000000,0.000000}%
\pgfsetfillcolor{currentfill}%
\pgfsetlinewidth{0.803000pt}%
\definecolor{currentstroke}{rgb}{0.000000,0.000000,0.000000}%
\pgfsetstrokecolor{currentstroke}%
\pgfsetdash{}{0pt}%
\pgfsys@defobject{currentmarker}{\pgfqpoint{0.000000in}{-0.048611in}}{\pgfqpoint{0.000000in}{0.000000in}}{%
\pgfpathmoveto{\pgfqpoint{0.000000in}{0.000000in}}%
\pgfpathlineto{\pgfqpoint{0.000000in}{-0.048611in}}%
\pgfusepath{stroke,fill}%
}%
\begin{pgfscope}%
\pgfsys@transformshift{0.839736in}{3.102083in}%
\pgfsys@useobject{currentmarker}{}%
\end{pgfscope}%
\end{pgfscope}%
\begin{pgfscope}%
\definecolor{textcolor}{rgb}{0.000000,0.000000,0.000000}%
\pgfsetstrokecolor{textcolor}%
\pgfsetfillcolor{textcolor}%
\pgftext[x=0.839736in,y=3.004861in,,top]{\color{textcolor}\rmfamily\fontsize{10.000000}{12.000000}\selectfont \(\displaystyle {0.0}\)}%
\end{pgfscope}%
\begin{pgfscope}%
\pgfsetbuttcap%
\pgfsetroundjoin%
\definecolor{currentfill}{rgb}{0.000000,0.000000,0.000000}%
\pgfsetfillcolor{currentfill}%
\pgfsetlinewidth{0.803000pt}%
\definecolor{currentstroke}{rgb}{0.000000,0.000000,0.000000}%
\pgfsetstrokecolor{currentstroke}%
\pgfsetdash{}{0pt}%
\pgfsys@defobject{currentmarker}{\pgfqpoint{0.000000in}{-0.048611in}}{\pgfqpoint{0.000000in}{0.000000in}}{%
\pgfpathmoveto{\pgfqpoint{0.000000in}{0.000000in}}%
\pgfpathlineto{\pgfqpoint{0.000000in}{-0.048611in}}%
\pgfusepath{stroke,fill}%
}%
\begin{pgfscope}%
\pgfsys@transformshift{1.606693in}{3.102083in}%
\pgfsys@useobject{currentmarker}{}%
\end{pgfscope}%
\end{pgfscope}%
\begin{pgfscope}%
\definecolor{textcolor}{rgb}{0.000000,0.000000,0.000000}%
\pgfsetstrokecolor{textcolor}%
\pgfsetfillcolor{textcolor}%
\pgftext[x=1.606693in,y=3.004861in,,top]{\color{textcolor}\rmfamily\fontsize{10.000000}{12.000000}\selectfont \(\displaystyle {0.5}\)}%
\end{pgfscope}%
\begin{pgfscope}%
\pgfsetbuttcap%
\pgfsetroundjoin%
\definecolor{currentfill}{rgb}{0.000000,0.000000,0.000000}%
\pgfsetfillcolor{currentfill}%
\pgfsetlinewidth{0.803000pt}%
\definecolor{currentstroke}{rgb}{0.000000,0.000000,0.000000}%
\pgfsetstrokecolor{currentstroke}%
\pgfsetdash{}{0pt}%
\pgfsys@defobject{currentmarker}{\pgfqpoint{0.000000in}{-0.048611in}}{\pgfqpoint{0.000000in}{0.000000in}}{%
\pgfpathmoveto{\pgfqpoint{0.000000in}{0.000000in}}%
\pgfpathlineto{\pgfqpoint{0.000000in}{-0.048611in}}%
\pgfusepath{stroke,fill}%
}%
\begin{pgfscope}%
\pgfsys@transformshift{2.373650in}{3.102083in}%
\pgfsys@useobject{currentmarker}{}%
\end{pgfscope}%
\end{pgfscope}%
\begin{pgfscope}%
\definecolor{textcolor}{rgb}{0.000000,0.000000,0.000000}%
\pgfsetstrokecolor{textcolor}%
\pgfsetfillcolor{textcolor}%
\pgftext[x=2.373650in,y=3.004861in,,top]{\color{textcolor}\rmfamily\fontsize{10.000000}{12.000000}\selectfont \(\displaystyle {1.0}\)}%
\end{pgfscope}%
\begin{pgfscope}%
\definecolor{textcolor}{rgb}{0.000000,0.000000,0.000000}%
\pgfsetstrokecolor{textcolor}%
\pgfsetfillcolor{textcolor}%
\pgftext[x=1.606693in,y=2.825849in,,top]{\color{textcolor}\rmfamily\fontsize{16.000000}{19.200000}\selectfont FPR}%
\end{pgfscope}%
\begin{pgfscope}%
\pgfsetbuttcap%
\pgfsetroundjoin%
\definecolor{currentfill}{rgb}{0.000000,0.000000,0.000000}%
\pgfsetfillcolor{currentfill}%
\pgfsetlinewidth{0.803000pt}%
\definecolor{currentstroke}{rgb}{0.000000,0.000000,0.000000}%
\pgfsetstrokecolor{currentstroke}%
\pgfsetdash{}{0pt}%
\pgfsys@defobject{currentmarker}{\pgfqpoint{-0.048611in}{0.000000in}}{\pgfqpoint{-0.000000in}{0.000000in}}{%
\pgfpathmoveto{\pgfqpoint{-0.000000in}{0.000000in}}%
\pgfpathlineto{\pgfqpoint{-0.048611in}{0.000000in}}%
\pgfusepath{stroke,fill}%
}%
\begin{pgfscope}%
\pgfsys@transformshift{0.763041in}{3.172062in}%
\pgfsys@useobject{currentmarker}{}%
\end{pgfscope}%
\end{pgfscope}%
\begin{pgfscope}%
\definecolor{textcolor}{rgb}{0.000000,0.000000,0.000000}%
\pgfsetstrokecolor{textcolor}%
\pgfsetfillcolor{textcolor}%
\pgftext[x=0.418904in, y=3.123837in, left, base]{\color{textcolor}\rmfamily\fontsize{10.000000}{12.000000}\selectfont \(\displaystyle {0.00}\)}%
\end{pgfscope}%
\begin{pgfscope}%
\pgfsetbuttcap%
\pgfsetroundjoin%
\definecolor{currentfill}{rgb}{0.000000,0.000000,0.000000}%
\pgfsetfillcolor{currentfill}%
\pgfsetlinewidth{0.803000pt}%
\definecolor{currentstroke}{rgb}{0.000000,0.000000,0.000000}%
\pgfsetstrokecolor{currentstroke}%
\pgfsetdash{}{0pt}%
\pgfsys@defobject{currentmarker}{\pgfqpoint{-0.048611in}{0.000000in}}{\pgfqpoint{-0.000000in}{0.000000in}}{%
\pgfpathmoveto{\pgfqpoint{-0.000000in}{0.000000in}}%
\pgfpathlineto{\pgfqpoint{-0.048611in}{0.000000in}}%
\pgfusepath{stroke,fill}%
}%
\begin{pgfscope}%
\pgfsys@transformshift{0.763041in}{3.521959in}%
\pgfsys@useobject{currentmarker}{}%
\end{pgfscope}%
\end{pgfscope}%
\begin{pgfscope}%
\definecolor{textcolor}{rgb}{0.000000,0.000000,0.000000}%
\pgfsetstrokecolor{textcolor}%
\pgfsetfillcolor{textcolor}%
\pgftext[x=0.418904in, y=3.473734in, left, base]{\color{textcolor}\rmfamily\fontsize{10.000000}{12.000000}\selectfont \(\displaystyle {0.25}\)}%
\end{pgfscope}%
\begin{pgfscope}%
\pgfsetbuttcap%
\pgfsetroundjoin%
\definecolor{currentfill}{rgb}{0.000000,0.000000,0.000000}%
\pgfsetfillcolor{currentfill}%
\pgfsetlinewidth{0.803000pt}%
\definecolor{currentstroke}{rgb}{0.000000,0.000000,0.000000}%
\pgfsetstrokecolor{currentstroke}%
\pgfsetdash{}{0pt}%
\pgfsys@defobject{currentmarker}{\pgfqpoint{-0.048611in}{0.000000in}}{\pgfqpoint{-0.000000in}{0.000000in}}{%
\pgfpathmoveto{\pgfqpoint{-0.000000in}{0.000000in}}%
\pgfpathlineto{\pgfqpoint{-0.048611in}{0.000000in}}%
\pgfusepath{stroke,fill}%
}%
\begin{pgfscope}%
\pgfsys@transformshift{0.763041in}{3.871856in}%
\pgfsys@useobject{currentmarker}{}%
\end{pgfscope}%
\end{pgfscope}%
\begin{pgfscope}%
\definecolor{textcolor}{rgb}{0.000000,0.000000,0.000000}%
\pgfsetstrokecolor{textcolor}%
\pgfsetfillcolor{textcolor}%
\pgftext[x=0.418904in, y=3.823630in, left, base]{\color{textcolor}\rmfamily\fontsize{10.000000}{12.000000}\selectfont \(\displaystyle {0.50}\)}%
\end{pgfscope}%
\begin{pgfscope}%
\pgfsetbuttcap%
\pgfsetroundjoin%
\definecolor{currentfill}{rgb}{0.000000,0.000000,0.000000}%
\pgfsetfillcolor{currentfill}%
\pgfsetlinewidth{0.803000pt}%
\definecolor{currentstroke}{rgb}{0.000000,0.000000,0.000000}%
\pgfsetstrokecolor{currentstroke}%
\pgfsetdash{}{0pt}%
\pgfsys@defobject{currentmarker}{\pgfqpoint{-0.048611in}{0.000000in}}{\pgfqpoint{-0.000000in}{0.000000in}}{%
\pgfpathmoveto{\pgfqpoint{-0.000000in}{0.000000in}}%
\pgfpathlineto{\pgfqpoint{-0.048611in}{0.000000in}}%
\pgfusepath{stroke,fill}%
}%
\begin{pgfscope}%
\pgfsys@transformshift{0.763041in}{4.221752in}%
\pgfsys@useobject{currentmarker}{}%
\end{pgfscope}%
\end{pgfscope}%
\begin{pgfscope}%
\definecolor{textcolor}{rgb}{0.000000,0.000000,0.000000}%
\pgfsetstrokecolor{textcolor}%
\pgfsetfillcolor{textcolor}%
\pgftext[x=0.418904in, y=4.173527in, left, base]{\color{textcolor}\rmfamily\fontsize{10.000000}{12.000000}\selectfont \(\displaystyle {0.75}\)}%
\end{pgfscope}%
\begin{pgfscope}%
\pgfsetbuttcap%
\pgfsetroundjoin%
\definecolor{currentfill}{rgb}{0.000000,0.000000,0.000000}%
\pgfsetfillcolor{currentfill}%
\pgfsetlinewidth{0.803000pt}%
\definecolor{currentstroke}{rgb}{0.000000,0.000000,0.000000}%
\pgfsetstrokecolor{currentstroke}%
\pgfsetdash{}{0pt}%
\pgfsys@defobject{currentmarker}{\pgfqpoint{-0.048611in}{0.000000in}}{\pgfqpoint{-0.000000in}{0.000000in}}{%
\pgfpathmoveto{\pgfqpoint{-0.000000in}{0.000000in}}%
\pgfpathlineto{\pgfqpoint{-0.048611in}{0.000000in}}%
\pgfusepath{stroke,fill}%
}%
\begin{pgfscope}%
\pgfsys@transformshift{0.763041in}{4.571649in}%
\pgfsys@useobject{currentmarker}{}%
\end{pgfscope}%
\end{pgfscope}%
\begin{pgfscope}%
\definecolor{textcolor}{rgb}{0.000000,0.000000,0.000000}%
\pgfsetstrokecolor{textcolor}%
\pgfsetfillcolor{textcolor}%
\pgftext[x=0.418904in, y=4.523424in, left, base]{\color{textcolor}\rmfamily\fontsize{10.000000}{12.000000}\selectfont \(\displaystyle {1.00}\)}%
\end{pgfscope}%
\begin{pgfscope}%
\definecolor{textcolor}{rgb}{0.000000,0.000000,0.000000}%
\pgfsetstrokecolor{textcolor}%
\pgfsetfillcolor{textcolor}%
\pgftext[x=0.363349in,y=3.871856in,,bottom,rotate=90.000000]{\color{textcolor}\rmfamily\fontsize{16.000000}{19.200000}\selectfont TPR}%
\end{pgfscope}%
\begin{pgfscope}%
\pgfpathrectangle{\pgfqpoint{0.763041in}{3.102083in}}{\pgfqpoint{1.687305in}{1.539545in}}%
\pgfusepath{clip}%
\pgfsetrectcap%
\pgfsetroundjoin%
\pgfsetlinewidth{1.505625pt}%
\definecolor{currentstroke}{rgb}{0.000000,0.501961,0.000000}%
\pgfsetstrokecolor{currentstroke}%
\pgfsetdash{}{0pt}%
\pgfpathmoveto{\pgfqpoint{0.839736in}{3.172062in}}%
\pgfpathlineto{\pgfqpoint{0.853082in}{3.325413in}}%
\pgfpathlineto{\pgfqpoint{0.858185in}{3.386333in}}%
\pgfpathlineto{\pgfqpoint{0.862334in}{3.421520in}}%
\pgfpathlineto{\pgfqpoint{0.865026in}{3.443577in}}%
\pgfpathlineto{\pgfqpoint{0.865082in}{3.443577in}}%
\pgfpathlineto{\pgfqpoint{0.883306in}{3.600079in}}%
\pgfpathlineto{\pgfqpoint{0.883979in}{3.601129in}}%
\pgfpathlineto{\pgfqpoint{0.885493in}{3.609006in}}%
\pgfpathlineto{\pgfqpoint{0.885830in}{3.610057in}}%
\pgfpathlineto{\pgfqpoint{0.895362in}{3.686207in}}%
\pgfpathlineto{\pgfqpoint{0.895643in}{3.687257in}}%
\pgfpathlineto{\pgfqpoint{0.896988in}{3.698811in}}%
\pgfpathlineto{\pgfqpoint{0.897269in}{3.699336in}}%
\pgfpathlineto{\pgfqpoint{0.898671in}{3.708789in}}%
\pgfpathlineto{\pgfqpoint{0.899063in}{3.709840in}}%
\pgfpathlineto{\pgfqpoint{0.900577in}{3.722969in}}%
\pgfpathlineto{\pgfqpoint{0.900857in}{3.724019in}}%
\pgfpathlineto{\pgfqpoint{0.902371in}{3.729271in}}%
\pgfpathlineto{\pgfqpoint{0.902596in}{3.730321in}}%
\pgfpathlineto{\pgfqpoint{0.904110in}{3.744501in}}%
\pgfpathlineto{\pgfqpoint{0.905007in}{3.745551in}}%
\pgfpathlineto{\pgfqpoint{0.906465in}{3.755530in}}%
\pgfpathlineto{\pgfqpoint{0.906857in}{3.756580in}}%
\pgfpathlineto{\pgfqpoint{0.908315in}{3.770760in}}%
\pgfpathlineto{\pgfqpoint{0.908484in}{3.770760in}}%
\pgfpathlineto{\pgfqpoint{0.908484in}{3.771810in}}%
\pgfpathlineto{\pgfqpoint{0.914540in}{3.814349in}}%
\pgfpathlineto{\pgfqpoint{0.914708in}{3.814874in}}%
\pgfpathlineto{\pgfqpoint{0.915998in}{3.827478in}}%
\pgfpathlineto{\pgfqpoint{0.916446in}{3.828529in}}%
\pgfpathlineto{\pgfqpoint{0.917960in}{3.843759in}}%
\pgfpathlineto{\pgfqpoint{0.918521in}{3.844284in}}%
\pgfpathlineto{\pgfqpoint{0.918521in}{3.844809in}}%
\pgfpathlineto{\pgfqpoint{0.923231in}{3.877370in}}%
\pgfpathlineto{\pgfqpoint{0.923792in}{3.878420in}}%
\pgfpathlineto{\pgfqpoint{0.925138in}{3.888924in}}%
\pgfpathlineto{\pgfqpoint{0.925754in}{3.889449in}}%
\pgfpathlineto{\pgfqpoint{0.927100in}{3.897852in}}%
\pgfpathlineto{\pgfqpoint{0.927549in}{3.897852in}}%
\pgfpathlineto{\pgfqpoint{0.929063in}{3.909406in}}%
\pgfpathlineto{\pgfqpoint{0.929399in}{3.910456in}}%
\pgfpathlineto{\pgfqpoint{0.930913in}{3.918859in}}%
\pgfpathlineto{\pgfqpoint{0.931474in}{3.919909in}}%
\pgfpathlineto{\pgfqpoint{0.932932in}{3.928837in}}%
\pgfpathlineto{\pgfqpoint{0.933381in}{3.929887in}}%
\pgfpathlineto{\pgfqpoint{0.934895in}{3.937240in}}%
\pgfpathlineto{\pgfqpoint{0.935343in}{3.937765in}}%
\pgfpathlineto{\pgfqpoint{0.936745in}{3.951419in}}%
\pgfpathlineto{\pgfqpoint{0.937306in}{3.951945in}}%
\pgfpathlineto{\pgfqpoint{0.938820in}{3.961398in}}%
\pgfpathlineto{\pgfqpoint{0.938932in}{3.962448in}}%
\pgfpathlineto{\pgfqpoint{0.940446in}{3.974002in}}%
\pgfpathlineto{\pgfqpoint{0.940838in}{3.975052in}}%
\pgfpathlineto{\pgfqpoint{0.942128in}{3.980829in}}%
\pgfpathlineto{\pgfqpoint{0.943025in}{3.981879in}}%
\pgfpathlineto{\pgfqpoint{0.944539in}{3.991858in}}%
\pgfpathlineto{\pgfqpoint{0.944876in}{3.992908in}}%
\pgfpathlineto{\pgfqpoint{0.946390in}{4.001311in}}%
\pgfpathlineto{\pgfqpoint{0.946838in}{4.002361in}}%
\pgfpathlineto{\pgfqpoint{0.948296in}{4.009714in}}%
\pgfpathlineto{\pgfqpoint{0.949474in}{4.010764in}}%
\pgfpathlineto{\pgfqpoint{0.950988in}{4.018116in}}%
\pgfpathlineto{\pgfqpoint{0.951324in}{4.019167in}}%
\pgfpathlineto{\pgfqpoint{0.952838in}{4.023893in}}%
\pgfpathlineto{\pgfqpoint{0.953175in}{4.024418in}}%
\pgfpathlineto{\pgfqpoint{0.954521in}{4.033346in}}%
\pgfpathlineto{\pgfqpoint{0.955194in}{4.034397in}}%
\pgfpathlineto{\pgfqpoint{0.956651in}{4.042274in}}%
\pgfpathlineto{\pgfqpoint{0.957493in}{4.043325in}}%
\pgfpathlineto{\pgfqpoint{0.959007in}{4.049102in}}%
\pgfpathlineto{\pgfqpoint{0.959231in}{4.050152in}}%
\pgfpathlineto{\pgfqpoint{0.960745in}{4.058555in}}%
\pgfpathlineto{\pgfqpoint{0.960969in}{4.058555in}}%
\pgfpathlineto{\pgfqpoint{0.962371in}{4.063806in}}%
\pgfpathlineto{\pgfqpoint{0.962876in}{4.064857in}}%
\pgfpathlineto{\pgfqpoint{0.964390in}{4.073785in}}%
\pgfpathlineto{\pgfqpoint{0.965063in}{4.074310in}}%
\pgfpathlineto{\pgfqpoint{0.966408in}{4.080612in}}%
\pgfpathlineto{\pgfqpoint{0.966745in}{4.081662in}}%
\pgfpathlineto{\pgfqpoint{0.968203in}{4.089540in}}%
\pgfpathlineto{\pgfqpoint{0.968595in}{4.090590in}}%
\pgfpathlineto{\pgfqpoint{0.970109in}{4.096892in}}%
\pgfpathlineto{\pgfqpoint{0.970278in}{4.097418in}}%
\pgfpathlineto{\pgfqpoint{0.971679in}{4.107921in}}%
\pgfpathlineto{\pgfqpoint{0.972240in}{4.108971in}}%
\pgfpathlineto{\pgfqpoint{0.973754in}{4.114223in}}%
\pgfpathlineto{\pgfqpoint{0.974091in}{4.115273in}}%
\pgfpathlineto{\pgfqpoint{0.975605in}{4.121050in}}%
\pgfpathlineto{\pgfqpoint{0.976165in}{4.122101in}}%
\pgfpathlineto{\pgfqpoint{0.977623in}{4.132604in}}%
\pgfpathlineto{\pgfqpoint{0.977904in}{4.133129in}}%
\pgfpathlineto{\pgfqpoint{0.979418in}{4.137331in}}%
\pgfpathlineto{\pgfqpoint{0.979810in}{4.138381in}}%
\pgfpathlineto{\pgfqpoint{0.981268in}{4.144683in}}%
\pgfpathlineto{\pgfqpoint{0.981773in}{4.145733in}}%
\pgfpathlineto{\pgfqpoint{0.983062in}{4.150985in}}%
\pgfpathlineto{\pgfqpoint{0.983904in}{4.152036in}}%
\pgfpathlineto{\pgfqpoint{0.985305in}{4.153611in}}%
\pgfpathlineto{\pgfqpoint{0.986090in}{4.154661in}}%
\pgfpathlineto{\pgfqpoint{0.987548in}{4.160438in}}%
\pgfpathlineto{\pgfqpoint{0.988109in}{4.160963in}}%
\pgfpathlineto{\pgfqpoint{0.989399in}{4.166215in}}%
\pgfpathlineto{\pgfqpoint{0.990576in}{4.167266in}}%
\pgfpathlineto{\pgfqpoint{0.991586in}{4.169366in}}%
\pgfpathlineto{\pgfqpoint{0.992259in}{4.169891in}}%
\pgfpathlineto{\pgfqpoint{0.993773in}{4.176719in}}%
\pgfpathlineto{\pgfqpoint{0.994950in}{4.177769in}}%
\pgfpathlineto{\pgfqpoint{0.996296in}{4.185121in}}%
\pgfpathlineto{\pgfqpoint{0.996969in}{4.186172in}}%
\pgfpathlineto{\pgfqpoint{0.998259in}{4.192474in}}%
\pgfpathlineto{\pgfqpoint{0.999044in}{4.193524in}}%
\pgfpathlineto{\pgfqpoint{1.000558in}{4.196150in}}%
\pgfpathlineto{\pgfqpoint{1.001118in}{4.197200in}}%
\pgfpathlineto{\pgfqpoint{1.002408in}{4.198776in}}%
\pgfpathlineto{\pgfqpoint{1.002745in}{4.199826in}}%
\pgfpathlineto{\pgfqpoint{1.004146in}{4.206654in}}%
\pgfpathlineto{\pgfqpoint{1.004371in}{4.207179in}}%
\pgfpathlineto{\pgfqpoint{1.005829in}{4.212956in}}%
\pgfpathlineto{\pgfqpoint{1.006389in}{4.214006in}}%
\pgfpathlineto{\pgfqpoint{1.007735in}{4.219783in}}%
\pgfpathlineto{\pgfqpoint{1.008240in}{4.220833in}}%
\pgfpathlineto{\pgfqpoint{1.009698in}{4.223459in}}%
\pgfpathlineto{\pgfqpoint{1.010034in}{4.224509in}}%
\pgfpathlineto{\pgfqpoint{1.011548in}{4.228186in}}%
\pgfpathlineto{\pgfqpoint{1.012670in}{4.229236in}}%
\pgfpathlineto{\pgfqpoint{1.013959in}{4.231337in}}%
\pgfpathlineto{\pgfqpoint{1.015025in}{4.232387in}}%
\pgfpathlineto{\pgfqpoint{1.016539in}{4.236063in}}%
\pgfpathlineto{\pgfqpoint{1.017492in}{4.236588in}}%
\pgfpathlineto{\pgfqpoint{1.018501in}{4.239214in}}%
\pgfpathlineto{\pgfqpoint{1.019623in}{4.240265in}}%
\pgfpathlineto{\pgfqpoint{1.021137in}{4.243941in}}%
\pgfpathlineto{\pgfqpoint{1.021922in}{4.244991in}}%
\pgfpathlineto{\pgfqpoint{1.023380in}{4.249718in}}%
\pgfpathlineto{\pgfqpoint{1.023716in}{4.250768in}}%
\pgfpathlineto{\pgfqpoint{1.025174in}{4.252869in}}%
\pgfpathlineto{\pgfqpoint{1.026240in}{4.253919in}}%
\pgfpathlineto{\pgfqpoint{1.027473in}{4.256545in}}%
\pgfpathlineto{\pgfqpoint{1.028371in}{4.257595in}}%
\pgfpathlineto{\pgfqpoint{1.029716in}{4.260221in}}%
\pgfpathlineto{\pgfqpoint{1.031230in}{4.261272in}}%
\pgfpathlineto{\pgfqpoint{1.032576in}{4.267574in}}%
\pgfpathlineto{\pgfqpoint{1.033305in}{4.268099in}}%
\pgfpathlineto{\pgfqpoint{1.034707in}{4.271250in}}%
\pgfpathlineto{\pgfqpoint{1.035548in}{4.272300in}}%
\pgfpathlineto{\pgfqpoint{1.036053in}{4.273876in}}%
\pgfpathlineto{\pgfqpoint{1.038015in}{4.274401in}}%
\pgfpathlineto{\pgfqpoint{1.039361in}{4.281228in}}%
\pgfpathlineto{\pgfqpoint{1.041212in}{4.282278in}}%
\pgfpathlineto{\pgfqpoint{1.042557in}{4.287005in}}%
\pgfpathlineto{\pgfqpoint{1.043903in}{4.287530in}}%
\pgfpathlineto{\pgfqpoint{1.045137in}{4.290681in}}%
\pgfpathlineto{\pgfqpoint{1.045698in}{4.291206in}}%
\pgfpathlineto{\pgfqpoint{1.047099in}{4.296458in}}%
\pgfpathlineto{\pgfqpoint{1.048445in}{4.297508in}}%
\pgfpathlineto{\pgfqpoint{1.049679in}{4.300134in}}%
\pgfpathlineto{\pgfqpoint{1.050408in}{4.301185in}}%
\pgfpathlineto{\pgfqpoint{1.051922in}{4.305386in}}%
\pgfpathlineto{\pgfqpoint{1.053772in}{4.306436in}}%
\pgfpathlineto{\pgfqpoint{1.055118in}{4.309587in}}%
\pgfpathlineto{\pgfqpoint{1.056520in}{4.310638in}}%
\pgfpathlineto{\pgfqpoint{1.057305in}{4.311688in}}%
\pgfpathlineto{\pgfqpoint{1.058539in}{4.312738in}}%
\pgfpathlineto{\pgfqpoint{1.059436in}{4.314839in}}%
\pgfpathlineto{\pgfqpoint{1.061735in}{4.315890in}}%
\pgfpathlineto{\pgfqpoint{1.063137in}{4.320091in}}%
\pgfpathlineto{\pgfqpoint{1.064034in}{4.321141in}}%
\pgfpathlineto{\pgfqpoint{1.065492in}{4.325343in}}%
\pgfpathlineto{\pgfqpoint{1.066725in}{4.326393in}}%
\pgfpathlineto{\pgfqpoint{1.068071in}{4.328494in}}%
\pgfpathlineto{\pgfqpoint{1.068968in}{4.329544in}}%
\pgfpathlineto{\pgfqpoint{1.070314in}{4.331645in}}%
\pgfpathlineto{\pgfqpoint{1.070931in}{4.332695in}}%
\pgfpathlineto{\pgfqpoint{1.072221in}{4.335321in}}%
\pgfpathlineto{\pgfqpoint{1.073174in}{4.336371in}}%
\pgfpathlineto{\pgfqpoint{1.074632in}{4.339522in}}%
\pgfpathlineto{\pgfqpoint{1.075922in}{4.340573in}}%
\pgfpathlineto{\pgfqpoint{1.075922in}{4.341098in}}%
\pgfpathlineto{\pgfqpoint{1.078837in}{4.342148in}}%
\pgfpathlineto{\pgfqpoint{1.079735in}{4.343724in}}%
\pgfpathlineto{\pgfqpoint{1.082314in}{4.344774in}}%
\pgfpathlineto{\pgfqpoint{1.083772in}{4.346350in}}%
\pgfpathlineto{\pgfqpoint{1.085174in}{4.347400in}}%
\pgfpathlineto{\pgfqpoint{1.086576in}{4.351076in}}%
\pgfpathlineto{\pgfqpoint{1.087978in}{4.351601in}}%
\pgfpathlineto{\pgfqpoint{1.089043in}{4.354752in}}%
\pgfpathlineto{\pgfqpoint{1.090781in}{4.355803in}}%
\pgfpathlineto{\pgfqpoint{1.092183in}{4.359479in}}%
\pgfpathlineto{\pgfqpoint{1.092856in}{4.360529in}}%
\pgfpathlineto{\pgfqpoint{1.094370in}{4.363155in}}%
\pgfpathlineto{\pgfqpoint{1.097847in}{4.364205in}}%
\pgfpathlineto{\pgfqpoint{1.098463in}{4.365256in}}%
\pgfpathlineto{\pgfqpoint{1.101660in}{4.366306in}}%
\pgfpathlineto{\pgfqpoint{1.102837in}{4.367882in}}%
\pgfpathlineto{\pgfqpoint{1.104463in}{4.368932in}}%
\pgfpathlineto{\pgfqpoint{1.105529in}{4.370508in}}%
\pgfpathlineto{\pgfqpoint{1.107043in}{4.371558in}}%
\pgfpathlineto{\pgfqpoint{1.108276in}{4.374184in}}%
\pgfpathlineto{\pgfqpoint{1.109398in}{4.375234in}}%
\pgfpathlineto{\pgfqpoint{1.110407in}{4.376284in}}%
\pgfpathlineto{\pgfqpoint{1.113211in}{4.377335in}}%
\pgfpathlineto{\pgfqpoint{1.114389in}{4.378910in}}%
\pgfpathlineto{\pgfqpoint{1.115454in}{4.379961in}}%
\pgfpathlineto{\pgfqpoint{1.116912in}{4.382587in}}%
\pgfpathlineto{\pgfqpoint{1.118650in}{4.383637in}}%
\pgfpathlineto{\pgfqpoint{1.119772in}{4.384687in}}%
\pgfpathlineto{\pgfqpoint{1.121510in}{4.385738in}}%
\pgfpathlineto{\pgfqpoint{1.122351in}{4.387838in}}%
\pgfpathlineto{\pgfqpoint{1.125660in}{4.388889in}}%
\pgfpathlineto{\pgfqpoint{1.127061in}{4.390464in}}%
\pgfpathlineto{\pgfqpoint{1.128407in}{4.391514in}}%
\pgfpathlineto{\pgfqpoint{1.129809in}{4.393615in}}%
\pgfpathlineto{\pgfqpoint{1.131828in}{4.394666in}}%
\pgfpathlineto{\pgfqpoint{1.133005in}{4.397291in}}%
\pgfpathlineto{\pgfqpoint{1.135248in}{4.398342in}}%
\pgfpathlineto{\pgfqpoint{1.136538in}{4.400442in}}%
\pgfpathlineto{\pgfqpoint{1.137435in}{4.401493in}}%
\pgfpathlineto{\pgfqpoint{1.137547in}{4.402543in}}%
\pgfpathlineto{\pgfqpoint{1.140239in}{4.403593in}}%
\pgfpathlineto{\pgfqpoint{1.141248in}{4.405169in}}%
\pgfpathlineto{\pgfqpoint{1.143828in}{4.406219in}}%
\pgfpathlineto{\pgfqpoint{1.144276in}{4.407270in}}%
\pgfpathlineto{\pgfqpoint{1.148762in}{4.408320in}}%
\pgfpathlineto{\pgfqpoint{1.149211in}{4.409896in}}%
\pgfpathlineto{\pgfqpoint{1.150837in}{4.410946in}}%
\pgfpathlineto{\pgfqpoint{1.152183in}{4.412521in}}%
\pgfpathlineto{\pgfqpoint{1.155996in}{4.413572in}}%
\pgfpathlineto{\pgfqpoint{1.155996in}{4.414097in}}%
\pgfpathlineto{\pgfqpoint{1.160370in}{4.415147in}}%
\pgfpathlineto{\pgfqpoint{1.161547in}{4.416198in}}%
\pgfpathlineto{\pgfqpoint{1.164463in}{4.417248in}}%
\pgfpathlineto{\pgfqpoint{1.165753in}{4.418298in}}%
\pgfpathlineto{\pgfqpoint{1.166874in}{4.419349in}}%
\pgfpathlineto{\pgfqpoint{1.167996in}{4.420399in}}%
\pgfpathlineto{\pgfqpoint{1.171248in}{4.421449in}}%
\pgfpathlineto{\pgfqpoint{1.171304in}{4.422500in}}%
\pgfpathlineto{\pgfqpoint{1.175285in}{4.423550in}}%
\pgfpathlineto{\pgfqpoint{1.175846in}{4.424600in}}%
\pgfpathlineto{\pgfqpoint{1.179154in}{4.425651in}}%
\pgfpathlineto{\pgfqpoint{1.180108in}{4.427226in}}%
\pgfpathlineto{\pgfqpoint{1.183528in}{4.428277in}}%
\pgfpathlineto{\pgfqpoint{1.183528in}{4.428802in}}%
\pgfpathlineto{\pgfqpoint{1.189080in}{4.429852in}}%
\pgfpathlineto{\pgfqpoint{1.190594in}{4.431428in}}%
\pgfpathlineto{\pgfqpoint{1.194911in}{4.432478in}}%
\pgfpathlineto{\pgfqpoint{1.194911in}{4.433003in}}%
\pgfpathlineto{\pgfqpoint{1.200855in}{4.434054in}}%
\pgfpathlineto{\pgfqpoint{1.202369in}{4.436154in}}%
\pgfpathlineto{\pgfqpoint{1.205061in}{4.437205in}}%
\pgfpathlineto{\pgfqpoint{1.205453in}{4.438780in}}%
\pgfpathlineto{\pgfqpoint{1.208986in}{4.439830in}}%
\pgfpathlineto{\pgfqpoint{1.209771in}{4.441931in}}%
\pgfpathlineto{\pgfqpoint{1.211902in}{4.442981in}}%
\pgfpathlineto{\pgfqpoint{1.212631in}{4.444032in}}%
\pgfpathlineto{\pgfqpoint{1.215547in}{4.445082in}}%
\pgfpathlineto{\pgfqpoint{1.216220in}{4.447183in}}%
\pgfpathlineto{\pgfqpoint{1.218911in}{4.448233in}}%
\pgfpathlineto{\pgfqpoint{1.220033in}{4.449809in}}%
\pgfpathlineto{\pgfqpoint{1.226369in}{4.450859in}}%
\pgfpathlineto{\pgfqpoint{1.226369in}{4.451384in}}%
\pgfpathlineto{\pgfqpoint{1.230687in}{4.452435in}}%
\pgfpathlineto{\pgfqpoint{1.232201in}{4.454010in}}%
\pgfpathlineto{\pgfqpoint{1.235565in}{4.455060in}}%
\pgfpathlineto{\pgfqpoint{1.235565in}{4.455586in}}%
\pgfpathlineto{\pgfqpoint{1.239603in}{4.456636in}}%
\pgfpathlineto{\pgfqpoint{1.241005in}{4.457686in}}%
\pgfpathlineto{\pgfqpoint{1.245771in}{4.458737in}}%
\pgfpathlineto{\pgfqpoint{1.245827in}{4.459787in}}%
\pgfpathlineto{\pgfqpoint{1.248238in}{4.460837in}}%
\pgfpathlineto{\pgfqpoint{1.249247in}{4.461888in}}%
\pgfpathlineto{\pgfqpoint{1.252051in}{4.462938in}}%
\pgfpathlineto{\pgfqpoint{1.252948in}{4.464514in}}%
\pgfpathlineto{\pgfqpoint{1.257546in}{4.465564in}}%
\pgfpathlineto{\pgfqpoint{1.258388in}{4.467139in}}%
\pgfpathlineto{\pgfqpoint{1.262649in}{4.468190in}}%
\pgfpathlineto{\pgfqpoint{1.263827in}{4.469765in}}%
\pgfpathlineto{\pgfqpoint{1.270500in}{4.470816in}}%
\pgfpathlineto{\pgfqpoint{1.271789in}{4.471866in}}%
\pgfpathlineto{\pgfqpoint{1.272743in}{4.472916in}}%
\pgfpathlineto{\pgfqpoint{1.272743in}{4.473441in}}%
\pgfpathlineto{\pgfqpoint{1.278686in}{4.474492in}}%
\pgfpathlineto{\pgfqpoint{1.279920in}{4.477118in}}%
\pgfpathlineto{\pgfqpoint{1.292537in}{4.478168in}}%
\pgfpathlineto{\pgfqpoint{1.292537in}{4.478693in}}%
\pgfpathlineto{\pgfqpoint{1.297920in}{4.479744in}}%
\pgfpathlineto{\pgfqpoint{1.298200in}{4.480794in}}%
\pgfpathlineto{\pgfqpoint{1.302630in}{4.481844in}}%
\pgfpathlineto{\pgfqpoint{1.302798in}{4.482895in}}%
\pgfpathlineto{\pgfqpoint{1.308518in}{4.483945in}}%
\pgfpathlineto{\pgfqpoint{1.308518in}{4.484470in}}%
\pgfpathlineto{\pgfqpoint{1.311882in}{4.485520in}}%
\pgfpathlineto{\pgfqpoint{1.312892in}{4.486571in}}%
\pgfpathlineto{\pgfqpoint{1.315191in}{4.486571in}}%
\pgfpathlineto{\pgfqpoint{1.316481in}{4.489197in}}%
\pgfpathlineto{\pgfqpoint{1.321023in}{4.490247in}}%
\pgfpathlineto{\pgfqpoint{1.322312in}{4.491823in}}%
\pgfpathlineto{\pgfqpoint{1.324780in}{4.492873in}}%
\pgfpathlineto{\pgfqpoint{1.325396in}{4.494448in}}%
\pgfpathlineto{\pgfqpoint{1.331396in}{4.495499in}}%
\pgfpathlineto{\pgfqpoint{1.331396in}{4.496024in}}%
\pgfpathlineto{\pgfqpoint{1.338574in}{4.497074in}}%
\pgfpathlineto{\pgfqpoint{1.338910in}{4.498650in}}%
\pgfpathlineto{\pgfqpoint{1.343901in}{4.499700in}}%
\pgfpathlineto{\pgfqpoint{1.344406in}{4.500750in}}%
\pgfpathlineto{\pgfqpoint{1.355340in}{4.501801in}}%
\pgfpathlineto{\pgfqpoint{1.355340in}{4.502326in}}%
\pgfpathlineto{\pgfqpoint{1.358368in}{4.503376in}}%
\pgfpathlineto{\pgfqpoint{1.358368in}{4.503902in}}%
\pgfpathlineto{\pgfqpoint{1.366331in}{4.504952in}}%
\pgfpathlineto{\pgfqpoint{1.366331in}{4.505477in}}%
\pgfpathlineto{\pgfqpoint{1.375807in}{4.506527in}}%
\pgfpathlineto{\pgfqpoint{1.376704in}{4.507578in}}%
\pgfpathlineto{\pgfqpoint{1.383097in}{4.508628in}}%
\pgfpathlineto{\pgfqpoint{1.383938in}{4.510204in}}%
\pgfpathlineto{\pgfqpoint{1.390387in}{4.511254in}}%
\pgfpathlineto{\pgfqpoint{1.390387in}{4.512304in}}%
\pgfpathlineto{\pgfqpoint{1.395097in}{4.513355in}}%
\pgfpathlineto{\pgfqpoint{1.395097in}{4.513880in}}%
\pgfpathlineto{\pgfqpoint{1.406704in}{4.514930in}}%
\pgfpathlineto{\pgfqpoint{1.406704in}{4.515455in}}%
\pgfpathlineto{\pgfqpoint{1.418985in}{4.516506in}}%
\pgfpathlineto{\pgfqpoint{1.418985in}{4.517556in}}%
\pgfpathlineto{\pgfqpoint{1.434125in}{4.518606in}}%
\pgfpathlineto{\pgfqpoint{1.434125in}{4.519132in}}%
\pgfpathlineto{\pgfqpoint{1.440068in}{4.520182in}}%
\pgfpathlineto{\pgfqpoint{1.440517in}{4.521232in}}%
\pgfpathlineto{\pgfqpoint{1.452124in}{4.522283in}}%
\pgfpathlineto{\pgfqpoint{1.452124in}{4.522808in}}%
\pgfpathlineto{\pgfqpoint{1.459919in}{4.523858in}}%
\pgfpathlineto{\pgfqpoint{1.459919in}{4.524383in}}%
\pgfpathlineto{\pgfqpoint{1.471134in}{4.525434in}}%
\pgfpathlineto{\pgfqpoint{1.471134in}{4.525959in}}%
\pgfpathlineto{\pgfqpoint{1.476965in}{4.526484in}}%
\pgfpathlineto{\pgfqpoint{1.477807in}{4.528585in}}%
\pgfpathlineto{\pgfqpoint{1.496087in}{4.529635in}}%
\pgfpathlineto{\pgfqpoint{1.496087in}{4.530160in}}%
\pgfpathlineto{\pgfqpoint{1.513582in}{4.531211in}}%
\pgfpathlineto{\pgfqpoint{1.513582in}{4.531736in}}%
\pgfpathlineto{\pgfqpoint{1.525133in}{4.532786in}}%
\pgfpathlineto{\pgfqpoint{1.525133in}{4.533311in}}%
\pgfpathlineto{\pgfqpoint{1.531470in}{4.534362in}}%
\pgfpathlineto{\pgfqpoint{1.531918in}{4.535412in}}%
\pgfpathlineto{\pgfqpoint{1.544871in}{4.536462in}}%
\pgfpathlineto{\pgfqpoint{1.545208in}{4.537513in}}%
\pgfpathlineto{\pgfqpoint{1.556086in}{4.538563in}}%
\pgfpathlineto{\pgfqpoint{1.556086in}{4.539088in}}%
\pgfpathlineto{\pgfqpoint{1.571058in}{4.540138in}}%
\pgfpathlineto{\pgfqpoint{1.571058in}{4.540664in}}%
\pgfpathlineto{\pgfqpoint{1.594553in}{4.541714in}}%
\pgfpathlineto{\pgfqpoint{1.595394in}{4.542764in}}%
\pgfpathlineto{\pgfqpoint{1.610535in}{4.543815in}}%
\pgfpathlineto{\pgfqpoint{1.610535in}{4.544340in}}%
\pgfpathlineto{\pgfqpoint{1.636273in}{4.545390in}}%
\pgfpathlineto{\pgfqpoint{1.636273in}{4.545915in}}%
\pgfpathlineto{\pgfqpoint{1.640703in}{4.546966in}}%
\pgfpathlineto{\pgfqpoint{1.640703in}{4.547491in}}%
\pgfpathlineto{\pgfqpoint{1.663357in}{4.548541in}}%
\pgfpathlineto{\pgfqpoint{1.663525in}{4.549592in}}%
\pgfpathlineto{\pgfqpoint{1.680964in}{4.550642in}}%
\pgfpathlineto{\pgfqpoint{1.680964in}{4.551167in}}%
\pgfpathlineto{\pgfqpoint{1.699581in}{4.552217in}}%
\pgfpathlineto{\pgfqpoint{1.699581in}{4.552743in}}%
\pgfpathlineto{\pgfqpoint{1.762216in}{4.553793in}}%
\pgfpathlineto{\pgfqpoint{1.762216in}{4.554318in}}%
\pgfpathlineto{\pgfqpoint{1.773711in}{4.555369in}}%
\pgfpathlineto{\pgfqpoint{1.773711in}{4.555894in}}%
\pgfpathlineto{\pgfqpoint{1.779991in}{4.556944in}}%
\pgfpathlineto{\pgfqpoint{1.779991in}{4.557469in}}%
\pgfpathlineto{\pgfqpoint{1.800010in}{4.558520in}}%
\pgfpathlineto{\pgfqpoint{1.800010in}{4.559045in}}%
\pgfpathlineto{\pgfqpoint{1.831411in}{4.560095in}}%
\pgfpathlineto{\pgfqpoint{1.831411in}{4.560620in}}%
\pgfpathlineto{\pgfqpoint{1.876103in}{4.561671in}}%
\pgfpathlineto{\pgfqpoint{1.876103in}{4.562196in}}%
\pgfpathlineto{\pgfqpoint{1.877168in}{4.562196in}}%
\pgfpathlineto{\pgfqpoint{1.916701in}{4.563246in}}%
\pgfpathlineto{\pgfqpoint{1.916701in}{4.563771in}}%
\pgfpathlineto{\pgfqpoint{1.968289in}{4.564822in}}%
\pgfpathlineto{\pgfqpoint{1.968289in}{4.565347in}}%
\pgfpathlineto{\pgfqpoint{1.969803in}{4.565347in}}%
\pgfpathlineto{\pgfqpoint{2.003167in}{4.566397in}}%
\pgfpathlineto{\pgfqpoint{2.003167in}{4.566922in}}%
\pgfpathlineto{\pgfqpoint{2.050270in}{4.567973in}}%
\pgfpathlineto{\pgfqpoint{2.050270in}{4.568498in}}%
\pgfpathlineto{\pgfqpoint{2.156474in}{4.569548in}}%
\pgfpathlineto{\pgfqpoint{2.156474in}{4.570073in}}%
\pgfpathlineto{\pgfqpoint{2.318137in}{4.571124in}}%
\pgfpathlineto{\pgfqpoint{2.318137in}{4.571649in}}%
\pgfpathlineto{\pgfqpoint{2.373650in}{4.571649in}}%
\pgfpathlineto{\pgfqpoint{2.373650in}{4.571649in}}%
\pgfusepath{stroke}%
\end{pgfscope}%
\begin{pgfscope}%
\pgfpathrectangle{\pgfqpoint{0.763041in}{3.102083in}}{\pgfqpoint{1.687305in}{1.539545in}}%
\pgfusepath{clip}%
\pgfsetrectcap%
\pgfsetroundjoin%
\pgfsetlinewidth{1.505625pt}%
\definecolor{currentstroke}{rgb}{0.501961,0.501961,0.501961}%
\pgfsetstrokecolor{currentstroke}%
\pgfsetdash{}{0pt}%
\pgfpathmoveto{\pgfqpoint{0.839736in}{3.172062in}}%
\pgfpathlineto{\pgfqpoint{2.373650in}{4.571649in}}%
\pgfusepath{stroke}%
\end{pgfscope}%
\begin{pgfscope}%
\pgfsetrectcap%
\pgfsetmiterjoin%
\pgfsetlinewidth{0.803000pt}%
\definecolor{currentstroke}{rgb}{0.000000,0.000000,0.000000}%
\pgfsetstrokecolor{currentstroke}%
\pgfsetdash{}{0pt}%
\pgfpathmoveto{\pgfqpoint{0.763041in}{3.102083in}}%
\pgfpathlineto{\pgfqpoint{0.763041in}{4.641628in}}%
\pgfusepath{stroke}%
\end{pgfscope}%
\begin{pgfscope}%
\pgfsetrectcap%
\pgfsetmiterjoin%
\pgfsetlinewidth{0.803000pt}%
\definecolor{currentstroke}{rgb}{0.000000,0.000000,0.000000}%
\pgfsetstrokecolor{currentstroke}%
\pgfsetdash{}{0pt}%
\pgfpathmoveto{\pgfqpoint{2.450346in}{3.102083in}}%
\pgfpathlineto{\pgfqpoint{2.450346in}{4.641628in}}%
\pgfusepath{stroke}%
\end{pgfscope}%
\begin{pgfscope}%
\pgfsetrectcap%
\pgfsetmiterjoin%
\pgfsetlinewidth{0.803000pt}%
\definecolor{currentstroke}{rgb}{0.000000,0.000000,0.000000}%
\pgfsetstrokecolor{currentstroke}%
\pgfsetdash{}{0pt}%
\pgfpathmoveto{\pgfqpoint{0.763041in}{3.102083in}}%
\pgfpathlineto{\pgfqpoint{2.450346in}{3.102083in}}%
\pgfusepath{stroke}%
\end{pgfscope}%
\begin{pgfscope}%
\pgfsetrectcap%
\pgfsetmiterjoin%
\pgfsetlinewidth{0.803000pt}%
\definecolor{currentstroke}{rgb}{0.000000,0.000000,0.000000}%
\pgfsetstrokecolor{currentstroke}%
\pgfsetdash{}{0pt}%
\pgfpathmoveto{\pgfqpoint{0.763041in}{4.641628in}}%
\pgfpathlineto{\pgfqpoint{2.450346in}{4.641628in}}%
\pgfusepath{stroke}%
\end{pgfscope}%
\begin{pgfscope}%
\definecolor{textcolor}{rgb}{0.000000,0.000000,0.000000}%
\pgfsetstrokecolor{textcolor}%
\pgfsetfillcolor{textcolor}%
\pgftext[x=1.606693in,y=4.724962in,,base]{\color{textcolor}\rmfamily\fontsize{20.000000}{24.000000}\selectfont Pneumothorax}%
\end{pgfscope}%
\begin{pgfscope}%
\pgfsetbuttcap%
\pgfsetmiterjoin%
\definecolor{currentfill}{rgb}{1.000000,1.000000,1.000000}%
\pgfsetfillcolor{currentfill}%
\pgfsetfillopacity{0.800000}%
\pgfsetlinewidth{1.003750pt}%
\definecolor{currentstroke}{rgb}{0.800000,0.800000,0.800000}%
\pgfsetstrokecolor{currentstroke}%
\pgfsetstrokeopacity{0.800000}%
\pgfsetdash{}{0pt}%
\pgfpathmoveto{\pgfqpoint{1.241240in}{3.171527in}}%
\pgfpathlineto{\pgfqpoint{2.353124in}{3.171527in}}%
\pgfpathquadraticcurveto{\pgfqpoint{2.380902in}{3.171527in}}{\pgfqpoint{2.380902in}{3.199305in}}%
\pgfpathlineto{\pgfqpoint{2.380902in}{3.379089in}}%
\pgfpathquadraticcurveto{\pgfqpoint{2.380902in}{3.406867in}}{\pgfqpoint{2.353124in}{3.406867in}}%
\pgfpathlineto{\pgfqpoint{1.241240in}{3.406867in}}%
\pgfpathquadraticcurveto{\pgfqpoint{1.213462in}{3.406867in}}{\pgfqpoint{1.213462in}{3.379089in}}%
\pgfpathlineto{\pgfqpoint{1.213462in}{3.199305in}}%
\pgfpathquadraticcurveto{\pgfqpoint{1.213462in}{3.171527in}}{\pgfqpoint{1.241240in}{3.171527in}}%
\pgfpathclose%
\pgfusepath{stroke,fill}%
\end{pgfscope}%
\begin{pgfscope}%
\pgfsetrectcap%
\pgfsetroundjoin%
\pgfsetlinewidth{1.505625pt}%
\definecolor{currentstroke}{rgb}{0.000000,0.501961,0.000000}%
\pgfsetstrokecolor{currentstroke}%
\pgfsetdash{}{0pt}%
\pgfpathmoveto{\pgfqpoint{1.269018in}{3.302700in}}%
\pgfpathlineto{\pgfqpoint{1.546795in}{3.302700in}}%
\pgfusepath{stroke}%
\end{pgfscope}%
\begin{pgfscope}%
\definecolor{textcolor}{rgb}{0.000000,0.000000,0.000000}%
\pgfsetstrokecolor{textcolor}%
\pgfsetfillcolor{textcolor}%
\pgftext[x=1.657907in,y=3.254089in,left,base]{\color{textcolor}\rmfamily\fontsize{10.000000}{12.000000}\selectfont AUC 0.906}%
\end{pgfscope}%
\begin{pgfscope}%
\pgfsetbuttcap%
\pgfsetmiterjoin%
\definecolor{currentfill}{rgb}{1.000000,1.000000,1.000000}%
\pgfsetfillcolor{currentfill}%
\pgfsetlinewidth{0.000000pt}%
\definecolor{currentstroke}{rgb}{0.000000,0.000000,0.000000}%
\pgfsetstrokecolor{currentstroke}%
\pgfsetstrokeopacity{0.000000}%
\pgfsetdash{}{0pt}%
\pgfpathmoveto{\pgfqpoint{3.225541in}{3.102083in}}%
\pgfpathlineto{\pgfqpoint{4.912846in}{3.102083in}}%
\pgfpathlineto{\pgfqpoint{4.912846in}{4.641628in}}%
\pgfpathlineto{\pgfqpoint{3.225541in}{4.641628in}}%
\pgfpathclose%
\pgfusepath{fill}%
\end{pgfscope}%
\begin{pgfscope}%
\pgfsetbuttcap%
\pgfsetroundjoin%
\definecolor{currentfill}{rgb}{0.000000,0.000000,0.000000}%
\pgfsetfillcolor{currentfill}%
\pgfsetlinewidth{0.803000pt}%
\definecolor{currentstroke}{rgb}{0.000000,0.000000,0.000000}%
\pgfsetstrokecolor{currentstroke}%
\pgfsetdash{}{0pt}%
\pgfsys@defobject{currentmarker}{\pgfqpoint{0.000000in}{-0.048611in}}{\pgfqpoint{0.000000in}{0.000000in}}{%
\pgfpathmoveto{\pgfqpoint{0.000000in}{0.000000in}}%
\pgfpathlineto{\pgfqpoint{0.000000in}{-0.048611in}}%
\pgfusepath{stroke,fill}%
}%
\begin{pgfscope}%
\pgfsys@transformshift{3.302236in}{3.102083in}%
\pgfsys@useobject{currentmarker}{}%
\end{pgfscope}%
\end{pgfscope}%
\begin{pgfscope}%
\definecolor{textcolor}{rgb}{0.000000,0.000000,0.000000}%
\pgfsetstrokecolor{textcolor}%
\pgfsetfillcolor{textcolor}%
\pgftext[x=3.302236in,y=3.004861in,,top]{\color{textcolor}\rmfamily\fontsize{10.000000}{12.000000}\selectfont \(\displaystyle {0.0}\)}%
\end{pgfscope}%
\begin{pgfscope}%
\pgfsetbuttcap%
\pgfsetroundjoin%
\definecolor{currentfill}{rgb}{0.000000,0.000000,0.000000}%
\pgfsetfillcolor{currentfill}%
\pgfsetlinewidth{0.803000pt}%
\definecolor{currentstroke}{rgb}{0.000000,0.000000,0.000000}%
\pgfsetstrokecolor{currentstroke}%
\pgfsetdash{}{0pt}%
\pgfsys@defobject{currentmarker}{\pgfqpoint{0.000000in}{-0.048611in}}{\pgfqpoint{0.000000in}{0.000000in}}{%
\pgfpathmoveto{\pgfqpoint{0.000000in}{0.000000in}}%
\pgfpathlineto{\pgfqpoint{0.000000in}{-0.048611in}}%
\pgfusepath{stroke,fill}%
}%
\begin{pgfscope}%
\pgfsys@transformshift{4.069193in}{3.102083in}%
\pgfsys@useobject{currentmarker}{}%
\end{pgfscope}%
\end{pgfscope}%
\begin{pgfscope}%
\definecolor{textcolor}{rgb}{0.000000,0.000000,0.000000}%
\pgfsetstrokecolor{textcolor}%
\pgfsetfillcolor{textcolor}%
\pgftext[x=4.069193in,y=3.004861in,,top]{\color{textcolor}\rmfamily\fontsize{10.000000}{12.000000}\selectfont \(\displaystyle {0.5}\)}%
\end{pgfscope}%
\begin{pgfscope}%
\pgfsetbuttcap%
\pgfsetroundjoin%
\definecolor{currentfill}{rgb}{0.000000,0.000000,0.000000}%
\pgfsetfillcolor{currentfill}%
\pgfsetlinewidth{0.803000pt}%
\definecolor{currentstroke}{rgb}{0.000000,0.000000,0.000000}%
\pgfsetstrokecolor{currentstroke}%
\pgfsetdash{}{0pt}%
\pgfsys@defobject{currentmarker}{\pgfqpoint{0.000000in}{-0.048611in}}{\pgfqpoint{0.000000in}{0.000000in}}{%
\pgfpathmoveto{\pgfqpoint{0.000000in}{0.000000in}}%
\pgfpathlineto{\pgfqpoint{0.000000in}{-0.048611in}}%
\pgfusepath{stroke,fill}%
}%
\begin{pgfscope}%
\pgfsys@transformshift{4.836150in}{3.102083in}%
\pgfsys@useobject{currentmarker}{}%
\end{pgfscope}%
\end{pgfscope}%
\begin{pgfscope}%
\definecolor{textcolor}{rgb}{0.000000,0.000000,0.000000}%
\pgfsetstrokecolor{textcolor}%
\pgfsetfillcolor{textcolor}%
\pgftext[x=4.836150in,y=3.004861in,,top]{\color{textcolor}\rmfamily\fontsize{10.000000}{12.000000}\selectfont \(\displaystyle {1.0}\)}%
\end{pgfscope}%
\begin{pgfscope}%
\definecolor{textcolor}{rgb}{0.000000,0.000000,0.000000}%
\pgfsetstrokecolor{textcolor}%
\pgfsetfillcolor{textcolor}%
\pgftext[x=4.069193in,y=2.825849in,,top]{\color{textcolor}\rmfamily\fontsize{16.000000}{19.200000}\selectfont FPR}%
\end{pgfscope}%
\begin{pgfscope}%
\pgfsetbuttcap%
\pgfsetroundjoin%
\definecolor{currentfill}{rgb}{0.000000,0.000000,0.000000}%
\pgfsetfillcolor{currentfill}%
\pgfsetlinewidth{0.803000pt}%
\definecolor{currentstroke}{rgb}{0.000000,0.000000,0.000000}%
\pgfsetstrokecolor{currentstroke}%
\pgfsetdash{}{0pt}%
\pgfsys@defobject{currentmarker}{\pgfqpoint{-0.048611in}{0.000000in}}{\pgfqpoint{-0.000000in}{0.000000in}}{%
\pgfpathmoveto{\pgfqpoint{-0.000000in}{0.000000in}}%
\pgfpathlineto{\pgfqpoint{-0.048611in}{0.000000in}}%
\pgfusepath{stroke,fill}%
}%
\begin{pgfscope}%
\pgfsys@transformshift{3.225541in}{3.172062in}%
\pgfsys@useobject{currentmarker}{}%
\end{pgfscope}%
\end{pgfscope}%
\begin{pgfscope}%
\definecolor{textcolor}{rgb}{0.000000,0.000000,0.000000}%
\pgfsetstrokecolor{textcolor}%
\pgfsetfillcolor{textcolor}%
\pgftext[x=2.881404in, y=3.123837in, left, base]{\color{textcolor}\rmfamily\fontsize{10.000000}{12.000000}\selectfont \(\displaystyle {0.00}\)}%
\end{pgfscope}%
\begin{pgfscope}%
\pgfsetbuttcap%
\pgfsetroundjoin%
\definecolor{currentfill}{rgb}{0.000000,0.000000,0.000000}%
\pgfsetfillcolor{currentfill}%
\pgfsetlinewidth{0.803000pt}%
\definecolor{currentstroke}{rgb}{0.000000,0.000000,0.000000}%
\pgfsetstrokecolor{currentstroke}%
\pgfsetdash{}{0pt}%
\pgfsys@defobject{currentmarker}{\pgfqpoint{-0.048611in}{0.000000in}}{\pgfqpoint{-0.000000in}{0.000000in}}{%
\pgfpathmoveto{\pgfqpoint{-0.000000in}{0.000000in}}%
\pgfpathlineto{\pgfqpoint{-0.048611in}{0.000000in}}%
\pgfusepath{stroke,fill}%
}%
\begin{pgfscope}%
\pgfsys@transformshift{3.225541in}{3.521959in}%
\pgfsys@useobject{currentmarker}{}%
\end{pgfscope}%
\end{pgfscope}%
\begin{pgfscope}%
\definecolor{textcolor}{rgb}{0.000000,0.000000,0.000000}%
\pgfsetstrokecolor{textcolor}%
\pgfsetfillcolor{textcolor}%
\pgftext[x=2.881404in, y=3.473734in, left, base]{\color{textcolor}\rmfamily\fontsize{10.000000}{12.000000}\selectfont \(\displaystyle {0.25}\)}%
\end{pgfscope}%
\begin{pgfscope}%
\pgfsetbuttcap%
\pgfsetroundjoin%
\definecolor{currentfill}{rgb}{0.000000,0.000000,0.000000}%
\pgfsetfillcolor{currentfill}%
\pgfsetlinewidth{0.803000pt}%
\definecolor{currentstroke}{rgb}{0.000000,0.000000,0.000000}%
\pgfsetstrokecolor{currentstroke}%
\pgfsetdash{}{0pt}%
\pgfsys@defobject{currentmarker}{\pgfqpoint{-0.048611in}{0.000000in}}{\pgfqpoint{-0.000000in}{0.000000in}}{%
\pgfpathmoveto{\pgfqpoint{-0.000000in}{0.000000in}}%
\pgfpathlineto{\pgfqpoint{-0.048611in}{0.000000in}}%
\pgfusepath{stroke,fill}%
}%
\begin{pgfscope}%
\pgfsys@transformshift{3.225541in}{3.871856in}%
\pgfsys@useobject{currentmarker}{}%
\end{pgfscope}%
\end{pgfscope}%
\begin{pgfscope}%
\definecolor{textcolor}{rgb}{0.000000,0.000000,0.000000}%
\pgfsetstrokecolor{textcolor}%
\pgfsetfillcolor{textcolor}%
\pgftext[x=2.881404in, y=3.823630in, left, base]{\color{textcolor}\rmfamily\fontsize{10.000000}{12.000000}\selectfont \(\displaystyle {0.50}\)}%
\end{pgfscope}%
\begin{pgfscope}%
\pgfsetbuttcap%
\pgfsetroundjoin%
\definecolor{currentfill}{rgb}{0.000000,0.000000,0.000000}%
\pgfsetfillcolor{currentfill}%
\pgfsetlinewidth{0.803000pt}%
\definecolor{currentstroke}{rgb}{0.000000,0.000000,0.000000}%
\pgfsetstrokecolor{currentstroke}%
\pgfsetdash{}{0pt}%
\pgfsys@defobject{currentmarker}{\pgfqpoint{-0.048611in}{0.000000in}}{\pgfqpoint{-0.000000in}{0.000000in}}{%
\pgfpathmoveto{\pgfqpoint{-0.000000in}{0.000000in}}%
\pgfpathlineto{\pgfqpoint{-0.048611in}{0.000000in}}%
\pgfusepath{stroke,fill}%
}%
\begin{pgfscope}%
\pgfsys@transformshift{3.225541in}{4.221752in}%
\pgfsys@useobject{currentmarker}{}%
\end{pgfscope}%
\end{pgfscope}%
\begin{pgfscope}%
\definecolor{textcolor}{rgb}{0.000000,0.000000,0.000000}%
\pgfsetstrokecolor{textcolor}%
\pgfsetfillcolor{textcolor}%
\pgftext[x=2.881404in, y=4.173527in, left, base]{\color{textcolor}\rmfamily\fontsize{10.000000}{12.000000}\selectfont \(\displaystyle {0.75}\)}%
\end{pgfscope}%
\begin{pgfscope}%
\pgfsetbuttcap%
\pgfsetroundjoin%
\definecolor{currentfill}{rgb}{0.000000,0.000000,0.000000}%
\pgfsetfillcolor{currentfill}%
\pgfsetlinewidth{0.803000pt}%
\definecolor{currentstroke}{rgb}{0.000000,0.000000,0.000000}%
\pgfsetstrokecolor{currentstroke}%
\pgfsetdash{}{0pt}%
\pgfsys@defobject{currentmarker}{\pgfqpoint{-0.048611in}{0.000000in}}{\pgfqpoint{-0.000000in}{0.000000in}}{%
\pgfpathmoveto{\pgfqpoint{-0.000000in}{0.000000in}}%
\pgfpathlineto{\pgfqpoint{-0.048611in}{0.000000in}}%
\pgfusepath{stroke,fill}%
}%
\begin{pgfscope}%
\pgfsys@transformshift{3.225541in}{4.571649in}%
\pgfsys@useobject{currentmarker}{}%
\end{pgfscope}%
\end{pgfscope}%
\begin{pgfscope}%
\definecolor{textcolor}{rgb}{0.000000,0.000000,0.000000}%
\pgfsetstrokecolor{textcolor}%
\pgfsetfillcolor{textcolor}%
\pgftext[x=2.881404in, y=4.523424in, left, base]{\color{textcolor}\rmfamily\fontsize{10.000000}{12.000000}\selectfont \(\displaystyle {1.00}\)}%
\end{pgfscope}%
\begin{pgfscope}%
\definecolor{textcolor}{rgb}{0.000000,0.000000,0.000000}%
\pgfsetstrokecolor{textcolor}%
\pgfsetfillcolor{textcolor}%
\pgftext[x=2.825849in,y=3.871856in,,bottom,rotate=90.000000]{\color{textcolor}\rmfamily\fontsize{16.000000}{19.200000}\selectfont TPR}%
\end{pgfscope}%
\begin{pgfscope}%
\pgfpathrectangle{\pgfqpoint{3.225541in}{3.102083in}}{\pgfqpoint{1.687305in}{1.539545in}}%
\pgfusepath{clip}%
\pgfsetrectcap%
\pgfsetroundjoin%
\pgfsetlinewidth{1.505625pt}%
\definecolor{currentstroke}{rgb}{0.000000,0.501961,0.000000}%
\pgfsetstrokecolor{currentstroke}%
\pgfsetdash{}{0pt}%
\pgfpathmoveto{\pgfqpoint{3.302236in}{3.172062in}}%
\pgfpathlineto{\pgfqpoint{3.305742in}{3.203899in}}%
\pgfpathlineto{\pgfqpoint{3.306592in}{3.205123in}}%
\pgfpathlineto{\pgfqpoint{3.308929in}{3.229613in}}%
\pgfpathlineto{\pgfqpoint{3.309514in}{3.230838in}}%
\pgfpathlineto{\pgfqpoint{3.311692in}{3.246756in}}%
\pgfpathlineto{\pgfqpoint{3.312595in}{3.247980in}}%
\pgfpathlineto{\pgfqpoint{3.314082in}{3.262674in}}%
\pgfpathlineto{\pgfqpoint{3.314294in}{3.263899in}}%
\pgfpathlineto{\pgfqpoint{3.315729in}{3.273695in}}%
\pgfpathlineto{\pgfqpoint{3.315835in}{3.273695in}}%
\pgfpathlineto{\pgfqpoint{3.316366in}{3.274919in}}%
\pgfpathlineto{\pgfqpoint{3.317535in}{3.284715in}}%
\pgfpathlineto{\pgfqpoint{3.318650in}{3.285939in}}%
\pgfpathlineto{\pgfqpoint{3.319659in}{3.296960in}}%
\pgfpathlineto{\pgfqpoint{3.320191in}{3.296960in}}%
\pgfpathlineto{\pgfqpoint{3.320403in}{3.298184in}}%
\pgfpathlineto{\pgfqpoint{3.321678in}{3.316552in}}%
\pgfpathlineto{\pgfqpoint{3.322740in}{3.316552in}}%
\pgfpathlineto{\pgfqpoint{3.324068in}{3.328796in}}%
\pgfpathlineto{\pgfqpoint{3.324228in}{3.328796in}}%
\pgfpathlineto{\pgfqpoint{3.324546in}{3.330021in}}%
\pgfpathlineto{\pgfqpoint{3.325768in}{3.338592in}}%
\pgfpathlineto{\pgfqpoint{3.326034in}{3.338592in}}%
\pgfpathlineto{\pgfqpoint{3.326405in}{3.339817in}}%
\pgfpathlineto{\pgfqpoint{3.327787in}{3.344715in}}%
\pgfpathlineto{\pgfqpoint{3.328530in}{3.345939in}}%
\pgfpathlineto{\pgfqpoint{3.330974in}{3.371653in}}%
\pgfpathlineto{\pgfqpoint{3.331611in}{3.372878in}}%
\pgfpathlineto{\pgfqpoint{3.333098in}{3.391245in}}%
\pgfpathlineto{\pgfqpoint{3.333683in}{3.392470in}}%
\pgfpathlineto{\pgfqpoint{3.335223in}{3.405939in}}%
\pgfpathlineto{\pgfqpoint{3.335542in}{3.407164in}}%
\pgfpathlineto{\pgfqpoint{3.336870in}{3.421857in}}%
\pgfpathlineto{\pgfqpoint{3.337454in}{3.423082in}}%
\pgfpathlineto{\pgfqpoint{3.338092in}{3.430429in}}%
\pgfpathlineto{\pgfqpoint{3.340323in}{3.431653in}}%
\pgfpathlineto{\pgfqpoint{3.341597in}{3.439000in}}%
\pgfpathlineto{\pgfqpoint{3.341810in}{3.439000in}}%
\pgfpathlineto{\pgfqpoint{3.342819in}{3.440225in}}%
\pgfpathlineto{\pgfqpoint{3.344360in}{3.445123in}}%
\pgfpathlineto{\pgfqpoint{3.345050in}{3.446347in}}%
\pgfpathlineto{\pgfqpoint{3.346431in}{3.458592in}}%
\pgfpathlineto{\pgfqpoint{3.348184in}{3.459816in}}%
\pgfpathlineto{\pgfqpoint{3.349140in}{3.472061in}}%
\pgfpathlineto{\pgfqpoint{3.350309in}{3.473286in}}%
\pgfpathlineto{\pgfqpoint{3.351690in}{3.483082in}}%
\pgfpathlineto{\pgfqpoint{3.352221in}{3.484306in}}%
\pgfpathlineto{\pgfqpoint{3.353496in}{3.494102in}}%
\pgfpathlineto{\pgfqpoint{3.354399in}{3.495326in}}%
\pgfpathlineto{\pgfqpoint{3.355037in}{3.499000in}}%
\pgfpathlineto{\pgfqpoint{3.355993in}{3.499000in}}%
\pgfpathlineto{\pgfqpoint{3.357480in}{3.505122in}}%
\pgfpathlineto{\pgfqpoint{3.357905in}{3.506347in}}%
\pgfpathlineto{\pgfqpoint{3.359339in}{3.519816in}}%
\pgfpathlineto{\pgfqpoint{3.360455in}{3.519816in}}%
\pgfpathlineto{\pgfqpoint{3.361942in}{3.532061in}}%
\pgfpathlineto{\pgfqpoint{3.362208in}{3.533285in}}%
\pgfpathlineto{\pgfqpoint{3.364226in}{3.545530in}}%
\pgfpathlineto{\pgfqpoint{3.365129in}{3.546755in}}%
\pgfpathlineto{\pgfqpoint{3.367041in}{3.557775in}}%
\pgfpathlineto{\pgfqpoint{3.368157in}{3.559000in}}%
\pgfpathlineto{\pgfqpoint{3.369697in}{3.562673in}}%
\pgfpathlineto{\pgfqpoint{3.370335in}{3.563898in}}%
\pgfpathlineto{\pgfqpoint{3.371132in}{3.572469in}}%
\pgfpathlineto{\pgfqpoint{3.372513in}{3.573693in}}%
\pgfpathlineto{\pgfqpoint{3.374053in}{3.577367in}}%
\pgfpathlineto{\pgfqpoint{3.374372in}{3.577367in}}%
\pgfpathlineto{\pgfqpoint{3.375594in}{3.585938in}}%
\pgfpathlineto{\pgfqpoint{3.376762in}{3.587163in}}%
\pgfpathlineto{\pgfqpoint{3.378250in}{3.593285in}}%
\pgfpathlineto{\pgfqpoint{3.379259in}{3.594510in}}%
\pgfpathlineto{\pgfqpoint{3.380640in}{3.607979in}}%
\pgfpathlineto{\pgfqpoint{3.380746in}{3.607979in}}%
\pgfpathlineto{\pgfqpoint{3.381490in}{3.609204in}}%
\pgfpathlineto{\pgfqpoint{3.382765in}{3.615326in}}%
\pgfpathlineto{\pgfqpoint{3.383615in}{3.616550in}}%
\pgfpathlineto{\pgfqpoint{3.385049in}{3.626346in}}%
\pgfpathlineto{\pgfqpoint{3.386695in}{3.627571in}}%
\pgfpathlineto{\pgfqpoint{3.388130in}{3.637367in}}%
\pgfpathlineto{\pgfqpoint{3.389404in}{3.638591in}}%
\pgfpathlineto{\pgfqpoint{3.390786in}{3.649612in}}%
\pgfpathlineto{\pgfqpoint{3.391848in}{3.650836in}}%
\pgfpathlineto{\pgfqpoint{3.393229in}{3.653285in}}%
\pgfpathlineto{\pgfqpoint{3.393760in}{3.654510in}}%
\pgfpathlineto{\pgfqpoint{3.394716in}{3.665530in}}%
\pgfpathlineto{\pgfqpoint{3.395885in}{3.666754in}}%
\pgfpathlineto{\pgfqpoint{3.397425in}{3.677775in}}%
\pgfpathlineto{\pgfqpoint{3.398328in}{3.678999in}}%
\pgfpathlineto{\pgfqpoint{3.399497in}{3.683897in}}%
\pgfpathlineto{\pgfqpoint{3.400188in}{3.685122in}}%
\pgfpathlineto{\pgfqpoint{3.401462in}{3.687571in}}%
\pgfpathlineto{\pgfqpoint{3.403959in}{3.688795in}}%
\pgfpathlineto{\pgfqpoint{3.404862in}{3.694918in}}%
\pgfpathlineto{\pgfqpoint{3.405606in}{3.694918in}}%
\pgfpathlineto{\pgfqpoint{3.406934in}{3.702264in}}%
\pgfpathlineto{\pgfqpoint{3.407412in}{3.703489in}}%
\pgfpathlineto{\pgfqpoint{3.408474in}{3.710836in}}%
\pgfpathlineto{\pgfqpoint{3.410015in}{3.712060in}}%
\pgfpathlineto{\pgfqpoint{3.411396in}{3.715734in}}%
\pgfpathlineto{\pgfqpoint{3.412192in}{3.715734in}}%
\pgfpathlineto{\pgfqpoint{3.413042in}{3.719407in}}%
\pgfpathlineto{\pgfqpoint{3.414583in}{3.720632in}}%
\pgfpathlineto{\pgfqpoint{3.415858in}{3.727979in}}%
\pgfpathlineto{\pgfqpoint{3.417079in}{3.729203in}}%
\pgfpathlineto{\pgfqpoint{3.418195in}{3.732877in}}%
\pgfpathlineto{\pgfqpoint{3.418939in}{3.734101in}}%
\pgfpathlineto{\pgfqpoint{3.419629in}{3.738999in}}%
\pgfpathlineto{\pgfqpoint{3.422232in}{3.740223in}}%
\pgfpathlineto{\pgfqpoint{3.423454in}{3.742672in}}%
\pgfpathlineto{\pgfqpoint{3.426003in}{3.743897in}}%
\pgfpathlineto{\pgfqpoint{3.427491in}{3.748795in}}%
\pgfpathlineto{\pgfqpoint{3.427756in}{3.750019in}}%
\pgfpathlineto{\pgfqpoint{3.428978in}{3.758591in}}%
\pgfpathlineto{\pgfqpoint{3.430094in}{3.759815in}}%
\pgfpathlineto{\pgfqpoint{3.431528in}{3.765938in}}%
\pgfpathlineto{\pgfqpoint{3.432537in}{3.767162in}}%
\pgfpathlineto{\pgfqpoint{3.433918in}{3.770836in}}%
\pgfpathlineto{\pgfqpoint{3.434290in}{3.770836in}}%
\pgfpathlineto{\pgfqpoint{3.435618in}{3.775734in}}%
\pgfpathlineto{\pgfqpoint{3.436415in}{3.776958in}}%
\pgfpathlineto{\pgfqpoint{3.437318in}{3.780631in}}%
\pgfpathlineto{\pgfqpoint{3.438433in}{3.781856in}}%
\pgfpathlineto{\pgfqpoint{3.439655in}{3.785529in}}%
\pgfpathlineto{\pgfqpoint{3.440133in}{3.786754in}}%
\pgfpathlineto{\pgfqpoint{3.441620in}{3.789203in}}%
\pgfpathlineto{\pgfqpoint{3.442417in}{3.790427in}}%
\pgfpathlineto{\pgfqpoint{3.443798in}{3.797774in}}%
\pgfpathlineto{\pgfqpoint{3.445870in}{3.798999in}}%
\pgfpathlineto{\pgfqpoint{3.447092in}{3.802672in}}%
\pgfpathlineto{\pgfqpoint{3.448473in}{3.803897in}}%
\pgfpathlineto{\pgfqpoint{3.449535in}{3.808795in}}%
\pgfpathlineto{\pgfqpoint{3.451500in}{3.810019in}}%
\pgfpathlineto{\pgfqpoint{3.452244in}{3.813693in}}%
\pgfpathlineto{\pgfqpoint{3.455591in}{3.814917in}}%
\pgfpathlineto{\pgfqpoint{3.455591in}{3.817366in}}%
\pgfpathlineto{\pgfqpoint{3.457184in}{3.817366in}}%
\pgfpathlineto{\pgfqpoint{3.458618in}{3.822264in}}%
\pgfpathlineto{\pgfqpoint{3.461221in}{3.823488in}}%
\pgfpathlineto{\pgfqpoint{3.462762in}{3.828386in}}%
\pgfpathlineto{\pgfqpoint{3.463612in}{3.829611in}}%
\pgfpathlineto{\pgfqpoint{3.463771in}{3.832060in}}%
\pgfpathlineto{\pgfqpoint{3.465843in}{3.833284in}}%
\pgfpathlineto{\pgfqpoint{3.466533in}{3.836958in}}%
\pgfpathlineto{\pgfqpoint{3.468817in}{3.838182in}}%
\pgfpathlineto{\pgfqpoint{3.470251in}{3.844305in}}%
\pgfpathlineto{\pgfqpoint{3.470676in}{3.845529in}}%
\pgfpathlineto{\pgfqpoint{3.471845in}{3.852876in}}%
\pgfpathlineto{\pgfqpoint{3.472589in}{3.854101in}}%
\pgfpathlineto{\pgfqpoint{3.474023in}{3.857774in}}%
\pgfpathlineto{\pgfqpoint{3.474607in}{3.858999in}}%
\pgfpathlineto{\pgfqpoint{3.476095in}{3.861448in}}%
\pgfpathlineto{\pgfqpoint{3.476785in}{3.862672in}}%
\pgfpathlineto{\pgfqpoint{3.478272in}{3.870019in}}%
\pgfpathlineto{\pgfqpoint{3.480185in}{3.871243in}}%
\pgfpathlineto{\pgfqpoint{3.481725in}{3.874917in}}%
\pgfpathlineto{\pgfqpoint{3.483744in}{3.874917in}}%
\pgfpathlineto{\pgfqpoint{3.484594in}{3.882264in}}%
\pgfpathlineto{\pgfqpoint{3.487728in}{3.883488in}}%
\pgfpathlineto{\pgfqpoint{3.487728in}{3.884713in}}%
\pgfpathlineto{\pgfqpoint{3.491658in}{3.885937in}}%
\pgfpathlineto{\pgfqpoint{3.492349in}{3.888386in}}%
\pgfpathlineto{\pgfqpoint{3.494261in}{3.889611in}}%
\pgfpathlineto{\pgfqpoint{3.495642in}{3.896958in}}%
\pgfpathlineto{\pgfqpoint{3.496652in}{3.898182in}}%
\pgfpathlineto{\pgfqpoint{3.497873in}{3.903080in}}%
\pgfpathlineto{\pgfqpoint{3.499732in}{3.904304in}}%
\pgfpathlineto{\pgfqpoint{3.501273in}{3.910427in}}%
\pgfpathlineto{\pgfqpoint{3.501432in}{3.911651in}}%
\pgfpathlineto{\pgfqpoint{3.502282in}{3.915325in}}%
\pgfpathlineto{\pgfqpoint{3.504832in}{3.916549in}}%
\pgfpathlineto{\pgfqpoint{3.505575in}{3.921447in}}%
\pgfpathlineto{\pgfqpoint{3.507541in}{3.922672in}}%
\pgfpathlineto{\pgfqpoint{3.508019in}{3.926345in}}%
\pgfpathlineto{\pgfqpoint{3.510356in}{3.927570in}}%
\pgfpathlineto{\pgfqpoint{3.510887in}{3.930019in}}%
\pgfpathlineto{\pgfqpoint{3.513650in}{3.931243in}}%
\pgfpathlineto{\pgfqpoint{3.514924in}{3.938590in}}%
\pgfpathlineto{\pgfqpoint{3.515774in}{3.939815in}}%
\pgfpathlineto{\pgfqpoint{3.516890in}{3.945937in}}%
\pgfpathlineto{\pgfqpoint{3.518005in}{3.947161in}}%
\pgfpathlineto{\pgfqpoint{3.519174in}{3.953284in}}%
\pgfpathlineto{\pgfqpoint{3.519971in}{3.954508in}}%
\pgfpathlineto{\pgfqpoint{3.520555in}{3.958182in}}%
\pgfpathlineto{\pgfqpoint{3.521671in}{3.959406in}}%
\pgfpathlineto{\pgfqpoint{3.523052in}{3.964304in}}%
\pgfpathlineto{\pgfqpoint{3.523795in}{3.965529in}}%
\pgfpathlineto{\pgfqpoint{3.524326in}{3.969202in}}%
\pgfpathlineto{\pgfqpoint{3.527992in}{3.970427in}}%
\pgfpathlineto{\pgfqpoint{3.528788in}{3.975325in}}%
\pgfpathlineto{\pgfqpoint{3.532135in}{3.976549in}}%
\pgfpathlineto{\pgfqpoint{3.533463in}{3.983896in}}%
\pgfpathlineto{\pgfqpoint{3.535056in}{3.985120in}}%
\pgfpathlineto{\pgfqpoint{3.536172in}{3.993692in}}%
\pgfpathlineto{\pgfqpoint{3.537234in}{3.994916in}}%
\pgfpathlineto{\pgfqpoint{3.538244in}{3.999814in}}%
\pgfpathlineto{\pgfqpoint{3.539359in}{3.999814in}}%
\pgfpathlineto{\pgfqpoint{3.539518in}{4.004712in}}%
\pgfpathlineto{\pgfqpoint{3.542440in}{4.005937in}}%
\pgfpathlineto{\pgfqpoint{3.543343in}{4.009610in}}%
\pgfpathlineto{\pgfqpoint{3.544618in}{4.010835in}}%
\pgfpathlineto{\pgfqpoint{3.545149in}{4.013284in}}%
\pgfpathlineto{\pgfqpoint{3.547061in}{4.014508in}}%
\pgfpathlineto{\pgfqpoint{3.548389in}{4.018182in}}%
\pgfpathlineto{\pgfqpoint{3.550833in}{4.019406in}}%
\pgfpathlineto{\pgfqpoint{3.551576in}{4.021855in}}%
\pgfpathlineto{\pgfqpoint{3.555560in}{4.023080in}}%
\pgfpathlineto{\pgfqpoint{3.555879in}{4.026753in}}%
\pgfpathlineto{\pgfqpoint{3.559172in}{4.027977in}}%
\pgfpathlineto{\pgfqpoint{3.559969in}{4.032875in}}%
\pgfpathlineto{\pgfqpoint{3.561085in}{4.034100in}}%
\pgfpathlineto{\pgfqpoint{3.561457in}{4.037773in}}%
\pgfpathlineto{\pgfqpoint{3.562997in}{4.038998in}}%
\pgfpathlineto{\pgfqpoint{3.563369in}{4.041447in}}%
\pgfpathlineto{\pgfqpoint{3.565387in}{4.042671in}}%
\pgfpathlineto{\pgfqpoint{3.566184in}{4.046345in}}%
\pgfpathlineto{\pgfqpoint{3.570593in}{4.047569in}}%
\pgfpathlineto{\pgfqpoint{3.570752in}{4.050018in}}%
\pgfpathlineto{\pgfqpoint{3.574046in}{4.051243in}}%
\pgfpathlineto{\pgfqpoint{3.575427in}{4.059814in}}%
\pgfpathlineto{\pgfqpoint{3.578189in}{4.061039in}}%
\pgfpathlineto{\pgfqpoint{3.578189in}{4.062263in}}%
\pgfpathlineto{\pgfqpoint{3.580261in}{4.063488in}}%
\pgfpathlineto{\pgfqpoint{3.580579in}{4.067161in}}%
\pgfpathlineto{\pgfqpoint{3.583288in}{4.068385in}}%
\pgfpathlineto{\pgfqpoint{3.584457in}{4.075732in}}%
\pgfpathlineto{\pgfqpoint{3.587272in}{4.076957in}}%
\pgfpathlineto{\pgfqpoint{3.587272in}{4.078181in}}%
\pgfpathlineto{\pgfqpoint{3.590884in}{4.079406in}}%
\pgfpathlineto{\pgfqpoint{3.592212in}{4.083079in}}%
\pgfpathlineto{\pgfqpoint{3.593222in}{4.084304in}}%
\pgfpathlineto{\pgfqpoint{3.594178in}{4.089202in}}%
\pgfpathlineto{\pgfqpoint{3.595187in}{4.090426in}}%
\pgfpathlineto{\pgfqpoint{3.595187in}{4.091651in}}%
\pgfpathlineto{\pgfqpoint{3.597152in}{4.092875in}}%
\pgfpathlineto{\pgfqpoint{3.597790in}{4.095324in}}%
\pgfpathlineto{\pgfqpoint{3.598215in}{4.095324in}}%
\pgfpathlineto{\pgfqpoint{3.600499in}{4.096549in}}%
\pgfpathlineto{\pgfqpoint{3.601561in}{4.098998in}}%
\pgfpathlineto{\pgfqpoint{3.602358in}{4.100222in}}%
\pgfpathlineto{\pgfqpoint{3.603686in}{4.107569in}}%
\pgfpathlineto{\pgfqpoint{3.607245in}{4.108793in}}%
\pgfpathlineto{\pgfqpoint{3.608785in}{4.112467in}}%
\pgfpathlineto{\pgfqpoint{3.609370in}{4.113691in}}%
\pgfpathlineto{\pgfqpoint{3.610538in}{4.118589in}}%
\pgfpathlineto{\pgfqpoint{3.611813in}{4.119814in}}%
\pgfpathlineto{\pgfqpoint{3.611813in}{4.121038in}}%
\pgfpathlineto{\pgfqpoint{3.614682in}{4.122263in}}%
\pgfpathlineto{\pgfqpoint{3.616010in}{4.127161in}}%
\pgfpathlineto{\pgfqpoint{3.616806in}{4.127161in}}%
\pgfpathlineto{\pgfqpoint{3.617603in}{4.133283in}}%
\pgfpathlineto{\pgfqpoint{3.622490in}{4.134508in}}%
\pgfpathlineto{\pgfqpoint{3.623234in}{4.138181in}}%
\pgfpathlineto{\pgfqpoint{3.629342in}{4.139406in}}%
\pgfpathlineto{\pgfqpoint{3.630352in}{4.141855in}}%
\pgfpathlineto{\pgfqpoint{3.633698in}{4.143079in}}%
\pgfpathlineto{\pgfqpoint{3.633698in}{4.145528in}}%
\pgfpathlineto{\pgfqpoint{3.637735in}{4.146753in}}%
\pgfpathlineto{\pgfqpoint{3.638745in}{4.152875in}}%
\pgfpathlineto{\pgfqpoint{3.639648in}{4.152875in}}%
\pgfpathlineto{\pgfqpoint{3.639648in}{4.155324in}}%
\pgfpathlineto{\pgfqpoint{3.642941in}{4.156548in}}%
\pgfpathlineto{\pgfqpoint{3.644375in}{4.161446in}}%
\pgfpathlineto{\pgfqpoint{3.646234in}{4.162671in}}%
\pgfpathlineto{\pgfqpoint{3.646553in}{4.165120in}}%
\pgfpathlineto{\pgfqpoint{3.649475in}{4.166344in}}%
\pgfpathlineto{\pgfqpoint{3.650271in}{4.170018in}}%
\pgfpathlineto{\pgfqpoint{3.653671in}{4.171242in}}%
\pgfpathlineto{\pgfqpoint{3.654255in}{4.173691in}}%
\pgfpathlineto{\pgfqpoint{3.655955in}{4.174916in}}%
\pgfpathlineto{\pgfqpoint{3.656539in}{4.177365in}}%
\pgfpathlineto{\pgfqpoint{3.658399in}{4.178589in}}%
\pgfpathlineto{\pgfqpoint{3.659302in}{4.181038in}}%
\pgfpathlineto{\pgfqpoint{3.663392in}{4.182263in}}%
\pgfpathlineto{\pgfqpoint{3.664401in}{4.184712in}}%
\pgfpathlineto{\pgfqpoint{3.668650in}{4.185936in}}%
\pgfpathlineto{\pgfqpoint{3.668650in}{4.187161in}}%
\pgfpathlineto{\pgfqpoint{3.671625in}{4.188385in}}%
\pgfpathlineto{\pgfqpoint{3.672156in}{4.190834in}}%
\pgfpathlineto{\pgfqpoint{3.677681in}{4.192058in}}%
\pgfpathlineto{\pgfqpoint{3.679168in}{4.195732in}}%
\pgfpathlineto{\pgfqpoint{3.680284in}{4.196956in}}%
\pgfpathlineto{\pgfqpoint{3.680974in}{4.199405in}}%
\pgfpathlineto{\pgfqpoint{3.682302in}{4.200630in}}%
\pgfpathlineto{\pgfqpoint{3.682780in}{4.203079in}}%
\pgfpathlineto{\pgfqpoint{3.684480in}{4.204303in}}%
\pgfpathlineto{\pgfqpoint{3.685967in}{4.209201in}}%
\pgfpathlineto{\pgfqpoint{3.690642in}{4.210426in}}%
\pgfpathlineto{\pgfqpoint{3.690642in}{4.211650in}}%
\pgfpathlineto{\pgfqpoint{3.696325in}{4.212875in}}%
\pgfpathlineto{\pgfqpoint{3.697175in}{4.215324in}}%
\pgfpathlineto{\pgfqpoint{3.697919in}{4.215324in}}%
\pgfpathlineto{\pgfqpoint{3.699300in}{4.221446in}}%
\pgfpathlineto{\pgfqpoint{3.703550in}{4.222671in}}%
\pgfpathlineto{\pgfqpoint{3.704824in}{4.226344in}}%
\pgfpathlineto{\pgfqpoint{3.706684in}{4.227569in}}%
\pgfpathlineto{\pgfqpoint{3.706684in}{4.228793in}}%
\pgfpathlineto{\pgfqpoint{3.708702in}{4.228793in}}%
\pgfpathlineto{\pgfqpoint{3.709818in}{4.234915in}}%
\pgfpathlineto{\pgfqpoint{3.712208in}{4.236140in}}%
\pgfpathlineto{\pgfqpoint{3.712686in}{4.238589in}}%
\pgfpathlineto{\pgfqpoint{3.716139in}{4.239813in}}%
\pgfpathlineto{\pgfqpoint{3.717520in}{4.243487in}}%
\pgfpathlineto{\pgfqpoint{3.717998in}{4.244711in}}%
\pgfpathlineto{\pgfqpoint{3.719273in}{4.247160in}}%
\pgfpathlineto{\pgfqpoint{3.722088in}{4.248385in}}%
\pgfpathlineto{\pgfqpoint{3.722938in}{4.252058in}}%
\pgfpathlineto{\pgfqpoint{3.724903in}{4.253283in}}%
\pgfpathlineto{\pgfqpoint{3.725860in}{4.258181in}}%
\pgfpathlineto{\pgfqpoint{3.727559in}{4.259405in}}%
\pgfpathlineto{\pgfqpoint{3.728197in}{4.263079in}}%
\pgfpathlineto{\pgfqpoint{3.735687in}{4.264303in}}%
\pgfpathlineto{\pgfqpoint{3.735687in}{4.265528in}}%
\pgfpathlineto{\pgfqpoint{3.740839in}{4.266752in}}%
\pgfpathlineto{\pgfqpoint{3.740839in}{4.267977in}}%
\pgfpathlineto{\pgfqpoint{3.743176in}{4.269201in}}%
\pgfpathlineto{\pgfqpoint{3.743176in}{4.270426in}}%
\pgfpathlineto{\pgfqpoint{3.748913in}{4.271650in}}%
\pgfpathlineto{\pgfqpoint{3.749126in}{4.274099in}}%
\pgfpathlineto{\pgfqpoint{3.754756in}{4.275323in}}%
\pgfpathlineto{\pgfqpoint{3.756031in}{4.277772in}}%
\pgfpathlineto{\pgfqpoint{3.758793in}{4.278997in}}%
\pgfpathlineto{\pgfqpoint{3.759749in}{4.281446in}}%
\pgfpathlineto{\pgfqpoint{3.771489in}{4.282670in}}%
\pgfpathlineto{\pgfqpoint{3.771914in}{4.285119in}}%
\pgfpathlineto{\pgfqpoint{3.773188in}{4.286344in}}%
\pgfpathlineto{\pgfqpoint{3.773188in}{4.287568in}}%
\pgfpathlineto{\pgfqpoint{3.778182in}{4.288793in}}%
\pgfpathlineto{\pgfqpoint{3.779191in}{4.291242in}}%
\pgfpathlineto{\pgfqpoint{3.781103in}{4.292466in}}%
\pgfpathlineto{\pgfqpoint{3.781103in}{4.293691in}}%
\pgfpathlineto{\pgfqpoint{3.784928in}{4.294915in}}%
\pgfpathlineto{\pgfqpoint{3.785937in}{4.297364in}}%
\pgfpathlineto{\pgfqpoint{3.790240in}{4.298589in}}%
\pgfpathlineto{\pgfqpoint{3.791780in}{4.301038in}}%
\pgfpathlineto{\pgfqpoint{3.792736in}{4.302262in}}%
\pgfpathlineto{\pgfqpoint{3.792736in}{4.303487in}}%
\pgfpathlineto{\pgfqpoint{3.798154in}{4.304711in}}%
\pgfpathlineto{\pgfqpoint{3.799535in}{4.307160in}}%
\pgfpathlineto{\pgfqpoint{3.800438in}{4.308385in}}%
\pgfpathlineto{\pgfqpoint{3.800438in}{4.309609in}}%
\pgfpathlineto{\pgfqpoint{3.806919in}{4.309609in}}%
\pgfpathlineto{\pgfqpoint{3.806919in}{4.312058in}}%
\pgfpathlineto{\pgfqpoint{3.808459in}{4.312058in}}%
\pgfpathlineto{\pgfqpoint{3.811700in}{4.313283in}}%
\pgfpathlineto{\pgfqpoint{3.811700in}{4.314507in}}%
\pgfpathlineto{\pgfqpoint{3.815630in}{4.315731in}}%
\pgfpathlineto{\pgfqpoint{3.817118in}{4.318180in}}%
\pgfpathlineto{\pgfqpoint{3.818552in}{4.319405in}}%
\pgfpathlineto{\pgfqpoint{3.819667in}{4.321854in}}%
\pgfpathlineto{\pgfqpoint{3.820358in}{4.323078in}}%
\pgfpathlineto{\pgfqpoint{3.820517in}{4.325527in}}%
\pgfpathlineto{\pgfqpoint{3.826573in}{4.326752in}}%
\pgfpathlineto{\pgfqpoint{3.828060in}{4.329201in}}%
\pgfpathlineto{\pgfqpoint{3.834859in}{4.330425in}}%
\pgfpathlineto{\pgfqpoint{3.834859in}{4.331650in}}%
\pgfpathlineto{\pgfqpoint{3.838897in}{4.332874in}}%
\pgfpathlineto{\pgfqpoint{3.839534in}{4.337772in}}%
\pgfpathlineto{\pgfqpoint{3.840968in}{4.338997in}}%
\pgfpathlineto{\pgfqpoint{3.840968in}{4.340221in}}%
\pgfpathlineto{\pgfqpoint{3.845218in}{4.341446in}}%
\pgfpathlineto{\pgfqpoint{3.845218in}{4.342670in}}%
\pgfpathlineto{\pgfqpoint{3.849839in}{4.343895in}}%
\pgfpathlineto{\pgfqpoint{3.849839in}{4.345119in}}%
\pgfpathlineto{\pgfqpoint{3.856585in}{4.346344in}}%
\pgfpathlineto{\pgfqpoint{3.856585in}{4.347568in}}%
\pgfpathlineto{\pgfqpoint{3.860941in}{4.348793in}}%
\pgfpathlineto{\pgfqpoint{3.861950in}{4.351242in}}%
\pgfpathlineto{\pgfqpoint{3.864340in}{4.351242in}}%
\pgfpathlineto{\pgfqpoint{3.865509in}{4.354915in}}%
\pgfpathlineto{\pgfqpoint{3.868537in}{4.356139in}}%
\pgfpathlineto{\pgfqpoint{3.869281in}{4.359813in}}%
\pgfpathlineto{\pgfqpoint{3.872361in}{4.361037in}}%
\pgfpathlineto{\pgfqpoint{3.873477in}{4.364711in}}%
\pgfpathlineto{\pgfqpoint{3.874752in}{4.365935in}}%
\pgfpathlineto{\pgfqpoint{3.874752in}{4.367160in}}%
\pgfpathlineto{\pgfqpoint{3.879479in}{4.368384in}}%
\pgfpathlineto{\pgfqpoint{3.880860in}{4.370833in}}%
\pgfpathlineto{\pgfqpoint{3.885057in}{4.372058in}}%
\pgfpathlineto{\pgfqpoint{3.885057in}{4.373282in}}%
\pgfpathlineto{\pgfqpoint{3.887341in}{4.374507in}}%
\pgfpathlineto{\pgfqpoint{3.887341in}{4.375731in}}%
\pgfpathlineto{\pgfqpoint{3.890687in}{4.376956in}}%
\pgfpathlineto{\pgfqpoint{3.890687in}{4.378180in}}%
\pgfpathlineto{\pgfqpoint{3.893556in}{4.379405in}}%
\pgfpathlineto{\pgfqpoint{3.893556in}{4.380629in}}%
\pgfpathlineto{\pgfqpoint{3.899399in}{4.381854in}}%
\pgfpathlineto{\pgfqpoint{3.899399in}{4.383078in}}%
\pgfpathlineto{\pgfqpoint{3.907898in}{4.384303in}}%
\pgfpathlineto{\pgfqpoint{3.908695in}{4.386752in}}%
\pgfpathlineto{\pgfqpoint{3.916238in}{4.387976in}}%
\pgfpathlineto{\pgfqpoint{3.916238in}{4.389201in}}%
\pgfpathlineto{\pgfqpoint{3.922081in}{4.390425in}}%
\pgfpathlineto{\pgfqpoint{3.922081in}{4.391650in}}%
\pgfpathlineto{\pgfqpoint{3.928136in}{4.392874in}}%
\pgfpathlineto{\pgfqpoint{3.928136in}{4.394099in}}%
\pgfpathlineto{\pgfqpoint{3.931111in}{4.395323in}}%
\pgfpathlineto{\pgfqpoint{3.931111in}{4.396547in}}%
\pgfpathlineto{\pgfqpoint{3.935679in}{4.397772in}}%
\pgfpathlineto{\pgfqpoint{3.935679in}{4.398996in}}%
\pgfpathlineto{\pgfqpoint{3.941628in}{4.400221in}}%
\pgfpathlineto{\pgfqpoint{3.941628in}{4.402670in}}%
\pgfpathlineto{\pgfqpoint{3.947950in}{4.403894in}}%
\pgfpathlineto{\pgfqpoint{3.949490in}{4.406343in}}%
\pgfpathlineto{\pgfqpoint{3.955014in}{4.407568in}}%
\pgfpathlineto{\pgfqpoint{3.955014in}{4.408792in}}%
\pgfpathlineto{\pgfqpoint{3.959529in}{4.410017in}}%
\pgfpathlineto{\pgfqpoint{3.959529in}{4.411241in}}%
\pgfpathlineto{\pgfqpoint{3.966913in}{4.412466in}}%
\pgfpathlineto{\pgfqpoint{3.966913in}{4.413690in}}%
\pgfpathlineto{\pgfqpoint{3.971375in}{4.414915in}}%
\pgfpathlineto{\pgfqpoint{3.972225in}{4.418588in}}%
\pgfpathlineto{\pgfqpoint{3.975996in}{4.419813in}}%
\pgfpathlineto{\pgfqpoint{3.975996in}{4.421037in}}%
\pgfpathlineto{\pgfqpoint{3.979608in}{4.422262in}}%
\pgfpathlineto{\pgfqpoint{3.980140in}{4.424711in}}%
\pgfpathlineto{\pgfqpoint{3.983911in}{4.425935in}}%
\pgfpathlineto{\pgfqpoint{3.985186in}{4.428384in}}%
\pgfpathlineto{\pgfqpoint{3.990126in}{4.429609in}}%
\pgfpathlineto{\pgfqpoint{3.990126in}{4.430833in}}%
\pgfpathlineto{\pgfqpoint{3.995438in}{4.432058in}}%
\pgfpathlineto{\pgfqpoint{3.996500in}{4.434507in}}%
\pgfpathlineto{\pgfqpoint{4.001281in}{4.434507in}}%
\pgfpathlineto{\pgfqpoint{4.001281in}{4.436956in}}%
\pgfpathlineto{\pgfqpoint{4.013073in}{4.438180in}}%
\pgfpathlineto{\pgfqpoint{4.013073in}{4.439404in}}%
\pgfpathlineto{\pgfqpoint{4.021466in}{4.440629in}}%
\pgfpathlineto{\pgfqpoint{4.022741in}{4.443078in}}%
\pgfpathlineto{\pgfqpoint{4.024972in}{4.444302in}}%
\pgfpathlineto{\pgfqpoint{4.026353in}{4.447976in}}%
\pgfpathlineto{\pgfqpoint{4.036286in}{4.449200in}}%
\pgfpathlineto{\pgfqpoint{4.036658in}{4.451649in}}%
\pgfpathlineto{\pgfqpoint{4.041120in}{4.452874in}}%
\pgfpathlineto{\pgfqpoint{4.041386in}{4.455323in}}%
\pgfpathlineto{\pgfqpoint{4.058224in}{4.456547in}}%
\pgfpathlineto{\pgfqpoint{4.058224in}{4.457772in}}%
\pgfpathlineto{\pgfqpoint{4.061040in}{4.458996in}}%
\pgfpathlineto{\pgfqpoint{4.061199in}{4.461445in}}%
\pgfpathlineto{\pgfqpoint{4.067095in}{4.462670in}}%
\pgfpathlineto{\pgfqpoint{4.067095in}{4.463894in}}%
\pgfpathlineto{\pgfqpoint{4.078888in}{4.465119in}}%
\pgfpathlineto{\pgfqpoint{4.078888in}{4.466343in}}%
\pgfpathlineto{\pgfqpoint{4.088980in}{4.467568in}}%
\pgfpathlineto{\pgfqpoint{4.088980in}{4.468792in}}%
\pgfpathlineto{\pgfqpoint{4.095833in}{4.470017in}}%
\pgfpathlineto{\pgfqpoint{4.096682in}{4.472466in}}%
\pgfpathlineto{\pgfqpoint{4.106456in}{4.473690in}}%
\pgfpathlineto{\pgfqpoint{4.107997in}{4.476139in}}%
\pgfpathlineto{\pgfqpoint{4.110440in}{4.477364in}}%
\pgfpathlineto{\pgfqpoint{4.110440in}{4.478588in}}%
\pgfpathlineto{\pgfqpoint{4.116496in}{4.479812in}}%
\pgfpathlineto{\pgfqpoint{4.116496in}{4.481037in}}%
\pgfpathlineto{\pgfqpoint{4.130679in}{4.482261in}}%
\pgfpathlineto{\pgfqpoint{4.130679in}{4.483486in}}%
\pgfpathlineto{\pgfqpoint{4.141143in}{4.484710in}}%
\pgfpathlineto{\pgfqpoint{4.141143in}{4.485935in}}%
\pgfpathlineto{\pgfqpoint{4.168446in}{4.487159in}}%
\pgfpathlineto{\pgfqpoint{4.168446in}{4.488384in}}%
\pgfpathlineto{\pgfqpoint{4.178751in}{4.489608in}}%
\pgfpathlineto{\pgfqpoint{4.179973in}{4.493282in}}%
\pgfpathlineto{\pgfqpoint{4.193040in}{4.494506in}}%
\pgfpathlineto{\pgfqpoint{4.193040in}{4.495731in}}%
\pgfpathlineto{\pgfqpoint{4.204354in}{4.496955in}}%
\pgfpathlineto{\pgfqpoint{4.204354in}{4.498180in}}%
\pgfpathlineto{\pgfqpoint{4.216837in}{4.499404in}}%
\pgfpathlineto{\pgfqpoint{4.216837in}{4.500629in}}%
\pgfpathlineto{\pgfqpoint{4.217740in}{4.500629in}}%
\pgfpathlineto{\pgfqpoint{4.225124in}{4.501853in}}%
\pgfpathlineto{\pgfqpoint{4.225124in}{4.503078in}}%
\pgfpathlineto{\pgfqpoint{4.244672in}{4.504302in}}%
\pgfpathlineto{\pgfqpoint{4.245309in}{4.506751in}}%
\pgfpathlineto{\pgfqpoint{4.250780in}{4.507976in}}%
\pgfpathlineto{\pgfqpoint{4.250780in}{4.509200in}}%
\pgfpathlineto{\pgfqpoint{4.275799in}{4.510425in}}%
\pgfpathlineto{\pgfqpoint{4.275799in}{4.511649in}}%
\pgfpathlineto{\pgfqpoint{4.282333in}{4.511649in}}%
\pgfpathlineto{\pgfqpoint{4.282333in}{4.514098in}}%
\pgfpathlineto{\pgfqpoint{4.292213in}{4.515323in}}%
\pgfpathlineto{\pgfqpoint{4.292744in}{4.517772in}}%
\pgfpathlineto{\pgfqpoint{4.300181in}{4.518996in}}%
\pgfpathlineto{\pgfqpoint{4.300181in}{4.520220in}}%
\pgfpathlineto{\pgfqpoint{4.318879in}{4.521445in}}%
\pgfpathlineto{\pgfqpoint{4.318879in}{4.522669in}}%
\pgfpathlineto{\pgfqpoint{4.324934in}{4.523894in}}%
\pgfpathlineto{\pgfqpoint{4.325625in}{4.527567in}}%
\pgfpathlineto{\pgfqpoint{4.340126in}{4.528792in}}%
\pgfpathlineto{\pgfqpoint{4.340126in}{4.530016in}}%
\pgfpathlineto{\pgfqpoint{4.351122in}{4.531241in}}%
\pgfpathlineto{\pgfqpoint{4.351122in}{4.532465in}}%
\pgfpathlineto{\pgfqpoint{4.369714in}{4.533690in}}%
\pgfpathlineto{\pgfqpoint{4.369714in}{4.534914in}}%
\pgfpathlineto{\pgfqpoint{4.383153in}{4.536139in}}%
\pgfpathlineto{\pgfqpoint{4.383153in}{4.537363in}}%
\pgfpathlineto{\pgfqpoint{4.389474in}{4.537363in}}%
\pgfpathlineto{\pgfqpoint{4.389474in}{4.539812in}}%
\pgfpathlineto{\pgfqpoint{4.390749in}{4.539812in}}%
\pgfpathlineto{\pgfqpoint{4.406472in}{4.541037in}}%
\pgfpathlineto{\pgfqpoint{4.408012in}{4.543486in}}%
\pgfpathlineto{\pgfqpoint{4.421292in}{4.544710in}}%
\pgfpathlineto{\pgfqpoint{4.421292in}{4.545935in}}%
\pgfpathlineto{\pgfqpoint{4.436484in}{4.547159in}}%
\pgfpathlineto{\pgfqpoint{4.436484in}{4.548384in}}%
\pgfpathlineto{\pgfqpoint{4.481635in}{4.549608in}}%
\pgfpathlineto{\pgfqpoint{4.481635in}{4.550833in}}%
\pgfpathlineto{\pgfqpoint{4.502936in}{4.552057in}}%
\pgfpathlineto{\pgfqpoint{4.502936in}{4.553282in}}%
\pgfpathlineto{\pgfqpoint{4.508407in}{4.554506in}}%
\pgfpathlineto{\pgfqpoint{4.508407in}{4.555731in}}%
\pgfpathlineto{\pgfqpoint{4.508779in}{4.555731in}}%
\pgfpathlineto{\pgfqpoint{4.521740in}{4.556955in}}%
\pgfpathlineto{\pgfqpoint{4.521740in}{4.558180in}}%
\pgfpathlineto{\pgfqpoint{4.581977in}{4.559404in}}%
\pgfpathlineto{\pgfqpoint{4.581977in}{4.560629in}}%
\pgfpathlineto{\pgfqpoint{4.618947in}{4.561853in}}%
\pgfpathlineto{\pgfqpoint{4.618947in}{4.563077in}}%
\pgfpathlineto{\pgfqpoint{4.698201in}{4.564302in}}%
\pgfpathlineto{\pgfqpoint{4.698201in}{4.565526in}}%
\pgfpathlineto{\pgfqpoint{4.713765in}{4.566751in}}%
\pgfpathlineto{\pgfqpoint{4.713765in}{4.567975in}}%
\pgfpathlineto{\pgfqpoint{4.714561in}{4.567975in}}%
\pgfpathlineto{\pgfqpoint{4.746486in}{4.569200in}}%
\pgfpathlineto{\pgfqpoint{4.746486in}{4.570424in}}%
\pgfpathlineto{\pgfqpoint{4.836150in}{4.571649in}}%
\pgfpathlineto{\pgfqpoint{4.836150in}{4.571649in}}%
\pgfusepath{stroke}%
\end{pgfscope}%
\begin{pgfscope}%
\pgfpathrectangle{\pgfqpoint{3.225541in}{3.102083in}}{\pgfqpoint{1.687305in}{1.539545in}}%
\pgfusepath{clip}%
\pgfsetrectcap%
\pgfsetroundjoin%
\pgfsetlinewidth{1.505625pt}%
\definecolor{currentstroke}{rgb}{0.501961,0.501961,0.501961}%
\pgfsetstrokecolor{currentstroke}%
\pgfsetdash{}{0pt}%
\pgfpathmoveto{\pgfqpoint{3.302236in}{3.172062in}}%
\pgfpathlineto{\pgfqpoint{4.836150in}{4.571649in}}%
\pgfusepath{stroke}%
\end{pgfscope}%
\begin{pgfscope}%
\pgfsetrectcap%
\pgfsetmiterjoin%
\pgfsetlinewidth{0.803000pt}%
\definecolor{currentstroke}{rgb}{0.000000,0.000000,0.000000}%
\pgfsetstrokecolor{currentstroke}%
\pgfsetdash{}{0pt}%
\pgfpathmoveto{\pgfqpoint{3.225541in}{3.102083in}}%
\pgfpathlineto{\pgfqpoint{3.225541in}{4.641628in}}%
\pgfusepath{stroke}%
\end{pgfscope}%
\begin{pgfscope}%
\pgfsetrectcap%
\pgfsetmiterjoin%
\pgfsetlinewidth{0.803000pt}%
\definecolor{currentstroke}{rgb}{0.000000,0.000000,0.000000}%
\pgfsetstrokecolor{currentstroke}%
\pgfsetdash{}{0pt}%
\pgfpathmoveto{\pgfqpoint{4.912846in}{3.102083in}}%
\pgfpathlineto{\pgfqpoint{4.912846in}{4.641628in}}%
\pgfusepath{stroke}%
\end{pgfscope}%
\begin{pgfscope}%
\pgfsetrectcap%
\pgfsetmiterjoin%
\pgfsetlinewidth{0.803000pt}%
\definecolor{currentstroke}{rgb}{0.000000,0.000000,0.000000}%
\pgfsetstrokecolor{currentstroke}%
\pgfsetdash{}{0pt}%
\pgfpathmoveto{\pgfqpoint{3.225541in}{3.102083in}}%
\pgfpathlineto{\pgfqpoint{4.912846in}{3.102083in}}%
\pgfusepath{stroke}%
\end{pgfscope}%
\begin{pgfscope}%
\pgfsetrectcap%
\pgfsetmiterjoin%
\pgfsetlinewidth{0.803000pt}%
\definecolor{currentstroke}{rgb}{0.000000,0.000000,0.000000}%
\pgfsetstrokecolor{currentstroke}%
\pgfsetdash{}{0pt}%
\pgfpathmoveto{\pgfqpoint{3.225541in}{4.641628in}}%
\pgfpathlineto{\pgfqpoint{4.912846in}{4.641628in}}%
\pgfusepath{stroke}%
\end{pgfscope}%
\begin{pgfscope}%
\definecolor{textcolor}{rgb}{0.000000,0.000000,0.000000}%
\pgfsetstrokecolor{textcolor}%
\pgfsetfillcolor{textcolor}%
\pgftext[x=4.069193in,y=4.724962in,,base]{\color{textcolor}\rmfamily\fontsize{20.000000}{24.000000}\selectfont Pleural-Thickening}%
\end{pgfscope}%
\begin{pgfscope}%
\pgfsetbuttcap%
\pgfsetmiterjoin%
\definecolor{currentfill}{rgb}{1.000000,1.000000,1.000000}%
\pgfsetfillcolor{currentfill}%
\pgfsetfillopacity{0.800000}%
\pgfsetlinewidth{1.003750pt}%
\definecolor{currentstroke}{rgb}{0.800000,0.800000,0.800000}%
\pgfsetstrokecolor{currentstroke}%
\pgfsetstrokeopacity{0.800000}%
\pgfsetdash{}{0pt}%
\pgfpathmoveto{\pgfqpoint{3.703740in}{3.171527in}}%
\pgfpathlineto{\pgfqpoint{4.815624in}{3.171527in}}%
\pgfpathquadraticcurveto{\pgfqpoint{4.843402in}{3.171527in}}{\pgfqpoint{4.843402in}{3.199305in}}%
\pgfpathlineto{\pgfqpoint{4.843402in}{3.379089in}}%
\pgfpathquadraticcurveto{\pgfqpoint{4.843402in}{3.406867in}}{\pgfqpoint{4.815624in}{3.406867in}}%
\pgfpathlineto{\pgfqpoint{3.703740in}{3.406867in}}%
\pgfpathquadraticcurveto{\pgfqpoint{3.675962in}{3.406867in}}{\pgfqpoint{3.675962in}{3.379089in}}%
\pgfpathlineto{\pgfqpoint{3.675962in}{3.199305in}}%
\pgfpathquadraticcurveto{\pgfqpoint{3.675962in}{3.171527in}}{\pgfqpoint{3.703740in}{3.171527in}}%
\pgfpathclose%
\pgfusepath{stroke,fill}%
\end{pgfscope}%
\begin{pgfscope}%
\pgfsetrectcap%
\pgfsetroundjoin%
\pgfsetlinewidth{1.505625pt}%
\definecolor{currentstroke}{rgb}{0.000000,0.501961,0.000000}%
\pgfsetstrokecolor{currentstroke}%
\pgfsetdash{}{0pt}%
\pgfpathmoveto{\pgfqpoint{3.731518in}{3.302700in}}%
\pgfpathlineto{\pgfqpoint{4.009295in}{3.302700in}}%
\pgfusepath{stroke}%
\end{pgfscope}%
\begin{pgfscope}%
\definecolor{textcolor}{rgb}{0.000000,0.000000,0.000000}%
\pgfsetstrokecolor{textcolor}%
\pgfsetfillcolor{textcolor}%
\pgftext[x=4.120407in,y=3.254089in,left,base]{\color{textcolor}\rmfamily\fontsize{10.000000}{12.000000}\selectfont AUC 0.820}%
\end{pgfscope}%
\begin{pgfscope}%
\pgfsetbuttcap%
\pgfsetmiterjoin%
\definecolor{currentfill}{rgb}{1.000000,1.000000,1.000000}%
\pgfsetfillcolor{currentfill}%
\pgfsetlinewidth{0.000000pt}%
\definecolor{currentstroke}{rgb}{0.000000,0.000000,0.000000}%
\pgfsetstrokecolor{currentstroke}%
\pgfsetstrokeopacity{0.000000}%
\pgfsetdash{}{0pt}%
\pgfpathmoveto{\pgfqpoint{5.688041in}{3.102083in}}%
\pgfpathlineto{\pgfqpoint{7.375346in}{3.102083in}}%
\pgfpathlineto{\pgfqpoint{7.375346in}{4.641628in}}%
\pgfpathlineto{\pgfqpoint{5.688041in}{4.641628in}}%
\pgfpathclose%
\pgfusepath{fill}%
\end{pgfscope}%
\begin{pgfscope}%
\pgfsetbuttcap%
\pgfsetroundjoin%
\definecolor{currentfill}{rgb}{0.000000,0.000000,0.000000}%
\pgfsetfillcolor{currentfill}%
\pgfsetlinewidth{0.803000pt}%
\definecolor{currentstroke}{rgb}{0.000000,0.000000,0.000000}%
\pgfsetstrokecolor{currentstroke}%
\pgfsetdash{}{0pt}%
\pgfsys@defobject{currentmarker}{\pgfqpoint{0.000000in}{-0.048611in}}{\pgfqpoint{0.000000in}{0.000000in}}{%
\pgfpathmoveto{\pgfqpoint{0.000000in}{0.000000in}}%
\pgfpathlineto{\pgfqpoint{0.000000in}{-0.048611in}}%
\pgfusepath{stroke,fill}%
}%
\begin{pgfscope}%
\pgfsys@transformshift{5.764736in}{3.102083in}%
\pgfsys@useobject{currentmarker}{}%
\end{pgfscope}%
\end{pgfscope}%
\begin{pgfscope}%
\definecolor{textcolor}{rgb}{0.000000,0.000000,0.000000}%
\pgfsetstrokecolor{textcolor}%
\pgfsetfillcolor{textcolor}%
\pgftext[x=5.764736in,y=3.004861in,,top]{\color{textcolor}\rmfamily\fontsize{10.000000}{12.000000}\selectfont \(\displaystyle {0.0}\)}%
\end{pgfscope}%
\begin{pgfscope}%
\pgfsetbuttcap%
\pgfsetroundjoin%
\definecolor{currentfill}{rgb}{0.000000,0.000000,0.000000}%
\pgfsetfillcolor{currentfill}%
\pgfsetlinewidth{0.803000pt}%
\definecolor{currentstroke}{rgb}{0.000000,0.000000,0.000000}%
\pgfsetstrokecolor{currentstroke}%
\pgfsetdash{}{0pt}%
\pgfsys@defobject{currentmarker}{\pgfqpoint{0.000000in}{-0.048611in}}{\pgfqpoint{0.000000in}{0.000000in}}{%
\pgfpathmoveto{\pgfqpoint{0.000000in}{0.000000in}}%
\pgfpathlineto{\pgfqpoint{0.000000in}{-0.048611in}}%
\pgfusepath{stroke,fill}%
}%
\begin{pgfscope}%
\pgfsys@transformshift{6.531693in}{3.102083in}%
\pgfsys@useobject{currentmarker}{}%
\end{pgfscope}%
\end{pgfscope}%
\begin{pgfscope}%
\definecolor{textcolor}{rgb}{0.000000,0.000000,0.000000}%
\pgfsetstrokecolor{textcolor}%
\pgfsetfillcolor{textcolor}%
\pgftext[x=6.531693in,y=3.004861in,,top]{\color{textcolor}\rmfamily\fontsize{10.000000}{12.000000}\selectfont \(\displaystyle {0.5}\)}%
\end{pgfscope}%
\begin{pgfscope}%
\pgfsetbuttcap%
\pgfsetroundjoin%
\definecolor{currentfill}{rgb}{0.000000,0.000000,0.000000}%
\pgfsetfillcolor{currentfill}%
\pgfsetlinewidth{0.803000pt}%
\definecolor{currentstroke}{rgb}{0.000000,0.000000,0.000000}%
\pgfsetstrokecolor{currentstroke}%
\pgfsetdash{}{0pt}%
\pgfsys@defobject{currentmarker}{\pgfqpoint{0.000000in}{-0.048611in}}{\pgfqpoint{0.000000in}{0.000000in}}{%
\pgfpathmoveto{\pgfqpoint{0.000000in}{0.000000in}}%
\pgfpathlineto{\pgfqpoint{0.000000in}{-0.048611in}}%
\pgfusepath{stroke,fill}%
}%
\begin{pgfscope}%
\pgfsys@transformshift{7.298650in}{3.102083in}%
\pgfsys@useobject{currentmarker}{}%
\end{pgfscope}%
\end{pgfscope}%
\begin{pgfscope}%
\definecolor{textcolor}{rgb}{0.000000,0.000000,0.000000}%
\pgfsetstrokecolor{textcolor}%
\pgfsetfillcolor{textcolor}%
\pgftext[x=7.298650in,y=3.004861in,,top]{\color{textcolor}\rmfamily\fontsize{10.000000}{12.000000}\selectfont \(\displaystyle {1.0}\)}%
\end{pgfscope}%
\begin{pgfscope}%
\definecolor{textcolor}{rgb}{0.000000,0.000000,0.000000}%
\pgfsetstrokecolor{textcolor}%
\pgfsetfillcolor{textcolor}%
\pgftext[x=6.531693in,y=2.825849in,,top]{\color{textcolor}\rmfamily\fontsize{16.000000}{19.200000}\selectfont FPR}%
\end{pgfscope}%
\begin{pgfscope}%
\pgfsetbuttcap%
\pgfsetroundjoin%
\definecolor{currentfill}{rgb}{0.000000,0.000000,0.000000}%
\pgfsetfillcolor{currentfill}%
\pgfsetlinewidth{0.803000pt}%
\definecolor{currentstroke}{rgb}{0.000000,0.000000,0.000000}%
\pgfsetstrokecolor{currentstroke}%
\pgfsetdash{}{0pt}%
\pgfsys@defobject{currentmarker}{\pgfqpoint{-0.048611in}{0.000000in}}{\pgfqpoint{-0.000000in}{0.000000in}}{%
\pgfpathmoveto{\pgfqpoint{-0.000000in}{0.000000in}}%
\pgfpathlineto{\pgfqpoint{-0.048611in}{0.000000in}}%
\pgfusepath{stroke,fill}%
}%
\begin{pgfscope}%
\pgfsys@transformshift{5.688041in}{3.172062in}%
\pgfsys@useobject{currentmarker}{}%
\end{pgfscope}%
\end{pgfscope}%
\begin{pgfscope}%
\definecolor{textcolor}{rgb}{0.000000,0.000000,0.000000}%
\pgfsetstrokecolor{textcolor}%
\pgfsetfillcolor{textcolor}%
\pgftext[x=5.343904in, y=3.123837in, left, base]{\color{textcolor}\rmfamily\fontsize{10.000000}{12.000000}\selectfont \(\displaystyle {0.00}\)}%
\end{pgfscope}%
\begin{pgfscope}%
\pgfsetbuttcap%
\pgfsetroundjoin%
\definecolor{currentfill}{rgb}{0.000000,0.000000,0.000000}%
\pgfsetfillcolor{currentfill}%
\pgfsetlinewidth{0.803000pt}%
\definecolor{currentstroke}{rgb}{0.000000,0.000000,0.000000}%
\pgfsetstrokecolor{currentstroke}%
\pgfsetdash{}{0pt}%
\pgfsys@defobject{currentmarker}{\pgfqpoint{-0.048611in}{0.000000in}}{\pgfqpoint{-0.000000in}{0.000000in}}{%
\pgfpathmoveto{\pgfqpoint{-0.000000in}{0.000000in}}%
\pgfpathlineto{\pgfqpoint{-0.048611in}{0.000000in}}%
\pgfusepath{stroke,fill}%
}%
\begin{pgfscope}%
\pgfsys@transformshift{5.688041in}{3.521959in}%
\pgfsys@useobject{currentmarker}{}%
\end{pgfscope}%
\end{pgfscope}%
\begin{pgfscope}%
\definecolor{textcolor}{rgb}{0.000000,0.000000,0.000000}%
\pgfsetstrokecolor{textcolor}%
\pgfsetfillcolor{textcolor}%
\pgftext[x=5.343904in, y=3.473734in, left, base]{\color{textcolor}\rmfamily\fontsize{10.000000}{12.000000}\selectfont \(\displaystyle {0.25}\)}%
\end{pgfscope}%
\begin{pgfscope}%
\pgfsetbuttcap%
\pgfsetroundjoin%
\definecolor{currentfill}{rgb}{0.000000,0.000000,0.000000}%
\pgfsetfillcolor{currentfill}%
\pgfsetlinewidth{0.803000pt}%
\definecolor{currentstroke}{rgb}{0.000000,0.000000,0.000000}%
\pgfsetstrokecolor{currentstroke}%
\pgfsetdash{}{0pt}%
\pgfsys@defobject{currentmarker}{\pgfqpoint{-0.048611in}{0.000000in}}{\pgfqpoint{-0.000000in}{0.000000in}}{%
\pgfpathmoveto{\pgfqpoint{-0.000000in}{0.000000in}}%
\pgfpathlineto{\pgfqpoint{-0.048611in}{0.000000in}}%
\pgfusepath{stroke,fill}%
}%
\begin{pgfscope}%
\pgfsys@transformshift{5.688041in}{3.871856in}%
\pgfsys@useobject{currentmarker}{}%
\end{pgfscope}%
\end{pgfscope}%
\begin{pgfscope}%
\definecolor{textcolor}{rgb}{0.000000,0.000000,0.000000}%
\pgfsetstrokecolor{textcolor}%
\pgfsetfillcolor{textcolor}%
\pgftext[x=5.343904in, y=3.823630in, left, base]{\color{textcolor}\rmfamily\fontsize{10.000000}{12.000000}\selectfont \(\displaystyle {0.50}\)}%
\end{pgfscope}%
\begin{pgfscope}%
\pgfsetbuttcap%
\pgfsetroundjoin%
\definecolor{currentfill}{rgb}{0.000000,0.000000,0.000000}%
\pgfsetfillcolor{currentfill}%
\pgfsetlinewidth{0.803000pt}%
\definecolor{currentstroke}{rgb}{0.000000,0.000000,0.000000}%
\pgfsetstrokecolor{currentstroke}%
\pgfsetdash{}{0pt}%
\pgfsys@defobject{currentmarker}{\pgfqpoint{-0.048611in}{0.000000in}}{\pgfqpoint{-0.000000in}{0.000000in}}{%
\pgfpathmoveto{\pgfqpoint{-0.000000in}{0.000000in}}%
\pgfpathlineto{\pgfqpoint{-0.048611in}{0.000000in}}%
\pgfusepath{stroke,fill}%
}%
\begin{pgfscope}%
\pgfsys@transformshift{5.688041in}{4.221752in}%
\pgfsys@useobject{currentmarker}{}%
\end{pgfscope}%
\end{pgfscope}%
\begin{pgfscope}%
\definecolor{textcolor}{rgb}{0.000000,0.000000,0.000000}%
\pgfsetstrokecolor{textcolor}%
\pgfsetfillcolor{textcolor}%
\pgftext[x=5.343904in, y=4.173527in, left, base]{\color{textcolor}\rmfamily\fontsize{10.000000}{12.000000}\selectfont \(\displaystyle {0.75}\)}%
\end{pgfscope}%
\begin{pgfscope}%
\pgfsetbuttcap%
\pgfsetroundjoin%
\definecolor{currentfill}{rgb}{0.000000,0.000000,0.000000}%
\pgfsetfillcolor{currentfill}%
\pgfsetlinewidth{0.803000pt}%
\definecolor{currentstroke}{rgb}{0.000000,0.000000,0.000000}%
\pgfsetstrokecolor{currentstroke}%
\pgfsetdash{}{0pt}%
\pgfsys@defobject{currentmarker}{\pgfqpoint{-0.048611in}{0.000000in}}{\pgfqpoint{-0.000000in}{0.000000in}}{%
\pgfpathmoveto{\pgfqpoint{-0.000000in}{0.000000in}}%
\pgfpathlineto{\pgfqpoint{-0.048611in}{0.000000in}}%
\pgfusepath{stroke,fill}%
}%
\begin{pgfscope}%
\pgfsys@transformshift{5.688041in}{4.571649in}%
\pgfsys@useobject{currentmarker}{}%
\end{pgfscope}%
\end{pgfscope}%
\begin{pgfscope}%
\definecolor{textcolor}{rgb}{0.000000,0.000000,0.000000}%
\pgfsetstrokecolor{textcolor}%
\pgfsetfillcolor{textcolor}%
\pgftext[x=5.343904in, y=4.523424in, left, base]{\color{textcolor}\rmfamily\fontsize{10.000000}{12.000000}\selectfont \(\displaystyle {1.00}\)}%
\end{pgfscope}%
\begin{pgfscope}%
\definecolor{textcolor}{rgb}{0.000000,0.000000,0.000000}%
\pgfsetstrokecolor{textcolor}%
\pgfsetfillcolor{textcolor}%
\pgftext[x=5.288349in,y=3.871856in,,bottom,rotate=90.000000]{\color{textcolor}\rmfamily\fontsize{16.000000}{19.200000}\selectfont TPR}%
\end{pgfscope}%
\begin{pgfscope}%
\pgfpathrectangle{\pgfqpoint{5.688041in}{3.102083in}}{\pgfqpoint{1.687305in}{1.539545in}}%
\pgfusepath{clip}%
\pgfsetrectcap%
\pgfsetroundjoin%
\pgfsetlinewidth{1.505625pt}%
\definecolor{currentstroke}{rgb}{0.000000,0.501961,0.000000}%
\pgfsetstrokecolor{currentstroke}%
\pgfsetdash{}{0pt}%
\pgfpathmoveto{\pgfqpoint{5.764736in}{3.172062in}}%
\pgfpathlineto{\pgfqpoint{5.766277in}{3.370611in}}%
\pgfpathlineto{\pgfqpoint{5.766330in}{3.370611in}}%
\pgfpathlineto{\pgfqpoint{5.767871in}{3.435170in}}%
\pgfpathlineto{\pgfqpoint{5.768030in}{3.436388in}}%
\pgfpathlineto{\pgfqpoint{5.772334in}{3.566724in}}%
\pgfpathlineto{\pgfqpoint{5.772759in}{3.566724in}}%
\pgfpathlineto{\pgfqpoint{5.772759in}{3.567942in}}%
\pgfpathlineto{\pgfqpoint{5.777541in}{3.645900in}}%
\pgfpathlineto{\pgfqpoint{5.777753in}{3.645900in}}%
\pgfpathlineto{\pgfqpoint{5.779081in}{3.658081in}}%
\pgfpathlineto{\pgfqpoint{5.779400in}{3.658081in}}%
\pgfpathlineto{\pgfqpoint{5.780941in}{3.672698in}}%
\pgfpathlineto{\pgfqpoint{5.781260in}{3.673916in}}%
\pgfpathlineto{\pgfqpoint{5.782801in}{3.692187in}}%
\pgfpathlineto{\pgfqpoint{5.783013in}{3.693405in}}%
\pgfpathlineto{\pgfqpoint{5.784448in}{3.705586in}}%
\pgfpathlineto{\pgfqpoint{5.784820in}{3.706804in}}%
\pgfpathlineto{\pgfqpoint{5.786041in}{3.720203in}}%
\pgfpathlineto{\pgfqpoint{5.787157in}{3.721421in}}%
\pgfpathlineto{\pgfqpoint{5.788592in}{3.736038in}}%
\pgfpathlineto{\pgfqpoint{5.789123in}{3.736038in}}%
\pgfpathlineto{\pgfqpoint{5.790611in}{3.751874in}}%
\pgfpathlineto{\pgfqpoint{5.791089in}{3.753092in}}%
\pgfpathlineto{\pgfqpoint{5.792364in}{3.765273in}}%
\pgfpathlineto{\pgfqpoint{5.792576in}{3.765273in}}%
\pgfpathlineto{\pgfqpoint{5.792736in}{3.766491in}}%
\pgfpathlineto{\pgfqpoint{5.794224in}{3.775017in}}%
\pgfpathlineto{\pgfqpoint{5.794595in}{3.775017in}}%
\pgfpathlineto{\pgfqpoint{5.796083in}{3.785980in}}%
\pgfpathlineto{\pgfqpoint{5.796242in}{3.785980in}}%
\pgfpathlineto{\pgfqpoint{5.797677in}{3.793289in}}%
\pgfpathlineto{\pgfqpoint{5.798952in}{3.794507in}}%
\pgfpathlineto{\pgfqpoint{5.800174in}{3.796943in}}%
\pgfpathlineto{\pgfqpoint{5.800758in}{3.796943in}}%
\pgfpathlineto{\pgfqpoint{5.801980in}{3.801815in}}%
\pgfpathlineto{\pgfqpoint{5.802459in}{3.803033in}}%
\pgfpathlineto{\pgfqpoint{5.803946in}{3.806688in}}%
\pgfpathlineto{\pgfqpoint{5.804637in}{3.807906in}}%
\pgfpathlineto{\pgfqpoint{5.806071in}{3.823741in}}%
\pgfpathlineto{\pgfqpoint{5.809950in}{3.824959in}}%
\pgfpathlineto{\pgfqpoint{5.810322in}{3.828613in}}%
\pgfpathlineto{\pgfqpoint{5.812394in}{3.829831in}}%
\pgfpathlineto{\pgfqpoint{5.814094in}{3.837140in}}%
\pgfpathlineto{\pgfqpoint{5.815422in}{3.838358in}}%
\pgfpathlineto{\pgfqpoint{5.816804in}{3.843230in}}%
\pgfpathlineto{\pgfqpoint{5.817388in}{3.844449in}}%
\pgfpathlineto{\pgfqpoint{5.818132in}{3.849321in}}%
\pgfpathlineto{\pgfqpoint{5.818823in}{3.849321in}}%
\pgfpathlineto{\pgfqpoint{5.819726in}{3.850539in}}%
\pgfpathlineto{\pgfqpoint{5.820842in}{3.855411in}}%
\pgfpathlineto{\pgfqpoint{5.821692in}{3.856629in}}%
\pgfpathlineto{\pgfqpoint{5.822648in}{3.860284in}}%
\pgfpathlineto{\pgfqpoint{5.823657in}{3.861502in}}%
\pgfpathlineto{\pgfqpoint{5.825145in}{3.868810in}}%
\pgfpathlineto{\pgfqpoint{5.825836in}{3.870028in}}%
\pgfpathlineto{\pgfqpoint{5.827164in}{3.874901in}}%
\pgfpathlineto{\pgfqpoint{5.828280in}{3.876119in}}%
\pgfpathlineto{\pgfqpoint{5.829449in}{3.883427in}}%
\pgfpathlineto{\pgfqpoint{5.830989in}{3.884646in}}%
\pgfpathlineto{\pgfqpoint{5.831946in}{3.888300in}}%
\pgfpathlineto{\pgfqpoint{5.833699in}{3.889518in}}%
\pgfpathlineto{\pgfqpoint{5.834921in}{3.894390in}}%
\pgfpathlineto{\pgfqpoint{5.836568in}{3.895608in}}%
\pgfpathlineto{\pgfqpoint{5.837949in}{3.901699in}}%
\pgfpathlineto{\pgfqpoint{5.838056in}{3.901699in}}%
\pgfpathlineto{\pgfqpoint{5.838587in}{3.902917in}}%
\pgfpathlineto{\pgfqpoint{5.839703in}{3.905353in}}%
\pgfpathlineto{\pgfqpoint{5.841084in}{3.906571in}}%
\pgfpathlineto{\pgfqpoint{5.842572in}{3.912662in}}%
\pgfpathlineto{\pgfqpoint{5.844166in}{3.913880in}}%
\pgfpathlineto{\pgfqpoint{5.844484in}{3.916316in}}%
\pgfpathlineto{\pgfqpoint{5.846556in}{3.917534in}}%
\pgfpathlineto{\pgfqpoint{5.847778in}{3.921188in}}%
\pgfpathlineto{\pgfqpoint{5.848947in}{3.922406in}}%
\pgfpathlineto{\pgfqpoint{5.849425in}{3.927279in}}%
\pgfpathlineto{\pgfqpoint{5.850966in}{3.928497in}}%
\pgfpathlineto{\pgfqpoint{5.851604in}{3.933369in}}%
\pgfpathlineto{\pgfqpoint{5.853942in}{3.934587in}}%
\pgfpathlineto{\pgfqpoint{5.855376in}{3.938242in}}%
\pgfpathlineto{\pgfqpoint{5.857236in}{3.939460in}}%
\pgfpathlineto{\pgfqpoint{5.858404in}{3.944332in}}%
\pgfpathlineto{\pgfqpoint{5.860105in}{3.945550in}}%
\pgfpathlineto{\pgfqpoint{5.860742in}{3.949204in}}%
\pgfpathlineto{\pgfqpoint{5.862230in}{3.950422in}}%
\pgfpathlineto{\pgfqpoint{5.863558in}{3.954077in}}%
\pgfpathlineto{\pgfqpoint{5.864886in}{3.955295in}}%
\pgfpathlineto{\pgfqpoint{5.866055in}{3.960167in}}%
\pgfpathlineto{\pgfqpoint{5.867012in}{3.961385in}}%
\pgfpathlineto{\pgfqpoint{5.868499in}{3.967476in}}%
\pgfpathlineto{\pgfqpoint{5.869615in}{3.968694in}}%
\pgfpathlineto{\pgfqpoint{5.871103in}{3.974784in}}%
\pgfpathlineto{\pgfqpoint{5.872059in}{3.976002in}}%
\pgfpathlineto{\pgfqpoint{5.873600in}{3.982093in}}%
\pgfpathlineto{\pgfqpoint{5.874768in}{3.983311in}}%
\pgfpathlineto{\pgfqpoint{5.875300in}{3.988183in}}%
\pgfpathlineto{\pgfqpoint{5.879816in}{3.989401in}}%
\pgfpathlineto{\pgfqpoint{5.880560in}{3.991838in}}%
\pgfpathlineto{\pgfqpoint{5.883588in}{3.993056in}}%
\pgfpathlineto{\pgfqpoint{5.885129in}{4.002800in}}%
\pgfpathlineto{\pgfqpoint{5.886298in}{4.004018in}}%
\pgfpathlineto{\pgfqpoint{5.887307in}{4.010109in}}%
\pgfpathlineto{\pgfqpoint{5.889273in}{4.011327in}}%
\pgfpathlineto{\pgfqpoint{5.890495in}{4.019854in}}%
\pgfpathlineto{\pgfqpoint{5.891186in}{4.021072in}}%
\pgfpathlineto{\pgfqpoint{5.892567in}{4.027162in}}%
\pgfpathlineto{\pgfqpoint{5.893045in}{4.027162in}}%
\pgfpathlineto{\pgfqpoint{5.893789in}{4.034471in}}%
\pgfpathlineto{\pgfqpoint{5.896074in}{4.035689in}}%
\pgfpathlineto{\pgfqpoint{5.896924in}{4.040561in}}%
\pgfpathlineto{\pgfqpoint{5.898571in}{4.041779in}}%
\pgfpathlineto{\pgfqpoint{5.899846in}{4.047870in}}%
\pgfpathlineto{\pgfqpoint{5.900377in}{4.049088in}}%
\pgfpathlineto{\pgfqpoint{5.901440in}{4.056396in}}%
\pgfpathlineto{\pgfqpoint{5.902874in}{4.057614in}}%
\pgfpathlineto{\pgfqpoint{5.902874in}{4.058833in}}%
\pgfpathlineto{\pgfqpoint{5.905690in}{4.060051in}}%
\pgfpathlineto{\pgfqpoint{5.906806in}{4.062487in}}%
\pgfpathlineto{\pgfqpoint{5.907656in}{4.063705in}}%
\pgfpathlineto{\pgfqpoint{5.908665in}{4.069795in}}%
\pgfpathlineto{\pgfqpoint{5.911641in}{4.071013in}}%
\pgfpathlineto{\pgfqpoint{5.912172in}{4.073450in}}%
\pgfpathlineto{\pgfqpoint{5.915519in}{4.074668in}}%
\pgfpathlineto{\pgfqpoint{5.917060in}{4.080758in}}%
\pgfpathlineto{\pgfqpoint{5.918388in}{4.081976in}}%
\pgfpathlineto{\pgfqpoint{5.918601in}{4.084412in}}%
\pgfpathlineto{\pgfqpoint{5.920620in}{4.085631in}}%
\pgfpathlineto{\pgfqpoint{5.920938in}{4.089285in}}%
\pgfpathlineto{\pgfqpoint{5.922532in}{4.090503in}}%
\pgfpathlineto{\pgfqpoint{5.922532in}{4.092939in}}%
\pgfpathlineto{\pgfqpoint{5.924657in}{4.094157in}}%
\pgfpathlineto{\pgfqpoint{5.926145in}{4.096593in}}%
\pgfpathlineto{\pgfqpoint{5.928589in}{4.097811in}}%
\pgfpathlineto{\pgfqpoint{5.929652in}{4.101466in}}%
\pgfpathlineto{\pgfqpoint{5.930449in}{4.101466in}}%
\pgfpathlineto{\pgfqpoint{5.930449in}{4.105120in}}%
\pgfpathlineto{\pgfqpoint{5.937462in}{4.106338in}}%
\pgfpathlineto{\pgfqpoint{5.938578in}{4.112429in}}%
\pgfpathlineto{\pgfqpoint{5.940278in}{4.113647in}}%
\pgfpathlineto{\pgfqpoint{5.941393in}{4.119737in}}%
\pgfpathlineto{\pgfqpoint{5.943837in}{4.120955in}}%
\pgfpathlineto{\pgfqpoint{5.944475in}{4.124609in}}%
\pgfpathlineto{\pgfqpoint{5.947875in}{4.125828in}}%
\pgfpathlineto{\pgfqpoint{5.949044in}{4.131918in}}%
\pgfpathlineto{\pgfqpoint{5.951222in}{4.133136in}}%
\pgfpathlineto{\pgfqpoint{5.951222in}{4.134354in}}%
\pgfpathlineto{\pgfqpoint{5.953613in}{4.134354in}}%
\pgfpathlineto{\pgfqpoint{5.955048in}{4.142881in}}%
\pgfpathlineto{\pgfqpoint{5.955260in}{4.142881in}}%
\pgfpathlineto{\pgfqpoint{5.956323in}{4.147753in}}%
\pgfpathlineto{\pgfqpoint{5.959245in}{4.148971in}}%
\pgfpathlineto{\pgfqpoint{5.959298in}{4.151407in}}%
\pgfpathlineto{\pgfqpoint{5.964771in}{4.152626in}}%
\pgfpathlineto{\pgfqpoint{5.965993in}{4.157498in}}%
\pgfpathlineto{\pgfqpoint{5.969127in}{4.158716in}}%
\pgfpathlineto{\pgfqpoint{5.969127in}{4.159934in}}%
\pgfpathlineto{\pgfqpoint{5.972262in}{4.161152in}}%
\pgfpathlineto{\pgfqpoint{5.972740in}{4.163588in}}%
\pgfpathlineto{\pgfqpoint{5.976193in}{4.164806in}}%
\pgfpathlineto{\pgfqpoint{5.977522in}{4.170897in}}%
\pgfpathlineto{\pgfqpoint{5.978425in}{4.172115in}}%
\pgfpathlineto{\pgfqpoint{5.979541in}{4.176987in}}%
\pgfpathlineto{\pgfqpoint{5.979966in}{4.176987in}}%
\pgfpathlineto{\pgfqpoint{5.981400in}{4.178205in}}%
\pgfpathlineto{\pgfqpoint{5.982250in}{4.180642in}}%
\pgfpathlineto{\pgfqpoint{5.984216in}{4.181860in}}%
\pgfpathlineto{\pgfqpoint{5.985013in}{4.184296in}}%
\pgfpathlineto{\pgfqpoint{5.987563in}{4.185514in}}%
\pgfpathlineto{\pgfqpoint{5.987563in}{4.186732in}}%
\pgfpathlineto{\pgfqpoint{5.991123in}{4.187950in}}%
\pgfpathlineto{\pgfqpoint{5.991123in}{4.189168in}}%
\pgfpathlineto{\pgfqpoint{5.996861in}{4.190386in}}%
\pgfpathlineto{\pgfqpoint{5.996861in}{4.191604in}}%
\pgfpathlineto{\pgfqpoint{6.000793in}{4.192823in}}%
\pgfpathlineto{\pgfqpoint{6.002015in}{4.197695in}}%
\pgfpathlineto{\pgfqpoint{6.003502in}{4.198913in}}%
\pgfpathlineto{\pgfqpoint{6.004937in}{4.202567in}}%
\pgfpathlineto{\pgfqpoint{6.005946in}{4.203785in}}%
\pgfpathlineto{\pgfqpoint{6.007115in}{4.206222in}}%
\pgfpathlineto{\pgfqpoint{6.008922in}{4.207440in}}%
\pgfpathlineto{\pgfqpoint{6.009718in}{4.211094in}}%
\pgfpathlineto{\pgfqpoint{6.012216in}{4.212312in}}%
\pgfpathlineto{\pgfqpoint{6.012853in}{4.214748in}}%
\pgfpathlineto{\pgfqpoint{6.017422in}{4.215966in}}%
\pgfpathlineto{\pgfqpoint{6.018644in}{4.218402in}}%
\pgfpathlineto{\pgfqpoint{6.024807in}{4.219621in}}%
\pgfpathlineto{\pgfqpoint{6.024807in}{4.220839in}}%
\pgfpathlineto{\pgfqpoint{6.038196in}{4.222057in}}%
\pgfpathlineto{\pgfqpoint{6.038462in}{4.224493in}}%
\pgfpathlineto{\pgfqpoint{6.045900in}{4.225711in}}%
\pgfpathlineto{\pgfqpoint{6.047281in}{4.230583in}}%
\pgfpathlineto{\pgfqpoint{6.054454in}{4.231801in}}%
\pgfpathlineto{\pgfqpoint{6.054454in}{4.233020in}}%
\pgfpathlineto{\pgfqpoint{6.059448in}{4.234238in}}%
\pgfpathlineto{\pgfqpoint{6.060670in}{4.239110in}}%
\pgfpathlineto{\pgfqpoint{6.064974in}{4.240328in}}%
\pgfpathlineto{\pgfqpoint{6.064974in}{4.241546in}}%
\pgfpathlineto{\pgfqpoint{6.069171in}{4.242764in}}%
\pgfpathlineto{\pgfqpoint{6.069171in}{4.243982in}}%
\pgfpathlineto{\pgfqpoint{6.071934in}{4.245200in}}%
\pgfpathlineto{\pgfqpoint{6.073315in}{4.247637in}}%
\pgfpathlineto{\pgfqpoint{6.076875in}{4.248855in}}%
\pgfpathlineto{\pgfqpoint{6.078309in}{4.251291in}}%
\pgfpathlineto{\pgfqpoint{6.081656in}{4.252509in}}%
\pgfpathlineto{\pgfqpoint{6.083091in}{4.256163in}}%
\pgfpathlineto{\pgfqpoint{6.087660in}{4.257381in}}%
\pgfpathlineto{\pgfqpoint{6.087660in}{4.258599in}}%
\pgfpathlineto{\pgfqpoint{6.091432in}{4.259818in}}%
\pgfpathlineto{\pgfqpoint{6.091432in}{4.261036in}}%
\pgfpathlineto{\pgfqpoint{6.096745in}{4.262254in}}%
\pgfpathlineto{\pgfqpoint{6.096745in}{4.263472in}}%
\pgfpathlineto{\pgfqpoint{6.098977in}{4.264690in}}%
\pgfpathlineto{\pgfqpoint{6.098977in}{4.265908in}}%
\pgfpathlineto{\pgfqpoint{6.102536in}{4.267126in}}%
\pgfpathlineto{\pgfqpoint{6.104024in}{4.273217in}}%
\pgfpathlineto{\pgfqpoint{6.107531in}{4.274435in}}%
\pgfpathlineto{\pgfqpoint{6.108753in}{4.278089in}}%
\pgfpathlineto{\pgfqpoint{6.109868in}{4.279307in}}%
\pgfpathlineto{\pgfqpoint{6.110506in}{4.281743in}}%
\pgfpathlineto{\pgfqpoint{6.114013in}{4.282961in}}%
\pgfpathlineto{\pgfqpoint{6.114013in}{4.284179in}}%
\pgfpathlineto{\pgfqpoint{6.119060in}{4.285397in}}%
\pgfpathlineto{\pgfqpoint{6.119060in}{4.286616in}}%
\pgfpathlineto{\pgfqpoint{6.122141in}{4.287834in}}%
\pgfpathlineto{\pgfqpoint{6.122141in}{4.289052in}}%
\pgfpathlineto{\pgfqpoint{6.125223in}{4.290270in}}%
\pgfpathlineto{\pgfqpoint{6.125223in}{4.291488in}}%
\pgfpathlineto{\pgfqpoint{6.129208in}{4.292706in}}%
\pgfpathlineto{\pgfqpoint{6.129208in}{4.293924in}}%
\pgfpathlineto{\pgfqpoint{6.140152in}{4.295142in}}%
\pgfpathlineto{\pgfqpoint{6.140152in}{4.296360in}}%
\pgfpathlineto{\pgfqpoint{6.143500in}{4.296360in}}%
\pgfpathlineto{\pgfqpoint{6.144456in}{4.301233in}}%
\pgfpathlineto{\pgfqpoint{6.150725in}{4.302451in}}%
\pgfpathlineto{\pgfqpoint{6.151682in}{4.304887in}}%
\pgfpathlineto{\pgfqpoint{6.154976in}{4.306105in}}%
\pgfpathlineto{\pgfqpoint{6.154976in}{4.307323in}}%
\pgfpathlineto{\pgfqpoint{6.160076in}{4.308541in}}%
\pgfpathlineto{\pgfqpoint{6.160820in}{4.310977in}}%
\pgfpathlineto{\pgfqpoint{6.164433in}{4.312195in}}%
\pgfpathlineto{\pgfqpoint{6.164433in}{4.313414in}}%
\pgfpathlineto{\pgfqpoint{6.169693in}{4.314632in}}%
\pgfpathlineto{\pgfqpoint{6.169693in}{4.315850in}}%
\pgfpathlineto{\pgfqpoint{6.174740in}{4.317068in}}%
\pgfpathlineto{\pgfqpoint{6.174740in}{4.318286in}}%
\pgfpathlineto{\pgfqpoint{6.175909in}{4.318286in}}%
\pgfpathlineto{\pgfqpoint{6.176653in}{4.319504in}}%
\pgfpathlineto{\pgfqpoint{6.178087in}{4.321940in}}%
\pgfpathlineto{\pgfqpoint{6.180903in}{4.323158in}}%
\pgfpathlineto{\pgfqpoint{6.181169in}{4.325595in}}%
\pgfpathlineto{\pgfqpoint{6.190095in}{4.326813in}}%
\pgfpathlineto{\pgfqpoint{6.190095in}{4.328031in}}%
\pgfpathlineto{\pgfqpoint{6.196842in}{4.329249in}}%
\pgfpathlineto{\pgfqpoint{6.196842in}{4.330467in}}%
\pgfpathlineto{\pgfqpoint{6.203590in}{4.331685in}}%
\pgfpathlineto{\pgfqpoint{6.203590in}{4.332903in}}%
\pgfpathlineto{\pgfqpoint{6.207043in}{4.334121in}}%
\pgfpathlineto{\pgfqpoint{6.207043in}{4.335339in}}%
\pgfpathlineto{\pgfqpoint{6.218147in}{4.336557in}}%
\pgfpathlineto{\pgfqpoint{6.218147in}{4.337775in}}%
\pgfpathlineto{\pgfqpoint{6.220113in}{4.338994in}}%
\pgfpathlineto{\pgfqpoint{6.220113in}{4.340212in}}%
\pgfpathlineto{\pgfqpoint{6.225638in}{4.341430in}}%
\pgfpathlineto{\pgfqpoint{6.225638in}{4.342648in}}%
\pgfpathlineto{\pgfqpoint{6.226010in}{4.342648in}}%
\pgfpathlineto{\pgfqpoint{6.230739in}{4.343866in}}%
\pgfpathlineto{\pgfqpoint{6.231323in}{4.346302in}}%
\pgfpathlineto{\pgfqpoint{6.237646in}{4.347520in}}%
\pgfpathlineto{\pgfqpoint{6.237646in}{4.348738in}}%
\pgfpathlineto{\pgfqpoint{6.240621in}{4.349956in}}%
\pgfpathlineto{\pgfqpoint{6.240621in}{4.351174in}}%
\pgfpathlineto{\pgfqpoint{6.241949in}{4.351174in}}%
\pgfpathlineto{\pgfqpoint{6.246147in}{4.352393in}}%
\pgfpathlineto{\pgfqpoint{6.246147in}{4.353611in}}%
\pgfpathlineto{\pgfqpoint{6.254435in}{4.354829in}}%
\pgfpathlineto{\pgfqpoint{6.254435in}{4.356047in}}%
\pgfpathlineto{\pgfqpoint{6.263414in}{4.357265in}}%
\pgfpathlineto{\pgfqpoint{6.263414in}{4.358483in}}%
\pgfpathlineto{\pgfqpoint{6.269843in}{4.359701in}}%
\pgfpathlineto{\pgfqpoint{6.269843in}{4.360919in}}%
\pgfpathlineto{\pgfqpoint{6.275793in}{4.362137in}}%
\pgfpathlineto{\pgfqpoint{6.275793in}{4.363355in}}%
\pgfpathlineto{\pgfqpoint{6.279831in}{4.364573in}}%
\pgfpathlineto{\pgfqpoint{6.280150in}{4.367010in}}%
\pgfpathlineto{\pgfqpoint{6.286791in}{4.368228in}}%
\pgfpathlineto{\pgfqpoint{6.286791in}{4.369446in}}%
\pgfpathlineto{\pgfqpoint{6.295026in}{4.370664in}}%
\pgfpathlineto{\pgfqpoint{6.295504in}{4.373100in}}%
\pgfpathlineto{\pgfqpoint{6.297470in}{4.374318in}}%
\pgfpathlineto{\pgfqpoint{6.297683in}{4.376754in}}%
\pgfpathlineto{\pgfqpoint{6.301189in}{4.377972in}}%
\pgfpathlineto{\pgfqpoint{6.302411in}{4.381627in}}%
\pgfpathlineto{\pgfqpoint{6.305546in}{4.382845in}}%
\pgfpathlineto{\pgfqpoint{6.306555in}{4.385281in}}%
\pgfpathlineto{\pgfqpoint{6.309478in}{4.386499in}}%
\pgfpathlineto{\pgfqpoint{6.310753in}{4.388935in}}%
\pgfpathlineto{\pgfqpoint{6.315216in}{4.390153in}}%
\pgfpathlineto{\pgfqpoint{6.316650in}{4.393808in}}%
\pgfpathlineto{\pgfqpoint{6.321697in}{4.395026in}}%
\pgfpathlineto{\pgfqpoint{6.321697in}{4.396244in}}%
\pgfpathlineto{\pgfqpoint{6.327382in}{4.397462in}}%
\pgfpathlineto{\pgfqpoint{6.327435in}{4.399898in}}%
\pgfpathlineto{\pgfqpoint{6.334289in}{4.401116in}}%
\pgfpathlineto{\pgfqpoint{6.334289in}{4.402334in}}%
\pgfpathlineto{\pgfqpoint{6.339974in}{4.403552in}}%
\pgfpathlineto{\pgfqpoint{6.339974in}{4.404770in}}%
\pgfpathlineto{\pgfqpoint{6.352035in}{4.405989in}}%
\pgfpathlineto{\pgfqpoint{6.352035in}{4.407207in}}%
\pgfpathlineto{\pgfqpoint{6.361810in}{4.408425in}}%
\pgfpathlineto{\pgfqpoint{6.363192in}{4.410861in}}%
\pgfpathlineto{\pgfqpoint{6.366858in}{4.412079in}}%
\pgfpathlineto{\pgfqpoint{6.367017in}{4.414515in}}%
\pgfpathlineto{\pgfqpoint{6.383806in}{4.415733in}}%
\pgfpathlineto{\pgfqpoint{6.384816in}{4.419388in}}%
\pgfpathlineto{\pgfqpoint{6.389013in}{4.420606in}}%
\pgfpathlineto{\pgfqpoint{6.389013in}{4.421824in}}%
\pgfpathlineto{\pgfqpoint{6.389757in}{4.421824in}}%
\pgfpathlineto{\pgfqpoint{6.397567in}{4.423042in}}%
\pgfpathlineto{\pgfqpoint{6.397567in}{4.424260in}}%
\pgfpathlineto{\pgfqpoint{6.404952in}{4.425478in}}%
\pgfpathlineto{\pgfqpoint{6.404952in}{4.426696in}}%
\pgfpathlineto{\pgfqpoint{6.407343in}{4.427914in}}%
\pgfpathlineto{\pgfqpoint{6.407343in}{4.429132in}}%
\pgfpathlineto{\pgfqpoint{6.408193in}{4.429132in}}%
\pgfpathlineto{\pgfqpoint{6.416641in}{4.430350in}}%
\pgfpathlineto{\pgfqpoint{6.416641in}{4.431568in}}%
\pgfpathlineto{\pgfqpoint{6.424344in}{4.432787in}}%
\pgfpathlineto{\pgfqpoint{6.424344in}{4.434005in}}%
\pgfpathlineto{\pgfqpoint{6.429551in}{4.435223in}}%
\pgfpathlineto{\pgfqpoint{6.429551in}{4.436441in}}%
\pgfpathlineto{\pgfqpoint{6.447243in}{4.437659in}}%
\pgfpathlineto{\pgfqpoint{6.447243in}{4.438877in}}%
\pgfpathlineto{\pgfqpoint{6.455372in}{4.440095in}}%
\pgfpathlineto{\pgfqpoint{6.455372in}{4.441313in}}%
\pgfpathlineto{\pgfqpoint{6.463129in}{4.442531in}}%
\pgfpathlineto{\pgfqpoint{6.463129in}{4.443749in}}%
\pgfpathlineto{\pgfqpoint{6.478378in}{4.444967in}}%
\pgfpathlineto{\pgfqpoint{6.478378in}{4.446186in}}%
\pgfpathlineto{\pgfqpoint{6.491501in}{4.447404in}}%
\pgfpathlineto{\pgfqpoint{6.491501in}{4.448622in}}%
\pgfpathlineto{\pgfqpoint{6.503986in}{4.449840in}}%
\pgfpathlineto{\pgfqpoint{6.503986in}{4.451058in}}%
\pgfpathlineto{\pgfqpoint{6.533845in}{4.452276in}}%
\pgfpathlineto{\pgfqpoint{6.533845in}{4.453494in}}%
\pgfpathlineto{\pgfqpoint{6.551112in}{4.454712in}}%
\pgfpathlineto{\pgfqpoint{6.551112in}{4.455930in}}%
\pgfpathlineto{\pgfqpoint{6.558179in}{4.457148in}}%
\pgfpathlineto{\pgfqpoint{6.559454in}{4.460803in}}%
\pgfpathlineto{\pgfqpoint{6.568486in}{4.462021in}}%
\pgfpathlineto{\pgfqpoint{6.568486in}{4.463239in}}%
\pgfpathlineto{\pgfqpoint{6.585594in}{4.464457in}}%
\pgfpathlineto{\pgfqpoint{6.585594in}{4.465675in}}%
\pgfpathlineto{\pgfqpoint{6.611521in}{4.466893in}}%
\pgfpathlineto{\pgfqpoint{6.611521in}{4.468111in}}%
\pgfpathlineto{\pgfqpoint{6.617950in}{4.469329in}}%
\pgfpathlineto{\pgfqpoint{6.617950in}{4.470547in}}%
\pgfpathlineto{\pgfqpoint{6.628257in}{4.471765in}}%
\pgfpathlineto{\pgfqpoint{6.628257in}{4.472984in}}%
\pgfpathlineto{\pgfqpoint{6.632242in}{4.474202in}}%
\pgfpathlineto{\pgfqpoint{6.632242in}{4.475420in}}%
\pgfpathlineto{\pgfqpoint{6.640424in}{4.476638in}}%
\pgfpathlineto{\pgfqpoint{6.640424in}{4.477856in}}%
\pgfpathlineto{\pgfqpoint{6.650678in}{4.479074in}}%
\pgfpathlineto{\pgfqpoint{6.650678in}{4.480292in}}%
\pgfpathlineto{\pgfqpoint{6.660773in}{4.481510in}}%
\pgfpathlineto{\pgfqpoint{6.661995in}{4.483946in}}%
\pgfpathlineto{\pgfqpoint{6.678146in}{4.485164in}}%
\pgfpathlineto{\pgfqpoint{6.678146in}{4.486383in}}%
\pgfpathlineto{\pgfqpoint{6.687072in}{4.487601in}}%
\pgfpathlineto{\pgfqpoint{6.687072in}{4.488819in}}%
\pgfpathlineto{\pgfqpoint{6.688613in}{4.488819in}}%
\pgfpathlineto{\pgfqpoint{6.697379in}{4.490037in}}%
\pgfpathlineto{\pgfqpoint{6.697485in}{4.492473in}}%
\pgfpathlineto{\pgfqpoint{6.701258in}{4.493691in}}%
\pgfpathlineto{\pgfqpoint{6.701258in}{4.494909in}}%
\pgfpathlineto{\pgfqpoint{6.735420in}{4.496127in}}%
\pgfpathlineto{\pgfqpoint{6.736323in}{4.498563in}}%
\pgfpathlineto{\pgfqpoint{6.757841in}{4.499782in}}%
\pgfpathlineto{\pgfqpoint{6.757841in}{4.501000in}}%
\pgfpathlineto{\pgfqpoint{6.758372in}{4.501000in}}%
\pgfpathlineto{\pgfqpoint{6.762038in}{4.502218in}}%
\pgfpathlineto{\pgfqpoint{6.763154in}{4.504654in}}%
\pgfpathlineto{\pgfqpoint{6.813627in}{4.505872in}}%
\pgfpathlineto{\pgfqpoint{6.813627in}{4.507090in}}%
\pgfpathlineto{\pgfqpoint{6.822659in}{4.508308in}}%
\pgfpathlineto{\pgfqpoint{6.822659in}{4.509526in}}%
\pgfpathlineto{\pgfqpoint{6.830523in}{4.510744in}}%
\pgfpathlineto{\pgfqpoint{6.831638in}{4.513181in}}%
\pgfpathlineto{\pgfqpoint{6.854059in}{4.514399in}}%
\pgfpathlineto{\pgfqpoint{6.854059in}{4.515617in}}%
\pgfpathlineto{\pgfqpoint{6.856769in}{4.515617in}}%
\pgfpathlineto{\pgfqpoint{6.856769in}{4.518053in}}%
\pgfpathlineto{\pgfqpoint{6.865482in}{4.519271in}}%
\pgfpathlineto{\pgfqpoint{6.865482in}{4.520489in}}%
\pgfpathlineto{\pgfqpoint{6.875099in}{4.521707in}}%
\pgfpathlineto{\pgfqpoint{6.875099in}{4.522925in}}%
\pgfpathlineto{\pgfqpoint{6.888541in}{4.524143in}}%
\pgfpathlineto{\pgfqpoint{6.888541in}{4.525361in}}%
\pgfpathlineto{\pgfqpoint{6.889975in}{4.525361in}}%
\pgfpathlineto{\pgfqpoint{6.897732in}{4.526580in}}%
\pgfpathlineto{\pgfqpoint{6.897732in}{4.527798in}}%
\pgfpathlineto{\pgfqpoint{6.918134in}{4.529016in}}%
\pgfpathlineto{\pgfqpoint{6.918400in}{4.531452in}}%
\pgfpathlineto{\pgfqpoint{6.959947in}{4.532670in}}%
\pgfpathlineto{\pgfqpoint{6.961382in}{4.535106in}}%
\pgfpathlineto{\pgfqpoint{6.974186in}{4.536324in}}%
\pgfpathlineto{\pgfqpoint{6.974186in}{4.537542in}}%
\pgfpathlineto{\pgfqpoint{7.002398in}{4.538760in}}%
\pgfpathlineto{\pgfqpoint{7.003089in}{4.541197in}}%
\pgfpathlineto{\pgfqpoint{7.033213in}{4.542415in}}%
\pgfpathlineto{\pgfqpoint{7.034382in}{4.544851in}}%
\pgfpathlineto{\pgfqpoint{7.052765in}{4.546069in}}%
\pgfpathlineto{\pgfqpoint{7.052765in}{4.547287in}}%
\pgfpathlineto{\pgfqpoint{7.058025in}{4.548505in}}%
\pgfpathlineto{\pgfqpoint{7.058025in}{4.549723in}}%
\pgfpathlineto{\pgfqpoint{7.074920in}{4.550941in}}%
\pgfpathlineto{\pgfqpoint{7.074920in}{4.552159in}}%
\pgfpathlineto{\pgfqpoint{7.082252in}{4.553378in}}%
\pgfpathlineto{\pgfqpoint{7.082252in}{4.554596in}}%
\pgfpathlineto{\pgfqpoint{7.097235in}{4.555814in}}%
\pgfpathlineto{\pgfqpoint{7.097235in}{4.557032in}}%
\pgfpathlineto{\pgfqpoint{7.115512in}{4.558250in}}%
\pgfpathlineto{\pgfqpoint{7.115512in}{4.559468in}}%
\pgfpathlineto{\pgfqpoint{7.183465in}{4.560686in}}%
\pgfpathlineto{\pgfqpoint{7.183571in}{4.563122in}}%
\pgfpathlineto{\pgfqpoint{7.202857in}{4.564340in}}%
\pgfpathlineto{\pgfqpoint{7.202857in}{4.565558in}}%
\pgfpathlineto{\pgfqpoint{7.211942in}{4.566777in}}%
\pgfpathlineto{\pgfqpoint{7.211942in}{4.567995in}}%
\pgfpathlineto{\pgfqpoint{7.215821in}{4.569213in}}%
\pgfpathlineto{\pgfqpoint{7.215821in}{4.570431in}}%
\pgfpathlineto{\pgfqpoint{7.298650in}{4.571649in}}%
\pgfpathlineto{\pgfqpoint{7.298650in}{4.571649in}}%
\pgfusepath{stroke}%
\end{pgfscope}%
\begin{pgfscope}%
\pgfpathrectangle{\pgfqpoint{5.688041in}{3.102083in}}{\pgfqpoint{1.687305in}{1.539545in}}%
\pgfusepath{clip}%
\pgfsetrectcap%
\pgfsetroundjoin%
\pgfsetlinewidth{1.505625pt}%
\definecolor{currentstroke}{rgb}{0.501961,0.501961,0.501961}%
\pgfsetstrokecolor{currentstroke}%
\pgfsetdash{}{0pt}%
\pgfpathmoveto{\pgfqpoint{5.764736in}{3.172062in}}%
\pgfpathlineto{\pgfqpoint{7.298650in}{4.571649in}}%
\pgfusepath{stroke}%
\end{pgfscope}%
\begin{pgfscope}%
\pgfsetrectcap%
\pgfsetmiterjoin%
\pgfsetlinewidth{0.803000pt}%
\definecolor{currentstroke}{rgb}{0.000000,0.000000,0.000000}%
\pgfsetstrokecolor{currentstroke}%
\pgfsetdash{}{0pt}%
\pgfpathmoveto{\pgfqpoint{5.688041in}{3.102083in}}%
\pgfpathlineto{\pgfqpoint{5.688041in}{4.641628in}}%
\pgfusepath{stroke}%
\end{pgfscope}%
\begin{pgfscope}%
\pgfsetrectcap%
\pgfsetmiterjoin%
\pgfsetlinewidth{0.803000pt}%
\definecolor{currentstroke}{rgb}{0.000000,0.000000,0.000000}%
\pgfsetstrokecolor{currentstroke}%
\pgfsetdash{}{0pt}%
\pgfpathmoveto{\pgfqpoint{7.375346in}{3.102083in}}%
\pgfpathlineto{\pgfqpoint{7.375346in}{4.641628in}}%
\pgfusepath{stroke}%
\end{pgfscope}%
\begin{pgfscope}%
\pgfsetrectcap%
\pgfsetmiterjoin%
\pgfsetlinewidth{0.803000pt}%
\definecolor{currentstroke}{rgb}{0.000000,0.000000,0.000000}%
\pgfsetstrokecolor{currentstroke}%
\pgfsetdash{}{0pt}%
\pgfpathmoveto{\pgfqpoint{5.688041in}{3.102083in}}%
\pgfpathlineto{\pgfqpoint{7.375346in}{3.102083in}}%
\pgfusepath{stroke}%
\end{pgfscope}%
\begin{pgfscope}%
\pgfsetrectcap%
\pgfsetmiterjoin%
\pgfsetlinewidth{0.803000pt}%
\definecolor{currentstroke}{rgb}{0.000000,0.000000,0.000000}%
\pgfsetstrokecolor{currentstroke}%
\pgfsetdash{}{0pt}%
\pgfpathmoveto{\pgfqpoint{5.688041in}{4.641628in}}%
\pgfpathlineto{\pgfqpoint{7.375346in}{4.641628in}}%
\pgfusepath{stroke}%
\end{pgfscope}%
\begin{pgfscope}%
\definecolor{textcolor}{rgb}{0.000000,0.000000,0.000000}%
\pgfsetstrokecolor{textcolor}%
\pgfsetfillcolor{textcolor}%
\pgftext[x=6.531693in,y=4.724962in,,base]{\color{textcolor}\rmfamily\fontsize{20.000000}{24.000000}\selectfont Pneumonia}%
\end{pgfscope}%
\begin{pgfscope}%
\pgfsetbuttcap%
\pgfsetmiterjoin%
\definecolor{currentfill}{rgb}{1.000000,1.000000,1.000000}%
\pgfsetfillcolor{currentfill}%
\pgfsetfillopacity{0.800000}%
\pgfsetlinewidth{1.003750pt}%
\definecolor{currentstroke}{rgb}{0.800000,0.800000,0.800000}%
\pgfsetstrokecolor{currentstroke}%
\pgfsetstrokeopacity{0.800000}%
\pgfsetdash{}{0pt}%
\pgfpathmoveto{\pgfqpoint{6.166240in}{3.171527in}}%
\pgfpathlineto{\pgfqpoint{7.278124in}{3.171527in}}%
\pgfpathquadraticcurveto{\pgfqpoint{7.305902in}{3.171527in}}{\pgfqpoint{7.305902in}{3.199305in}}%
\pgfpathlineto{\pgfqpoint{7.305902in}{3.379089in}}%
\pgfpathquadraticcurveto{\pgfqpoint{7.305902in}{3.406867in}}{\pgfqpoint{7.278124in}{3.406867in}}%
\pgfpathlineto{\pgfqpoint{6.166240in}{3.406867in}}%
\pgfpathquadraticcurveto{\pgfqpoint{6.138462in}{3.406867in}}{\pgfqpoint{6.138462in}{3.379089in}}%
\pgfpathlineto{\pgfqpoint{6.138462in}{3.199305in}}%
\pgfpathquadraticcurveto{\pgfqpoint{6.138462in}{3.171527in}}{\pgfqpoint{6.166240in}{3.171527in}}%
\pgfpathclose%
\pgfusepath{stroke,fill}%
\end{pgfscope}%
\begin{pgfscope}%
\pgfsetrectcap%
\pgfsetroundjoin%
\pgfsetlinewidth{1.505625pt}%
\definecolor{currentstroke}{rgb}{0.000000,0.501961,0.000000}%
\pgfsetstrokecolor{currentstroke}%
\pgfsetdash{}{0pt}%
\pgfpathmoveto{\pgfqpoint{6.194018in}{3.302700in}}%
\pgfpathlineto{\pgfqpoint{6.471795in}{3.302700in}}%
\pgfusepath{stroke}%
\end{pgfscope}%
\begin{pgfscope}%
\definecolor{textcolor}{rgb}{0.000000,0.000000,0.000000}%
\pgfsetstrokecolor{textcolor}%
\pgfsetfillcolor{textcolor}%
\pgftext[x=6.582907in,y=3.254089in,left,base]{\color{textcolor}\rmfamily\fontsize{10.000000}{12.000000}\selectfont AUC 0.863}%
\end{pgfscope}%
\begin{pgfscope}%
\pgfsetbuttcap%
\pgfsetmiterjoin%
\definecolor{currentfill}{rgb}{1.000000,1.000000,1.000000}%
\pgfsetfillcolor{currentfill}%
\pgfsetlinewidth{0.000000pt}%
\definecolor{currentstroke}{rgb}{0.000000,0.000000,0.000000}%
\pgfsetstrokecolor{currentstroke}%
\pgfsetstrokeopacity{0.000000}%
\pgfsetdash{}{0pt}%
\pgfpathmoveto{\pgfqpoint{8.150541in}{3.102083in}}%
\pgfpathlineto{\pgfqpoint{9.837846in}{3.102083in}}%
\pgfpathlineto{\pgfqpoint{9.837846in}{4.641628in}}%
\pgfpathlineto{\pgfqpoint{8.150541in}{4.641628in}}%
\pgfpathclose%
\pgfusepath{fill}%
\end{pgfscope}%
\begin{pgfscope}%
\pgfsetbuttcap%
\pgfsetroundjoin%
\definecolor{currentfill}{rgb}{0.000000,0.000000,0.000000}%
\pgfsetfillcolor{currentfill}%
\pgfsetlinewidth{0.803000pt}%
\definecolor{currentstroke}{rgb}{0.000000,0.000000,0.000000}%
\pgfsetstrokecolor{currentstroke}%
\pgfsetdash{}{0pt}%
\pgfsys@defobject{currentmarker}{\pgfqpoint{0.000000in}{-0.048611in}}{\pgfqpoint{0.000000in}{0.000000in}}{%
\pgfpathmoveto{\pgfqpoint{0.000000in}{0.000000in}}%
\pgfpathlineto{\pgfqpoint{0.000000in}{-0.048611in}}%
\pgfusepath{stroke,fill}%
}%
\begin{pgfscope}%
\pgfsys@transformshift{8.227236in}{3.102083in}%
\pgfsys@useobject{currentmarker}{}%
\end{pgfscope}%
\end{pgfscope}%
\begin{pgfscope}%
\definecolor{textcolor}{rgb}{0.000000,0.000000,0.000000}%
\pgfsetstrokecolor{textcolor}%
\pgfsetfillcolor{textcolor}%
\pgftext[x=8.227236in,y=3.004861in,,top]{\color{textcolor}\rmfamily\fontsize{10.000000}{12.000000}\selectfont \(\displaystyle {0.0}\)}%
\end{pgfscope}%
\begin{pgfscope}%
\pgfsetbuttcap%
\pgfsetroundjoin%
\definecolor{currentfill}{rgb}{0.000000,0.000000,0.000000}%
\pgfsetfillcolor{currentfill}%
\pgfsetlinewidth{0.803000pt}%
\definecolor{currentstroke}{rgb}{0.000000,0.000000,0.000000}%
\pgfsetstrokecolor{currentstroke}%
\pgfsetdash{}{0pt}%
\pgfsys@defobject{currentmarker}{\pgfqpoint{0.000000in}{-0.048611in}}{\pgfqpoint{0.000000in}{0.000000in}}{%
\pgfpathmoveto{\pgfqpoint{0.000000in}{0.000000in}}%
\pgfpathlineto{\pgfqpoint{0.000000in}{-0.048611in}}%
\pgfusepath{stroke,fill}%
}%
\begin{pgfscope}%
\pgfsys@transformshift{8.994193in}{3.102083in}%
\pgfsys@useobject{currentmarker}{}%
\end{pgfscope}%
\end{pgfscope}%
\begin{pgfscope}%
\definecolor{textcolor}{rgb}{0.000000,0.000000,0.000000}%
\pgfsetstrokecolor{textcolor}%
\pgfsetfillcolor{textcolor}%
\pgftext[x=8.994193in,y=3.004861in,,top]{\color{textcolor}\rmfamily\fontsize{10.000000}{12.000000}\selectfont \(\displaystyle {0.5}\)}%
\end{pgfscope}%
\begin{pgfscope}%
\pgfsetbuttcap%
\pgfsetroundjoin%
\definecolor{currentfill}{rgb}{0.000000,0.000000,0.000000}%
\pgfsetfillcolor{currentfill}%
\pgfsetlinewidth{0.803000pt}%
\definecolor{currentstroke}{rgb}{0.000000,0.000000,0.000000}%
\pgfsetstrokecolor{currentstroke}%
\pgfsetdash{}{0pt}%
\pgfsys@defobject{currentmarker}{\pgfqpoint{0.000000in}{-0.048611in}}{\pgfqpoint{0.000000in}{0.000000in}}{%
\pgfpathmoveto{\pgfqpoint{0.000000in}{0.000000in}}%
\pgfpathlineto{\pgfqpoint{0.000000in}{-0.048611in}}%
\pgfusepath{stroke,fill}%
}%
\begin{pgfscope}%
\pgfsys@transformshift{9.761150in}{3.102083in}%
\pgfsys@useobject{currentmarker}{}%
\end{pgfscope}%
\end{pgfscope}%
\begin{pgfscope}%
\definecolor{textcolor}{rgb}{0.000000,0.000000,0.000000}%
\pgfsetstrokecolor{textcolor}%
\pgfsetfillcolor{textcolor}%
\pgftext[x=9.761150in,y=3.004861in,,top]{\color{textcolor}\rmfamily\fontsize{10.000000}{12.000000}\selectfont \(\displaystyle {1.0}\)}%
\end{pgfscope}%
\begin{pgfscope}%
\definecolor{textcolor}{rgb}{0.000000,0.000000,0.000000}%
\pgfsetstrokecolor{textcolor}%
\pgfsetfillcolor{textcolor}%
\pgftext[x=8.994193in,y=2.825849in,,top]{\color{textcolor}\rmfamily\fontsize{16.000000}{19.200000}\selectfont FPR}%
\end{pgfscope}%
\begin{pgfscope}%
\pgfsetbuttcap%
\pgfsetroundjoin%
\definecolor{currentfill}{rgb}{0.000000,0.000000,0.000000}%
\pgfsetfillcolor{currentfill}%
\pgfsetlinewidth{0.803000pt}%
\definecolor{currentstroke}{rgb}{0.000000,0.000000,0.000000}%
\pgfsetstrokecolor{currentstroke}%
\pgfsetdash{}{0pt}%
\pgfsys@defobject{currentmarker}{\pgfqpoint{-0.048611in}{0.000000in}}{\pgfqpoint{-0.000000in}{0.000000in}}{%
\pgfpathmoveto{\pgfqpoint{-0.000000in}{0.000000in}}%
\pgfpathlineto{\pgfqpoint{-0.048611in}{0.000000in}}%
\pgfusepath{stroke,fill}%
}%
\begin{pgfscope}%
\pgfsys@transformshift{8.150541in}{3.172062in}%
\pgfsys@useobject{currentmarker}{}%
\end{pgfscope}%
\end{pgfscope}%
\begin{pgfscope}%
\definecolor{textcolor}{rgb}{0.000000,0.000000,0.000000}%
\pgfsetstrokecolor{textcolor}%
\pgfsetfillcolor{textcolor}%
\pgftext[x=7.806404in, y=3.123837in, left, base]{\color{textcolor}\rmfamily\fontsize{10.000000}{12.000000}\selectfont \(\displaystyle {0.00}\)}%
\end{pgfscope}%
\begin{pgfscope}%
\pgfsetbuttcap%
\pgfsetroundjoin%
\definecolor{currentfill}{rgb}{0.000000,0.000000,0.000000}%
\pgfsetfillcolor{currentfill}%
\pgfsetlinewidth{0.803000pt}%
\definecolor{currentstroke}{rgb}{0.000000,0.000000,0.000000}%
\pgfsetstrokecolor{currentstroke}%
\pgfsetdash{}{0pt}%
\pgfsys@defobject{currentmarker}{\pgfqpoint{-0.048611in}{0.000000in}}{\pgfqpoint{-0.000000in}{0.000000in}}{%
\pgfpathmoveto{\pgfqpoint{-0.000000in}{0.000000in}}%
\pgfpathlineto{\pgfqpoint{-0.048611in}{0.000000in}}%
\pgfusepath{stroke,fill}%
}%
\begin{pgfscope}%
\pgfsys@transformshift{8.150541in}{3.521959in}%
\pgfsys@useobject{currentmarker}{}%
\end{pgfscope}%
\end{pgfscope}%
\begin{pgfscope}%
\definecolor{textcolor}{rgb}{0.000000,0.000000,0.000000}%
\pgfsetstrokecolor{textcolor}%
\pgfsetfillcolor{textcolor}%
\pgftext[x=7.806404in, y=3.473734in, left, base]{\color{textcolor}\rmfamily\fontsize{10.000000}{12.000000}\selectfont \(\displaystyle {0.25}\)}%
\end{pgfscope}%
\begin{pgfscope}%
\pgfsetbuttcap%
\pgfsetroundjoin%
\definecolor{currentfill}{rgb}{0.000000,0.000000,0.000000}%
\pgfsetfillcolor{currentfill}%
\pgfsetlinewidth{0.803000pt}%
\definecolor{currentstroke}{rgb}{0.000000,0.000000,0.000000}%
\pgfsetstrokecolor{currentstroke}%
\pgfsetdash{}{0pt}%
\pgfsys@defobject{currentmarker}{\pgfqpoint{-0.048611in}{0.000000in}}{\pgfqpoint{-0.000000in}{0.000000in}}{%
\pgfpathmoveto{\pgfqpoint{-0.000000in}{0.000000in}}%
\pgfpathlineto{\pgfqpoint{-0.048611in}{0.000000in}}%
\pgfusepath{stroke,fill}%
}%
\begin{pgfscope}%
\pgfsys@transformshift{8.150541in}{3.871856in}%
\pgfsys@useobject{currentmarker}{}%
\end{pgfscope}%
\end{pgfscope}%
\begin{pgfscope}%
\definecolor{textcolor}{rgb}{0.000000,0.000000,0.000000}%
\pgfsetstrokecolor{textcolor}%
\pgfsetfillcolor{textcolor}%
\pgftext[x=7.806404in, y=3.823630in, left, base]{\color{textcolor}\rmfamily\fontsize{10.000000}{12.000000}\selectfont \(\displaystyle {0.50}\)}%
\end{pgfscope}%
\begin{pgfscope}%
\pgfsetbuttcap%
\pgfsetroundjoin%
\definecolor{currentfill}{rgb}{0.000000,0.000000,0.000000}%
\pgfsetfillcolor{currentfill}%
\pgfsetlinewidth{0.803000pt}%
\definecolor{currentstroke}{rgb}{0.000000,0.000000,0.000000}%
\pgfsetstrokecolor{currentstroke}%
\pgfsetdash{}{0pt}%
\pgfsys@defobject{currentmarker}{\pgfqpoint{-0.048611in}{0.000000in}}{\pgfqpoint{-0.000000in}{0.000000in}}{%
\pgfpathmoveto{\pgfqpoint{-0.000000in}{0.000000in}}%
\pgfpathlineto{\pgfqpoint{-0.048611in}{0.000000in}}%
\pgfusepath{stroke,fill}%
}%
\begin{pgfscope}%
\pgfsys@transformshift{8.150541in}{4.221752in}%
\pgfsys@useobject{currentmarker}{}%
\end{pgfscope}%
\end{pgfscope}%
\begin{pgfscope}%
\definecolor{textcolor}{rgb}{0.000000,0.000000,0.000000}%
\pgfsetstrokecolor{textcolor}%
\pgfsetfillcolor{textcolor}%
\pgftext[x=7.806404in, y=4.173527in, left, base]{\color{textcolor}\rmfamily\fontsize{10.000000}{12.000000}\selectfont \(\displaystyle {0.75}\)}%
\end{pgfscope}%
\begin{pgfscope}%
\pgfsetbuttcap%
\pgfsetroundjoin%
\definecolor{currentfill}{rgb}{0.000000,0.000000,0.000000}%
\pgfsetfillcolor{currentfill}%
\pgfsetlinewidth{0.803000pt}%
\definecolor{currentstroke}{rgb}{0.000000,0.000000,0.000000}%
\pgfsetstrokecolor{currentstroke}%
\pgfsetdash{}{0pt}%
\pgfsys@defobject{currentmarker}{\pgfqpoint{-0.048611in}{0.000000in}}{\pgfqpoint{-0.000000in}{0.000000in}}{%
\pgfpathmoveto{\pgfqpoint{-0.000000in}{0.000000in}}%
\pgfpathlineto{\pgfqpoint{-0.048611in}{0.000000in}}%
\pgfusepath{stroke,fill}%
}%
\begin{pgfscope}%
\pgfsys@transformshift{8.150541in}{4.571649in}%
\pgfsys@useobject{currentmarker}{}%
\end{pgfscope}%
\end{pgfscope}%
\begin{pgfscope}%
\definecolor{textcolor}{rgb}{0.000000,0.000000,0.000000}%
\pgfsetstrokecolor{textcolor}%
\pgfsetfillcolor{textcolor}%
\pgftext[x=7.806404in, y=4.523424in, left, base]{\color{textcolor}\rmfamily\fontsize{10.000000}{12.000000}\selectfont \(\displaystyle {1.00}\)}%
\end{pgfscope}%
\begin{pgfscope}%
\definecolor{textcolor}{rgb}{0.000000,0.000000,0.000000}%
\pgfsetstrokecolor{textcolor}%
\pgfsetfillcolor{textcolor}%
\pgftext[x=7.750849in,y=3.871856in,,bottom,rotate=90.000000]{\color{textcolor}\rmfamily\fontsize{16.000000}{19.200000}\selectfont TPR}%
\end{pgfscope}%
\begin{pgfscope}%
\pgfpathrectangle{\pgfqpoint{8.150541in}{3.102083in}}{\pgfqpoint{1.687305in}{1.539545in}}%
\pgfusepath{clip}%
\pgfsetrectcap%
\pgfsetroundjoin%
\pgfsetlinewidth{1.505625pt}%
\definecolor{currentstroke}{rgb}{0.000000,0.501961,0.000000}%
\pgfsetstrokecolor{currentstroke}%
\pgfsetdash{}{0pt}%
\pgfpathmoveto{\pgfqpoint{8.227236in}{3.172062in}}%
\pgfpathlineto{\pgfqpoint{8.240872in}{3.365109in}}%
\pgfpathlineto{\pgfqpoint{8.242013in}{3.365109in}}%
\pgfpathlineto{\pgfqpoint{8.243828in}{3.387631in}}%
\pgfpathlineto{\pgfqpoint{8.244035in}{3.387631in}}%
\pgfpathlineto{\pgfqpoint{8.244916in}{3.400501in}}%
\pgfpathlineto{\pgfqpoint{8.245020in}{3.400501in}}%
\pgfpathlineto{\pgfqpoint{8.245746in}{3.400501in}}%
\pgfpathlineto{\pgfqpoint{8.246990in}{3.426240in}}%
\pgfpathlineto{\pgfqpoint{8.247405in}{3.426240in}}%
\pgfpathlineto{\pgfqpoint{8.248701in}{3.445545in}}%
\pgfpathlineto{\pgfqpoint{8.250101in}{3.445545in}}%
\pgfpathlineto{\pgfqpoint{8.251294in}{3.455197in}}%
\pgfpathlineto{\pgfqpoint{8.251760in}{3.455197in}}%
\pgfpathlineto{\pgfqpoint{8.253264in}{3.474502in}}%
\pgfpathlineto{\pgfqpoint{8.254094in}{3.474502in}}%
\pgfpathlineto{\pgfqpoint{8.255338in}{3.484154in}}%
\pgfpathlineto{\pgfqpoint{8.255649in}{3.484154in}}%
\pgfpathlineto{\pgfqpoint{8.256530in}{3.497024in}}%
\pgfpathlineto{\pgfqpoint{8.257827in}{3.497024in}}%
\pgfpathlineto{\pgfqpoint{8.258449in}{3.509894in}}%
\pgfpathlineto{\pgfqpoint{8.259382in}{3.509894in}}%
\pgfpathlineto{\pgfqpoint{8.260523in}{3.525981in}}%
\pgfpathlineto{\pgfqpoint{8.260989in}{3.525981in}}%
\pgfpathlineto{\pgfqpoint{8.262234in}{3.548503in}}%
\pgfpathlineto{\pgfqpoint{8.262700in}{3.548503in}}%
\pgfpathlineto{\pgfqpoint{8.264100in}{3.564590in}}%
\pgfpathlineto{\pgfqpoint{8.264567in}{3.564590in}}%
\pgfpathlineto{\pgfqpoint{8.265967in}{3.587112in}}%
\pgfpathlineto{\pgfqpoint{8.266900in}{3.587112in}}%
\pgfpathlineto{\pgfqpoint{8.267159in}{3.593547in}}%
\pgfpathlineto{\pgfqpoint{8.268455in}{3.593547in}}%
\pgfpathlineto{\pgfqpoint{8.269803in}{3.606417in}}%
\pgfpathlineto{\pgfqpoint{8.270166in}{3.606417in}}%
\pgfpathlineto{\pgfqpoint{8.271048in}{3.619287in}}%
\pgfpathlineto{\pgfqpoint{8.273277in}{3.619287in}}%
\pgfpathlineto{\pgfqpoint{8.274729in}{3.645026in}}%
\pgfpathlineto{\pgfqpoint{8.275610in}{3.645026in}}%
\pgfpathlineto{\pgfqpoint{8.276699in}{3.654678in}}%
\pgfpathlineto{\pgfqpoint{8.277373in}{3.654678in}}%
\pgfpathlineto{\pgfqpoint{8.278773in}{3.670766in}}%
\pgfpathlineto{\pgfqpoint{8.279291in}{3.670766in}}%
\pgfpathlineto{\pgfqpoint{8.280536in}{3.683635in}}%
\pgfpathlineto{\pgfqpoint{8.281625in}{3.683635in}}%
\pgfpathlineto{\pgfqpoint{8.281780in}{3.690070in}}%
\pgfpathlineto{\pgfqpoint{8.283284in}{3.690070in}}%
\pgfpathlineto{\pgfqpoint{8.284373in}{3.706157in}}%
\pgfpathlineto{\pgfqpoint{8.285047in}{3.706157in}}%
\pgfpathlineto{\pgfqpoint{8.285254in}{3.712592in}}%
\pgfpathlineto{\pgfqpoint{8.287639in}{3.712592in}}%
\pgfpathlineto{\pgfqpoint{8.288780in}{3.725462in}}%
\pgfpathlineto{\pgfqpoint{8.289246in}{3.725462in}}%
\pgfpathlineto{\pgfqpoint{8.289868in}{3.731897in}}%
\pgfpathlineto{\pgfqpoint{8.291839in}{3.731897in}}%
\pgfpathlineto{\pgfqpoint{8.293187in}{3.741549in}}%
\pgfpathlineto{\pgfqpoint{8.295261in}{3.741549in}}%
\pgfpathlineto{\pgfqpoint{8.295831in}{3.751202in}}%
\pgfpathlineto{\pgfqpoint{8.297023in}{3.751202in}}%
\pgfpathlineto{\pgfqpoint{8.297386in}{3.757636in}}%
\pgfpathlineto{\pgfqpoint{8.298579in}{3.757636in}}%
\pgfpathlineto{\pgfqpoint{8.299927in}{3.773724in}}%
\pgfpathlineto{\pgfqpoint{8.300290in}{3.773724in}}%
\pgfpathlineto{\pgfqpoint{8.301793in}{3.783376in}}%
\pgfpathlineto{\pgfqpoint{8.302208in}{3.783376in}}%
\pgfpathlineto{\pgfqpoint{8.303297in}{3.793028in}}%
\pgfpathlineto{\pgfqpoint{8.303919in}{3.793028in}}%
\pgfpathlineto{\pgfqpoint{8.305423in}{3.802681in}}%
\pgfpathlineto{\pgfqpoint{8.305682in}{3.802681in}}%
\pgfpathlineto{\pgfqpoint{8.307186in}{3.818768in}}%
\pgfpathlineto{\pgfqpoint{8.309882in}{3.818768in}}%
\pgfpathlineto{\pgfqpoint{8.310867in}{3.828420in}}%
\pgfpathlineto{\pgfqpoint{8.312370in}{3.828420in}}%
\pgfpathlineto{\pgfqpoint{8.312992in}{3.834855in}}%
\pgfpathlineto{\pgfqpoint{8.313926in}{3.834855in}}%
\pgfpathlineto{\pgfqpoint{8.315222in}{3.844507in}}%
\pgfpathlineto{\pgfqpoint{8.319525in}{3.844507in}}%
\pgfpathlineto{\pgfqpoint{8.320147in}{3.854160in}}%
\pgfpathlineto{\pgfqpoint{8.322481in}{3.854160in}}%
\pgfpathlineto{\pgfqpoint{8.322481in}{3.860595in}}%
\pgfpathlineto{\pgfqpoint{8.324451in}{3.860595in}}%
\pgfpathlineto{\pgfqpoint{8.324555in}{3.867029in}}%
\pgfpathlineto{\pgfqpoint{8.327769in}{3.867029in}}%
\pgfpathlineto{\pgfqpoint{8.328132in}{3.873464in}}%
\pgfpathlineto{\pgfqpoint{8.330569in}{3.873464in}}%
\pgfpathlineto{\pgfqpoint{8.330569in}{3.876682in}}%
\pgfpathlineto{\pgfqpoint{8.332539in}{3.876682in}}%
\pgfpathlineto{\pgfqpoint{8.332850in}{3.886334in}}%
\pgfpathlineto{\pgfqpoint{8.335546in}{3.886334in}}%
\pgfpathlineto{\pgfqpoint{8.336376in}{3.899204in}}%
\pgfpathlineto{\pgfqpoint{8.337983in}{3.899204in}}%
\pgfpathlineto{\pgfqpoint{8.337983in}{3.902421in}}%
\pgfpathlineto{\pgfqpoint{8.340627in}{3.902421in}}%
\pgfpathlineto{\pgfqpoint{8.341457in}{3.908856in}}%
\pgfpathlineto{\pgfqpoint{8.343323in}{3.908856in}}%
\pgfpathlineto{\pgfqpoint{8.344671in}{3.915291in}}%
\pgfpathlineto{\pgfqpoint{8.347471in}{3.915291in}}%
\pgfpathlineto{\pgfqpoint{8.348871in}{3.928161in}}%
\pgfpathlineto{\pgfqpoint{8.349234in}{3.928161in}}%
\pgfpathlineto{\pgfqpoint{8.349701in}{3.941031in}}%
\pgfpathlineto{\pgfqpoint{8.351723in}{3.941031in}}%
\pgfpathlineto{\pgfqpoint{8.353174in}{3.953900in}}%
\pgfpathlineto{\pgfqpoint{8.355663in}{3.953900in}}%
\pgfpathlineto{\pgfqpoint{8.357115in}{3.963553in}}%
\pgfpathlineto{\pgfqpoint{8.358359in}{3.963553in}}%
\pgfpathlineto{\pgfqpoint{8.359448in}{3.982857in}}%
\pgfpathlineto{\pgfqpoint{8.361418in}{3.982857in}}%
\pgfpathlineto{\pgfqpoint{8.362611in}{3.989292in}}%
\pgfpathlineto{\pgfqpoint{8.363337in}{3.989292in}}%
\pgfpathlineto{\pgfqpoint{8.364788in}{3.998945in}}%
\pgfpathlineto{\pgfqpoint{8.368470in}{3.998945in}}%
\pgfpathlineto{\pgfqpoint{8.368470in}{4.002162in}}%
\pgfpathlineto{\pgfqpoint{8.370336in}{4.002162in}}%
\pgfpathlineto{\pgfqpoint{8.370647in}{4.011814in}}%
\pgfpathlineto{\pgfqpoint{8.372773in}{4.011814in}}%
\pgfpathlineto{\pgfqpoint{8.372773in}{4.015032in}}%
\pgfpathlineto{\pgfqpoint{8.377906in}{4.015032in}}%
\pgfpathlineto{\pgfqpoint{8.377906in}{4.018249in}}%
\pgfpathlineto{\pgfqpoint{8.379669in}{4.018249in}}%
\pgfpathlineto{\pgfqpoint{8.380187in}{4.024684in}}%
\pgfpathlineto{\pgfqpoint{8.381794in}{4.024684in}}%
\pgfpathlineto{\pgfqpoint{8.382831in}{4.031119in}}%
\pgfpathlineto{\pgfqpoint{8.384335in}{4.031119in}}%
\pgfpathlineto{\pgfqpoint{8.384335in}{4.034336in}}%
\pgfpathlineto{\pgfqpoint{8.386046in}{4.034336in}}%
\pgfpathlineto{\pgfqpoint{8.386927in}{4.043989in}}%
\pgfpathlineto{\pgfqpoint{8.392682in}{4.043989in}}%
\pgfpathlineto{\pgfqpoint{8.393097in}{4.050424in}}%
\pgfpathlineto{\pgfqpoint{8.395016in}{4.050424in}}%
\pgfpathlineto{\pgfqpoint{8.395016in}{4.053641in}}%
\pgfpathlineto{\pgfqpoint{8.399371in}{4.053641in}}%
\pgfpathlineto{\pgfqpoint{8.399371in}{4.056858in}}%
\pgfpathlineto{\pgfqpoint{8.403415in}{4.056858in}}%
\pgfpathlineto{\pgfqpoint{8.403415in}{4.060076in}}%
\pgfpathlineto{\pgfqpoint{8.405022in}{4.060076in}}%
\pgfpathlineto{\pgfqpoint{8.405022in}{4.063293in}}%
\pgfpathlineto{\pgfqpoint{8.406941in}{4.063293in}}%
\pgfpathlineto{\pgfqpoint{8.406941in}{4.066511in}}%
\pgfpathlineto{\pgfqpoint{8.408859in}{4.066511in}}%
\pgfpathlineto{\pgfqpoint{8.408859in}{4.069728in}}%
\pgfpathlineto{\pgfqpoint{8.412073in}{4.069728in}}%
\pgfpathlineto{\pgfqpoint{8.412073in}{4.072946in}}%
\pgfpathlineto{\pgfqpoint{8.414147in}{4.072946in}}%
\pgfpathlineto{\pgfqpoint{8.414925in}{4.079381in}}%
\pgfpathlineto{\pgfqpoint{8.418606in}{4.079381in}}%
\pgfpathlineto{\pgfqpoint{8.420058in}{4.098685in}}%
\pgfpathlineto{\pgfqpoint{8.420265in}{4.098685in}}%
\pgfpathlineto{\pgfqpoint{8.420525in}{4.105120in}}%
\pgfpathlineto{\pgfqpoint{8.423273in}{4.105120in}}%
\pgfpathlineto{\pgfqpoint{8.423273in}{4.108337in}}%
\pgfpathlineto{\pgfqpoint{8.428561in}{4.108337in}}%
\pgfpathlineto{\pgfqpoint{8.428820in}{4.114772in}}%
\pgfpathlineto{\pgfqpoint{8.430376in}{4.114772in}}%
\pgfpathlineto{\pgfqpoint{8.430376in}{4.117990in}}%
\pgfpathlineto{\pgfqpoint{8.434575in}{4.117990in}}%
\pgfpathlineto{\pgfqpoint{8.435146in}{4.124425in}}%
\pgfpathlineto{\pgfqpoint{8.443441in}{4.124425in}}%
\pgfpathlineto{\pgfqpoint{8.443441in}{4.127642in}}%
\pgfpathlineto{\pgfqpoint{8.445878in}{4.127642in}}%
\pgfpathlineto{\pgfqpoint{8.447174in}{4.134077in}}%
\pgfpathlineto{\pgfqpoint{8.449974in}{4.134077in}}%
\pgfpathlineto{\pgfqpoint{8.449974in}{4.137294in}}%
\pgfpathlineto{\pgfqpoint{8.452670in}{4.137294in}}%
\pgfpathlineto{\pgfqpoint{8.452670in}{4.140512in}}%
\pgfpathlineto{\pgfqpoint{8.454744in}{4.140512in}}%
\pgfpathlineto{\pgfqpoint{8.455107in}{4.146947in}}%
\pgfpathlineto{\pgfqpoint{8.459462in}{4.146947in}}%
\pgfpathlineto{\pgfqpoint{8.459462in}{4.150164in}}%
\pgfpathlineto{\pgfqpoint{8.462936in}{4.150164in}}%
\pgfpathlineto{\pgfqpoint{8.462936in}{4.153382in}}%
\pgfpathlineto{\pgfqpoint{8.465165in}{4.153382in}}%
\pgfpathlineto{\pgfqpoint{8.465477in}{4.163034in}}%
\pgfpathlineto{\pgfqpoint{8.474031in}{4.163034in}}%
\pgfpathlineto{\pgfqpoint{8.475017in}{4.169469in}}%
\pgfpathlineto{\pgfqpoint{8.484816in}{4.169469in}}%
\pgfpathlineto{\pgfqpoint{8.484816in}{4.172686in}}%
\pgfpathlineto{\pgfqpoint{8.488445in}{4.172686in}}%
\pgfpathlineto{\pgfqpoint{8.488445in}{4.175904in}}%
\pgfpathlineto{\pgfqpoint{8.490830in}{4.175904in}}%
\pgfpathlineto{\pgfqpoint{8.491193in}{4.182339in}}%
\pgfpathlineto{\pgfqpoint{8.492800in}{4.182339in}}%
\pgfpathlineto{\pgfqpoint{8.492800in}{4.185556in}}%
\pgfpathlineto{\pgfqpoint{8.499281in}{4.185556in}}%
\pgfpathlineto{\pgfqpoint{8.500163in}{4.191991in}}%
\pgfpathlineto{\pgfqpoint{8.516909in}{4.191991in}}%
\pgfpathlineto{\pgfqpoint{8.516909in}{4.195208in}}%
\pgfpathlineto{\pgfqpoint{8.520124in}{4.195208in}}%
\pgfpathlineto{\pgfqpoint{8.520280in}{4.201643in}}%
\pgfpathlineto{\pgfqpoint{8.522457in}{4.201643in}}%
\pgfpathlineto{\pgfqpoint{8.522457in}{4.204861in}}%
\pgfpathlineto{\pgfqpoint{8.524324in}{4.204861in}}%
\pgfpathlineto{\pgfqpoint{8.525101in}{4.220948in}}%
\pgfpathlineto{\pgfqpoint{8.529871in}{4.220948in}}%
\pgfpathlineto{\pgfqpoint{8.529871in}{4.224165in}}%
\pgfpathlineto{\pgfqpoint{8.531738in}{4.224165in}}%
\pgfpathlineto{\pgfqpoint{8.531738in}{4.227383in}}%
\pgfpathlineto{\pgfqpoint{8.532930in}{4.227383in}}%
\pgfpathlineto{\pgfqpoint{8.542004in}{4.227383in}}%
\pgfpathlineto{\pgfqpoint{8.542004in}{4.230600in}}%
\pgfpathlineto{\pgfqpoint{8.544285in}{4.230600in}}%
\pgfpathlineto{\pgfqpoint{8.544752in}{4.237035in}}%
\pgfpathlineto{\pgfqpoint{8.546411in}{4.237035in}}%
\pgfpathlineto{\pgfqpoint{8.546826in}{4.243470in}}%
\pgfpathlineto{\pgfqpoint{8.549366in}{4.243470in}}%
\pgfpathlineto{\pgfqpoint{8.549366in}{4.246687in}}%
\pgfpathlineto{\pgfqpoint{8.555588in}{4.246687in}}%
\pgfpathlineto{\pgfqpoint{8.555588in}{4.249905in}}%
\pgfpathlineto{\pgfqpoint{8.558284in}{4.249905in}}%
\pgfpathlineto{\pgfqpoint{8.558284in}{4.253122in}}%
\pgfpathlineto{\pgfqpoint{8.563158in}{4.253122in}}%
\pgfpathlineto{\pgfqpoint{8.563158in}{4.256340in}}%
\pgfpathlineto{\pgfqpoint{8.565232in}{4.256340in}}%
\pgfpathlineto{\pgfqpoint{8.565232in}{4.259557in}}%
\pgfpathlineto{\pgfqpoint{8.568394in}{4.259557in}}%
\pgfpathlineto{\pgfqpoint{8.569846in}{4.265992in}}%
\pgfpathlineto{\pgfqpoint{8.570779in}{4.265992in}}%
\pgfpathlineto{\pgfqpoint{8.570779in}{4.269209in}}%
\pgfpathlineto{\pgfqpoint{8.574201in}{4.269209in}}%
\pgfpathlineto{\pgfqpoint{8.574201in}{4.272427in}}%
\pgfpathlineto{\pgfqpoint{8.575238in}{4.272427in}}%
\pgfpathlineto{\pgfqpoint{8.579386in}{4.272427in}}%
\pgfpathlineto{\pgfqpoint{8.579386in}{4.275644in}}%
\pgfpathlineto{\pgfqpoint{8.582082in}{4.275644in}}%
\pgfpathlineto{\pgfqpoint{8.582082in}{4.278862in}}%
\pgfpathlineto{\pgfqpoint{8.583741in}{4.278862in}}%
\pgfpathlineto{\pgfqpoint{8.584000in}{4.285297in}}%
\pgfpathlineto{\pgfqpoint{8.593333in}{4.285297in}}%
\pgfpathlineto{\pgfqpoint{8.593592in}{4.291732in}}%
\pgfpathlineto{\pgfqpoint{8.595822in}{4.291732in}}%
\pgfpathlineto{\pgfqpoint{8.595822in}{4.294949in}}%
\pgfpathlineto{\pgfqpoint{8.602458in}{4.294949in}}%
\pgfpathlineto{\pgfqpoint{8.602458in}{4.298166in}}%
\pgfpathlineto{\pgfqpoint{8.613191in}{4.298166in}}%
\pgfpathlineto{\pgfqpoint{8.613191in}{4.301384in}}%
\pgfpathlineto{\pgfqpoint{8.626619in}{4.301384in}}%
\pgfpathlineto{\pgfqpoint{8.626619in}{4.304601in}}%
\pgfpathlineto{\pgfqpoint{8.628589in}{4.304601in}}%
\pgfpathlineto{\pgfqpoint{8.629834in}{4.311036in}}%
\pgfpathlineto{\pgfqpoint{8.639892in}{4.311036in}}%
\pgfpathlineto{\pgfqpoint{8.640463in}{4.317471in}}%
\pgfpathlineto{\pgfqpoint{8.644870in}{4.317471in}}%
\pgfpathlineto{\pgfqpoint{8.644870in}{4.320689in}}%
\pgfpathlineto{\pgfqpoint{8.656017in}{4.320689in}}%
\pgfpathlineto{\pgfqpoint{8.656017in}{4.323906in}}%
\pgfpathlineto{\pgfqpoint{8.657935in}{4.323906in}}%
\pgfpathlineto{\pgfqpoint{8.658920in}{4.330341in}}%
\pgfpathlineto{\pgfqpoint{8.665401in}{4.330341in}}%
\pgfpathlineto{\pgfqpoint{8.665401in}{4.333558in}}%
\pgfpathlineto{\pgfqpoint{8.669808in}{4.333558in}}%
\pgfpathlineto{\pgfqpoint{8.669808in}{4.336776in}}%
\pgfpathlineto{\pgfqpoint{8.679556in}{4.336776in}}%
\pgfpathlineto{\pgfqpoint{8.679556in}{4.339993in}}%
\pgfpathlineto{\pgfqpoint{8.682511in}{4.339993in}}%
\pgfpathlineto{\pgfqpoint{8.682511in}{4.343211in}}%
\pgfpathlineto{\pgfqpoint{8.688836in}{4.343211in}}%
\pgfpathlineto{\pgfqpoint{8.688836in}{4.346428in}}%
\pgfpathlineto{\pgfqpoint{8.693814in}{4.346428in}}%
\pgfpathlineto{\pgfqpoint{8.693814in}{4.349645in}}%
\pgfpathlineto{\pgfqpoint{8.695940in}{4.349645in}}%
\pgfpathlineto{\pgfqpoint{8.695940in}{4.352863in}}%
\pgfpathlineto{\pgfqpoint{8.697443in}{4.352863in}}%
\pgfpathlineto{\pgfqpoint{8.706879in}{4.352863in}}%
\pgfpathlineto{\pgfqpoint{8.706879in}{4.356080in}}%
\pgfpathlineto{\pgfqpoint{8.717560in}{4.356080in}}%
\pgfpathlineto{\pgfqpoint{8.717560in}{4.359298in}}%
\pgfpathlineto{\pgfqpoint{8.721086in}{4.359298in}}%
\pgfpathlineto{\pgfqpoint{8.721086in}{4.362515in}}%
\pgfpathlineto{\pgfqpoint{8.732337in}{4.362515in}}%
\pgfpathlineto{\pgfqpoint{8.732337in}{4.365733in}}%
\pgfpathlineto{\pgfqpoint{8.739129in}{4.365733in}}%
\pgfpathlineto{\pgfqpoint{8.739129in}{4.368950in}}%
\pgfpathlineto{\pgfqpoint{8.742706in}{4.368950in}}%
\pgfpathlineto{\pgfqpoint{8.742706in}{4.372168in}}%
\pgfpathlineto{\pgfqpoint{8.752454in}{4.372168in}}%
\pgfpathlineto{\pgfqpoint{8.752454in}{4.375385in}}%
\pgfpathlineto{\pgfqpoint{8.758779in}{4.375385in}}%
\pgfpathlineto{\pgfqpoint{8.758779in}{4.378602in}}%
\pgfpathlineto{\pgfqpoint{8.764586in}{4.378602in}}%
\pgfpathlineto{\pgfqpoint{8.764586in}{4.381820in}}%
\pgfpathlineto{\pgfqpoint{8.780348in}{4.381820in}}%
\pgfpathlineto{\pgfqpoint{8.780348in}{4.385037in}}%
\pgfpathlineto{\pgfqpoint{8.782992in}{4.385037in}}%
\pgfpathlineto{\pgfqpoint{8.782992in}{4.388255in}}%
\pgfpathlineto{\pgfqpoint{8.784858in}{4.388255in}}%
\pgfpathlineto{\pgfqpoint{8.784858in}{4.391472in}}%
\pgfpathlineto{\pgfqpoint{8.788125in}{4.391472in}}%
\pgfpathlineto{\pgfqpoint{8.788125in}{4.394690in}}%
\pgfpathlineto{\pgfqpoint{8.791702in}{4.394690in}}%
\pgfpathlineto{\pgfqpoint{8.791702in}{4.397907in}}%
\pgfpathlineto{\pgfqpoint{8.798339in}{4.397907in}}%
\pgfpathlineto{\pgfqpoint{8.798339in}{4.401125in}}%
\pgfpathlineto{\pgfqpoint{8.808656in}{4.401125in}}%
\pgfpathlineto{\pgfqpoint{8.810056in}{4.407559in}}%
\pgfpathlineto{\pgfqpoint{8.830847in}{4.407559in}}%
\pgfpathlineto{\pgfqpoint{8.830847in}{4.410777in}}%
\pgfpathlineto{\pgfqpoint{8.832195in}{4.410777in}}%
\pgfpathlineto{\pgfqpoint{8.838210in}{4.410777in}}%
\pgfpathlineto{\pgfqpoint{8.838210in}{4.413994in}}%
\pgfpathlineto{\pgfqpoint{8.850186in}{4.413994in}}%
\pgfpathlineto{\pgfqpoint{8.850186in}{4.417212in}}%
\pgfpathlineto{\pgfqpoint{8.853401in}{4.417212in}}%
\pgfpathlineto{\pgfqpoint{8.853401in}{4.420429in}}%
\pgfpathlineto{\pgfqpoint{8.858275in}{4.420429in}}%
\pgfpathlineto{\pgfqpoint{8.859467in}{4.430082in}}%
\pgfpathlineto{\pgfqpoint{8.860297in}{4.430082in}}%
\pgfpathlineto{\pgfqpoint{8.860297in}{4.433299in}}%
\pgfpathlineto{\pgfqpoint{8.868852in}{4.433299in}}%
\pgfpathlineto{\pgfqpoint{8.870096in}{4.439734in}}%
\pgfpathlineto{\pgfqpoint{8.882643in}{4.439734in}}%
\pgfpathlineto{\pgfqpoint{8.882643in}{4.442951in}}%
\pgfpathlineto{\pgfqpoint{8.897420in}{4.442951in}}%
\pgfpathlineto{\pgfqpoint{8.897420in}{4.446169in}}%
\pgfpathlineto{\pgfqpoint{8.899960in}{4.446169in}}%
\pgfpathlineto{\pgfqpoint{8.899960in}{4.449386in}}%
\pgfpathlineto{\pgfqpoint{8.913233in}{4.449386in}}%
\pgfpathlineto{\pgfqpoint{8.913233in}{4.452604in}}%
\pgfpathlineto{\pgfqpoint{8.924899in}{4.452604in}}%
\pgfpathlineto{\pgfqpoint{8.924899in}{4.455821in}}%
\pgfpathlineto{\pgfqpoint{8.926610in}{4.455821in}}%
\pgfpathlineto{\pgfqpoint{8.926610in}{4.459038in}}%
\pgfpathlineto{\pgfqpoint{8.936202in}{4.459038in}}%
\pgfpathlineto{\pgfqpoint{8.936202in}{4.462256in}}%
\pgfpathlineto{\pgfqpoint{8.942009in}{4.462256in}}%
\pgfpathlineto{\pgfqpoint{8.942631in}{4.468691in}}%
\pgfpathlineto{\pgfqpoint{8.944290in}{4.468691in}}%
\pgfpathlineto{\pgfqpoint{8.944290in}{4.471908in}}%
\pgfpathlineto{\pgfqpoint{8.984057in}{4.471908in}}%
\pgfpathlineto{\pgfqpoint{8.984057in}{4.475126in}}%
\pgfpathlineto{\pgfqpoint{8.987738in}{4.475126in}}%
\pgfpathlineto{\pgfqpoint{8.987738in}{4.478343in}}%
\pgfpathlineto{\pgfqpoint{8.988568in}{4.478343in}}%
\pgfpathlineto{\pgfqpoint{9.030565in}{4.478343in}}%
\pgfpathlineto{\pgfqpoint{9.030565in}{4.481561in}}%
\pgfpathlineto{\pgfqpoint{9.043993in}{4.481561in}}%
\pgfpathlineto{\pgfqpoint{9.043993in}{4.484778in}}%
\pgfpathlineto{\pgfqpoint{9.068310in}{4.484778in}}%
\pgfpathlineto{\pgfqpoint{9.068310in}{4.487995in}}%
\pgfpathlineto{\pgfqpoint{9.086975in}{4.487995in}}%
\pgfpathlineto{\pgfqpoint{9.086975in}{4.491213in}}%
\pgfpathlineto{\pgfqpoint{9.097396in}{4.491213in}}%
\pgfpathlineto{\pgfqpoint{9.097396in}{4.494430in}}%
\pgfpathlineto{\pgfqpoint{9.108855in}{4.494430in}}%
\pgfpathlineto{\pgfqpoint{9.108855in}{4.497648in}}%
\pgfpathlineto{\pgfqpoint{9.147689in}{4.497648in}}%
\pgfpathlineto{\pgfqpoint{9.147689in}{4.500865in}}%
\pgfpathlineto{\pgfqpoint{9.165058in}{4.500865in}}%
\pgfpathlineto{\pgfqpoint{9.165058in}{4.504083in}}%
\pgfpathlineto{\pgfqpoint{9.183308in}{4.504083in}}%
\pgfpathlineto{\pgfqpoint{9.183308in}{4.507300in}}%
\pgfpathlineto{\pgfqpoint{9.203321in}{4.507300in}}%
\pgfpathlineto{\pgfqpoint{9.203321in}{4.510518in}}%
\pgfpathlineto{\pgfqpoint{9.224682in}{4.510518in}}%
\pgfpathlineto{\pgfqpoint{9.224682in}{4.513735in}}%
\pgfpathlineto{\pgfqpoint{9.238474in}{4.513735in}}%
\pgfpathlineto{\pgfqpoint{9.238474in}{4.516952in}}%
\pgfpathlineto{\pgfqpoint{9.281196in}{4.516952in}}%
\pgfpathlineto{\pgfqpoint{9.281196in}{4.520170in}}%
\pgfpathlineto{\pgfqpoint{9.308676in}{4.520170in}}%
\pgfpathlineto{\pgfqpoint{9.308676in}{4.523387in}}%
\pgfpathlineto{\pgfqpoint{9.320652in}{4.523387in}}%
\pgfpathlineto{\pgfqpoint{9.320652in}{4.526605in}}%
\pgfpathlineto{\pgfqpoint{9.332214in}{4.526605in}}%
\pgfpathlineto{\pgfqpoint{9.332214in}{4.529822in}}%
\pgfpathlineto{\pgfqpoint{9.339940in}{4.529822in}}%
\pgfpathlineto{\pgfqpoint{9.339940in}{4.533040in}}%
\pgfpathlineto{\pgfqpoint{9.391321in}{4.533040in}}%
\pgfpathlineto{\pgfqpoint{9.391321in}{4.536257in}}%
\pgfpathlineto{\pgfqpoint{9.425696in}{4.536257in}}%
\pgfpathlineto{\pgfqpoint{9.425696in}{4.539474in}}%
\pgfpathlineto{\pgfqpoint{9.463959in}{4.539474in}}%
\pgfpathlineto{\pgfqpoint{9.463959in}{4.542692in}}%
\pgfpathlineto{\pgfqpoint{9.467485in}{4.542692in}}%
\pgfpathlineto{\pgfqpoint{9.467485in}{4.545909in}}%
\pgfpathlineto{\pgfqpoint{9.472929in}{4.545909in}}%
\pgfpathlineto{\pgfqpoint{9.472929in}{4.549127in}}%
\pgfpathlineto{\pgfqpoint{9.491283in}{4.549127in}}%
\pgfpathlineto{\pgfqpoint{9.491283in}{4.552344in}}%
\pgfpathlineto{\pgfqpoint{9.512282in}{4.552344in}}%
\pgfpathlineto{\pgfqpoint{9.512282in}{4.555562in}}%
\pgfpathlineto{\pgfqpoint{9.530999in}{4.555562in}}%
\pgfpathlineto{\pgfqpoint{9.530999in}{4.558779in}}%
\pgfpathlineto{\pgfqpoint{9.555678in}{4.558779in}}%
\pgfpathlineto{\pgfqpoint{9.555678in}{4.561997in}}%
\pgfpathlineto{\pgfqpoint{9.590572in}{4.561997in}}%
\pgfpathlineto{\pgfqpoint{9.590572in}{4.565214in}}%
\pgfpathlineto{\pgfqpoint{9.646515in}{4.565214in}}%
\pgfpathlineto{\pgfqpoint{9.646515in}{4.568431in}}%
\pgfpathlineto{\pgfqpoint{9.703548in}{4.568431in}}%
\pgfpathlineto{\pgfqpoint{9.703548in}{4.571649in}}%
\pgfpathlineto{\pgfqpoint{9.761150in}{4.571649in}}%
\pgfpathlineto{\pgfqpoint{9.761150in}{4.571649in}}%
\pgfusepath{stroke}%
\end{pgfscope}%
\begin{pgfscope}%
\pgfpathrectangle{\pgfqpoint{8.150541in}{3.102083in}}{\pgfqpoint{1.687305in}{1.539545in}}%
\pgfusepath{clip}%
\pgfsetrectcap%
\pgfsetroundjoin%
\pgfsetlinewidth{1.505625pt}%
\definecolor{currentstroke}{rgb}{0.501961,0.501961,0.501961}%
\pgfsetstrokecolor{currentstroke}%
\pgfsetdash{}{0pt}%
\pgfpathmoveto{\pgfqpoint{8.227236in}{3.172062in}}%
\pgfpathlineto{\pgfqpoint{9.761150in}{4.571649in}}%
\pgfusepath{stroke}%
\end{pgfscope}%
\begin{pgfscope}%
\pgfsetrectcap%
\pgfsetmiterjoin%
\pgfsetlinewidth{0.803000pt}%
\definecolor{currentstroke}{rgb}{0.000000,0.000000,0.000000}%
\pgfsetstrokecolor{currentstroke}%
\pgfsetdash{}{0pt}%
\pgfpathmoveto{\pgfqpoint{8.150541in}{3.102083in}}%
\pgfpathlineto{\pgfqpoint{8.150541in}{4.641628in}}%
\pgfusepath{stroke}%
\end{pgfscope}%
\begin{pgfscope}%
\pgfsetrectcap%
\pgfsetmiterjoin%
\pgfsetlinewidth{0.803000pt}%
\definecolor{currentstroke}{rgb}{0.000000,0.000000,0.000000}%
\pgfsetstrokecolor{currentstroke}%
\pgfsetdash{}{0pt}%
\pgfpathmoveto{\pgfqpoint{9.837846in}{3.102083in}}%
\pgfpathlineto{\pgfqpoint{9.837846in}{4.641628in}}%
\pgfusepath{stroke}%
\end{pgfscope}%
\begin{pgfscope}%
\pgfsetrectcap%
\pgfsetmiterjoin%
\pgfsetlinewidth{0.803000pt}%
\definecolor{currentstroke}{rgb}{0.000000,0.000000,0.000000}%
\pgfsetstrokecolor{currentstroke}%
\pgfsetdash{}{0pt}%
\pgfpathmoveto{\pgfqpoint{8.150541in}{3.102083in}}%
\pgfpathlineto{\pgfqpoint{9.837846in}{3.102083in}}%
\pgfusepath{stroke}%
\end{pgfscope}%
\begin{pgfscope}%
\pgfsetrectcap%
\pgfsetmiterjoin%
\pgfsetlinewidth{0.803000pt}%
\definecolor{currentstroke}{rgb}{0.000000,0.000000,0.000000}%
\pgfsetstrokecolor{currentstroke}%
\pgfsetdash{}{0pt}%
\pgfpathmoveto{\pgfqpoint{8.150541in}{4.641628in}}%
\pgfpathlineto{\pgfqpoint{9.837846in}{4.641628in}}%
\pgfusepath{stroke}%
\end{pgfscope}%
\begin{pgfscope}%
\definecolor{textcolor}{rgb}{0.000000,0.000000,0.000000}%
\pgfsetstrokecolor{textcolor}%
\pgfsetfillcolor{textcolor}%
\pgftext[x=8.994193in,y=4.724962in,,base]{\color{textcolor}\rmfamily\fontsize{20.000000}{24.000000}\selectfont Fibrosis}%
\end{pgfscope}%
\begin{pgfscope}%
\pgfsetbuttcap%
\pgfsetmiterjoin%
\definecolor{currentfill}{rgb}{1.000000,1.000000,1.000000}%
\pgfsetfillcolor{currentfill}%
\pgfsetfillopacity{0.800000}%
\pgfsetlinewidth{1.003750pt}%
\definecolor{currentstroke}{rgb}{0.800000,0.800000,0.800000}%
\pgfsetstrokecolor{currentstroke}%
\pgfsetstrokeopacity{0.800000}%
\pgfsetdash{}{0pt}%
\pgfpathmoveto{\pgfqpoint{8.628740in}{3.171527in}}%
\pgfpathlineto{\pgfqpoint{9.740624in}{3.171527in}}%
\pgfpathquadraticcurveto{\pgfqpoint{9.768402in}{3.171527in}}{\pgfqpoint{9.768402in}{3.199305in}}%
\pgfpathlineto{\pgfqpoint{9.768402in}{3.379089in}}%
\pgfpathquadraticcurveto{\pgfqpoint{9.768402in}{3.406867in}}{\pgfqpoint{9.740624in}{3.406867in}}%
\pgfpathlineto{\pgfqpoint{8.628740in}{3.406867in}}%
\pgfpathquadraticcurveto{\pgfqpoint{8.600962in}{3.406867in}}{\pgfqpoint{8.600962in}{3.379089in}}%
\pgfpathlineto{\pgfqpoint{8.600962in}{3.199305in}}%
\pgfpathquadraticcurveto{\pgfqpoint{8.600962in}{3.171527in}}{\pgfqpoint{8.628740in}{3.171527in}}%
\pgfpathclose%
\pgfusepath{stroke,fill}%
\end{pgfscope}%
\begin{pgfscope}%
\pgfsetrectcap%
\pgfsetroundjoin%
\pgfsetlinewidth{1.505625pt}%
\definecolor{currentstroke}{rgb}{0.000000,0.501961,0.000000}%
\pgfsetstrokecolor{currentstroke}%
\pgfsetdash{}{0pt}%
\pgfpathmoveto{\pgfqpoint{8.656518in}{3.302700in}}%
\pgfpathlineto{\pgfqpoint{8.934295in}{3.302700in}}%
\pgfusepath{stroke}%
\end{pgfscope}%
\begin{pgfscope}%
\definecolor{textcolor}{rgb}{0.000000,0.000000,0.000000}%
\pgfsetstrokecolor{textcolor}%
\pgfsetfillcolor{textcolor}%
\pgftext[x=9.045407in,y=3.254089in,left,base]{\color{textcolor}\rmfamily\fontsize{10.000000}{12.000000}\selectfont AUC 0.852}%
\end{pgfscope}%
\begin{pgfscope}%
\pgfsetbuttcap%
\pgfsetmiterjoin%
\definecolor{currentfill}{rgb}{1.000000,1.000000,1.000000}%
\pgfsetfillcolor{currentfill}%
\pgfsetlinewidth{0.000000pt}%
\definecolor{currentstroke}{rgb}{0.000000,0.000000,0.000000}%
\pgfsetstrokecolor{currentstroke}%
\pgfsetstrokeopacity{0.000000}%
\pgfsetdash{}{0pt}%
\pgfpathmoveto{\pgfqpoint{0.763041in}{0.639583in}}%
\pgfpathlineto{\pgfqpoint{2.450346in}{0.639583in}}%
\pgfpathlineto{\pgfqpoint{2.450346in}{2.179128in}}%
\pgfpathlineto{\pgfqpoint{0.763041in}{2.179128in}}%
\pgfpathclose%
\pgfusepath{fill}%
\end{pgfscope}%
\begin{pgfscope}%
\pgfsetbuttcap%
\pgfsetroundjoin%
\definecolor{currentfill}{rgb}{0.000000,0.000000,0.000000}%
\pgfsetfillcolor{currentfill}%
\pgfsetlinewidth{0.803000pt}%
\definecolor{currentstroke}{rgb}{0.000000,0.000000,0.000000}%
\pgfsetstrokecolor{currentstroke}%
\pgfsetdash{}{0pt}%
\pgfsys@defobject{currentmarker}{\pgfqpoint{0.000000in}{-0.048611in}}{\pgfqpoint{0.000000in}{0.000000in}}{%
\pgfpathmoveto{\pgfqpoint{0.000000in}{0.000000in}}%
\pgfpathlineto{\pgfqpoint{0.000000in}{-0.048611in}}%
\pgfusepath{stroke,fill}%
}%
\begin{pgfscope}%
\pgfsys@transformshift{0.839736in}{0.639583in}%
\pgfsys@useobject{currentmarker}{}%
\end{pgfscope}%
\end{pgfscope}%
\begin{pgfscope}%
\definecolor{textcolor}{rgb}{0.000000,0.000000,0.000000}%
\pgfsetstrokecolor{textcolor}%
\pgfsetfillcolor{textcolor}%
\pgftext[x=0.839736in,y=0.542361in,,top]{\color{textcolor}\rmfamily\fontsize{10.000000}{12.000000}\selectfont \(\displaystyle {0.0}\)}%
\end{pgfscope}%
\begin{pgfscope}%
\pgfsetbuttcap%
\pgfsetroundjoin%
\definecolor{currentfill}{rgb}{0.000000,0.000000,0.000000}%
\pgfsetfillcolor{currentfill}%
\pgfsetlinewidth{0.803000pt}%
\definecolor{currentstroke}{rgb}{0.000000,0.000000,0.000000}%
\pgfsetstrokecolor{currentstroke}%
\pgfsetdash{}{0pt}%
\pgfsys@defobject{currentmarker}{\pgfqpoint{0.000000in}{-0.048611in}}{\pgfqpoint{0.000000in}{0.000000in}}{%
\pgfpathmoveto{\pgfqpoint{0.000000in}{0.000000in}}%
\pgfpathlineto{\pgfqpoint{0.000000in}{-0.048611in}}%
\pgfusepath{stroke,fill}%
}%
\begin{pgfscope}%
\pgfsys@transformshift{1.606693in}{0.639583in}%
\pgfsys@useobject{currentmarker}{}%
\end{pgfscope}%
\end{pgfscope}%
\begin{pgfscope}%
\definecolor{textcolor}{rgb}{0.000000,0.000000,0.000000}%
\pgfsetstrokecolor{textcolor}%
\pgfsetfillcolor{textcolor}%
\pgftext[x=1.606693in,y=0.542361in,,top]{\color{textcolor}\rmfamily\fontsize{10.000000}{12.000000}\selectfont \(\displaystyle {0.5}\)}%
\end{pgfscope}%
\begin{pgfscope}%
\pgfsetbuttcap%
\pgfsetroundjoin%
\definecolor{currentfill}{rgb}{0.000000,0.000000,0.000000}%
\pgfsetfillcolor{currentfill}%
\pgfsetlinewidth{0.803000pt}%
\definecolor{currentstroke}{rgb}{0.000000,0.000000,0.000000}%
\pgfsetstrokecolor{currentstroke}%
\pgfsetdash{}{0pt}%
\pgfsys@defobject{currentmarker}{\pgfqpoint{0.000000in}{-0.048611in}}{\pgfqpoint{0.000000in}{0.000000in}}{%
\pgfpathmoveto{\pgfqpoint{0.000000in}{0.000000in}}%
\pgfpathlineto{\pgfqpoint{0.000000in}{-0.048611in}}%
\pgfusepath{stroke,fill}%
}%
\begin{pgfscope}%
\pgfsys@transformshift{2.373650in}{0.639583in}%
\pgfsys@useobject{currentmarker}{}%
\end{pgfscope}%
\end{pgfscope}%
\begin{pgfscope}%
\definecolor{textcolor}{rgb}{0.000000,0.000000,0.000000}%
\pgfsetstrokecolor{textcolor}%
\pgfsetfillcolor{textcolor}%
\pgftext[x=2.373650in,y=0.542361in,,top]{\color{textcolor}\rmfamily\fontsize{10.000000}{12.000000}\selectfont \(\displaystyle {1.0}\)}%
\end{pgfscope}%
\begin{pgfscope}%
\definecolor{textcolor}{rgb}{0.000000,0.000000,0.000000}%
\pgfsetstrokecolor{textcolor}%
\pgfsetfillcolor{textcolor}%
\pgftext[x=1.606693in,y=0.363349in,,top]{\color{textcolor}\rmfamily\fontsize{16.000000}{19.200000}\selectfont FPR}%
\end{pgfscope}%
\begin{pgfscope}%
\pgfsetbuttcap%
\pgfsetroundjoin%
\definecolor{currentfill}{rgb}{0.000000,0.000000,0.000000}%
\pgfsetfillcolor{currentfill}%
\pgfsetlinewidth{0.803000pt}%
\definecolor{currentstroke}{rgb}{0.000000,0.000000,0.000000}%
\pgfsetstrokecolor{currentstroke}%
\pgfsetdash{}{0pt}%
\pgfsys@defobject{currentmarker}{\pgfqpoint{-0.048611in}{0.000000in}}{\pgfqpoint{-0.000000in}{0.000000in}}{%
\pgfpathmoveto{\pgfqpoint{-0.000000in}{0.000000in}}%
\pgfpathlineto{\pgfqpoint{-0.048611in}{0.000000in}}%
\pgfusepath{stroke,fill}%
}%
\begin{pgfscope}%
\pgfsys@transformshift{0.763041in}{0.709562in}%
\pgfsys@useobject{currentmarker}{}%
\end{pgfscope}%
\end{pgfscope}%
\begin{pgfscope}%
\definecolor{textcolor}{rgb}{0.000000,0.000000,0.000000}%
\pgfsetstrokecolor{textcolor}%
\pgfsetfillcolor{textcolor}%
\pgftext[x=0.418904in, y=0.661337in, left, base]{\color{textcolor}\rmfamily\fontsize{10.000000}{12.000000}\selectfont \(\displaystyle {0.00}\)}%
\end{pgfscope}%
\begin{pgfscope}%
\pgfsetbuttcap%
\pgfsetroundjoin%
\definecolor{currentfill}{rgb}{0.000000,0.000000,0.000000}%
\pgfsetfillcolor{currentfill}%
\pgfsetlinewidth{0.803000pt}%
\definecolor{currentstroke}{rgb}{0.000000,0.000000,0.000000}%
\pgfsetstrokecolor{currentstroke}%
\pgfsetdash{}{0pt}%
\pgfsys@defobject{currentmarker}{\pgfqpoint{-0.048611in}{0.000000in}}{\pgfqpoint{-0.000000in}{0.000000in}}{%
\pgfpathmoveto{\pgfqpoint{-0.000000in}{0.000000in}}%
\pgfpathlineto{\pgfqpoint{-0.048611in}{0.000000in}}%
\pgfusepath{stroke,fill}%
}%
\begin{pgfscope}%
\pgfsys@transformshift{0.763041in}{1.059459in}%
\pgfsys@useobject{currentmarker}{}%
\end{pgfscope}%
\end{pgfscope}%
\begin{pgfscope}%
\definecolor{textcolor}{rgb}{0.000000,0.000000,0.000000}%
\pgfsetstrokecolor{textcolor}%
\pgfsetfillcolor{textcolor}%
\pgftext[x=0.418904in, y=1.011234in, left, base]{\color{textcolor}\rmfamily\fontsize{10.000000}{12.000000}\selectfont \(\displaystyle {0.25}\)}%
\end{pgfscope}%
\begin{pgfscope}%
\pgfsetbuttcap%
\pgfsetroundjoin%
\definecolor{currentfill}{rgb}{0.000000,0.000000,0.000000}%
\pgfsetfillcolor{currentfill}%
\pgfsetlinewidth{0.803000pt}%
\definecolor{currentstroke}{rgb}{0.000000,0.000000,0.000000}%
\pgfsetstrokecolor{currentstroke}%
\pgfsetdash{}{0pt}%
\pgfsys@defobject{currentmarker}{\pgfqpoint{-0.048611in}{0.000000in}}{\pgfqpoint{-0.000000in}{0.000000in}}{%
\pgfpathmoveto{\pgfqpoint{-0.000000in}{0.000000in}}%
\pgfpathlineto{\pgfqpoint{-0.048611in}{0.000000in}}%
\pgfusepath{stroke,fill}%
}%
\begin{pgfscope}%
\pgfsys@transformshift{0.763041in}{1.409356in}%
\pgfsys@useobject{currentmarker}{}%
\end{pgfscope}%
\end{pgfscope}%
\begin{pgfscope}%
\definecolor{textcolor}{rgb}{0.000000,0.000000,0.000000}%
\pgfsetstrokecolor{textcolor}%
\pgfsetfillcolor{textcolor}%
\pgftext[x=0.418904in, y=1.361130in, left, base]{\color{textcolor}\rmfamily\fontsize{10.000000}{12.000000}\selectfont \(\displaystyle {0.50}\)}%
\end{pgfscope}%
\begin{pgfscope}%
\pgfsetbuttcap%
\pgfsetroundjoin%
\definecolor{currentfill}{rgb}{0.000000,0.000000,0.000000}%
\pgfsetfillcolor{currentfill}%
\pgfsetlinewidth{0.803000pt}%
\definecolor{currentstroke}{rgb}{0.000000,0.000000,0.000000}%
\pgfsetstrokecolor{currentstroke}%
\pgfsetdash{}{0pt}%
\pgfsys@defobject{currentmarker}{\pgfqpoint{-0.048611in}{0.000000in}}{\pgfqpoint{-0.000000in}{0.000000in}}{%
\pgfpathmoveto{\pgfqpoint{-0.000000in}{0.000000in}}%
\pgfpathlineto{\pgfqpoint{-0.048611in}{0.000000in}}%
\pgfusepath{stroke,fill}%
}%
\begin{pgfscope}%
\pgfsys@transformshift{0.763041in}{1.759252in}%
\pgfsys@useobject{currentmarker}{}%
\end{pgfscope}%
\end{pgfscope}%
\begin{pgfscope}%
\definecolor{textcolor}{rgb}{0.000000,0.000000,0.000000}%
\pgfsetstrokecolor{textcolor}%
\pgfsetfillcolor{textcolor}%
\pgftext[x=0.418904in, y=1.711027in, left, base]{\color{textcolor}\rmfamily\fontsize{10.000000}{12.000000}\selectfont \(\displaystyle {0.75}\)}%
\end{pgfscope}%
\begin{pgfscope}%
\pgfsetbuttcap%
\pgfsetroundjoin%
\definecolor{currentfill}{rgb}{0.000000,0.000000,0.000000}%
\pgfsetfillcolor{currentfill}%
\pgfsetlinewidth{0.803000pt}%
\definecolor{currentstroke}{rgb}{0.000000,0.000000,0.000000}%
\pgfsetstrokecolor{currentstroke}%
\pgfsetdash{}{0pt}%
\pgfsys@defobject{currentmarker}{\pgfqpoint{-0.048611in}{0.000000in}}{\pgfqpoint{-0.000000in}{0.000000in}}{%
\pgfpathmoveto{\pgfqpoint{-0.000000in}{0.000000in}}%
\pgfpathlineto{\pgfqpoint{-0.048611in}{0.000000in}}%
\pgfusepath{stroke,fill}%
}%
\begin{pgfscope}%
\pgfsys@transformshift{0.763041in}{2.109149in}%
\pgfsys@useobject{currentmarker}{}%
\end{pgfscope}%
\end{pgfscope}%
\begin{pgfscope}%
\definecolor{textcolor}{rgb}{0.000000,0.000000,0.000000}%
\pgfsetstrokecolor{textcolor}%
\pgfsetfillcolor{textcolor}%
\pgftext[x=0.418904in, y=2.060924in, left, base]{\color{textcolor}\rmfamily\fontsize{10.000000}{12.000000}\selectfont \(\displaystyle {1.00}\)}%
\end{pgfscope}%
\begin{pgfscope}%
\definecolor{textcolor}{rgb}{0.000000,0.000000,0.000000}%
\pgfsetstrokecolor{textcolor}%
\pgfsetfillcolor{textcolor}%
\pgftext[x=0.363349in,y=1.409356in,,bottom,rotate=90.000000]{\color{textcolor}\rmfamily\fontsize{16.000000}{19.200000}\selectfont TPR}%
\end{pgfscope}%
\begin{pgfscope}%
\pgfpathrectangle{\pgfqpoint{0.763041in}{0.639583in}}{\pgfqpoint{1.687305in}{1.539545in}}%
\pgfusepath{clip}%
\pgfsetrectcap%
\pgfsetroundjoin%
\pgfsetlinewidth{1.505625pt}%
\definecolor{currentstroke}{rgb}{0.000000,0.501961,0.000000}%
\pgfsetstrokecolor{currentstroke}%
\pgfsetdash{}{0pt}%
\pgfpathmoveto{\pgfqpoint{0.839736in}{0.709562in}}%
\pgfpathlineto{\pgfqpoint{0.841160in}{0.732258in}}%
\pgfpathlineto{\pgfqpoint{0.841265in}{0.732258in}}%
\pgfpathlineto{\pgfqpoint{0.849121in}{0.824555in}}%
\pgfpathlineto{\pgfqpoint{0.849279in}{0.824555in}}%
\pgfpathlineto{\pgfqpoint{0.851335in}{0.856330in}}%
\pgfpathlineto{\pgfqpoint{0.851809in}{0.857843in}}%
\pgfpathlineto{\pgfqpoint{0.853128in}{0.880539in}}%
\pgfpathlineto{\pgfqpoint{0.853338in}{0.880539in}}%
\pgfpathlineto{\pgfqpoint{0.854762in}{0.882052in}}%
\pgfpathlineto{\pgfqpoint{0.855816in}{0.895670in}}%
\pgfpathlineto{\pgfqpoint{0.856502in}{0.897183in}}%
\pgfpathlineto{\pgfqpoint{0.858610in}{0.916852in}}%
\pgfpathlineto{\pgfqpoint{0.858874in}{0.918366in}}%
\pgfpathlineto{\pgfqpoint{0.858874in}{0.921392in}}%
\pgfpathlineto{\pgfqpoint{0.860930in}{0.922905in}}%
\pgfpathlineto{\pgfqpoint{0.862248in}{0.941062in}}%
\pgfpathlineto{\pgfqpoint{0.862459in}{0.941062in}}%
\pgfpathlineto{\pgfqpoint{0.862512in}{0.941062in}}%
\pgfpathlineto{\pgfqpoint{0.863935in}{0.957705in}}%
\pgfpathlineto{\pgfqpoint{0.864041in}{0.957705in}}%
\pgfpathlineto{\pgfqpoint{0.866097in}{0.974349in}}%
\pgfpathlineto{\pgfqpoint{0.866308in}{0.974349in}}%
\pgfpathlineto{\pgfqpoint{0.867468in}{0.983427in}}%
\pgfpathlineto{\pgfqpoint{0.868627in}{0.984940in}}%
\pgfpathlineto{\pgfqpoint{0.870104in}{0.992506in}}%
\pgfpathlineto{\pgfqpoint{0.870789in}{0.994019in}}%
\pgfpathlineto{\pgfqpoint{0.872318in}{1.006123in}}%
\pgfpathlineto{\pgfqpoint{0.873056in}{1.007636in}}%
\pgfpathlineto{\pgfqpoint{0.873636in}{1.012176in}}%
\pgfpathlineto{\pgfqpoint{0.874532in}{1.012176in}}%
\pgfpathlineto{\pgfqpoint{0.875112in}{1.013689in}}%
\pgfpathlineto{\pgfqpoint{0.876536in}{1.027306in}}%
\pgfpathlineto{\pgfqpoint{0.877168in}{1.028819in}}%
\pgfpathlineto{\pgfqpoint{0.878697in}{1.046976in}}%
\pgfpathlineto{\pgfqpoint{0.879224in}{1.048489in}}%
\pgfpathlineto{\pgfqpoint{0.880542in}{1.060594in}}%
\pgfpathlineto{\pgfqpoint{0.881386in}{1.062107in}}%
\pgfpathlineto{\pgfqpoint{0.882704in}{1.075724in}}%
\pgfpathlineto{\pgfqpoint{0.883337in}{1.077238in}}%
\pgfpathlineto{\pgfqpoint{0.884760in}{1.089342in}}%
\pgfpathlineto{\pgfqpoint{0.885709in}{1.090855in}}%
\pgfpathlineto{\pgfqpoint{0.886553in}{1.096907in}}%
\pgfpathlineto{\pgfqpoint{0.886658in}{1.096907in}}%
\pgfpathlineto{\pgfqpoint{0.887554in}{1.096907in}}%
\pgfpathlineto{\pgfqpoint{0.889083in}{1.109012in}}%
\pgfpathlineto{\pgfqpoint{0.889347in}{1.110525in}}%
\pgfpathlineto{\pgfqpoint{0.890770in}{1.128682in}}%
\pgfpathlineto{\pgfqpoint{0.891297in}{1.130195in}}%
\pgfpathlineto{\pgfqpoint{0.892774in}{1.149865in}}%
\pgfpathlineto{\pgfqpoint{0.893881in}{1.151378in}}%
\pgfpathlineto{\pgfqpoint{0.895410in}{1.157430in}}%
\pgfpathlineto{\pgfqpoint{0.895568in}{1.158943in}}%
\pgfpathlineto{\pgfqpoint{0.896675in}{1.171048in}}%
\pgfpathlineto{\pgfqpoint{0.898520in}{1.172561in}}%
\pgfpathlineto{\pgfqpoint{0.899838in}{1.178613in}}%
\pgfpathlineto{\pgfqpoint{0.900207in}{1.180126in}}%
\pgfpathlineto{\pgfqpoint{0.900840in}{1.186178in}}%
\pgfpathlineto{\pgfqpoint{0.902369in}{1.187691in}}%
\pgfpathlineto{\pgfqpoint{0.903581in}{1.202822in}}%
\pgfpathlineto{\pgfqpoint{0.904319in}{1.204335in}}%
\pgfpathlineto{\pgfqpoint{0.905163in}{1.208874in}}%
\pgfpathlineto{\pgfqpoint{0.906956in}{1.210387in}}%
\pgfpathlineto{\pgfqpoint{0.907957in}{1.217953in}}%
\pgfpathlineto{\pgfqpoint{0.908274in}{1.217953in}}%
\pgfpathlineto{\pgfqpoint{0.908853in}{1.219466in}}%
\pgfpathlineto{\pgfqpoint{0.909750in}{1.230057in}}%
\pgfpathlineto{\pgfqpoint{0.912702in}{1.231570in}}%
\pgfpathlineto{\pgfqpoint{0.914073in}{1.239136in}}%
\pgfpathlineto{\pgfqpoint{0.914705in}{1.239136in}}%
\pgfpathlineto{\pgfqpoint{0.916129in}{1.249727in}}%
\pgfpathlineto{\pgfqpoint{0.916445in}{1.249727in}}%
\pgfpathlineto{\pgfqpoint{0.917869in}{1.255779in}}%
\pgfpathlineto{\pgfqpoint{0.919292in}{1.257292in}}%
\pgfpathlineto{\pgfqpoint{0.920768in}{1.264858in}}%
\pgfpathlineto{\pgfqpoint{0.921137in}{1.264858in}}%
\pgfpathlineto{\pgfqpoint{0.922614in}{1.273936in}}%
\pgfpathlineto{\pgfqpoint{0.923563in}{1.275449in}}%
\pgfpathlineto{\pgfqpoint{0.924722in}{1.283015in}}%
\pgfpathlineto{\pgfqpoint{0.927728in}{1.284528in}}%
\pgfpathlineto{\pgfqpoint{0.927780in}{1.287554in}}%
\pgfpathlineto{\pgfqpoint{0.929151in}{1.287554in}}%
\pgfpathlineto{\pgfqpoint{0.929626in}{1.289067in}}%
\pgfpathlineto{\pgfqpoint{0.930996in}{1.299658in}}%
\pgfpathlineto{\pgfqpoint{0.932156in}{1.301171in}}%
\pgfpathlineto{\pgfqpoint{0.933474in}{1.308737in}}%
\pgfpathlineto{\pgfqpoint{0.934212in}{1.310250in}}%
\pgfpathlineto{\pgfqpoint{0.934845in}{1.313276in}}%
\pgfpathlineto{\pgfqpoint{0.935952in}{1.313276in}}%
\pgfpathlineto{\pgfqpoint{0.937481in}{1.320841in}}%
\pgfpathlineto{\pgfqpoint{0.937955in}{1.322354in}}%
\pgfpathlineto{\pgfqpoint{0.939326in}{1.337485in}}%
\pgfpathlineto{\pgfqpoint{0.940117in}{1.338998in}}%
\pgfpathlineto{\pgfqpoint{0.941013in}{1.345050in}}%
\pgfpathlineto{\pgfqpoint{0.942489in}{1.346563in}}%
\pgfpathlineto{\pgfqpoint{0.943807in}{1.358668in}}%
\pgfpathlineto{\pgfqpoint{0.944440in}{1.360181in}}%
\pgfpathlineto{\pgfqpoint{0.945864in}{1.372285in}}%
\pgfpathlineto{\pgfqpoint{0.946602in}{1.373799in}}%
\pgfpathlineto{\pgfqpoint{0.947814in}{1.379851in}}%
\pgfpathlineto{\pgfqpoint{0.948763in}{1.381364in}}%
\pgfpathlineto{\pgfqpoint{0.949976in}{1.388929in}}%
\pgfpathlineto{\pgfqpoint{0.950608in}{1.390442in}}%
\pgfpathlineto{\pgfqpoint{0.951241in}{1.391955in}}%
\pgfpathlineto{\pgfqpoint{0.951347in}{1.391955in}}%
\pgfpathlineto{\pgfqpoint{0.952875in}{1.393468in}}%
\pgfpathlineto{\pgfqpoint{0.954404in}{1.401034in}}%
\pgfpathlineto{\pgfqpoint{0.955090in}{1.402547in}}%
\pgfpathlineto{\pgfqpoint{0.956250in}{1.411625in}}%
\pgfpathlineto{\pgfqpoint{0.957726in}{1.413138in}}%
\pgfpathlineto{\pgfqpoint{0.959149in}{1.423730in}}%
\pgfpathlineto{\pgfqpoint{0.959518in}{1.425243in}}%
\pgfpathlineto{\pgfqpoint{0.960678in}{1.432808in}}%
\pgfpathlineto{\pgfqpoint{0.961258in}{1.434321in}}%
\pgfpathlineto{\pgfqpoint{0.961680in}{1.438860in}}%
\pgfpathlineto{\pgfqpoint{0.963261in}{1.438860in}}%
\pgfpathlineto{\pgfqpoint{0.963367in}{1.443400in}}%
\pgfpathlineto{\pgfqpoint{0.965212in}{1.444913in}}%
\pgfpathlineto{\pgfqpoint{0.966425in}{1.450965in}}%
\pgfpathlineto{\pgfqpoint{0.967848in}{1.452478in}}%
\pgfpathlineto{\pgfqpoint{0.969061in}{1.461556in}}%
\pgfpathlineto{\pgfqpoint{0.970326in}{1.461556in}}%
\pgfpathlineto{\pgfqpoint{0.971433in}{1.469122in}}%
\pgfpathlineto{\pgfqpoint{0.972171in}{1.470635in}}%
\pgfpathlineto{\pgfqpoint{0.973173in}{1.479713in}}%
\pgfpathlineto{\pgfqpoint{0.975440in}{1.481226in}}%
\pgfpathlineto{\pgfqpoint{0.976863in}{1.487279in}}%
\pgfpathlineto{\pgfqpoint{0.977180in}{1.488792in}}%
\pgfpathlineto{\pgfqpoint{0.977918in}{1.494844in}}%
\pgfpathlineto{\pgfqpoint{0.978287in}{1.494844in}}%
\pgfpathlineto{\pgfqpoint{0.979499in}{1.496357in}}%
\pgfpathlineto{\pgfqpoint{0.981028in}{1.502409in}}%
\pgfpathlineto{\pgfqpoint{0.984930in}{1.503922in}}%
\pgfpathlineto{\pgfqpoint{0.986037in}{1.506948in}}%
\pgfpathlineto{\pgfqpoint{0.987197in}{1.508461in}}%
\pgfpathlineto{\pgfqpoint{0.987988in}{1.520566in}}%
\pgfpathlineto{\pgfqpoint{0.989675in}{1.522079in}}%
\pgfpathlineto{\pgfqpoint{0.991203in}{1.528131in}}%
\pgfpathlineto{\pgfqpoint{0.991362in}{1.528131in}}%
\pgfpathlineto{\pgfqpoint{0.991362in}{1.531157in}}%
\pgfpathlineto{\pgfqpoint{0.994103in}{1.532671in}}%
\pgfpathlineto{\pgfqpoint{0.994947in}{1.538723in}}%
\pgfpathlineto{\pgfqpoint{0.996212in}{1.540236in}}%
\pgfpathlineto{\pgfqpoint{0.997635in}{1.549314in}}%
\pgfpathlineto{\pgfqpoint{1.000008in}{1.550827in}}%
\pgfpathlineto{\pgfqpoint{1.000008in}{1.552340in}}%
\pgfpathlineto{\pgfqpoint{1.001642in}{1.552340in}}%
\pgfpathlineto{\pgfqpoint{1.002433in}{1.556880in}}%
\pgfpathlineto{\pgfqpoint{1.006334in}{1.558393in}}%
\pgfpathlineto{\pgfqpoint{1.007231in}{1.561419in}}%
\pgfpathlineto{\pgfqpoint{1.008390in}{1.561419in}}%
\pgfpathlineto{\pgfqpoint{1.009867in}{1.567471in}}%
\pgfpathlineto{\pgfqpoint{1.012661in}{1.568984in}}%
\pgfpathlineto{\pgfqpoint{1.013926in}{1.575036in}}%
\pgfpathlineto{\pgfqpoint{1.015982in}{1.576549in}}%
\pgfpathlineto{\pgfqpoint{1.017353in}{1.579576in}}%
\pgfpathlineto{\pgfqpoint{1.018935in}{1.581089in}}%
\pgfpathlineto{\pgfqpoint{1.020305in}{1.587141in}}%
\pgfpathlineto{\pgfqpoint{1.022045in}{1.588654in}}%
\pgfpathlineto{\pgfqpoint{1.023258in}{1.596219in}}%
\pgfpathlineto{\pgfqpoint{1.023785in}{1.596219in}}%
\pgfpathlineto{\pgfqpoint{1.025208in}{1.602272in}}%
\pgfpathlineto{\pgfqpoint{1.027001in}{1.603785in}}%
\pgfpathlineto{\pgfqpoint{1.027159in}{1.606811in}}%
\pgfpathlineto{\pgfqpoint{1.028899in}{1.606811in}}%
\pgfpathlineto{\pgfqpoint{1.030006in}{1.611350in}}%
\pgfpathlineto{\pgfqpoint{1.031377in}{1.612863in}}%
\pgfpathlineto{\pgfqpoint{1.032484in}{1.617402in}}%
\pgfpathlineto{\pgfqpoint{1.033433in}{1.618915in}}%
\pgfpathlineto{\pgfqpoint{1.034962in}{1.629507in}}%
\pgfpathlineto{\pgfqpoint{1.035753in}{1.631020in}}%
\pgfpathlineto{\pgfqpoint{1.037071in}{1.637072in}}%
\pgfpathlineto{\pgfqpoint{1.038125in}{1.638585in}}%
\pgfpathlineto{\pgfqpoint{1.039549in}{1.644637in}}%
\pgfpathlineto{\pgfqpoint{1.041183in}{1.646151in}}%
\pgfpathlineto{\pgfqpoint{1.042659in}{1.650690in}}%
\pgfpathlineto{\pgfqpoint{1.045242in}{1.652203in}}%
\pgfpathlineto{\pgfqpoint{1.046244in}{1.655229in}}%
\pgfpathlineto{\pgfqpoint{1.047931in}{1.656742in}}%
\pgfpathlineto{\pgfqpoint{1.048986in}{1.662794in}}%
\pgfpathlineto{\pgfqpoint{1.050462in}{1.664307in}}%
\pgfpathlineto{\pgfqpoint{1.051569in}{1.671873in}}%
\pgfpathlineto{\pgfqpoint{1.052834in}{1.673386in}}%
\pgfpathlineto{\pgfqpoint{1.054152in}{1.677925in}}%
\pgfpathlineto{\pgfqpoint{1.054574in}{1.679438in}}%
\pgfpathlineto{\pgfqpoint{1.055945in}{1.685490in}}%
\pgfpathlineto{\pgfqpoint{1.056736in}{1.687003in}}%
\pgfpathlineto{\pgfqpoint{1.058001in}{1.690029in}}%
\pgfpathlineto{\pgfqpoint{1.059266in}{1.691543in}}%
\pgfpathlineto{\pgfqpoint{1.059793in}{1.696082in}}%
\pgfpathlineto{\pgfqpoint{1.062060in}{1.697595in}}%
\pgfpathlineto{\pgfqpoint{1.062957in}{1.702134in}}%
\pgfpathlineto{\pgfqpoint{1.064538in}{1.703647in}}%
\pgfpathlineto{\pgfqpoint{1.065065in}{1.706673in}}%
\pgfpathlineto{\pgfqpoint{1.068703in}{1.708186in}}%
\pgfpathlineto{\pgfqpoint{1.069230in}{1.711212in}}%
\pgfpathlineto{\pgfqpoint{1.073026in}{1.712725in}}%
\pgfpathlineto{\pgfqpoint{1.073501in}{1.717265in}}%
\pgfpathlineto{\pgfqpoint{1.075979in}{1.718778in}}%
\pgfpathlineto{\pgfqpoint{1.077349in}{1.726343in}}%
\pgfpathlineto{\pgfqpoint{1.079458in}{1.727856in}}%
\pgfpathlineto{\pgfqpoint{1.079458in}{1.729369in}}%
\pgfpathlineto{\pgfqpoint{1.082042in}{1.730882in}}%
\pgfpathlineto{\pgfqpoint{1.083412in}{1.735421in}}%
\pgfpathlineto{\pgfqpoint{1.084888in}{1.736935in}}%
\pgfpathlineto{\pgfqpoint{1.086207in}{1.739961in}}%
\pgfpathlineto{\pgfqpoint{1.088052in}{1.741474in}}%
\pgfpathlineto{\pgfqpoint{1.089159in}{1.746013in}}%
\pgfpathlineto{\pgfqpoint{1.091953in}{1.747526in}}%
\pgfpathlineto{\pgfqpoint{1.092955in}{1.753578in}}%
\pgfpathlineto{\pgfqpoint{1.098069in}{1.755091in}}%
\pgfpathlineto{\pgfqpoint{1.098596in}{1.758117in}}%
\pgfpathlineto{\pgfqpoint{1.105291in}{1.759631in}}%
\pgfpathlineto{\pgfqpoint{1.106610in}{1.767196in}}%
\pgfpathlineto{\pgfqpoint{1.107084in}{1.768709in}}%
\pgfpathlineto{\pgfqpoint{1.107611in}{1.771735in}}%
\pgfpathlineto{\pgfqpoint{1.109035in}{1.771735in}}%
\pgfpathlineto{\pgfqpoint{1.109667in}{1.776274in}}%
\pgfpathlineto{\pgfqpoint{1.111302in}{1.777787in}}%
\pgfpathlineto{\pgfqpoint{1.111302in}{1.779300in}}%
\pgfpathlineto{\pgfqpoint{1.114939in}{1.780813in}}%
\pgfpathlineto{\pgfqpoint{1.116363in}{1.783840in}}%
\pgfpathlineto{\pgfqpoint{1.117997in}{1.785353in}}%
\pgfpathlineto{\pgfqpoint{1.118103in}{1.788379in}}%
\pgfpathlineto{\pgfqpoint{1.118999in}{1.788379in}}%
\pgfpathlineto{\pgfqpoint{1.120581in}{1.789892in}}%
\pgfpathlineto{\pgfqpoint{1.121424in}{1.794431in}}%
\pgfpathlineto{\pgfqpoint{1.123691in}{1.795944in}}%
\pgfpathlineto{\pgfqpoint{1.125220in}{1.800483in}}%
\pgfpathlineto{\pgfqpoint{1.126327in}{1.801996in}}%
\pgfpathlineto{\pgfqpoint{1.127856in}{1.809562in}}%
\pgfpathlineto{\pgfqpoint{1.129332in}{1.811075in}}%
\pgfpathlineto{\pgfqpoint{1.129332in}{1.812588in}}%
\pgfpathlineto{\pgfqpoint{1.133603in}{1.814101in}}%
\pgfpathlineto{\pgfqpoint{1.134868in}{1.820153in}}%
\pgfpathlineto{\pgfqpoint{1.137557in}{1.821666in}}%
\pgfpathlineto{\pgfqpoint{1.138295in}{1.824692in}}%
\pgfpathlineto{\pgfqpoint{1.140351in}{1.826205in}}%
\pgfpathlineto{\pgfqpoint{1.141142in}{1.833771in}}%
\pgfpathlineto{\pgfqpoint{1.142091in}{1.835284in}}%
\pgfpathlineto{\pgfqpoint{1.143409in}{1.839823in}}%
\pgfpathlineto{\pgfqpoint{1.144305in}{1.841336in}}%
\pgfpathlineto{\pgfqpoint{1.144305in}{1.842849in}}%
\pgfpathlineto{\pgfqpoint{1.147732in}{1.844362in}}%
\pgfpathlineto{\pgfqpoint{1.147732in}{1.845875in}}%
\pgfpathlineto{\pgfqpoint{1.150737in}{1.847388in}}%
\pgfpathlineto{\pgfqpoint{1.150737in}{1.848901in}}%
\pgfpathlineto{\pgfqpoint{1.155587in}{1.850415in}}%
\pgfpathlineto{\pgfqpoint{1.156905in}{1.853441in}}%
\pgfpathlineto{\pgfqpoint{1.160068in}{1.854954in}}%
\pgfpathlineto{\pgfqpoint{1.161334in}{1.857980in}}%
\pgfpathlineto{\pgfqpoint{1.164972in}{1.859493in}}%
\pgfpathlineto{\pgfqpoint{1.165710in}{1.864032in}}%
\pgfpathlineto{\pgfqpoint{1.169927in}{1.865545in}}%
\pgfpathlineto{\pgfqpoint{1.170349in}{1.868571in}}%
\pgfpathlineto{\pgfqpoint{1.173301in}{1.870084in}}%
\pgfpathlineto{\pgfqpoint{1.174303in}{1.874624in}}%
\pgfpathlineto{\pgfqpoint{1.179892in}{1.876137in}}%
\pgfpathlineto{\pgfqpoint{1.179892in}{1.877650in}}%
\pgfpathlineto{\pgfqpoint{1.183477in}{1.879163in}}%
\pgfpathlineto{\pgfqpoint{1.184847in}{1.882189in}}%
\pgfpathlineto{\pgfqpoint{1.190436in}{1.883702in}}%
\pgfpathlineto{\pgfqpoint{1.190699in}{1.886728in}}%
\pgfpathlineto{\pgfqpoint{1.192914in}{1.888241in}}%
\pgfpathlineto{\pgfqpoint{1.193546in}{1.891267in}}%
\pgfpathlineto{\pgfqpoint{1.196235in}{1.892780in}}%
\pgfpathlineto{\pgfqpoint{1.197342in}{1.897320in}}%
\pgfpathlineto{\pgfqpoint{1.199187in}{1.898833in}}%
\pgfpathlineto{\pgfqpoint{1.199346in}{1.904885in}}%
\pgfpathlineto{\pgfqpoint{1.202351in}{1.906398in}}%
\pgfpathlineto{\pgfqpoint{1.203141in}{1.913963in}}%
\pgfpathlineto{\pgfqpoint{1.205830in}{1.915476in}}%
\pgfpathlineto{\pgfqpoint{1.206674in}{1.920016in}}%
\pgfpathlineto{\pgfqpoint{1.211630in}{1.921529in}}%
\pgfpathlineto{\pgfqpoint{1.211893in}{1.924555in}}%
\pgfpathlineto{\pgfqpoint{1.214635in}{1.926068in}}%
\pgfpathlineto{\pgfqpoint{1.214635in}{1.927581in}}%
\pgfpathlineto{\pgfqpoint{1.217112in}{1.929094in}}%
\pgfpathlineto{\pgfqpoint{1.217112in}{1.930607in}}%
\pgfpathlineto{\pgfqpoint{1.221541in}{1.932120in}}%
\pgfpathlineto{\pgfqpoint{1.222174in}{1.935146in}}%
\pgfpathlineto{\pgfqpoint{1.225179in}{1.936659in}}%
\pgfpathlineto{\pgfqpoint{1.225179in}{1.938172in}}%
\pgfpathlineto{\pgfqpoint{1.230767in}{1.939685in}}%
\pgfpathlineto{\pgfqpoint{1.230767in}{1.941198in}}%
\pgfpathlineto{\pgfqpoint{1.233140in}{1.942712in}}%
\pgfpathlineto{\pgfqpoint{1.233140in}{1.944225in}}%
\pgfpathlineto{\pgfqpoint{1.240204in}{1.945738in}}%
\pgfpathlineto{\pgfqpoint{1.240837in}{1.948764in}}%
\pgfpathlineto{\pgfqpoint{1.259500in}{1.950277in}}%
\pgfpathlineto{\pgfqpoint{1.259500in}{1.951790in}}%
\pgfpathlineto{\pgfqpoint{1.270097in}{1.953303in}}%
\pgfpathlineto{\pgfqpoint{1.271362in}{1.956329in}}%
\pgfpathlineto{\pgfqpoint{1.272575in}{1.957842in}}%
\pgfpathlineto{\pgfqpoint{1.272575in}{1.959355in}}%
\pgfpathlineto{\pgfqpoint{1.283224in}{1.960868in}}%
\pgfpathlineto{\pgfqpoint{1.283224in}{1.962381in}}%
\pgfpathlineto{\pgfqpoint{1.287126in}{1.963894in}}%
\pgfpathlineto{\pgfqpoint{1.287126in}{1.965408in}}%
\pgfpathlineto{\pgfqpoint{1.294559in}{1.966921in}}%
\pgfpathlineto{\pgfqpoint{1.294559in}{1.968434in}}%
\pgfpathlineto{\pgfqpoint{1.296985in}{1.969947in}}%
\pgfpathlineto{\pgfqpoint{1.296985in}{1.971460in}}%
\pgfpathlineto{\pgfqpoint{1.302257in}{1.972973in}}%
\pgfpathlineto{\pgfqpoint{1.302257in}{1.974486in}}%
\pgfpathlineto{\pgfqpoint{1.308689in}{1.975999in}}%
\pgfpathlineto{\pgfqpoint{1.309849in}{1.980538in}}%
\pgfpathlineto{\pgfqpoint{1.311588in}{1.982051in}}%
\pgfpathlineto{\pgfqpoint{1.312643in}{1.986590in}}%
\pgfpathlineto{\pgfqpoint{1.316702in}{1.988104in}}%
\pgfpathlineto{\pgfqpoint{1.317440in}{1.992643in}}%
\pgfpathlineto{\pgfqpoint{1.320235in}{1.994156in}}%
\pgfpathlineto{\pgfqpoint{1.320762in}{1.997182in}}%
\pgfpathlineto{\pgfqpoint{1.325085in}{1.998695in}}%
\pgfpathlineto{\pgfqpoint{1.325085in}{2.000208in}}%
\pgfpathlineto{\pgfqpoint{1.331991in}{2.001721in}}%
\pgfpathlineto{\pgfqpoint{1.331991in}{2.003234in}}%
\pgfpathlineto{\pgfqpoint{1.339847in}{2.004747in}}%
\pgfpathlineto{\pgfqpoint{1.341323in}{2.007773in}}%
\pgfpathlineto{\pgfqpoint{1.342799in}{2.009286in}}%
\pgfpathlineto{\pgfqpoint{1.342799in}{2.010800in}}%
\pgfpathlineto{\pgfqpoint{1.349284in}{2.012313in}}%
\pgfpathlineto{\pgfqpoint{1.349284in}{2.013826in}}%
\pgfpathlineto{\pgfqpoint{1.353080in}{2.015339in}}%
\pgfpathlineto{\pgfqpoint{1.353185in}{2.018365in}}%
\pgfpathlineto{\pgfqpoint{1.369898in}{2.019878in}}%
\pgfpathlineto{\pgfqpoint{1.369898in}{2.021391in}}%
\pgfpathlineto{\pgfqpoint{1.380811in}{2.022904in}}%
\pgfpathlineto{\pgfqpoint{1.380811in}{2.024417in}}%
\pgfpathlineto{\pgfqpoint{1.383710in}{2.025930in}}%
\pgfpathlineto{\pgfqpoint{1.383710in}{2.027443in}}%
\pgfpathlineto{\pgfqpoint{1.398420in}{2.028956in}}%
\pgfpathlineto{\pgfqpoint{1.398420in}{2.030469in}}%
\pgfpathlineto{\pgfqpoint{1.406011in}{2.031982in}}%
\pgfpathlineto{\pgfqpoint{1.406064in}{2.035009in}}%
\pgfpathlineto{\pgfqpoint{1.433110in}{2.036522in}}%
\pgfpathlineto{\pgfqpoint{1.433110in}{2.038035in}}%
\pgfpathlineto{\pgfqpoint{1.455095in}{2.039548in}}%
\pgfpathlineto{\pgfqpoint{1.456518in}{2.042574in}}%
\pgfpathlineto{\pgfqpoint{1.469224in}{2.044087in}}%
\pgfpathlineto{\pgfqpoint{1.469224in}{2.045600in}}%
\pgfpathlineto{\pgfqpoint{1.485567in}{2.047113in}}%
\pgfpathlineto{\pgfqpoint{1.485567in}{2.048626in}}%
\pgfpathlineto{\pgfqpoint{1.491419in}{2.050139in}}%
\pgfpathlineto{\pgfqpoint{1.491419in}{2.051652in}}%
\pgfpathlineto{\pgfqpoint{1.527059in}{2.053165in}}%
\pgfpathlineto{\pgfqpoint{1.527428in}{2.056192in}}%
\pgfpathlineto{\pgfqpoint{1.537919in}{2.057705in}}%
\pgfpathlineto{\pgfqpoint{1.538341in}{2.060731in}}%
\pgfpathlineto{\pgfqpoint{1.552101in}{2.062244in}}%
\pgfpathlineto{\pgfqpoint{1.552101in}{2.063757in}}%
\pgfpathlineto{\pgfqpoint{1.557215in}{2.065270in}}%
\pgfpathlineto{\pgfqpoint{1.557215in}{2.066783in}}%
\pgfpathlineto{\pgfqpoint{1.567865in}{2.068296in}}%
\pgfpathlineto{\pgfqpoint{1.567865in}{2.069809in}}%
\pgfpathlineto{\pgfqpoint{1.611728in}{2.071322in}}%
\pgfpathlineto{\pgfqpoint{1.611728in}{2.072835in}}%
\pgfpathlineto{\pgfqpoint{1.628546in}{2.074348in}}%
\pgfpathlineto{\pgfqpoint{1.628546in}{2.075861in}}%
\pgfpathlineto{\pgfqpoint{1.675731in}{2.077374in}}%
\pgfpathlineto{\pgfqpoint{1.675731in}{2.078888in}}%
\pgfpathlineto{\pgfqpoint{1.707153in}{2.080401in}}%
\pgfpathlineto{\pgfqpoint{1.707153in}{2.081914in}}%
\pgfpathlineto{\pgfqpoint{1.749699in}{2.083427in}}%
\pgfpathlineto{\pgfqpoint{1.749699in}{2.084940in}}%
\pgfpathlineto{\pgfqpoint{1.832945in}{2.086453in}}%
\pgfpathlineto{\pgfqpoint{1.832945in}{2.087966in}}%
\pgfpathlineto{\pgfqpoint{1.834000in}{2.087966in}}%
\pgfpathlineto{\pgfqpoint{1.889515in}{2.089479in}}%
\pgfpathlineto{\pgfqpoint{1.889515in}{2.090992in}}%
\pgfpathlineto{\pgfqpoint{1.914926in}{2.092505in}}%
\pgfpathlineto{\pgfqpoint{1.914926in}{2.094018in}}%
\pgfpathlineto{\pgfqpoint{1.940443in}{2.095531in}}%
\pgfpathlineto{\pgfqpoint{1.940443in}{2.097044in}}%
\pgfpathlineto{\pgfqpoint{1.971496in}{2.098557in}}%
\pgfpathlineto{\pgfqpoint{1.971496in}{2.100070in}}%
\pgfpathlineto{\pgfqpoint{2.003866in}{2.101584in}}%
\pgfpathlineto{\pgfqpoint{2.003866in}{2.103097in}}%
\pgfpathlineto{\pgfqpoint{2.101242in}{2.104610in}}%
\pgfpathlineto{\pgfqpoint{2.101242in}{2.106123in}}%
\pgfpathlineto{\pgfqpoint{2.269158in}{2.107636in}}%
\pgfpathlineto{\pgfqpoint{2.269158in}{2.109149in}}%
\pgfpathlineto{\pgfqpoint{2.373650in}{2.109149in}}%
\pgfpathlineto{\pgfqpoint{2.373650in}{2.109149in}}%
\pgfusepath{stroke}%
\end{pgfscope}%
\begin{pgfscope}%
\pgfpathrectangle{\pgfqpoint{0.763041in}{0.639583in}}{\pgfqpoint{1.687305in}{1.539545in}}%
\pgfusepath{clip}%
\pgfsetrectcap%
\pgfsetroundjoin%
\pgfsetlinewidth{1.505625pt}%
\definecolor{currentstroke}{rgb}{0.501961,0.501961,0.501961}%
\pgfsetstrokecolor{currentstroke}%
\pgfsetdash{}{0pt}%
\pgfpathmoveto{\pgfqpoint{0.839736in}{0.709562in}}%
\pgfpathlineto{\pgfqpoint{2.373650in}{2.109149in}}%
\pgfusepath{stroke}%
\end{pgfscope}%
\begin{pgfscope}%
\pgfsetrectcap%
\pgfsetmiterjoin%
\pgfsetlinewidth{0.803000pt}%
\definecolor{currentstroke}{rgb}{0.000000,0.000000,0.000000}%
\pgfsetstrokecolor{currentstroke}%
\pgfsetdash{}{0pt}%
\pgfpathmoveto{\pgfqpoint{0.763041in}{0.639583in}}%
\pgfpathlineto{\pgfqpoint{0.763041in}{2.179128in}}%
\pgfusepath{stroke}%
\end{pgfscope}%
\begin{pgfscope}%
\pgfsetrectcap%
\pgfsetmiterjoin%
\pgfsetlinewidth{0.803000pt}%
\definecolor{currentstroke}{rgb}{0.000000,0.000000,0.000000}%
\pgfsetstrokecolor{currentstroke}%
\pgfsetdash{}{0pt}%
\pgfpathmoveto{\pgfqpoint{2.450346in}{0.639583in}}%
\pgfpathlineto{\pgfqpoint{2.450346in}{2.179128in}}%
\pgfusepath{stroke}%
\end{pgfscope}%
\begin{pgfscope}%
\pgfsetrectcap%
\pgfsetmiterjoin%
\pgfsetlinewidth{0.803000pt}%
\definecolor{currentstroke}{rgb}{0.000000,0.000000,0.000000}%
\pgfsetstrokecolor{currentstroke}%
\pgfsetdash{}{0pt}%
\pgfpathmoveto{\pgfqpoint{0.763041in}{0.639583in}}%
\pgfpathlineto{\pgfqpoint{2.450346in}{0.639583in}}%
\pgfusepath{stroke}%
\end{pgfscope}%
\begin{pgfscope}%
\pgfsetrectcap%
\pgfsetmiterjoin%
\pgfsetlinewidth{0.803000pt}%
\definecolor{currentstroke}{rgb}{0.000000,0.000000,0.000000}%
\pgfsetstrokecolor{currentstroke}%
\pgfsetdash{}{0pt}%
\pgfpathmoveto{\pgfqpoint{0.763041in}{2.179128in}}%
\pgfpathlineto{\pgfqpoint{2.450346in}{2.179128in}}%
\pgfusepath{stroke}%
\end{pgfscope}%
\begin{pgfscope}%
\definecolor{textcolor}{rgb}{0.000000,0.000000,0.000000}%
\pgfsetstrokecolor{textcolor}%
\pgfsetfillcolor{textcolor}%
\pgftext[x=1.606693in,y=2.262462in,,base]{\color{textcolor}\rmfamily\fontsize{20.000000}{24.000000}\selectfont Edema}%
\end{pgfscope}%
\begin{pgfscope}%
\pgfsetbuttcap%
\pgfsetmiterjoin%
\definecolor{currentfill}{rgb}{1.000000,1.000000,1.000000}%
\pgfsetfillcolor{currentfill}%
\pgfsetfillopacity{0.800000}%
\pgfsetlinewidth{1.003750pt}%
\definecolor{currentstroke}{rgb}{0.800000,0.800000,0.800000}%
\pgfsetstrokecolor{currentstroke}%
\pgfsetstrokeopacity{0.800000}%
\pgfsetdash{}{0pt}%
\pgfpathmoveto{\pgfqpoint{1.241240in}{0.709028in}}%
\pgfpathlineto{\pgfqpoint{2.353124in}{0.709028in}}%
\pgfpathquadraticcurveto{\pgfqpoint{2.380902in}{0.709028in}}{\pgfqpoint{2.380902in}{0.736805in}}%
\pgfpathlineto{\pgfqpoint{2.380902in}{0.916589in}}%
\pgfpathquadraticcurveto{\pgfqpoint{2.380902in}{0.944367in}}{\pgfqpoint{2.353124in}{0.944367in}}%
\pgfpathlineto{\pgfqpoint{1.241240in}{0.944367in}}%
\pgfpathquadraticcurveto{\pgfqpoint{1.213462in}{0.944367in}}{\pgfqpoint{1.213462in}{0.916589in}}%
\pgfpathlineto{\pgfqpoint{1.213462in}{0.736805in}}%
\pgfpathquadraticcurveto{\pgfqpoint{1.213462in}{0.709028in}}{\pgfqpoint{1.241240in}{0.709028in}}%
\pgfpathclose%
\pgfusepath{stroke,fill}%
\end{pgfscope}%
\begin{pgfscope}%
\pgfsetrectcap%
\pgfsetroundjoin%
\pgfsetlinewidth{1.505625pt}%
\definecolor{currentstroke}{rgb}{0.000000,0.501961,0.000000}%
\pgfsetstrokecolor{currentstroke}%
\pgfsetdash{}{0pt}%
\pgfpathmoveto{\pgfqpoint{1.269018in}{0.840200in}}%
\pgfpathlineto{\pgfqpoint{1.546795in}{0.840200in}}%
\pgfusepath{stroke}%
\end{pgfscope}%
\begin{pgfscope}%
\definecolor{textcolor}{rgb}{0.000000,0.000000,0.000000}%
\pgfsetstrokecolor{textcolor}%
\pgfsetfillcolor{textcolor}%
\pgftext[x=1.657907in,y=0.791589in,left,base]{\color{textcolor}\rmfamily\fontsize{10.000000}{12.000000}\selectfont AUC 0.878}%
\end{pgfscope}%
\begin{pgfscope}%
\pgfsetbuttcap%
\pgfsetmiterjoin%
\definecolor{currentfill}{rgb}{1.000000,1.000000,1.000000}%
\pgfsetfillcolor{currentfill}%
\pgfsetlinewidth{0.000000pt}%
\definecolor{currentstroke}{rgb}{0.000000,0.000000,0.000000}%
\pgfsetstrokecolor{currentstroke}%
\pgfsetstrokeopacity{0.000000}%
\pgfsetdash{}{0pt}%
\pgfpathmoveto{\pgfqpoint{3.225541in}{0.639583in}}%
\pgfpathlineto{\pgfqpoint{4.912846in}{0.639583in}}%
\pgfpathlineto{\pgfqpoint{4.912846in}{2.179128in}}%
\pgfpathlineto{\pgfqpoint{3.225541in}{2.179128in}}%
\pgfpathclose%
\pgfusepath{fill}%
\end{pgfscope}%
\begin{pgfscope}%
\pgfsetbuttcap%
\pgfsetroundjoin%
\definecolor{currentfill}{rgb}{0.000000,0.000000,0.000000}%
\pgfsetfillcolor{currentfill}%
\pgfsetlinewidth{0.803000pt}%
\definecolor{currentstroke}{rgb}{0.000000,0.000000,0.000000}%
\pgfsetstrokecolor{currentstroke}%
\pgfsetdash{}{0pt}%
\pgfsys@defobject{currentmarker}{\pgfqpoint{0.000000in}{-0.048611in}}{\pgfqpoint{0.000000in}{0.000000in}}{%
\pgfpathmoveto{\pgfqpoint{0.000000in}{0.000000in}}%
\pgfpathlineto{\pgfqpoint{0.000000in}{-0.048611in}}%
\pgfusepath{stroke,fill}%
}%
\begin{pgfscope}%
\pgfsys@transformshift{3.302236in}{0.639583in}%
\pgfsys@useobject{currentmarker}{}%
\end{pgfscope}%
\end{pgfscope}%
\begin{pgfscope}%
\definecolor{textcolor}{rgb}{0.000000,0.000000,0.000000}%
\pgfsetstrokecolor{textcolor}%
\pgfsetfillcolor{textcolor}%
\pgftext[x=3.302236in,y=0.542361in,,top]{\color{textcolor}\rmfamily\fontsize{10.000000}{12.000000}\selectfont \(\displaystyle {0.0}\)}%
\end{pgfscope}%
\begin{pgfscope}%
\pgfsetbuttcap%
\pgfsetroundjoin%
\definecolor{currentfill}{rgb}{0.000000,0.000000,0.000000}%
\pgfsetfillcolor{currentfill}%
\pgfsetlinewidth{0.803000pt}%
\definecolor{currentstroke}{rgb}{0.000000,0.000000,0.000000}%
\pgfsetstrokecolor{currentstroke}%
\pgfsetdash{}{0pt}%
\pgfsys@defobject{currentmarker}{\pgfqpoint{0.000000in}{-0.048611in}}{\pgfqpoint{0.000000in}{0.000000in}}{%
\pgfpathmoveto{\pgfqpoint{0.000000in}{0.000000in}}%
\pgfpathlineto{\pgfqpoint{0.000000in}{-0.048611in}}%
\pgfusepath{stroke,fill}%
}%
\begin{pgfscope}%
\pgfsys@transformshift{4.069193in}{0.639583in}%
\pgfsys@useobject{currentmarker}{}%
\end{pgfscope}%
\end{pgfscope}%
\begin{pgfscope}%
\definecolor{textcolor}{rgb}{0.000000,0.000000,0.000000}%
\pgfsetstrokecolor{textcolor}%
\pgfsetfillcolor{textcolor}%
\pgftext[x=4.069193in,y=0.542361in,,top]{\color{textcolor}\rmfamily\fontsize{10.000000}{12.000000}\selectfont \(\displaystyle {0.5}\)}%
\end{pgfscope}%
\begin{pgfscope}%
\pgfsetbuttcap%
\pgfsetroundjoin%
\definecolor{currentfill}{rgb}{0.000000,0.000000,0.000000}%
\pgfsetfillcolor{currentfill}%
\pgfsetlinewidth{0.803000pt}%
\definecolor{currentstroke}{rgb}{0.000000,0.000000,0.000000}%
\pgfsetstrokecolor{currentstroke}%
\pgfsetdash{}{0pt}%
\pgfsys@defobject{currentmarker}{\pgfqpoint{0.000000in}{-0.048611in}}{\pgfqpoint{0.000000in}{0.000000in}}{%
\pgfpathmoveto{\pgfqpoint{0.000000in}{0.000000in}}%
\pgfpathlineto{\pgfqpoint{0.000000in}{-0.048611in}}%
\pgfusepath{stroke,fill}%
}%
\begin{pgfscope}%
\pgfsys@transformshift{4.836150in}{0.639583in}%
\pgfsys@useobject{currentmarker}{}%
\end{pgfscope}%
\end{pgfscope}%
\begin{pgfscope}%
\definecolor{textcolor}{rgb}{0.000000,0.000000,0.000000}%
\pgfsetstrokecolor{textcolor}%
\pgfsetfillcolor{textcolor}%
\pgftext[x=4.836150in,y=0.542361in,,top]{\color{textcolor}\rmfamily\fontsize{10.000000}{12.000000}\selectfont \(\displaystyle {1.0}\)}%
\end{pgfscope}%
\begin{pgfscope}%
\definecolor{textcolor}{rgb}{0.000000,0.000000,0.000000}%
\pgfsetstrokecolor{textcolor}%
\pgfsetfillcolor{textcolor}%
\pgftext[x=4.069193in,y=0.363349in,,top]{\color{textcolor}\rmfamily\fontsize{16.000000}{19.200000}\selectfont FPR}%
\end{pgfscope}%
\begin{pgfscope}%
\pgfsetbuttcap%
\pgfsetroundjoin%
\definecolor{currentfill}{rgb}{0.000000,0.000000,0.000000}%
\pgfsetfillcolor{currentfill}%
\pgfsetlinewidth{0.803000pt}%
\definecolor{currentstroke}{rgb}{0.000000,0.000000,0.000000}%
\pgfsetstrokecolor{currentstroke}%
\pgfsetdash{}{0pt}%
\pgfsys@defobject{currentmarker}{\pgfqpoint{-0.048611in}{0.000000in}}{\pgfqpoint{-0.000000in}{0.000000in}}{%
\pgfpathmoveto{\pgfqpoint{-0.000000in}{0.000000in}}%
\pgfpathlineto{\pgfqpoint{-0.048611in}{0.000000in}}%
\pgfusepath{stroke,fill}%
}%
\begin{pgfscope}%
\pgfsys@transformshift{3.225541in}{0.709562in}%
\pgfsys@useobject{currentmarker}{}%
\end{pgfscope}%
\end{pgfscope}%
\begin{pgfscope}%
\definecolor{textcolor}{rgb}{0.000000,0.000000,0.000000}%
\pgfsetstrokecolor{textcolor}%
\pgfsetfillcolor{textcolor}%
\pgftext[x=2.881404in, y=0.661337in, left, base]{\color{textcolor}\rmfamily\fontsize{10.000000}{12.000000}\selectfont \(\displaystyle {0.00}\)}%
\end{pgfscope}%
\begin{pgfscope}%
\pgfsetbuttcap%
\pgfsetroundjoin%
\definecolor{currentfill}{rgb}{0.000000,0.000000,0.000000}%
\pgfsetfillcolor{currentfill}%
\pgfsetlinewidth{0.803000pt}%
\definecolor{currentstroke}{rgb}{0.000000,0.000000,0.000000}%
\pgfsetstrokecolor{currentstroke}%
\pgfsetdash{}{0pt}%
\pgfsys@defobject{currentmarker}{\pgfqpoint{-0.048611in}{0.000000in}}{\pgfqpoint{-0.000000in}{0.000000in}}{%
\pgfpathmoveto{\pgfqpoint{-0.000000in}{0.000000in}}%
\pgfpathlineto{\pgfqpoint{-0.048611in}{0.000000in}}%
\pgfusepath{stroke,fill}%
}%
\begin{pgfscope}%
\pgfsys@transformshift{3.225541in}{1.059459in}%
\pgfsys@useobject{currentmarker}{}%
\end{pgfscope}%
\end{pgfscope}%
\begin{pgfscope}%
\definecolor{textcolor}{rgb}{0.000000,0.000000,0.000000}%
\pgfsetstrokecolor{textcolor}%
\pgfsetfillcolor{textcolor}%
\pgftext[x=2.881404in, y=1.011234in, left, base]{\color{textcolor}\rmfamily\fontsize{10.000000}{12.000000}\selectfont \(\displaystyle {0.25}\)}%
\end{pgfscope}%
\begin{pgfscope}%
\pgfsetbuttcap%
\pgfsetroundjoin%
\definecolor{currentfill}{rgb}{0.000000,0.000000,0.000000}%
\pgfsetfillcolor{currentfill}%
\pgfsetlinewidth{0.803000pt}%
\definecolor{currentstroke}{rgb}{0.000000,0.000000,0.000000}%
\pgfsetstrokecolor{currentstroke}%
\pgfsetdash{}{0pt}%
\pgfsys@defobject{currentmarker}{\pgfqpoint{-0.048611in}{0.000000in}}{\pgfqpoint{-0.000000in}{0.000000in}}{%
\pgfpathmoveto{\pgfqpoint{-0.000000in}{0.000000in}}%
\pgfpathlineto{\pgfqpoint{-0.048611in}{0.000000in}}%
\pgfusepath{stroke,fill}%
}%
\begin{pgfscope}%
\pgfsys@transformshift{3.225541in}{1.409356in}%
\pgfsys@useobject{currentmarker}{}%
\end{pgfscope}%
\end{pgfscope}%
\begin{pgfscope}%
\definecolor{textcolor}{rgb}{0.000000,0.000000,0.000000}%
\pgfsetstrokecolor{textcolor}%
\pgfsetfillcolor{textcolor}%
\pgftext[x=2.881404in, y=1.361130in, left, base]{\color{textcolor}\rmfamily\fontsize{10.000000}{12.000000}\selectfont \(\displaystyle {0.50}\)}%
\end{pgfscope}%
\begin{pgfscope}%
\pgfsetbuttcap%
\pgfsetroundjoin%
\definecolor{currentfill}{rgb}{0.000000,0.000000,0.000000}%
\pgfsetfillcolor{currentfill}%
\pgfsetlinewidth{0.803000pt}%
\definecolor{currentstroke}{rgb}{0.000000,0.000000,0.000000}%
\pgfsetstrokecolor{currentstroke}%
\pgfsetdash{}{0pt}%
\pgfsys@defobject{currentmarker}{\pgfqpoint{-0.048611in}{0.000000in}}{\pgfqpoint{-0.000000in}{0.000000in}}{%
\pgfpathmoveto{\pgfqpoint{-0.000000in}{0.000000in}}%
\pgfpathlineto{\pgfqpoint{-0.048611in}{0.000000in}}%
\pgfusepath{stroke,fill}%
}%
\begin{pgfscope}%
\pgfsys@transformshift{3.225541in}{1.759252in}%
\pgfsys@useobject{currentmarker}{}%
\end{pgfscope}%
\end{pgfscope}%
\begin{pgfscope}%
\definecolor{textcolor}{rgb}{0.000000,0.000000,0.000000}%
\pgfsetstrokecolor{textcolor}%
\pgfsetfillcolor{textcolor}%
\pgftext[x=2.881404in, y=1.711027in, left, base]{\color{textcolor}\rmfamily\fontsize{10.000000}{12.000000}\selectfont \(\displaystyle {0.75}\)}%
\end{pgfscope}%
\begin{pgfscope}%
\pgfsetbuttcap%
\pgfsetroundjoin%
\definecolor{currentfill}{rgb}{0.000000,0.000000,0.000000}%
\pgfsetfillcolor{currentfill}%
\pgfsetlinewidth{0.803000pt}%
\definecolor{currentstroke}{rgb}{0.000000,0.000000,0.000000}%
\pgfsetstrokecolor{currentstroke}%
\pgfsetdash{}{0pt}%
\pgfsys@defobject{currentmarker}{\pgfqpoint{-0.048611in}{0.000000in}}{\pgfqpoint{-0.000000in}{0.000000in}}{%
\pgfpathmoveto{\pgfqpoint{-0.000000in}{0.000000in}}%
\pgfpathlineto{\pgfqpoint{-0.048611in}{0.000000in}}%
\pgfusepath{stroke,fill}%
}%
\begin{pgfscope}%
\pgfsys@transformshift{3.225541in}{2.109149in}%
\pgfsys@useobject{currentmarker}{}%
\end{pgfscope}%
\end{pgfscope}%
\begin{pgfscope}%
\definecolor{textcolor}{rgb}{0.000000,0.000000,0.000000}%
\pgfsetstrokecolor{textcolor}%
\pgfsetfillcolor{textcolor}%
\pgftext[x=2.881404in, y=2.060924in, left, base]{\color{textcolor}\rmfamily\fontsize{10.000000}{12.000000}\selectfont \(\displaystyle {1.00}\)}%
\end{pgfscope}%
\begin{pgfscope}%
\definecolor{textcolor}{rgb}{0.000000,0.000000,0.000000}%
\pgfsetstrokecolor{textcolor}%
\pgfsetfillcolor{textcolor}%
\pgftext[x=2.825849in,y=1.409356in,,bottom,rotate=90.000000]{\color{textcolor}\rmfamily\fontsize{16.000000}{19.200000}\selectfont TPR}%
\end{pgfscope}%
\begin{pgfscope}%
\pgfpathrectangle{\pgfqpoint{3.225541in}{0.639583in}}{\pgfqpoint{1.687305in}{1.539545in}}%
\pgfusepath{clip}%
\pgfsetrectcap%
\pgfsetroundjoin%
\pgfsetlinewidth{1.505625pt}%
\definecolor{currentstroke}{rgb}{0.000000,0.501961,0.000000}%
\pgfsetstrokecolor{currentstroke}%
\pgfsetdash{}{0pt}%
\pgfpathmoveto{\pgfqpoint{3.302236in}{0.709562in}}%
\pgfpathlineto{\pgfqpoint{3.303270in}{0.711105in}}%
\pgfpathlineto{\pgfqpoint{3.304684in}{0.715731in}}%
\pgfpathlineto{\pgfqpoint{3.304738in}{0.715731in}}%
\pgfpathlineto{\pgfqpoint{3.305282in}{0.717274in}}%
\pgfpathlineto{\pgfqpoint{3.306370in}{0.721129in}}%
\pgfpathlineto{\pgfqpoint{3.306642in}{0.721129in}}%
\pgfpathlineto{\pgfqpoint{3.307729in}{0.722671in}}%
\pgfpathlineto{\pgfqpoint{3.309252in}{0.728069in}}%
\pgfpathlineto{\pgfqpoint{3.311427in}{0.729612in}}%
\pgfpathlineto{\pgfqpoint{3.313657in}{0.741178in}}%
\pgfpathlineto{\pgfqpoint{3.316485in}{0.742721in}}%
\pgfpathlineto{\pgfqpoint{3.320020in}{0.760456in}}%
\pgfpathlineto{\pgfqpoint{3.321108in}{0.761999in}}%
\pgfpathlineto{\pgfqpoint{3.322359in}{0.772794in}}%
\pgfpathlineto{\pgfqpoint{3.325894in}{0.774337in}}%
\pgfpathlineto{\pgfqpoint{3.327308in}{0.778192in}}%
\pgfpathlineto{\pgfqpoint{3.327362in}{0.778192in}}%
\pgfpathlineto{\pgfqpoint{3.328776in}{0.779734in}}%
\pgfpathlineto{\pgfqpoint{3.330244in}{0.788988in}}%
\pgfpathlineto{\pgfqpoint{3.331495in}{0.789759in}}%
\pgfpathlineto{\pgfqpoint{3.332637in}{0.799013in}}%
\pgfpathlineto{\pgfqpoint{3.333344in}{0.800555in}}%
\pgfpathlineto{\pgfqpoint{3.334650in}{0.805953in}}%
\pgfpathlineto{\pgfqpoint{3.335792in}{0.807495in}}%
\pgfpathlineto{\pgfqpoint{3.337314in}{0.813664in}}%
\pgfpathlineto{\pgfqpoint{3.339870in}{0.821375in}}%
\pgfpathlineto{\pgfqpoint{3.340414in}{0.822917in}}%
\pgfpathlineto{\pgfqpoint{3.341883in}{0.832942in}}%
\pgfpathlineto{\pgfqpoint{3.341937in}{0.832942in}}%
\pgfpathlineto{\pgfqpoint{3.342427in}{0.834484in}}%
\pgfpathlineto{\pgfqpoint{3.343786in}{0.839882in}}%
\pgfpathlineto{\pgfqpoint{3.343895in}{0.839882in}}%
\pgfpathlineto{\pgfqpoint{3.344058in}{0.839882in}}%
\pgfpathlineto{\pgfqpoint{3.345146in}{0.844509in}}%
\pgfpathlineto{\pgfqpoint{3.346125in}{0.844509in}}%
\pgfpathlineto{\pgfqpoint{3.347430in}{0.856076in}}%
\pgfpathlineto{\pgfqpoint{3.348354in}{0.857618in}}%
\pgfpathlineto{\pgfqpoint{3.350095in}{0.864558in}}%
\pgfpathlineto{\pgfqpoint{3.350639in}{0.865329in}}%
\pgfpathlineto{\pgfqpoint{3.352053in}{0.869185in}}%
\pgfpathlineto{\pgfqpoint{3.353575in}{0.870727in}}%
\pgfpathlineto{\pgfqpoint{3.355044in}{0.878438in}}%
\pgfpathlineto{\pgfqpoint{3.356349in}{0.879980in}}%
\pgfpathlineto{\pgfqpoint{3.357872in}{0.886149in}}%
\pgfpathlineto{\pgfqpoint{3.359068in}{0.887692in}}%
\pgfpathlineto{\pgfqpoint{3.360537in}{0.896174in}}%
\pgfpathlineto{\pgfqpoint{3.361951in}{0.897716in}}%
\pgfpathlineto{\pgfqpoint{3.363093in}{0.903885in}}%
\pgfpathlineto{\pgfqpoint{3.364833in}{0.905427in}}%
\pgfpathlineto{\pgfqpoint{3.366247in}{0.911596in}}%
\pgfpathlineto{\pgfqpoint{3.367063in}{0.912367in}}%
\pgfpathlineto{\pgfqpoint{3.368314in}{0.916994in}}%
\pgfpathlineto{\pgfqpoint{3.368477in}{0.916994in}}%
\pgfpathlineto{\pgfqpoint{3.370326in}{0.918536in}}%
\pgfpathlineto{\pgfqpoint{3.371794in}{0.923163in}}%
\pgfpathlineto{\pgfqpoint{3.372012in}{0.924705in}}%
\pgfpathlineto{\pgfqpoint{3.373208in}{0.928561in}}%
\pgfpathlineto{\pgfqpoint{3.374133in}{0.930103in}}%
\pgfpathlineto{\pgfqpoint{3.375547in}{0.937815in}}%
\pgfpathlineto{\pgfqpoint{3.375655in}{0.937815in}}%
\pgfpathlineto{\pgfqpoint{3.377668in}{0.946297in}}%
\pgfpathlineto{\pgfqpoint{3.379245in}{0.947839in}}%
\pgfpathlineto{\pgfqpoint{3.379680in}{0.950152in}}%
\pgfpathlineto{\pgfqpoint{3.381529in}{0.951695in}}%
\pgfpathlineto{\pgfqpoint{3.382889in}{0.961719in}}%
\pgfpathlineto{\pgfqpoint{3.384520in}{0.963262in}}%
\pgfpathlineto{\pgfqpoint{3.385608in}{0.969431in}}%
\pgfpathlineto{\pgfqpoint{3.385825in}{0.969431in}}%
\pgfpathlineto{\pgfqpoint{3.386260in}{0.970202in}}%
\pgfpathlineto{\pgfqpoint{3.387566in}{0.977142in}}%
\pgfpathlineto{\pgfqpoint{3.389252in}{0.978684in}}%
\pgfpathlineto{\pgfqpoint{3.390666in}{0.987166in}}%
\pgfpathlineto{\pgfqpoint{3.392134in}{0.988709in}}%
\pgfpathlineto{\pgfqpoint{3.393548in}{0.994106in}}%
\pgfpathlineto{\pgfqpoint{3.394309in}{0.994878in}}%
\pgfpathlineto{\pgfqpoint{3.395669in}{1.000275in}}%
\pgfpathlineto{\pgfqpoint{3.396974in}{1.001818in}}%
\pgfpathlineto{\pgfqpoint{3.398443in}{1.008758in}}%
\pgfpathlineto{\pgfqpoint{3.399095in}{1.010300in}}%
\pgfpathlineto{\pgfqpoint{3.400237in}{1.014156in}}%
\pgfpathlineto{\pgfqpoint{3.401434in}{1.015698in}}%
\pgfpathlineto{\pgfqpoint{3.402902in}{1.025722in}}%
\pgfpathlineto{\pgfqpoint{3.403283in}{1.027265in}}%
\pgfpathlineto{\pgfqpoint{3.404588in}{1.034976in}}%
\pgfpathlineto{\pgfqpoint{3.406818in}{1.036518in}}%
\pgfpathlineto{\pgfqpoint{3.408341in}{1.044229in}}%
\pgfpathlineto{\pgfqpoint{3.408612in}{1.045772in}}%
\pgfpathlineto{\pgfqpoint{3.410026in}{1.052712in}}%
\pgfpathlineto{\pgfqpoint{3.410625in}{1.053483in}}%
\pgfpathlineto{\pgfqpoint{3.411712in}{1.058881in}}%
\pgfpathlineto{\pgfqpoint{3.412746in}{1.060423in}}%
\pgfpathlineto{\pgfqpoint{3.414160in}{1.068134in}}%
\pgfpathlineto{\pgfqpoint{3.414812in}{1.069676in}}%
\pgfpathlineto{\pgfqpoint{3.416281in}{1.073532in}}%
\pgfpathlineto{\pgfqpoint{3.417260in}{1.075074in}}%
\pgfpathlineto{\pgfqpoint{3.418782in}{1.084328in}}%
\pgfpathlineto{\pgfqpoint{3.419054in}{1.085870in}}%
\pgfpathlineto{\pgfqpoint{3.420359in}{1.092039in}}%
\pgfpathlineto{\pgfqpoint{3.421284in}{1.093581in}}%
\pgfpathlineto{\pgfqpoint{3.422698in}{1.098208in}}%
\pgfpathlineto{\pgfqpoint{3.423514in}{1.099750in}}%
\pgfpathlineto{\pgfqpoint{3.424656in}{1.104377in}}%
\pgfpathlineto{\pgfqpoint{3.425689in}{1.105919in}}%
\pgfpathlineto{\pgfqpoint{3.427049in}{1.112088in}}%
\pgfpathlineto{\pgfqpoint{3.428245in}{1.113630in}}%
\pgfpathlineto{\pgfqpoint{3.429605in}{1.117486in}}%
\pgfpathlineto{\pgfqpoint{3.430584in}{1.119028in}}%
\pgfpathlineto{\pgfqpoint{3.432052in}{1.127511in}}%
\pgfpathlineto{\pgfqpoint{3.432433in}{1.129053in}}%
\pgfpathlineto{\pgfqpoint{3.433847in}{1.133679in}}%
\pgfpathlineto{\pgfqpoint{3.434499in}{1.135222in}}%
\pgfpathlineto{\pgfqpoint{3.435859in}{1.136764in}}%
\pgfpathlineto{\pgfqpoint{3.436838in}{1.138306in}}%
\pgfpathlineto{\pgfqpoint{3.438252in}{1.146789in}}%
\pgfpathlineto{\pgfqpoint{3.439177in}{1.148331in}}%
\pgfpathlineto{\pgfqpoint{3.440699in}{1.154500in}}%
\pgfpathlineto{\pgfqpoint{3.441189in}{1.156042in}}%
\pgfpathlineto{\pgfqpoint{3.441678in}{1.159127in}}%
\pgfpathlineto{\pgfqpoint{3.443799in}{1.160669in}}%
\pgfpathlineto{\pgfqpoint{3.445159in}{1.166067in}}%
\pgfpathlineto{\pgfqpoint{3.445594in}{1.167609in}}%
\pgfpathlineto{\pgfqpoint{3.447117in}{1.177633in}}%
\pgfpathlineto{\pgfqpoint{3.448585in}{1.179176in}}%
\pgfpathlineto{\pgfqpoint{3.450108in}{1.183031in}}%
\pgfpathlineto{\pgfqpoint{3.450543in}{1.184574in}}%
\pgfpathlineto{\pgfqpoint{3.451576in}{1.187658in}}%
\pgfpathlineto{\pgfqpoint{3.452555in}{1.189200in}}%
\pgfpathlineto{\pgfqpoint{3.454078in}{1.194598in}}%
\pgfpathlineto{\pgfqpoint{3.455601in}{1.196140in}}%
\pgfpathlineto{\pgfqpoint{3.456525in}{1.197683in}}%
\pgfpathlineto{\pgfqpoint{3.458211in}{1.199225in}}%
\pgfpathlineto{\pgfqpoint{3.459734in}{1.203080in}}%
\pgfpathlineto{\pgfqpoint{3.461257in}{1.204623in}}%
\pgfpathlineto{\pgfqpoint{3.462507in}{1.206936in}}%
\pgfpathlineto{\pgfqpoint{3.463486in}{1.207707in}}%
\pgfpathlineto{\pgfqpoint{3.465009in}{1.212334in}}%
\pgfpathlineto{\pgfqpoint{3.465934in}{1.213876in}}%
\pgfpathlineto{\pgfqpoint{3.467456in}{1.220816in}}%
\pgfpathlineto{\pgfqpoint{3.469632in}{1.222359in}}%
\pgfpathlineto{\pgfqpoint{3.470774in}{1.225443in}}%
\pgfpathlineto{\pgfqpoint{3.472242in}{1.226985in}}%
\pgfpathlineto{\pgfqpoint{3.473493in}{1.230841in}}%
\pgfpathlineto{\pgfqpoint{3.474798in}{1.232383in}}%
\pgfpathlineto{\pgfqpoint{3.474798in}{1.233154in}}%
\pgfpathlineto{\pgfqpoint{3.477789in}{1.234696in}}%
\pgfpathlineto{\pgfqpoint{3.478877in}{1.239323in}}%
\pgfpathlineto{\pgfqpoint{3.479910in}{1.240865in}}%
\pgfpathlineto{\pgfqpoint{3.481324in}{1.243950in}}%
\pgfpathlineto{\pgfqpoint{3.482847in}{1.245492in}}%
\pgfpathlineto{\pgfqpoint{3.484098in}{1.249348in}}%
\pgfpathlineto{\pgfqpoint{3.485077in}{1.250890in}}%
\pgfpathlineto{\pgfqpoint{3.486600in}{1.254746in}}%
\pgfpathlineto{\pgfqpoint{3.487252in}{1.256288in}}%
\pgfpathlineto{\pgfqpoint{3.487959in}{1.257830in}}%
\pgfpathlineto{\pgfqpoint{3.489156in}{1.259372in}}%
\pgfpathlineto{\pgfqpoint{3.489917in}{1.261686in}}%
\pgfpathlineto{\pgfqpoint{3.492691in}{1.263228in}}%
\pgfpathlineto{\pgfqpoint{3.493289in}{1.266312in}}%
\pgfpathlineto{\pgfqpoint{3.494921in}{1.267084in}}%
\pgfpathlineto{\pgfqpoint{3.496226in}{1.272481in}}%
\pgfpathlineto{\pgfqpoint{3.497803in}{1.273253in}}%
\pgfpathlineto{\pgfqpoint{3.499054in}{1.277108in}}%
\pgfpathlineto{\pgfqpoint{3.500903in}{1.278650in}}%
\pgfpathlineto{\pgfqpoint{3.502426in}{1.284048in}}%
\pgfpathlineto{\pgfqpoint{3.503622in}{1.285591in}}%
\pgfpathlineto{\pgfqpoint{3.505145in}{1.291760in}}%
\pgfpathlineto{\pgfqpoint{3.507483in}{1.293302in}}%
\pgfpathlineto{\pgfqpoint{3.508843in}{1.297928in}}%
\pgfpathlineto{\pgfqpoint{3.510746in}{1.298700in}}%
\pgfpathlineto{\pgfqpoint{3.511562in}{1.302555in}}%
\pgfpathlineto{\pgfqpoint{3.513738in}{1.304097in}}%
\pgfpathlineto{\pgfqpoint{3.514662in}{1.307953in}}%
\pgfpathlineto{\pgfqpoint{3.515804in}{1.309495in}}%
\pgfpathlineto{\pgfqpoint{3.517273in}{1.312580in}}%
\pgfpathlineto{\pgfqpoint{3.517599in}{1.312580in}}%
\pgfpathlineto{\pgfqpoint{3.519067in}{1.317207in}}%
\pgfpathlineto{\pgfqpoint{3.520373in}{1.318749in}}%
\pgfpathlineto{\pgfqpoint{3.521841in}{1.324918in}}%
\pgfpathlineto{\pgfqpoint{3.522385in}{1.326460in}}%
\pgfpathlineto{\pgfqpoint{3.523744in}{1.331087in}}%
\pgfpathlineto{\pgfqpoint{3.525593in}{1.332629in}}%
\pgfpathlineto{\pgfqpoint{3.527062in}{1.338798in}}%
\pgfpathlineto{\pgfqpoint{3.527660in}{1.339569in}}%
\pgfpathlineto{\pgfqpoint{3.528149in}{1.341882in}}%
\pgfpathlineto{\pgfqpoint{3.532391in}{1.343425in}}%
\pgfpathlineto{\pgfqpoint{3.533914in}{1.349594in}}%
\pgfpathlineto{\pgfqpoint{3.535872in}{1.351136in}}%
\pgfpathlineto{\pgfqpoint{3.537177in}{1.356534in}}%
\pgfpathlineto{\pgfqpoint{3.538102in}{1.358076in}}%
\pgfpathlineto{\pgfqpoint{3.539298in}{1.365787in}}%
\pgfpathlineto{\pgfqpoint{3.540604in}{1.367329in}}%
\pgfpathlineto{\pgfqpoint{3.541093in}{1.370414in}}%
\pgfpathlineto{\pgfqpoint{3.542833in}{1.370414in}}%
\pgfpathlineto{\pgfqpoint{3.544356in}{1.374270in}}%
\pgfpathlineto{\pgfqpoint{3.545117in}{1.375812in}}%
\pgfpathlineto{\pgfqpoint{3.546314in}{1.379667in}}%
\pgfpathlineto{\pgfqpoint{3.547347in}{1.381210in}}%
\pgfpathlineto{\pgfqpoint{3.548816in}{1.387379in}}%
\pgfpathlineto{\pgfqpoint{3.552677in}{1.388921in}}%
\pgfpathlineto{\pgfqpoint{3.554200in}{1.391234in}}%
\pgfpathlineto{\pgfqpoint{3.555070in}{1.392005in}}%
\pgfpathlineto{\pgfqpoint{3.556321in}{1.398945in}}%
\pgfpathlineto{\pgfqpoint{3.557408in}{1.400488in}}%
\pgfpathlineto{\pgfqpoint{3.558931in}{1.403572in}}%
\pgfpathlineto{\pgfqpoint{3.560236in}{1.405114in}}%
\pgfpathlineto{\pgfqpoint{3.561215in}{1.408970in}}%
\pgfpathlineto{\pgfqpoint{3.562412in}{1.409741in}}%
\pgfpathlineto{\pgfqpoint{3.563010in}{1.414368in}}%
\pgfpathlineto{\pgfqpoint{3.564587in}{1.415910in}}%
\pgfpathlineto{\pgfqpoint{3.565947in}{1.421308in}}%
\pgfpathlineto{\pgfqpoint{3.567361in}{1.422850in}}%
\pgfpathlineto{\pgfqpoint{3.568666in}{1.426706in}}%
\pgfpathlineto{\pgfqpoint{3.570624in}{1.428248in}}%
\pgfpathlineto{\pgfqpoint{3.572147in}{1.432104in}}%
\pgfpathlineto{\pgfqpoint{3.572690in}{1.432875in}}%
\pgfpathlineto{\pgfqpoint{3.573397in}{1.435188in}}%
\pgfpathlineto{\pgfqpoint{3.575790in}{1.436730in}}%
\pgfpathlineto{\pgfqpoint{3.577041in}{1.442899in}}%
\pgfpathlineto{\pgfqpoint{3.577639in}{1.444442in}}%
\pgfpathlineto{\pgfqpoint{3.578346in}{1.446755in}}%
\pgfpathlineto{\pgfqpoint{3.581392in}{1.448297in}}%
\pgfpathlineto{\pgfqpoint{3.582752in}{1.451382in}}%
\pgfpathlineto{\pgfqpoint{3.584601in}{1.452924in}}%
\pgfpathlineto{\pgfqpoint{3.586123in}{1.459864in}}%
\pgfpathlineto{\pgfqpoint{3.586993in}{1.461406in}}%
\pgfpathlineto{\pgfqpoint{3.588407in}{1.462949in}}%
\pgfpathlineto{\pgfqpoint{3.591399in}{1.464491in}}%
\pgfpathlineto{\pgfqpoint{3.592323in}{1.469118in}}%
\pgfpathlineto{\pgfqpoint{3.594009in}{1.470660in}}%
\pgfpathlineto{\pgfqpoint{3.595151in}{1.476058in}}%
\pgfpathlineto{\pgfqpoint{3.598305in}{1.477600in}}%
\pgfpathlineto{\pgfqpoint{3.598741in}{1.480684in}}%
\pgfpathlineto{\pgfqpoint{3.600046in}{1.481456in}}%
\pgfpathlineto{\pgfqpoint{3.601405in}{1.488396in}}%
\pgfpathlineto{\pgfqpoint{3.602928in}{1.489938in}}%
\pgfpathlineto{\pgfqpoint{3.604451in}{1.495336in}}%
\pgfpathlineto{\pgfqpoint{3.605811in}{1.496878in}}%
\pgfpathlineto{\pgfqpoint{3.606681in}{1.499962in}}%
\pgfpathlineto{\pgfqpoint{3.608639in}{1.501505in}}%
\pgfpathlineto{\pgfqpoint{3.608910in}{1.503818in}}%
\pgfpathlineto{\pgfqpoint{3.612010in}{1.505360in}}%
\pgfpathlineto{\pgfqpoint{3.613370in}{1.508445in}}%
\pgfpathlineto{\pgfqpoint{3.617231in}{1.509987in}}%
\pgfpathlineto{\pgfqpoint{3.618700in}{1.513843in}}%
\pgfpathlineto{\pgfqpoint{3.618972in}{1.515385in}}%
\pgfpathlineto{\pgfqpoint{3.619461in}{1.516927in}}%
\pgfpathlineto{\pgfqpoint{3.621310in}{1.518469in}}%
\pgfpathlineto{\pgfqpoint{3.622833in}{1.523867in}}%
\pgfpathlineto{\pgfqpoint{3.624519in}{1.525409in}}%
\pgfpathlineto{\pgfqpoint{3.625171in}{1.527723in}}%
\pgfpathlineto{\pgfqpoint{3.628054in}{1.529265in}}%
\pgfpathlineto{\pgfqpoint{3.629468in}{1.533121in}}%
\pgfpathlineto{\pgfqpoint{3.630827in}{1.534663in}}%
\pgfpathlineto{\pgfqpoint{3.631861in}{1.536205in}}%
\pgfpathlineto{\pgfqpoint{3.633547in}{1.537747in}}%
\pgfpathlineto{\pgfqpoint{3.634199in}{1.540061in}}%
\pgfpathlineto{\pgfqpoint{3.638332in}{1.541603in}}%
\pgfpathlineto{\pgfqpoint{3.638332in}{1.542374in}}%
\pgfpathlineto{\pgfqpoint{3.641324in}{1.543916in}}%
\pgfpathlineto{\pgfqpoint{3.642411in}{1.547001in}}%
\pgfpathlineto{\pgfqpoint{3.645348in}{1.548543in}}%
\pgfpathlineto{\pgfqpoint{3.646490in}{1.552399in}}%
\pgfpathlineto{\pgfqpoint{3.648992in}{1.553170in}}%
\pgfpathlineto{\pgfqpoint{3.650243in}{1.557797in}}%
\pgfpathlineto{\pgfqpoint{3.651385in}{1.559339in}}%
\pgfpathlineto{\pgfqpoint{3.652853in}{1.563194in}}%
\pgfpathlineto{\pgfqpoint{3.653560in}{1.564737in}}%
\pgfpathlineto{\pgfqpoint{3.653560in}{1.565508in}}%
\pgfpathlineto{\pgfqpoint{3.657476in}{1.567050in}}%
\pgfpathlineto{\pgfqpoint{3.658400in}{1.570135in}}%
\pgfpathlineto{\pgfqpoint{3.659651in}{1.571677in}}%
\pgfpathlineto{\pgfqpoint{3.660848in}{1.574761in}}%
\pgfpathlineto{\pgfqpoint{3.663567in}{1.576304in}}%
\pgfpathlineto{\pgfqpoint{3.663839in}{1.577846in}}%
\pgfpathlineto{\pgfqpoint{3.665960in}{1.579388in}}%
\pgfpathlineto{\pgfqpoint{3.666395in}{1.580930in}}%
\pgfpathlineto{\pgfqpoint{3.669821in}{1.582473in}}%
\pgfpathlineto{\pgfqpoint{3.671126in}{1.584786in}}%
\pgfpathlineto{\pgfqpoint{3.671833in}{1.586328in}}%
\pgfpathlineto{\pgfqpoint{3.673356in}{1.589413in}}%
\pgfpathlineto{\pgfqpoint{3.673628in}{1.590184in}}%
\pgfpathlineto{\pgfqpoint{3.675151in}{1.594039in}}%
\pgfpathlineto{\pgfqpoint{3.676456in}{1.595582in}}%
\pgfpathlineto{\pgfqpoint{3.676837in}{1.597895in}}%
\pgfpathlineto{\pgfqpoint{3.680644in}{1.599437in}}%
\pgfpathlineto{\pgfqpoint{3.682058in}{1.604064in}}%
\pgfpathlineto{\pgfqpoint{3.684450in}{1.605606in}}%
\pgfpathlineto{\pgfqpoint{3.685810in}{1.610233in}}%
\pgfpathlineto{\pgfqpoint{3.687387in}{1.611775in}}%
\pgfpathlineto{\pgfqpoint{3.688910in}{1.614089in}}%
\pgfpathlineto{\pgfqpoint{3.689399in}{1.615631in}}%
\pgfpathlineto{\pgfqpoint{3.689943in}{1.617173in}}%
\pgfpathlineto{\pgfqpoint{3.692499in}{1.618715in}}%
\pgfpathlineto{\pgfqpoint{3.693696in}{1.621800in}}%
\pgfpathlineto{\pgfqpoint{3.696198in}{1.623342in}}%
\pgfpathlineto{\pgfqpoint{3.697720in}{1.627198in}}%
\pgfpathlineto{\pgfqpoint{3.699624in}{1.628740in}}%
\pgfpathlineto{\pgfqpoint{3.700983in}{1.633367in}}%
\pgfpathlineto{\pgfqpoint{3.703050in}{1.634909in}}%
\pgfpathlineto{\pgfqpoint{3.704301in}{1.637222in}}%
\pgfpathlineto{\pgfqpoint{3.705987in}{1.638764in}}%
\pgfpathlineto{\pgfqpoint{3.706639in}{1.642620in}}%
\pgfpathlineto{\pgfqpoint{3.708652in}{1.644162in}}%
\pgfpathlineto{\pgfqpoint{3.710066in}{1.648789in}}%
\pgfpathlineto{\pgfqpoint{3.711480in}{1.650331in}}%
\pgfpathlineto{\pgfqpoint{3.712513in}{1.653416in}}%
\pgfpathlineto{\pgfqpoint{3.714688in}{1.654187in}}%
\pgfpathlineto{\pgfqpoint{3.716211in}{1.660356in}}%
\pgfpathlineto{\pgfqpoint{3.718114in}{1.661898in}}%
\pgfpathlineto{\pgfqpoint{3.718767in}{1.663440in}}%
\pgfpathlineto{\pgfqpoint{3.721378in}{1.664983in}}%
\pgfpathlineto{\pgfqpoint{3.722085in}{1.667296in}}%
\pgfpathlineto{\pgfqpoint{3.727577in}{1.668838in}}%
\pgfpathlineto{\pgfqpoint{3.728339in}{1.671152in}}%
\pgfpathlineto{\pgfqpoint{3.730514in}{1.672694in}}%
\pgfpathlineto{\pgfqpoint{3.731276in}{1.675778in}}%
\pgfpathlineto{\pgfqpoint{3.734702in}{1.677321in}}%
\pgfpathlineto{\pgfqpoint{3.735844in}{1.680405in}}%
\pgfpathlineto{\pgfqpoint{3.737475in}{1.681947in}}%
\pgfpathlineto{\pgfqpoint{3.738835in}{1.684261in}}%
\pgfpathlineto{\pgfqpoint{3.739868in}{1.685803in}}%
\pgfpathlineto{\pgfqpoint{3.741010in}{1.688116in}}%
\pgfpathlineto{\pgfqpoint{3.743458in}{1.689658in}}%
\pgfpathlineto{\pgfqpoint{3.744926in}{1.692743in}}%
\pgfpathlineto{\pgfqpoint{3.746829in}{1.694285in}}%
\pgfpathlineto{\pgfqpoint{3.748189in}{1.696599in}}%
\pgfpathlineto{\pgfqpoint{3.752921in}{1.698141in}}%
\pgfpathlineto{\pgfqpoint{3.754335in}{1.700454in}}%
\pgfpathlineto{\pgfqpoint{3.758685in}{1.701996in}}%
\pgfpathlineto{\pgfqpoint{3.759773in}{1.703539in}}%
\pgfpathlineto{\pgfqpoint{3.763961in}{1.705081in}}%
\pgfpathlineto{\pgfqpoint{3.765266in}{1.709708in}}%
\pgfpathlineto{\pgfqpoint{3.769345in}{1.711250in}}%
\pgfpathlineto{\pgfqpoint{3.769345in}{1.712021in}}%
\pgfpathlineto{\pgfqpoint{3.771846in}{1.713563in}}%
\pgfpathlineto{\pgfqpoint{3.773260in}{1.715106in}}%
\pgfpathlineto{\pgfqpoint{3.773695in}{1.716648in}}%
\pgfpathlineto{\pgfqpoint{3.774783in}{1.718190in}}%
\pgfpathlineto{\pgfqpoint{3.775708in}{1.719732in}}%
\pgfpathlineto{\pgfqpoint{3.777013in}{1.721274in}}%
\pgfpathlineto{\pgfqpoint{3.779732in}{1.722817in}}%
\pgfpathlineto{\pgfqpoint{3.780330in}{1.725901in}}%
\pgfpathlineto{\pgfqpoint{3.784083in}{1.727443in}}%
\pgfpathlineto{\pgfqpoint{3.784083in}{1.728215in}}%
\pgfpathlineto{\pgfqpoint{3.786421in}{1.729757in}}%
\pgfpathlineto{\pgfqpoint{3.787835in}{1.734384in}}%
\pgfpathlineto{\pgfqpoint{3.791642in}{1.735926in}}%
\pgfpathlineto{\pgfqpoint{3.793111in}{1.737468in}}%
\pgfpathlineto{\pgfqpoint{3.794198in}{1.739010in}}%
\pgfpathlineto{\pgfqpoint{3.795721in}{1.742095in}}%
\pgfpathlineto{\pgfqpoint{3.797081in}{1.743637in}}%
\pgfpathlineto{\pgfqpoint{3.797516in}{1.745179in}}%
\pgfpathlineto{\pgfqpoint{3.802791in}{1.746722in}}%
\pgfpathlineto{\pgfqpoint{3.803389in}{1.748264in}}%
\pgfpathlineto{\pgfqpoint{3.805402in}{1.749806in}}%
\pgfpathlineto{\pgfqpoint{3.806598in}{1.754433in}}%
\pgfpathlineto{\pgfqpoint{3.809752in}{1.755975in}}%
\pgfpathlineto{\pgfqpoint{3.810677in}{1.759059in}}%
\pgfpathlineto{\pgfqpoint{3.812308in}{1.760602in}}%
\pgfpathlineto{\pgfqpoint{3.813070in}{1.762144in}}%
\pgfpathlineto{\pgfqpoint{3.815082in}{1.762915in}}%
\pgfpathlineto{\pgfqpoint{3.815300in}{1.766771in}}%
\pgfpathlineto{\pgfqpoint{3.817257in}{1.768313in}}%
\pgfpathlineto{\pgfqpoint{3.818019in}{1.771397in}}%
\pgfpathlineto{\pgfqpoint{3.821282in}{1.772940in}}%
\pgfpathlineto{\pgfqpoint{3.822424in}{1.775253in}}%
\pgfpathlineto{\pgfqpoint{3.823729in}{1.776024in}}%
\pgfpathlineto{\pgfqpoint{3.825197in}{1.779109in}}%
\pgfpathlineto{\pgfqpoint{3.828569in}{1.780651in}}%
\pgfpathlineto{\pgfqpoint{3.828841in}{1.782193in}}%
\pgfpathlineto{\pgfqpoint{3.830527in}{1.782964in}}%
\pgfpathlineto{\pgfqpoint{3.830908in}{1.786820in}}%
\pgfpathlineto{\pgfqpoint{3.835802in}{1.788362in}}%
\pgfpathlineto{\pgfqpoint{3.837271in}{1.792218in}}%
\pgfpathlineto{\pgfqpoint{3.839501in}{1.793760in}}%
\pgfpathlineto{\pgfqpoint{3.840588in}{1.796073in}}%
\pgfpathlineto{\pgfqpoint{3.842764in}{1.797616in}}%
\pgfpathlineto{\pgfqpoint{3.843688in}{1.799929in}}%
\pgfpathlineto{\pgfqpoint{3.847060in}{1.801471in}}%
\pgfpathlineto{\pgfqpoint{3.848420in}{1.803785in}}%
\pgfpathlineto{\pgfqpoint{3.850214in}{1.805327in}}%
\pgfpathlineto{\pgfqpoint{3.850378in}{1.807640in}}%
\pgfpathlineto{\pgfqpoint{3.854456in}{1.809182in}}%
\pgfpathlineto{\pgfqpoint{3.855925in}{1.811496in}}%
\pgfpathlineto{\pgfqpoint{3.858318in}{1.813038in}}%
\pgfpathlineto{\pgfqpoint{3.859405in}{1.815351in}}%
\pgfpathlineto{\pgfqpoint{3.860384in}{1.816894in}}%
\pgfpathlineto{\pgfqpoint{3.861635in}{1.819207in}}%
\pgfpathlineto{\pgfqpoint{3.863974in}{1.820749in}}%
\pgfpathlineto{\pgfqpoint{3.864626in}{1.823063in}}%
\pgfpathlineto{\pgfqpoint{3.866312in}{1.824605in}}%
\pgfpathlineto{\pgfqpoint{3.867509in}{1.826918in}}%
\pgfpathlineto{\pgfqpoint{3.870337in}{1.828460in}}%
\pgfpathlineto{\pgfqpoint{3.871642in}{1.833858in}}%
\pgfpathlineto{\pgfqpoint{3.877352in}{1.835401in}}%
\pgfpathlineto{\pgfqpoint{3.878712in}{1.837714in}}%
\pgfpathlineto{\pgfqpoint{3.881159in}{1.839256in}}%
\pgfpathlineto{\pgfqpoint{3.881159in}{1.840027in}}%
\pgfpathlineto{\pgfqpoint{3.884259in}{1.841570in}}%
\pgfpathlineto{\pgfqpoint{3.885673in}{1.844654in}}%
\pgfpathlineto{\pgfqpoint{3.890133in}{1.846196in}}%
\pgfpathlineto{\pgfqpoint{3.891655in}{1.848510in}}%
\pgfpathlineto{\pgfqpoint{3.892417in}{1.850052in}}%
\pgfpathlineto{\pgfqpoint{3.892852in}{1.851594in}}%
\pgfpathlineto{\pgfqpoint{3.894483in}{1.852365in}}%
\pgfpathlineto{\pgfqpoint{3.895571in}{1.855450in}}%
\pgfpathlineto{\pgfqpoint{3.898562in}{1.856992in}}%
\pgfpathlineto{\pgfqpoint{3.898562in}{1.857763in}}%
\pgfpathlineto{\pgfqpoint{3.901227in}{1.859305in}}%
\pgfpathlineto{\pgfqpoint{3.901445in}{1.861619in}}%
\pgfpathlineto{\pgfqpoint{3.905850in}{1.863161in}}%
\pgfpathlineto{\pgfqpoint{3.907372in}{1.864703in}}%
\pgfpathlineto{\pgfqpoint{3.908243in}{1.866245in}}%
\pgfpathlineto{\pgfqpoint{3.908841in}{1.867788in}}%
\pgfpathlineto{\pgfqpoint{3.914660in}{1.869330in}}%
\pgfpathlineto{\pgfqpoint{3.915421in}{1.873186in}}%
\pgfpathlineto{\pgfqpoint{3.923960in}{1.874728in}}%
\pgfpathlineto{\pgfqpoint{3.925047in}{1.877041in}}%
\pgfpathlineto{\pgfqpoint{3.927005in}{1.878583in}}%
\pgfpathlineto{\pgfqpoint{3.927005in}{1.879355in}}%
\pgfpathlineto{\pgfqpoint{3.930377in}{1.880897in}}%
\pgfpathlineto{\pgfqpoint{3.930595in}{1.882439in}}%
\pgfpathlineto{\pgfqpoint{3.933640in}{1.883981in}}%
\pgfpathlineto{\pgfqpoint{3.934184in}{1.885523in}}%
\pgfpathlineto{\pgfqpoint{3.941743in}{1.887066in}}%
\pgfpathlineto{\pgfqpoint{3.943266in}{1.888608in}}%
\pgfpathlineto{\pgfqpoint{3.947399in}{1.890150in}}%
\pgfpathlineto{\pgfqpoint{3.948650in}{1.894006in}}%
\pgfpathlineto{\pgfqpoint{3.952620in}{1.895548in}}%
\pgfpathlineto{\pgfqpoint{3.953871in}{1.897090in}}%
\pgfpathlineto{\pgfqpoint{3.957080in}{1.898633in}}%
\pgfpathlineto{\pgfqpoint{3.957080in}{1.899404in}}%
\pgfpathlineto{\pgfqpoint{3.961974in}{1.900946in}}%
\pgfpathlineto{\pgfqpoint{3.961974in}{1.901717in}}%
\pgfpathlineto{\pgfqpoint{3.968120in}{1.903259in}}%
\pgfpathlineto{\pgfqpoint{3.968120in}{1.904030in}}%
\pgfpathlineto{\pgfqpoint{3.971709in}{1.905573in}}%
\pgfpathlineto{\pgfqpoint{3.973014in}{1.907115in}}%
\pgfpathlineto{\pgfqpoint{3.977311in}{1.908657in}}%
\pgfpathlineto{\pgfqpoint{3.978616in}{1.910199in}}%
\pgfpathlineto{\pgfqpoint{3.982151in}{1.911742in}}%
\pgfpathlineto{\pgfqpoint{3.983184in}{1.913284in}}%
\pgfpathlineto{\pgfqpoint{3.986937in}{1.914826in}}%
\pgfpathlineto{\pgfqpoint{3.988133in}{1.918682in}}%
\pgfpathlineto{\pgfqpoint{3.993572in}{1.920224in}}%
\pgfpathlineto{\pgfqpoint{3.994442in}{1.922537in}}%
\pgfpathlineto{\pgfqpoint{3.997216in}{1.923308in}}%
\pgfpathlineto{\pgfqpoint{3.997814in}{1.925622in}}%
\pgfpathlineto{\pgfqpoint{4.000261in}{1.926393in}}%
\pgfpathlineto{\pgfqpoint{4.000261in}{1.927935in}}%
\pgfpathlineto{\pgfqpoint{4.005808in}{1.929477in}}%
\pgfpathlineto{\pgfqpoint{4.005808in}{1.930249in}}%
\pgfpathlineto{\pgfqpoint{4.009398in}{1.931791in}}%
\pgfpathlineto{\pgfqpoint{4.009398in}{1.932562in}}%
\pgfpathlineto{\pgfqpoint{4.009833in}{1.932562in}}%
\pgfpathlineto{\pgfqpoint{4.011573in}{1.934104in}}%
\pgfpathlineto{\pgfqpoint{4.012715in}{1.935646in}}%
\pgfpathlineto{\pgfqpoint{4.018589in}{1.937189in}}%
\pgfpathlineto{\pgfqpoint{4.020003in}{1.940273in}}%
\pgfpathlineto{\pgfqpoint{4.023918in}{1.941815in}}%
\pgfpathlineto{\pgfqpoint{4.024843in}{1.944129in}}%
\pgfpathlineto{\pgfqpoint{4.026583in}{1.945671in}}%
\pgfpathlineto{\pgfqpoint{4.027888in}{1.947984in}}%
\pgfpathlineto{\pgfqpoint{4.031423in}{1.949527in}}%
\pgfpathlineto{\pgfqpoint{4.031423in}{1.950298in}}%
\pgfpathlineto{\pgfqpoint{4.034034in}{1.951840in}}%
\pgfpathlineto{\pgfqpoint{4.035176in}{1.954153in}}%
\pgfpathlineto{\pgfqpoint{4.039799in}{1.955696in}}%
\pgfpathlineto{\pgfqpoint{4.040941in}{1.958009in}}%
\pgfpathlineto{\pgfqpoint{4.050621in}{1.959551in}}%
\pgfpathlineto{\pgfqpoint{4.050621in}{1.960322in}}%
\pgfpathlineto{\pgfqpoint{4.059486in}{1.961865in}}%
\pgfpathlineto{\pgfqpoint{4.059486in}{1.962636in}}%
\pgfpathlineto{\pgfqpoint{4.066121in}{1.964178in}}%
\pgfpathlineto{\pgfqpoint{4.067317in}{1.965720in}}%
\pgfpathlineto{\pgfqpoint{4.073735in}{1.967262in}}%
\pgfpathlineto{\pgfqpoint{4.074822in}{1.969576in}}%
\pgfpathlineto{\pgfqpoint{4.083197in}{1.971118in}}%
\pgfpathlineto{\pgfqpoint{4.084611in}{1.972660in}}%
\pgfpathlineto{\pgfqpoint{4.088853in}{1.974203in}}%
\pgfpathlineto{\pgfqpoint{4.088853in}{1.974974in}}%
\pgfpathlineto{\pgfqpoint{4.094020in}{1.976516in}}%
\pgfpathlineto{\pgfqpoint{4.095434in}{1.978829in}}%
\pgfpathlineto{\pgfqpoint{4.102721in}{1.980371in}}%
\pgfpathlineto{\pgfqpoint{4.103483in}{1.981914in}}%
\pgfpathlineto{\pgfqpoint{4.106093in}{1.983456in}}%
\pgfpathlineto{\pgfqpoint{4.106093in}{1.984227in}}%
\pgfpathlineto{\pgfqpoint{4.109084in}{1.985769in}}%
\pgfpathlineto{\pgfqpoint{4.109628in}{1.987312in}}%
\pgfpathlineto{\pgfqpoint{4.115339in}{1.988854in}}%
\pgfpathlineto{\pgfqpoint{4.116426in}{1.990396in}}%
\pgfpathlineto{\pgfqpoint{4.132579in}{1.991938in}}%
\pgfpathlineto{\pgfqpoint{4.132579in}{1.992709in}}%
\pgfpathlineto{\pgfqpoint{4.138670in}{1.994252in}}%
\pgfpathlineto{\pgfqpoint{4.138670in}{1.995023in}}%
\pgfpathlineto{\pgfqpoint{4.146011in}{1.996565in}}%
\pgfpathlineto{\pgfqpoint{4.146718in}{1.998107in}}%
\pgfpathlineto{\pgfqpoint{4.148731in}{1.999650in}}%
\pgfpathlineto{\pgfqpoint{4.148731in}{2.000421in}}%
\pgfpathlineto{\pgfqpoint{4.155202in}{2.001963in}}%
\pgfpathlineto{\pgfqpoint{4.155202in}{2.002734in}}%
\pgfpathlineto{\pgfqpoint{4.156671in}{2.002734in}}%
\pgfpathlineto{\pgfqpoint{4.159771in}{2.004276in}}%
\pgfpathlineto{\pgfqpoint{4.160151in}{2.005819in}}%
\pgfpathlineto{\pgfqpoint{4.168527in}{2.006590in}}%
\pgfpathlineto{\pgfqpoint{4.168527in}{2.008132in}}%
\pgfpathlineto{\pgfqpoint{4.172660in}{2.008903in}}%
\pgfpathlineto{\pgfqpoint{4.172660in}{2.011216in}}%
\pgfpathlineto{\pgfqpoint{4.176902in}{2.012759in}}%
\pgfpathlineto{\pgfqpoint{4.177990in}{2.015072in}}%
\pgfpathlineto{\pgfqpoint{4.180763in}{2.016614in}}%
\pgfpathlineto{\pgfqpoint{4.181253in}{2.018156in}}%
\pgfpathlineto{\pgfqpoint{4.181796in}{2.018156in}}%
\pgfpathlineto{\pgfqpoint{4.195066in}{2.019699in}}%
\pgfpathlineto{\pgfqpoint{4.195991in}{2.021241in}}%
\pgfpathlineto{\pgfqpoint{4.201919in}{2.022783in}}%
\pgfpathlineto{\pgfqpoint{4.201919in}{2.023554in}}%
\pgfpathlineto{\pgfqpoint{4.208826in}{2.025097in}}%
\pgfpathlineto{\pgfqpoint{4.208826in}{2.025868in}}%
\pgfpathlineto{\pgfqpoint{4.219974in}{2.027410in}}%
\pgfpathlineto{\pgfqpoint{4.219974in}{2.028181in}}%
\pgfpathlineto{\pgfqpoint{4.227479in}{2.029723in}}%
\pgfpathlineto{\pgfqpoint{4.228893in}{2.032037in}}%
\pgfpathlineto{\pgfqpoint{4.238628in}{2.033579in}}%
\pgfpathlineto{\pgfqpoint{4.238628in}{2.034350in}}%
\pgfpathlineto{\pgfqpoint{4.255379in}{2.035892in}}%
\pgfpathlineto{\pgfqpoint{4.256793in}{2.037435in}}%
\pgfpathlineto{\pgfqpoint{4.271694in}{2.038977in}}%
\pgfpathlineto{\pgfqpoint{4.271694in}{2.039748in}}%
\pgfpathlineto{\pgfqpoint{4.280667in}{2.041290in}}%
\pgfpathlineto{\pgfqpoint{4.282136in}{2.042832in}}%
\pgfpathlineto{\pgfqpoint{4.288880in}{2.044375in}}%
\pgfpathlineto{\pgfqpoint{4.288880in}{2.045146in}}%
\pgfpathlineto{\pgfqpoint{4.306119in}{2.046688in}}%
\pgfpathlineto{\pgfqpoint{4.307044in}{2.048230in}}%
\pgfpathlineto{\pgfqpoint{4.314060in}{2.049772in}}%
\pgfpathlineto{\pgfqpoint{4.315365in}{2.051315in}}%
\pgfpathlineto{\pgfqpoint{4.322272in}{2.052857in}}%
\pgfpathlineto{\pgfqpoint{4.322761in}{2.054399in}}%
\pgfpathlineto{\pgfqpoint{4.335541in}{2.055941in}}%
\pgfpathlineto{\pgfqpoint{4.336683in}{2.058255in}}%
\pgfpathlineto{\pgfqpoint{4.340218in}{2.059797in}}%
\pgfpathlineto{\pgfqpoint{4.340218in}{2.060568in}}%
\pgfpathlineto{\pgfqpoint{4.340925in}{2.060568in}}%
\pgfpathlineto{\pgfqpoint{4.358383in}{2.062110in}}%
\pgfpathlineto{\pgfqpoint{4.358383in}{2.062882in}}%
\pgfpathlineto{\pgfqpoint{4.368770in}{2.064424in}}%
\pgfpathlineto{\pgfqpoint{4.368770in}{2.065195in}}%
\pgfpathlineto{\pgfqpoint{4.380681in}{2.066737in}}%
\pgfpathlineto{\pgfqpoint{4.380681in}{2.067508in}}%
\pgfpathlineto{\pgfqpoint{4.391285in}{2.069051in}}%
\pgfpathlineto{\pgfqpoint{4.391285in}{2.069822in}}%
\pgfpathlineto{\pgfqpoint{4.392482in}{2.069822in}}%
\pgfpathlineto{\pgfqpoint{4.404229in}{2.071364in}}%
\pgfpathlineto{\pgfqpoint{4.404229in}{2.072135in}}%
\pgfpathlineto{\pgfqpoint{4.411299in}{2.073677in}}%
\pgfpathlineto{\pgfqpoint{4.411408in}{2.075220in}}%
\pgfpathlineto{\pgfqpoint{4.411516in}{2.075220in}}%
\pgfpathlineto{\pgfqpoint{4.423807in}{2.076762in}}%
\pgfpathlineto{\pgfqpoint{4.423807in}{2.077533in}}%
\pgfpathlineto{\pgfqpoint{4.444202in}{2.079075in}}%
\pgfpathlineto{\pgfqpoint{4.444202in}{2.079846in}}%
\pgfpathlineto{\pgfqpoint{4.461224in}{2.081388in}}%
\pgfpathlineto{\pgfqpoint{4.461224in}{2.082160in}}%
\pgfpathlineto{\pgfqpoint{4.466934in}{2.083702in}}%
\pgfpathlineto{\pgfqpoint{4.466934in}{2.084473in}}%
\pgfpathlineto{\pgfqpoint{4.497118in}{2.086015in}}%
\pgfpathlineto{\pgfqpoint{4.497118in}{2.086786in}}%
\pgfpathlineto{\pgfqpoint{4.512454in}{2.088329in}}%
\pgfpathlineto{\pgfqpoint{4.512454in}{2.089100in}}%
\pgfpathlineto{\pgfqpoint{4.552318in}{2.090642in}}%
\pgfpathlineto{\pgfqpoint{4.553678in}{2.092184in}}%
\pgfpathlineto{\pgfqpoint{4.577552in}{2.093726in}}%
\pgfpathlineto{\pgfqpoint{4.577552in}{2.094498in}}%
\pgfpathlineto{\pgfqpoint{4.590768in}{2.096040in}}%
\pgfpathlineto{\pgfqpoint{4.590768in}{2.096811in}}%
\pgfpathlineto{\pgfqpoint{4.596043in}{2.098353in}}%
\pgfpathlineto{\pgfqpoint{4.596043in}{2.099124in}}%
\pgfpathlineto{\pgfqpoint{4.605995in}{2.100667in}}%
\pgfpathlineto{\pgfqpoint{4.606539in}{2.102209in}}%
\pgfpathlineto{\pgfqpoint{4.672780in}{2.103751in}}%
\pgfpathlineto{\pgfqpoint{4.672780in}{2.104522in}}%
\pgfpathlineto{\pgfqpoint{4.730264in}{2.106064in}}%
\pgfpathlineto{\pgfqpoint{4.730264in}{2.106836in}}%
\pgfpathlineto{\pgfqpoint{4.797918in}{2.108378in}}%
\pgfpathlineto{\pgfqpoint{4.797918in}{2.109149in}}%
\pgfpathlineto{\pgfqpoint{4.799060in}{2.109149in}}%
\pgfpathlineto{\pgfqpoint{4.836150in}{2.109149in}}%
\pgfpathlineto{\pgfqpoint{4.836150in}{2.109149in}}%
\pgfusepath{stroke}%
\end{pgfscope}%
\begin{pgfscope}%
\pgfpathrectangle{\pgfqpoint{3.225541in}{0.639583in}}{\pgfqpoint{1.687305in}{1.539545in}}%
\pgfusepath{clip}%
\pgfsetrectcap%
\pgfsetroundjoin%
\pgfsetlinewidth{1.505625pt}%
\definecolor{currentstroke}{rgb}{0.501961,0.501961,0.501961}%
\pgfsetstrokecolor{currentstroke}%
\pgfsetdash{}{0pt}%
\pgfpathmoveto{\pgfqpoint{3.302236in}{0.709562in}}%
\pgfpathlineto{\pgfqpoint{4.836150in}{2.109149in}}%
\pgfusepath{stroke}%
\end{pgfscope}%
\begin{pgfscope}%
\pgfsetrectcap%
\pgfsetmiterjoin%
\pgfsetlinewidth{0.803000pt}%
\definecolor{currentstroke}{rgb}{0.000000,0.000000,0.000000}%
\pgfsetstrokecolor{currentstroke}%
\pgfsetdash{}{0pt}%
\pgfpathmoveto{\pgfqpoint{3.225541in}{0.639583in}}%
\pgfpathlineto{\pgfqpoint{3.225541in}{2.179128in}}%
\pgfusepath{stroke}%
\end{pgfscope}%
\begin{pgfscope}%
\pgfsetrectcap%
\pgfsetmiterjoin%
\pgfsetlinewidth{0.803000pt}%
\definecolor{currentstroke}{rgb}{0.000000,0.000000,0.000000}%
\pgfsetstrokecolor{currentstroke}%
\pgfsetdash{}{0pt}%
\pgfpathmoveto{\pgfqpoint{4.912846in}{0.639583in}}%
\pgfpathlineto{\pgfqpoint{4.912846in}{2.179128in}}%
\pgfusepath{stroke}%
\end{pgfscope}%
\begin{pgfscope}%
\pgfsetrectcap%
\pgfsetmiterjoin%
\pgfsetlinewidth{0.803000pt}%
\definecolor{currentstroke}{rgb}{0.000000,0.000000,0.000000}%
\pgfsetstrokecolor{currentstroke}%
\pgfsetdash{}{0pt}%
\pgfpathmoveto{\pgfqpoint{3.225541in}{0.639583in}}%
\pgfpathlineto{\pgfqpoint{4.912846in}{0.639583in}}%
\pgfusepath{stroke}%
\end{pgfscope}%
\begin{pgfscope}%
\pgfsetrectcap%
\pgfsetmiterjoin%
\pgfsetlinewidth{0.803000pt}%
\definecolor{currentstroke}{rgb}{0.000000,0.000000,0.000000}%
\pgfsetstrokecolor{currentstroke}%
\pgfsetdash{}{0pt}%
\pgfpathmoveto{\pgfqpoint{3.225541in}{2.179128in}}%
\pgfpathlineto{\pgfqpoint{4.912846in}{2.179128in}}%
\pgfusepath{stroke}%
\end{pgfscope}%
\begin{pgfscope}%
\definecolor{textcolor}{rgb}{0.000000,0.000000,0.000000}%
\pgfsetstrokecolor{textcolor}%
\pgfsetfillcolor{textcolor}%
\pgftext[x=4.069193in,y=2.262462in,,base]{\color{textcolor}\rmfamily\fontsize{20.000000}{24.000000}\selectfont Consolidation}%
\end{pgfscope}%
\begin{pgfscope}%
\pgfsetbuttcap%
\pgfsetmiterjoin%
\definecolor{currentfill}{rgb}{1.000000,1.000000,1.000000}%
\pgfsetfillcolor{currentfill}%
\pgfsetfillopacity{0.800000}%
\pgfsetlinewidth{1.003750pt}%
\definecolor{currentstroke}{rgb}{0.800000,0.800000,0.800000}%
\pgfsetstrokecolor{currentstroke}%
\pgfsetstrokeopacity{0.800000}%
\pgfsetdash{}{0pt}%
\pgfpathmoveto{\pgfqpoint{3.703740in}{0.709028in}}%
\pgfpathlineto{\pgfqpoint{4.815624in}{0.709028in}}%
\pgfpathquadraticcurveto{\pgfqpoint{4.843402in}{0.709028in}}{\pgfqpoint{4.843402in}{0.736805in}}%
\pgfpathlineto{\pgfqpoint{4.843402in}{0.916589in}}%
\pgfpathquadraticcurveto{\pgfqpoint{4.843402in}{0.944367in}}{\pgfqpoint{4.815624in}{0.944367in}}%
\pgfpathlineto{\pgfqpoint{3.703740in}{0.944367in}}%
\pgfpathquadraticcurveto{\pgfqpoint{3.675962in}{0.944367in}}{\pgfqpoint{3.675962in}{0.916589in}}%
\pgfpathlineto{\pgfqpoint{3.675962in}{0.736805in}}%
\pgfpathquadraticcurveto{\pgfqpoint{3.675962in}{0.709028in}}{\pgfqpoint{3.703740in}{0.709028in}}%
\pgfpathclose%
\pgfusepath{stroke,fill}%
\end{pgfscope}%
\begin{pgfscope}%
\pgfsetrectcap%
\pgfsetroundjoin%
\pgfsetlinewidth{1.505625pt}%
\definecolor{currentstroke}{rgb}{0.000000,0.501961,0.000000}%
\pgfsetstrokecolor{currentstroke}%
\pgfsetdash{}{0pt}%
\pgfpathmoveto{\pgfqpoint{3.731518in}{0.840200in}}%
\pgfpathlineto{\pgfqpoint{4.009295in}{0.840200in}}%
\pgfusepath{stroke}%
\end{pgfscope}%
\begin{pgfscope}%
\definecolor{textcolor}{rgb}{0.000000,0.000000,0.000000}%
\pgfsetstrokecolor{textcolor}%
\pgfsetfillcolor{textcolor}%
\pgftext[x=4.120407in,y=0.791589in,left,base]{\color{textcolor}\rmfamily\fontsize{10.000000}{12.000000}\selectfont AUC 0.776}%
\end{pgfscope}%
\begin{pgfscope}%
\pgfsetbuttcap%
\pgfsetmiterjoin%
\definecolor{currentfill}{rgb}{1.000000,1.000000,1.000000}%
\pgfsetfillcolor{currentfill}%
\pgfsetlinewidth{0.000000pt}%
\definecolor{currentstroke}{rgb}{0.000000,0.000000,0.000000}%
\pgfsetstrokecolor{currentstroke}%
\pgfsetstrokeopacity{0.000000}%
\pgfsetdash{}{0pt}%
\pgfpathmoveto{\pgfqpoint{5.688041in}{0.639583in}}%
\pgfpathlineto{\pgfqpoint{7.375346in}{0.639583in}}%
\pgfpathlineto{\pgfqpoint{7.375346in}{2.179128in}}%
\pgfpathlineto{\pgfqpoint{5.688041in}{2.179128in}}%
\pgfpathclose%
\pgfusepath{fill}%
\end{pgfscope}%
\begin{pgfscope}%
\pgfsetbuttcap%
\pgfsetroundjoin%
\definecolor{currentfill}{rgb}{0.000000,0.000000,0.000000}%
\pgfsetfillcolor{currentfill}%
\pgfsetlinewidth{0.803000pt}%
\definecolor{currentstroke}{rgb}{0.000000,0.000000,0.000000}%
\pgfsetstrokecolor{currentstroke}%
\pgfsetdash{}{0pt}%
\pgfsys@defobject{currentmarker}{\pgfqpoint{0.000000in}{-0.048611in}}{\pgfqpoint{0.000000in}{0.000000in}}{%
\pgfpathmoveto{\pgfqpoint{0.000000in}{0.000000in}}%
\pgfpathlineto{\pgfqpoint{0.000000in}{-0.048611in}}%
\pgfusepath{stroke,fill}%
}%
\begin{pgfscope}%
\pgfsys@transformshift{5.764736in}{0.639583in}%
\pgfsys@useobject{currentmarker}{}%
\end{pgfscope}%
\end{pgfscope}%
\begin{pgfscope}%
\definecolor{textcolor}{rgb}{0.000000,0.000000,0.000000}%
\pgfsetstrokecolor{textcolor}%
\pgfsetfillcolor{textcolor}%
\pgftext[x=5.764736in,y=0.542361in,,top]{\color{textcolor}\rmfamily\fontsize{10.000000}{12.000000}\selectfont \(\displaystyle {0.0}\)}%
\end{pgfscope}%
\begin{pgfscope}%
\pgfsetbuttcap%
\pgfsetroundjoin%
\definecolor{currentfill}{rgb}{0.000000,0.000000,0.000000}%
\pgfsetfillcolor{currentfill}%
\pgfsetlinewidth{0.803000pt}%
\definecolor{currentstroke}{rgb}{0.000000,0.000000,0.000000}%
\pgfsetstrokecolor{currentstroke}%
\pgfsetdash{}{0pt}%
\pgfsys@defobject{currentmarker}{\pgfqpoint{0.000000in}{-0.048611in}}{\pgfqpoint{0.000000in}{0.000000in}}{%
\pgfpathmoveto{\pgfqpoint{0.000000in}{0.000000in}}%
\pgfpathlineto{\pgfqpoint{0.000000in}{-0.048611in}}%
\pgfusepath{stroke,fill}%
}%
\begin{pgfscope}%
\pgfsys@transformshift{6.531693in}{0.639583in}%
\pgfsys@useobject{currentmarker}{}%
\end{pgfscope}%
\end{pgfscope}%
\begin{pgfscope}%
\definecolor{textcolor}{rgb}{0.000000,0.000000,0.000000}%
\pgfsetstrokecolor{textcolor}%
\pgfsetfillcolor{textcolor}%
\pgftext[x=6.531693in,y=0.542361in,,top]{\color{textcolor}\rmfamily\fontsize{10.000000}{12.000000}\selectfont \(\displaystyle {0.5}\)}%
\end{pgfscope}%
\begin{pgfscope}%
\pgfsetbuttcap%
\pgfsetroundjoin%
\definecolor{currentfill}{rgb}{0.000000,0.000000,0.000000}%
\pgfsetfillcolor{currentfill}%
\pgfsetlinewidth{0.803000pt}%
\definecolor{currentstroke}{rgb}{0.000000,0.000000,0.000000}%
\pgfsetstrokecolor{currentstroke}%
\pgfsetdash{}{0pt}%
\pgfsys@defobject{currentmarker}{\pgfqpoint{0.000000in}{-0.048611in}}{\pgfqpoint{0.000000in}{0.000000in}}{%
\pgfpathmoveto{\pgfqpoint{0.000000in}{0.000000in}}%
\pgfpathlineto{\pgfqpoint{0.000000in}{-0.048611in}}%
\pgfusepath{stroke,fill}%
}%
\begin{pgfscope}%
\pgfsys@transformshift{7.298650in}{0.639583in}%
\pgfsys@useobject{currentmarker}{}%
\end{pgfscope}%
\end{pgfscope}%
\begin{pgfscope}%
\definecolor{textcolor}{rgb}{0.000000,0.000000,0.000000}%
\pgfsetstrokecolor{textcolor}%
\pgfsetfillcolor{textcolor}%
\pgftext[x=7.298650in,y=0.542361in,,top]{\color{textcolor}\rmfamily\fontsize{10.000000}{12.000000}\selectfont \(\displaystyle {1.0}\)}%
\end{pgfscope}%
\begin{pgfscope}%
\definecolor{textcolor}{rgb}{0.000000,0.000000,0.000000}%
\pgfsetstrokecolor{textcolor}%
\pgfsetfillcolor{textcolor}%
\pgftext[x=6.531693in,y=0.363349in,,top]{\color{textcolor}\rmfamily\fontsize{16.000000}{19.200000}\selectfont FPR}%
\end{pgfscope}%
\begin{pgfscope}%
\pgfsetbuttcap%
\pgfsetroundjoin%
\definecolor{currentfill}{rgb}{0.000000,0.000000,0.000000}%
\pgfsetfillcolor{currentfill}%
\pgfsetlinewidth{0.803000pt}%
\definecolor{currentstroke}{rgb}{0.000000,0.000000,0.000000}%
\pgfsetstrokecolor{currentstroke}%
\pgfsetdash{}{0pt}%
\pgfsys@defobject{currentmarker}{\pgfqpoint{-0.048611in}{0.000000in}}{\pgfqpoint{-0.000000in}{0.000000in}}{%
\pgfpathmoveto{\pgfqpoint{-0.000000in}{0.000000in}}%
\pgfpathlineto{\pgfqpoint{-0.048611in}{0.000000in}}%
\pgfusepath{stroke,fill}%
}%
\begin{pgfscope}%
\pgfsys@transformshift{5.688041in}{0.709562in}%
\pgfsys@useobject{currentmarker}{}%
\end{pgfscope}%
\end{pgfscope}%
\begin{pgfscope}%
\definecolor{textcolor}{rgb}{0.000000,0.000000,0.000000}%
\pgfsetstrokecolor{textcolor}%
\pgfsetfillcolor{textcolor}%
\pgftext[x=5.343904in, y=0.661337in, left, base]{\color{textcolor}\rmfamily\fontsize{10.000000}{12.000000}\selectfont \(\displaystyle {0.00}\)}%
\end{pgfscope}%
\begin{pgfscope}%
\pgfsetbuttcap%
\pgfsetroundjoin%
\definecolor{currentfill}{rgb}{0.000000,0.000000,0.000000}%
\pgfsetfillcolor{currentfill}%
\pgfsetlinewidth{0.803000pt}%
\definecolor{currentstroke}{rgb}{0.000000,0.000000,0.000000}%
\pgfsetstrokecolor{currentstroke}%
\pgfsetdash{}{0pt}%
\pgfsys@defobject{currentmarker}{\pgfqpoint{-0.048611in}{0.000000in}}{\pgfqpoint{-0.000000in}{0.000000in}}{%
\pgfpathmoveto{\pgfqpoint{-0.000000in}{0.000000in}}%
\pgfpathlineto{\pgfqpoint{-0.048611in}{0.000000in}}%
\pgfusepath{stroke,fill}%
}%
\begin{pgfscope}%
\pgfsys@transformshift{5.688041in}{1.059459in}%
\pgfsys@useobject{currentmarker}{}%
\end{pgfscope}%
\end{pgfscope}%
\begin{pgfscope}%
\definecolor{textcolor}{rgb}{0.000000,0.000000,0.000000}%
\pgfsetstrokecolor{textcolor}%
\pgfsetfillcolor{textcolor}%
\pgftext[x=5.343904in, y=1.011234in, left, base]{\color{textcolor}\rmfamily\fontsize{10.000000}{12.000000}\selectfont \(\displaystyle {0.25}\)}%
\end{pgfscope}%
\begin{pgfscope}%
\pgfsetbuttcap%
\pgfsetroundjoin%
\definecolor{currentfill}{rgb}{0.000000,0.000000,0.000000}%
\pgfsetfillcolor{currentfill}%
\pgfsetlinewidth{0.803000pt}%
\definecolor{currentstroke}{rgb}{0.000000,0.000000,0.000000}%
\pgfsetstrokecolor{currentstroke}%
\pgfsetdash{}{0pt}%
\pgfsys@defobject{currentmarker}{\pgfqpoint{-0.048611in}{0.000000in}}{\pgfqpoint{-0.000000in}{0.000000in}}{%
\pgfpathmoveto{\pgfqpoint{-0.000000in}{0.000000in}}%
\pgfpathlineto{\pgfqpoint{-0.048611in}{0.000000in}}%
\pgfusepath{stroke,fill}%
}%
\begin{pgfscope}%
\pgfsys@transformshift{5.688041in}{1.409356in}%
\pgfsys@useobject{currentmarker}{}%
\end{pgfscope}%
\end{pgfscope}%
\begin{pgfscope}%
\definecolor{textcolor}{rgb}{0.000000,0.000000,0.000000}%
\pgfsetstrokecolor{textcolor}%
\pgfsetfillcolor{textcolor}%
\pgftext[x=5.343904in, y=1.361130in, left, base]{\color{textcolor}\rmfamily\fontsize{10.000000}{12.000000}\selectfont \(\displaystyle {0.50}\)}%
\end{pgfscope}%
\begin{pgfscope}%
\pgfsetbuttcap%
\pgfsetroundjoin%
\definecolor{currentfill}{rgb}{0.000000,0.000000,0.000000}%
\pgfsetfillcolor{currentfill}%
\pgfsetlinewidth{0.803000pt}%
\definecolor{currentstroke}{rgb}{0.000000,0.000000,0.000000}%
\pgfsetstrokecolor{currentstroke}%
\pgfsetdash{}{0pt}%
\pgfsys@defobject{currentmarker}{\pgfqpoint{-0.048611in}{0.000000in}}{\pgfqpoint{-0.000000in}{0.000000in}}{%
\pgfpathmoveto{\pgfqpoint{-0.000000in}{0.000000in}}%
\pgfpathlineto{\pgfqpoint{-0.048611in}{0.000000in}}%
\pgfusepath{stroke,fill}%
}%
\begin{pgfscope}%
\pgfsys@transformshift{5.688041in}{1.759252in}%
\pgfsys@useobject{currentmarker}{}%
\end{pgfscope}%
\end{pgfscope}%
\begin{pgfscope}%
\definecolor{textcolor}{rgb}{0.000000,0.000000,0.000000}%
\pgfsetstrokecolor{textcolor}%
\pgfsetfillcolor{textcolor}%
\pgftext[x=5.343904in, y=1.711027in, left, base]{\color{textcolor}\rmfamily\fontsize{10.000000}{12.000000}\selectfont \(\displaystyle {0.75}\)}%
\end{pgfscope}%
\begin{pgfscope}%
\pgfsetbuttcap%
\pgfsetroundjoin%
\definecolor{currentfill}{rgb}{0.000000,0.000000,0.000000}%
\pgfsetfillcolor{currentfill}%
\pgfsetlinewidth{0.803000pt}%
\definecolor{currentstroke}{rgb}{0.000000,0.000000,0.000000}%
\pgfsetstrokecolor{currentstroke}%
\pgfsetdash{}{0pt}%
\pgfsys@defobject{currentmarker}{\pgfqpoint{-0.048611in}{0.000000in}}{\pgfqpoint{-0.000000in}{0.000000in}}{%
\pgfpathmoveto{\pgfqpoint{-0.000000in}{0.000000in}}%
\pgfpathlineto{\pgfqpoint{-0.048611in}{0.000000in}}%
\pgfusepath{stroke,fill}%
}%
\begin{pgfscope}%
\pgfsys@transformshift{5.688041in}{2.109149in}%
\pgfsys@useobject{currentmarker}{}%
\end{pgfscope}%
\end{pgfscope}%
\begin{pgfscope}%
\definecolor{textcolor}{rgb}{0.000000,0.000000,0.000000}%
\pgfsetstrokecolor{textcolor}%
\pgfsetfillcolor{textcolor}%
\pgftext[x=5.343904in, y=2.060924in, left, base]{\color{textcolor}\rmfamily\fontsize{10.000000}{12.000000}\selectfont \(\displaystyle {1.00}\)}%
\end{pgfscope}%
\begin{pgfscope}%
\definecolor{textcolor}{rgb}{0.000000,0.000000,0.000000}%
\pgfsetstrokecolor{textcolor}%
\pgfsetfillcolor{textcolor}%
\pgftext[x=5.288349in,y=1.409356in,,bottom,rotate=90.000000]{\color{textcolor}\rmfamily\fontsize{16.000000}{19.200000}\selectfont TPR}%
\end{pgfscope}%
\begin{pgfscope}%
\pgfpathrectangle{\pgfqpoint{5.688041in}{0.639583in}}{\pgfqpoint{1.687305in}{1.539545in}}%
\pgfusepath{clip}%
\pgfsetrectcap%
\pgfsetroundjoin%
\pgfsetlinewidth{1.505625pt}%
\definecolor{currentstroke}{rgb}{0.000000,0.501961,0.000000}%
\pgfsetstrokecolor{currentstroke}%
\pgfsetdash{}{0pt}%
\pgfpathmoveto{\pgfqpoint{5.764736in}{0.709562in}}%
\pgfpathlineto{\pgfqpoint{5.765718in}{0.844081in}}%
\pgfpathlineto{\pgfqpoint{5.767246in}{1.025645in}}%
\pgfpathlineto{\pgfqpoint{5.767301in}{1.025645in}}%
\pgfpathlineto{\pgfqpoint{5.768828in}{1.151344in}}%
\pgfpathlineto{\pgfqpoint{5.769374in}{1.207209in}}%
\pgfpathlineto{\pgfqpoint{5.770847in}{1.291008in}}%
\pgfpathlineto{\pgfqpoint{5.770956in}{1.291008in}}%
\pgfpathlineto{\pgfqpoint{5.772483in}{1.413031in}}%
\pgfpathlineto{\pgfqpoint{5.772538in}{1.413031in}}%
\pgfpathlineto{\pgfqpoint{5.774011in}{1.486539in}}%
\pgfpathlineto{\pgfqpoint{5.774175in}{1.488009in}}%
\pgfpathlineto{\pgfqpoint{5.775648in}{1.544610in}}%
\pgfpathlineto{\pgfqpoint{5.775811in}{1.546080in}}%
\pgfpathlineto{\pgfqpoint{5.777339in}{1.630614in}}%
\pgfpathlineto{\pgfqpoint{5.777448in}{1.630614in}}%
\pgfpathlineto{\pgfqpoint{5.778921in}{1.662957in}}%
\pgfpathlineto{\pgfqpoint{5.779085in}{1.664427in}}%
\pgfpathlineto{\pgfqpoint{5.780612in}{1.695301in}}%
\pgfpathlineto{\pgfqpoint{5.780885in}{1.696771in}}%
\pgfpathlineto{\pgfqpoint{5.782358in}{1.737200in}}%
\pgfpathlineto{\pgfqpoint{5.782467in}{1.737200in}}%
\pgfpathlineto{\pgfqpoint{5.783995in}{1.773954in}}%
\pgfpathlineto{\pgfqpoint{5.784213in}{1.774689in}}%
\pgfpathlineto{\pgfqpoint{5.785577in}{1.796006in}}%
\pgfpathlineto{\pgfqpoint{5.786013in}{1.796741in}}%
\pgfpathlineto{\pgfqpoint{5.787432in}{1.845256in}}%
\pgfpathlineto{\pgfqpoint{5.787596in}{1.845256in}}%
\pgfpathlineto{\pgfqpoint{5.789123in}{1.879070in}}%
\pgfpathlineto{\pgfqpoint{5.789178in}{1.879070in}}%
\pgfpathlineto{\pgfqpoint{5.790705in}{1.907003in}}%
\pgfpathlineto{\pgfqpoint{5.790869in}{1.908473in}}%
\pgfpathlineto{\pgfqpoint{5.792397in}{1.931260in}}%
\pgfpathlineto{\pgfqpoint{5.792997in}{1.931995in}}%
\pgfpathlineto{\pgfqpoint{5.794415in}{1.954048in}}%
\pgfpathlineto{\pgfqpoint{5.794524in}{1.954048in}}%
\pgfpathlineto{\pgfqpoint{5.794633in}{1.954048in}}%
\pgfpathlineto{\pgfqpoint{5.795943in}{1.968014in}}%
\pgfpathlineto{\pgfqpoint{5.796434in}{1.968014in}}%
\pgfpathlineto{\pgfqpoint{5.797743in}{1.984186in}}%
\pgfpathlineto{\pgfqpoint{5.798289in}{1.985656in}}%
\pgfpathlineto{\pgfqpoint{5.799816in}{1.999622in}}%
\pgfpathlineto{\pgfqpoint{5.799980in}{2.001093in}}%
\pgfpathlineto{\pgfqpoint{5.801289in}{2.013589in}}%
\pgfpathlineto{\pgfqpoint{5.801944in}{2.015059in}}%
\pgfpathlineto{\pgfqpoint{5.803144in}{2.023145in}}%
\pgfpathlineto{\pgfqpoint{5.803744in}{2.024615in}}%
\pgfpathlineto{\pgfqpoint{5.805272in}{2.031231in}}%
\pgfpathlineto{\pgfqpoint{5.806363in}{2.032701in}}%
\pgfpathlineto{\pgfqpoint{5.807891in}{2.047402in}}%
\pgfpathlineto{\pgfqpoint{5.808982in}{2.048873in}}%
\pgfpathlineto{\pgfqpoint{5.810509in}{2.053283in}}%
\pgfpathlineto{\pgfqpoint{5.810946in}{2.054753in}}%
\pgfpathlineto{\pgfqpoint{5.812092in}{2.062839in}}%
\pgfpathlineto{\pgfqpoint{5.813128in}{2.064309in}}%
\pgfpathlineto{\pgfqpoint{5.814383in}{2.071660in}}%
\pgfpathlineto{\pgfqpoint{5.815147in}{2.071660in}}%
\pgfpathlineto{\pgfqpoint{5.816674in}{2.079011in}}%
\pgfpathlineto{\pgfqpoint{5.817875in}{2.080481in}}%
\pgfpathlineto{\pgfqpoint{5.819348in}{2.084891in}}%
\pgfpathlineto{\pgfqpoint{5.820602in}{2.086361in}}%
\pgfpathlineto{\pgfqpoint{5.822130in}{2.088567in}}%
\pgfpathlineto{\pgfqpoint{5.825240in}{2.089302in}}%
\pgfpathlineto{\pgfqpoint{5.825894in}{2.092977in}}%
\pgfpathlineto{\pgfqpoint{5.830041in}{2.094447in}}%
\pgfpathlineto{\pgfqpoint{5.830641in}{2.096653in}}%
\pgfpathlineto{\pgfqpoint{5.837297in}{2.098123in}}%
\pgfpathlineto{\pgfqpoint{5.838224in}{2.099593in}}%
\pgfpathlineto{\pgfqpoint{5.845862in}{2.101063in}}%
\pgfpathlineto{\pgfqpoint{5.845862in}{2.101798in}}%
\pgfpathlineto{\pgfqpoint{5.888525in}{2.103268in}}%
\pgfpathlineto{\pgfqpoint{5.888525in}{2.104003in}}%
\pgfpathlineto{\pgfqpoint{5.917822in}{2.105473in}}%
\pgfpathlineto{\pgfqpoint{5.917822in}{2.106209in}}%
\pgfpathlineto{\pgfqpoint{5.983290in}{2.107679in}}%
\pgfpathlineto{\pgfqpoint{5.983290in}{2.108414in}}%
\pgfpathlineto{\pgfqpoint{7.298650in}{2.109149in}}%
\pgfpathlineto{\pgfqpoint{7.298650in}{2.109149in}}%
\pgfusepath{stroke}%
\end{pgfscope}%
\begin{pgfscope}%
\pgfpathrectangle{\pgfqpoint{5.688041in}{0.639583in}}{\pgfqpoint{1.687305in}{1.539545in}}%
\pgfusepath{clip}%
\pgfsetrectcap%
\pgfsetroundjoin%
\pgfsetlinewidth{1.505625pt}%
\definecolor{currentstroke}{rgb}{0.501961,0.501961,0.501961}%
\pgfsetstrokecolor{currentstroke}%
\pgfsetdash{}{0pt}%
\pgfpathmoveto{\pgfqpoint{5.764736in}{0.709562in}}%
\pgfpathlineto{\pgfqpoint{7.298650in}{2.109149in}}%
\pgfusepath{stroke}%
\end{pgfscope}%
\begin{pgfscope}%
\pgfsetrectcap%
\pgfsetmiterjoin%
\pgfsetlinewidth{0.803000pt}%
\definecolor{currentstroke}{rgb}{0.000000,0.000000,0.000000}%
\pgfsetstrokecolor{currentstroke}%
\pgfsetdash{}{0pt}%
\pgfpathmoveto{\pgfqpoint{5.688041in}{0.639583in}}%
\pgfpathlineto{\pgfqpoint{5.688041in}{2.179128in}}%
\pgfusepath{stroke}%
\end{pgfscope}%
\begin{pgfscope}%
\pgfsetrectcap%
\pgfsetmiterjoin%
\pgfsetlinewidth{0.803000pt}%
\definecolor{currentstroke}{rgb}{0.000000,0.000000,0.000000}%
\pgfsetstrokecolor{currentstroke}%
\pgfsetdash{}{0pt}%
\pgfpathmoveto{\pgfqpoint{7.375346in}{0.639583in}}%
\pgfpathlineto{\pgfqpoint{7.375346in}{2.179128in}}%
\pgfusepath{stroke}%
\end{pgfscope}%
\begin{pgfscope}%
\pgfsetrectcap%
\pgfsetmiterjoin%
\pgfsetlinewidth{0.803000pt}%
\definecolor{currentstroke}{rgb}{0.000000,0.000000,0.000000}%
\pgfsetstrokecolor{currentstroke}%
\pgfsetdash{}{0pt}%
\pgfpathmoveto{\pgfqpoint{5.688041in}{0.639583in}}%
\pgfpathlineto{\pgfqpoint{7.375346in}{0.639583in}}%
\pgfusepath{stroke}%
\end{pgfscope}%
\begin{pgfscope}%
\pgfsetrectcap%
\pgfsetmiterjoin%
\pgfsetlinewidth{0.803000pt}%
\definecolor{currentstroke}{rgb}{0.000000,0.000000,0.000000}%
\pgfsetstrokecolor{currentstroke}%
\pgfsetdash{}{0pt}%
\pgfpathmoveto{\pgfqpoint{5.688041in}{2.179128in}}%
\pgfpathlineto{\pgfqpoint{7.375346in}{2.179128in}}%
\pgfusepath{stroke}%
\end{pgfscope}%
\begin{pgfscope}%
\definecolor{textcolor}{rgb}{0.000000,0.000000,0.000000}%
\pgfsetstrokecolor{textcolor}%
\pgfsetfillcolor{textcolor}%
\pgftext[x=6.531693in,y=2.262462in,,base]{\color{textcolor}\rmfamily\fontsize{20.000000}{24.000000}\selectfont COVID-19}%
\end{pgfscope}%
\begin{pgfscope}%
\pgfsetbuttcap%
\pgfsetmiterjoin%
\definecolor{currentfill}{rgb}{1.000000,1.000000,1.000000}%
\pgfsetfillcolor{currentfill}%
\pgfsetfillopacity{0.800000}%
\pgfsetlinewidth{1.003750pt}%
\definecolor{currentstroke}{rgb}{0.800000,0.800000,0.800000}%
\pgfsetstrokecolor{currentstroke}%
\pgfsetstrokeopacity{0.800000}%
\pgfsetdash{}{0pt}%
\pgfpathmoveto{\pgfqpoint{6.166240in}{0.709028in}}%
\pgfpathlineto{\pgfqpoint{7.278124in}{0.709028in}}%
\pgfpathquadraticcurveto{\pgfqpoint{7.305902in}{0.709028in}}{\pgfqpoint{7.305902in}{0.736805in}}%
\pgfpathlineto{\pgfqpoint{7.305902in}{0.916589in}}%
\pgfpathquadraticcurveto{\pgfqpoint{7.305902in}{0.944367in}}{\pgfqpoint{7.278124in}{0.944367in}}%
\pgfpathlineto{\pgfqpoint{6.166240in}{0.944367in}}%
\pgfpathquadraticcurveto{\pgfqpoint{6.138462in}{0.944367in}}{\pgfqpoint{6.138462in}{0.916589in}}%
\pgfpathlineto{\pgfqpoint{6.138462in}{0.736805in}}%
\pgfpathquadraticcurveto{\pgfqpoint{6.138462in}{0.709028in}}{\pgfqpoint{6.166240in}{0.709028in}}%
\pgfpathclose%
\pgfusepath{stroke,fill}%
\end{pgfscope}%
\begin{pgfscope}%
\pgfsetrectcap%
\pgfsetroundjoin%
\pgfsetlinewidth{1.505625pt}%
\definecolor{currentstroke}{rgb}{0.000000,0.501961,0.000000}%
\pgfsetstrokecolor{currentstroke}%
\pgfsetdash{}{0pt}%
\pgfpathmoveto{\pgfqpoint{6.194018in}{0.840200in}}%
\pgfpathlineto{\pgfqpoint{6.471795in}{0.840200in}}%
\pgfusepath{stroke}%
\end{pgfscope}%
\begin{pgfscope}%
\definecolor{textcolor}{rgb}{0.000000,0.000000,0.000000}%
\pgfsetstrokecolor{textcolor}%
\pgfsetfillcolor{textcolor}%
\pgftext[x=6.582907in,y=0.791589in,left,base]{\color{textcolor}\rmfamily\fontsize{10.000000}{12.000000}\selectfont AUC 0.991}%
\end{pgfscope}%
\begin{pgfscope}%
\pgfsetbuttcap%
\pgfsetmiterjoin%
\definecolor{currentfill}{rgb}{1.000000,1.000000,1.000000}%
\pgfsetfillcolor{currentfill}%
\pgfsetlinewidth{0.000000pt}%
\definecolor{currentstroke}{rgb}{0.000000,0.000000,0.000000}%
\pgfsetstrokecolor{currentstroke}%
\pgfsetstrokeopacity{0.000000}%
\pgfsetdash{}{0pt}%
\pgfpathmoveto{\pgfqpoint{8.150541in}{0.639583in}}%
\pgfpathlineto{\pgfqpoint{9.837846in}{0.639583in}}%
\pgfpathlineto{\pgfqpoint{9.837846in}{2.179128in}}%
\pgfpathlineto{\pgfqpoint{8.150541in}{2.179128in}}%
\pgfpathclose%
\pgfusepath{fill}%
\end{pgfscope}%
\begin{pgfscope}%
\pgfsetbuttcap%
\pgfsetroundjoin%
\definecolor{currentfill}{rgb}{0.000000,0.000000,0.000000}%
\pgfsetfillcolor{currentfill}%
\pgfsetlinewidth{0.803000pt}%
\definecolor{currentstroke}{rgb}{0.000000,0.000000,0.000000}%
\pgfsetstrokecolor{currentstroke}%
\pgfsetdash{}{0pt}%
\pgfsys@defobject{currentmarker}{\pgfqpoint{0.000000in}{-0.048611in}}{\pgfqpoint{0.000000in}{0.000000in}}{%
\pgfpathmoveto{\pgfqpoint{0.000000in}{0.000000in}}%
\pgfpathlineto{\pgfqpoint{0.000000in}{-0.048611in}}%
\pgfusepath{stroke,fill}%
}%
\begin{pgfscope}%
\pgfsys@transformshift{8.227236in}{0.639583in}%
\pgfsys@useobject{currentmarker}{}%
\end{pgfscope}%
\end{pgfscope}%
\begin{pgfscope}%
\definecolor{textcolor}{rgb}{0.000000,0.000000,0.000000}%
\pgfsetstrokecolor{textcolor}%
\pgfsetfillcolor{textcolor}%
\pgftext[x=8.227236in,y=0.542361in,,top]{\color{textcolor}\rmfamily\fontsize{10.000000}{12.000000}\selectfont \(\displaystyle {0.0}\)}%
\end{pgfscope}%
\begin{pgfscope}%
\pgfsetbuttcap%
\pgfsetroundjoin%
\definecolor{currentfill}{rgb}{0.000000,0.000000,0.000000}%
\pgfsetfillcolor{currentfill}%
\pgfsetlinewidth{0.803000pt}%
\definecolor{currentstroke}{rgb}{0.000000,0.000000,0.000000}%
\pgfsetstrokecolor{currentstroke}%
\pgfsetdash{}{0pt}%
\pgfsys@defobject{currentmarker}{\pgfqpoint{0.000000in}{-0.048611in}}{\pgfqpoint{0.000000in}{0.000000in}}{%
\pgfpathmoveto{\pgfqpoint{0.000000in}{0.000000in}}%
\pgfpathlineto{\pgfqpoint{0.000000in}{-0.048611in}}%
\pgfusepath{stroke,fill}%
}%
\begin{pgfscope}%
\pgfsys@transformshift{8.994193in}{0.639583in}%
\pgfsys@useobject{currentmarker}{}%
\end{pgfscope}%
\end{pgfscope}%
\begin{pgfscope}%
\definecolor{textcolor}{rgb}{0.000000,0.000000,0.000000}%
\pgfsetstrokecolor{textcolor}%
\pgfsetfillcolor{textcolor}%
\pgftext[x=8.994193in,y=0.542361in,,top]{\color{textcolor}\rmfamily\fontsize{10.000000}{12.000000}\selectfont \(\displaystyle {0.5}\)}%
\end{pgfscope}%
\begin{pgfscope}%
\pgfsetbuttcap%
\pgfsetroundjoin%
\definecolor{currentfill}{rgb}{0.000000,0.000000,0.000000}%
\pgfsetfillcolor{currentfill}%
\pgfsetlinewidth{0.803000pt}%
\definecolor{currentstroke}{rgb}{0.000000,0.000000,0.000000}%
\pgfsetstrokecolor{currentstroke}%
\pgfsetdash{}{0pt}%
\pgfsys@defobject{currentmarker}{\pgfqpoint{0.000000in}{-0.048611in}}{\pgfqpoint{0.000000in}{0.000000in}}{%
\pgfpathmoveto{\pgfqpoint{0.000000in}{0.000000in}}%
\pgfpathlineto{\pgfqpoint{0.000000in}{-0.048611in}}%
\pgfusepath{stroke,fill}%
}%
\begin{pgfscope}%
\pgfsys@transformshift{9.761150in}{0.639583in}%
\pgfsys@useobject{currentmarker}{}%
\end{pgfscope}%
\end{pgfscope}%
\begin{pgfscope}%
\definecolor{textcolor}{rgb}{0.000000,0.000000,0.000000}%
\pgfsetstrokecolor{textcolor}%
\pgfsetfillcolor{textcolor}%
\pgftext[x=9.761150in,y=0.542361in,,top]{\color{textcolor}\rmfamily\fontsize{10.000000}{12.000000}\selectfont \(\displaystyle {1.0}\)}%
\end{pgfscope}%
\begin{pgfscope}%
\definecolor{textcolor}{rgb}{0.000000,0.000000,0.000000}%
\pgfsetstrokecolor{textcolor}%
\pgfsetfillcolor{textcolor}%
\pgftext[x=8.994193in,y=0.363349in,,top]{\color{textcolor}\rmfamily\fontsize{16.000000}{19.200000}\selectfont FPR}%
\end{pgfscope}%
\begin{pgfscope}%
\pgfsetbuttcap%
\pgfsetroundjoin%
\definecolor{currentfill}{rgb}{0.000000,0.000000,0.000000}%
\pgfsetfillcolor{currentfill}%
\pgfsetlinewidth{0.803000pt}%
\definecolor{currentstroke}{rgb}{0.000000,0.000000,0.000000}%
\pgfsetstrokecolor{currentstroke}%
\pgfsetdash{}{0pt}%
\pgfsys@defobject{currentmarker}{\pgfqpoint{-0.048611in}{0.000000in}}{\pgfqpoint{-0.000000in}{0.000000in}}{%
\pgfpathmoveto{\pgfqpoint{-0.000000in}{0.000000in}}%
\pgfpathlineto{\pgfqpoint{-0.048611in}{0.000000in}}%
\pgfusepath{stroke,fill}%
}%
\begin{pgfscope}%
\pgfsys@transformshift{8.150541in}{0.709562in}%
\pgfsys@useobject{currentmarker}{}%
\end{pgfscope}%
\end{pgfscope}%
\begin{pgfscope}%
\definecolor{textcolor}{rgb}{0.000000,0.000000,0.000000}%
\pgfsetstrokecolor{textcolor}%
\pgfsetfillcolor{textcolor}%
\pgftext[x=7.806404in, y=0.661337in, left, base]{\color{textcolor}\rmfamily\fontsize{10.000000}{12.000000}\selectfont \(\displaystyle {0.00}\)}%
\end{pgfscope}%
\begin{pgfscope}%
\pgfsetbuttcap%
\pgfsetroundjoin%
\definecolor{currentfill}{rgb}{0.000000,0.000000,0.000000}%
\pgfsetfillcolor{currentfill}%
\pgfsetlinewidth{0.803000pt}%
\definecolor{currentstroke}{rgb}{0.000000,0.000000,0.000000}%
\pgfsetstrokecolor{currentstroke}%
\pgfsetdash{}{0pt}%
\pgfsys@defobject{currentmarker}{\pgfqpoint{-0.048611in}{0.000000in}}{\pgfqpoint{-0.000000in}{0.000000in}}{%
\pgfpathmoveto{\pgfqpoint{-0.000000in}{0.000000in}}%
\pgfpathlineto{\pgfqpoint{-0.048611in}{0.000000in}}%
\pgfusepath{stroke,fill}%
}%
\begin{pgfscope}%
\pgfsys@transformshift{8.150541in}{1.059459in}%
\pgfsys@useobject{currentmarker}{}%
\end{pgfscope}%
\end{pgfscope}%
\begin{pgfscope}%
\definecolor{textcolor}{rgb}{0.000000,0.000000,0.000000}%
\pgfsetstrokecolor{textcolor}%
\pgfsetfillcolor{textcolor}%
\pgftext[x=7.806404in, y=1.011234in, left, base]{\color{textcolor}\rmfamily\fontsize{10.000000}{12.000000}\selectfont \(\displaystyle {0.25}\)}%
\end{pgfscope}%
\begin{pgfscope}%
\pgfsetbuttcap%
\pgfsetroundjoin%
\definecolor{currentfill}{rgb}{0.000000,0.000000,0.000000}%
\pgfsetfillcolor{currentfill}%
\pgfsetlinewidth{0.803000pt}%
\definecolor{currentstroke}{rgb}{0.000000,0.000000,0.000000}%
\pgfsetstrokecolor{currentstroke}%
\pgfsetdash{}{0pt}%
\pgfsys@defobject{currentmarker}{\pgfqpoint{-0.048611in}{0.000000in}}{\pgfqpoint{-0.000000in}{0.000000in}}{%
\pgfpathmoveto{\pgfqpoint{-0.000000in}{0.000000in}}%
\pgfpathlineto{\pgfqpoint{-0.048611in}{0.000000in}}%
\pgfusepath{stroke,fill}%
}%
\begin{pgfscope}%
\pgfsys@transformshift{8.150541in}{1.409356in}%
\pgfsys@useobject{currentmarker}{}%
\end{pgfscope}%
\end{pgfscope}%
\begin{pgfscope}%
\definecolor{textcolor}{rgb}{0.000000,0.000000,0.000000}%
\pgfsetstrokecolor{textcolor}%
\pgfsetfillcolor{textcolor}%
\pgftext[x=7.806404in, y=1.361130in, left, base]{\color{textcolor}\rmfamily\fontsize{10.000000}{12.000000}\selectfont \(\displaystyle {0.50}\)}%
\end{pgfscope}%
\begin{pgfscope}%
\pgfsetbuttcap%
\pgfsetroundjoin%
\definecolor{currentfill}{rgb}{0.000000,0.000000,0.000000}%
\pgfsetfillcolor{currentfill}%
\pgfsetlinewidth{0.803000pt}%
\definecolor{currentstroke}{rgb}{0.000000,0.000000,0.000000}%
\pgfsetstrokecolor{currentstroke}%
\pgfsetdash{}{0pt}%
\pgfsys@defobject{currentmarker}{\pgfqpoint{-0.048611in}{0.000000in}}{\pgfqpoint{-0.000000in}{0.000000in}}{%
\pgfpathmoveto{\pgfqpoint{-0.000000in}{0.000000in}}%
\pgfpathlineto{\pgfqpoint{-0.048611in}{0.000000in}}%
\pgfusepath{stroke,fill}%
}%
\begin{pgfscope}%
\pgfsys@transformshift{8.150541in}{1.759252in}%
\pgfsys@useobject{currentmarker}{}%
\end{pgfscope}%
\end{pgfscope}%
\begin{pgfscope}%
\definecolor{textcolor}{rgb}{0.000000,0.000000,0.000000}%
\pgfsetstrokecolor{textcolor}%
\pgfsetfillcolor{textcolor}%
\pgftext[x=7.806404in, y=1.711027in, left, base]{\color{textcolor}\rmfamily\fontsize{10.000000}{12.000000}\selectfont \(\displaystyle {0.75}\)}%
\end{pgfscope}%
\begin{pgfscope}%
\pgfsetbuttcap%
\pgfsetroundjoin%
\definecolor{currentfill}{rgb}{0.000000,0.000000,0.000000}%
\pgfsetfillcolor{currentfill}%
\pgfsetlinewidth{0.803000pt}%
\definecolor{currentstroke}{rgb}{0.000000,0.000000,0.000000}%
\pgfsetstrokecolor{currentstroke}%
\pgfsetdash{}{0pt}%
\pgfsys@defobject{currentmarker}{\pgfqpoint{-0.048611in}{0.000000in}}{\pgfqpoint{-0.000000in}{0.000000in}}{%
\pgfpathmoveto{\pgfqpoint{-0.000000in}{0.000000in}}%
\pgfpathlineto{\pgfqpoint{-0.048611in}{0.000000in}}%
\pgfusepath{stroke,fill}%
}%
\begin{pgfscope}%
\pgfsys@transformshift{8.150541in}{2.109149in}%
\pgfsys@useobject{currentmarker}{}%
\end{pgfscope}%
\end{pgfscope}%
\begin{pgfscope}%
\definecolor{textcolor}{rgb}{0.000000,0.000000,0.000000}%
\pgfsetstrokecolor{textcolor}%
\pgfsetfillcolor{textcolor}%
\pgftext[x=7.806404in, y=2.060924in, left, base]{\color{textcolor}\rmfamily\fontsize{10.000000}{12.000000}\selectfont \(\displaystyle {1.00}\)}%
\end{pgfscope}%
\begin{pgfscope}%
\definecolor{textcolor}{rgb}{0.000000,0.000000,0.000000}%
\pgfsetstrokecolor{textcolor}%
\pgfsetfillcolor{textcolor}%
\pgftext[x=7.750849in,y=1.409356in,,bottom,rotate=90.000000]{\color{textcolor}\rmfamily\fontsize{16.000000}{19.200000}\selectfont TPR}%
\end{pgfscope}%
\begin{pgfscope}%
\pgfpathrectangle{\pgfqpoint{8.150541in}{0.639583in}}{\pgfqpoint{1.687305in}{1.539545in}}%
\pgfusepath{clip}%
\pgfsetrectcap%
\pgfsetroundjoin%
\pgfsetlinewidth{1.505625pt}%
\definecolor{currentstroke}{rgb}{0.000000,0.501961,0.000000}%
\pgfsetstrokecolor{currentstroke}%
\pgfsetdash{}{0pt}%
\pgfpathmoveto{\pgfqpoint{8.227236in}{0.709562in}}%
\pgfpathlineto{\pgfqpoint{8.228751in}{0.781281in}}%
\pgfpathlineto{\pgfqpoint{8.228835in}{0.781281in}}%
\pgfpathlineto{\pgfqpoint{8.230349in}{0.803011in}}%
\pgfpathlineto{\pgfqpoint{8.230433in}{0.803011in}}%
\pgfpathlineto{\pgfqpoint{8.231948in}{0.818566in}}%
\pgfpathlineto{\pgfqpoint{8.232116in}{0.819159in}}%
\pgfpathlineto{\pgfqpoint{8.233630in}{0.839108in}}%
\pgfpathlineto{\pgfqpoint{8.233883in}{0.839939in}}%
\pgfpathlineto{\pgfqpoint{8.235397in}{0.861549in}}%
\pgfpathlineto{\pgfqpoint{8.235649in}{0.862737in}}%
\pgfpathlineto{\pgfqpoint{8.237164in}{0.877935in}}%
\pgfpathlineto{\pgfqpoint{8.237332in}{0.878885in}}%
\pgfpathlineto{\pgfqpoint{8.238846in}{0.895153in}}%
\pgfpathlineto{\pgfqpoint{8.239099in}{0.896340in}}%
\pgfpathlineto{\pgfqpoint{8.240613in}{0.909164in}}%
\pgfpathlineto{\pgfqpoint{8.240865in}{0.910470in}}%
\pgfpathlineto{\pgfqpoint{8.242380in}{0.921394in}}%
\pgfpathlineto{\pgfqpoint{8.242548in}{0.921869in}}%
\pgfpathlineto{\pgfqpoint{8.244062in}{0.933981in}}%
\pgfpathlineto{\pgfqpoint{8.244483in}{0.934931in}}%
\pgfpathlineto{\pgfqpoint{8.245997in}{0.942886in}}%
\pgfpathlineto{\pgfqpoint{8.246165in}{0.943242in}}%
\pgfpathlineto{\pgfqpoint{8.247595in}{0.952267in}}%
\pgfpathlineto{\pgfqpoint{8.247932in}{0.953454in}}%
\pgfpathlineto{\pgfqpoint{8.249278in}{0.963191in}}%
\pgfpathlineto{\pgfqpoint{8.249615in}{0.963191in}}%
\pgfpathlineto{\pgfqpoint{8.251129in}{0.974471in}}%
\pgfpathlineto{\pgfqpoint{8.251802in}{0.975777in}}%
\pgfpathlineto{\pgfqpoint{8.253316in}{0.979933in}}%
\pgfpathlineto{\pgfqpoint{8.253737in}{0.980408in}}%
\pgfpathlineto{\pgfqpoint{8.255251in}{0.992638in}}%
\pgfpathlineto{\pgfqpoint{8.255419in}{0.992994in}}%
\pgfpathlineto{\pgfqpoint{8.256934in}{1.001781in}}%
\pgfpathlineto{\pgfqpoint{8.257354in}{1.003206in}}%
\pgfpathlineto{\pgfqpoint{8.258785in}{1.007481in}}%
\pgfpathlineto{\pgfqpoint{8.259121in}{1.008787in}}%
\pgfpathlineto{\pgfqpoint{8.260551in}{1.014486in}}%
\pgfpathlineto{\pgfqpoint{8.260720in}{1.014486in}}%
\pgfpathlineto{\pgfqpoint{8.262066in}{1.023511in}}%
\pgfpathlineto{\pgfqpoint{8.263075in}{1.024817in}}%
\pgfpathlineto{\pgfqpoint{8.264589in}{1.030991in}}%
\pgfpathlineto{\pgfqpoint{8.264758in}{1.031941in}}%
\pgfpathlineto{\pgfqpoint{8.266272in}{1.040015in}}%
\pgfpathlineto{\pgfqpoint{8.266861in}{1.041321in}}%
\pgfpathlineto{\pgfqpoint{8.268375in}{1.050939in}}%
\pgfpathlineto{\pgfqpoint{8.269301in}{1.052246in}}%
\pgfpathlineto{\pgfqpoint{8.270647in}{1.062932in}}%
\pgfpathlineto{\pgfqpoint{8.271488in}{1.064357in}}%
\pgfpathlineto{\pgfqpoint{8.273002in}{1.070057in}}%
\pgfpathlineto{\pgfqpoint{8.273507in}{1.071481in}}%
\pgfpathlineto{\pgfqpoint{8.275021in}{1.077893in}}%
\pgfpathlineto{\pgfqpoint{8.275358in}{1.079081in}}%
\pgfpathlineto{\pgfqpoint{8.276872in}{1.085018in}}%
\pgfpathlineto{\pgfqpoint{8.277545in}{1.086205in}}%
\pgfpathlineto{\pgfqpoint{8.279060in}{1.095704in}}%
\pgfpathlineto{\pgfqpoint{8.279733in}{1.096892in}}%
\pgfpathlineto{\pgfqpoint{8.281247in}{1.100810in}}%
\pgfpathlineto{\pgfqpoint{8.281583in}{1.101285in}}%
\pgfpathlineto{\pgfqpoint{8.283098in}{1.108410in}}%
\pgfpathlineto{\pgfqpoint{8.283434in}{1.109478in}}%
\pgfpathlineto{\pgfqpoint{8.284949in}{1.112447in}}%
\pgfpathlineto{\pgfqpoint{8.285453in}{1.113159in}}%
\pgfpathlineto{\pgfqpoint{8.286968in}{1.119452in}}%
\pgfpathlineto{\pgfqpoint{8.287472in}{1.120640in}}%
\pgfpathlineto{\pgfqpoint{8.288903in}{1.126339in}}%
\pgfpathlineto{\pgfqpoint{8.289828in}{1.127645in}}%
\pgfpathlineto{\pgfqpoint{8.291006in}{1.130851in}}%
\pgfpathlineto{\pgfqpoint{8.291679in}{1.131801in}}%
\pgfpathlineto{\pgfqpoint{8.293193in}{1.136076in}}%
\pgfpathlineto{\pgfqpoint{8.294034in}{1.137501in}}%
\pgfpathlineto{\pgfqpoint{8.295465in}{1.141894in}}%
\pgfpathlineto{\pgfqpoint{8.295969in}{1.143082in}}%
\pgfpathlineto{\pgfqpoint{8.297484in}{1.149019in}}%
\pgfpathlineto{\pgfqpoint{8.297652in}{1.150206in}}%
\pgfpathlineto{\pgfqpoint{8.299166in}{1.153531in}}%
\pgfpathlineto{\pgfqpoint{8.300092in}{1.154956in}}%
\pgfpathlineto{\pgfqpoint{8.301438in}{1.157568in}}%
\pgfpathlineto{\pgfqpoint{8.302531in}{1.158874in}}%
\pgfpathlineto{\pgfqpoint{8.303962in}{1.162555in}}%
\pgfpathlineto{\pgfqpoint{8.304550in}{1.163742in}}%
\pgfpathlineto{\pgfqpoint{8.306065in}{1.168848in}}%
\pgfpathlineto{\pgfqpoint{8.306401in}{1.170154in}}%
\pgfpathlineto{\pgfqpoint{8.307916in}{1.178347in}}%
\pgfpathlineto{\pgfqpoint{8.308420in}{1.178703in}}%
\pgfpathlineto{\pgfqpoint{8.309851in}{1.183216in}}%
\pgfpathlineto{\pgfqpoint{8.310944in}{1.184640in}}%
\pgfpathlineto{\pgfqpoint{8.312459in}{1.189153in}}%
\pgfpathlineto{\pgfqpoint{8.313132in}{1.190459in}}%
\pgfpathlineto{\pgfqpoint{8.314646in}{1.194852in}}%
\pgfpathlineto{\pgfqpoint{8.314898in}{1.196158in}}%
\pgfpathlineto{\pgfqpoint{8.316413in}{1.200789in}}%
\pgfpathlineto{\pgfqpoint{8.317086in}{1.201976in}}%
\pgfpathlineto{\pgfqpoint{8.318600in}{1.205064in}}%
\pgfpathlineto{\pgfqpoint{8.318936in}{1.206370in}}%
\pgfpathlineto{\pgfqpoint{8.320451in}{1.212069in}}%
\pgfpathlineto{\pgfqpoint{8.320871in}{1.213376in}}%
\pgfpathlineto{\pgfqpoint{8.322386in}{1.217413in}}%
\pgfpathlineto{\pgfqpoint{8.323648in}{1.218600in}}%
\pgfpathlineto{\pgfqpoint{8.325162in}{1.222400in}}%
\pgfpathlineto{\pgfqpoint{8.326508in}{1.223825in}}%
\pgfpathlineto{\pgfqpoint{8.328022in}{1.226793in}}%
\pgfpathlineto{\pgfqpoint{8.328527in}{1.227624in}}%
\pgfpathlineto{\pgfqpoint{8.330041in}{1.233086in}}%
\pgfpathlineto{\pgfqpoint{8.330546in}{1.234036in}}%
\pgfpathlineto{\pgfqpoint{8.332060in}{1.237836in}}%
\pgfpathlineto{\pgfqpoint{8.333575in}{1.239261in}}%
\pgfpathlineto{\pgfqpoint{8.335089in}{1.244248in}}%
\pgfpathlineto{\pgfqpoint{8.335678in}{1.245673in}}%
\pgfpathlineto{\pgfqpoint{8.337192in}{1.248048in}}%
\pgfpathlineto{\pgfqpoint{8.337865in}{1.249472in}}%
\pgfpathlineto{\pgfqpoint{8.339380in}{1.255291in}}%
\pgfpathlineto{\pgfqpoint{8.340053in}{1.256359in}}%
\pgfpathlineto{\pgfqpoint{8.341567in}{1.260040in}}%
\pgfpathlineto{\pgfqpoint{8.343502in}{1.261109in}}%
\pgfpathlineto{\pgfqpoint{8.344596in}{1.263128in}}%
\pgfpathlineto{\pgfqpoint{8.345605in}{1.264434in}}%
\pgfpathlineto{\pgfqpoint{8.347119in}{1.270252in}}%
\pgfpathlineto{\pgfqpoint{8.348045in}{1.271677in}}%
\pgfpathlineto{\pgfqpoint{8.349559in}{1.276189in}}%
\pgfpathlineto{\pgfqpoint{8.350737in}{1.276783in}}%
\pgfpathlineto{\pgfqpoint{8.352251in}{1.282007in}}%
\pgfpathlineto{\pgfqpoint{8.353681in}{1.283195in}}%
\pgfpathlineto{\pgfqpoint{8.355196in}{1.285926in}}%
\pgfpathlineto{\pgfqpoint{8.355953in}{1.287350in}}%
\pgfpathlineto{\pgfqpoint{8.357467in}{1.290438in}}%
\pgfpathlineto{\pgfqpoint{8.358477in}{1.291744in}}%
\pgfpathlineto{\pgfqpoint{8.359991in}{1.294594in}}%
\pgfpathlineto{\pgfqpoint{8.360664in}{1.296018in}}%
\pgfpathlineto{\pgfqpoint{8.362178in}{1.297800in}}%
\pgfpathlineto{\pgfqpoint{8.362936in}{1.299224in}}%
\pgfpathlineto{\pgfqpoint{8.364366in}{1.301599in}}%
\pgfpathlineto{\pgfqpoint{8.365628in}{1.302905in}}%
\pgfpathlineto{\pgfqpoint{8.366890in}{1.306111in}}%
\pgfpathlineto{\pgfqpoint{8.368320in}{1.307536in}}%
\pgfpathlineto{\pgfqpoint{8.369834in}{1.309555in}}%
\pgfpathlineto{\pgfqpoint{8.370339in}{1.310861in}}%
\pgfpathlineto{\pgfqpoint{8.371180in}{1.312286in}}%
\pgfpathlineto{\pgfqpoint{8.372779in}{1.313711in}}%
\pgfpathlineto{\pgfqpoint{8.374293in}{1.316204in}}%
\pgfpathlineto{\pgfqpoint{8.375387in}{1.317154in}}%
\pgfpathlineto{\pgfqpoint{8.376817in}{1.319766in}}%
\pgfpathlineto{\pgfqpoint{8.377658in}{1.321191in}}%
\pgfpathlineto{\pgfqpoint{8.379172in}{1.324160in}}%
\pgfpathlineto{\pgfqpoint{8.380182in}{1.325466in}}%
\pgfpathlineto{\pgfqpoint{8.381528in}{1.326891in}}%
\pgfpathlineto{\pgfqpoint{8.382958in}{1.328316in}}%
\pgfpathlineto{\pgfqpoint{8.384388in}{1.330809in}}%
\pgfpathlineto{\pgfqpoint{8.385398in}{1.331759in}}%
\pgfpathlineto{\pgfqpoint{8.386828in}{1.334253in}}%
\pgfpathlineto{\pgfqpoint{8.387585in}{1.335440in}}%
\pgfpathlineto{\pgfqpoint{8.388679in}{1.337102in}}%
\pgfpathlineto{\pgfqpoint{8.390025in}{1.338171in}}%
\pgfpathlineto{\pgfqpoint{8.391203in}{1.339477in}}%
\pgfpathlineto{\pgfqpoint{8.392969in}{1.340665in}}%
\pgfpathlineto{\pgfqpoint{8.394400in}{1.342921in}}%
\pgfpathlineto{\pgfqpoint{8.395409in}{1.344108in}}%
\pgfpathlineto{\pgfqpoint{8.396839in}{1.347789in}}%
\pgfpathlineto{\pgfqpoint{8.397512in}{1.349095in}}%
\pgfpathlineto{\pgfqpoint{8.398943in}{1.350164in}}%
\pgfpathlineto{\pgfqpoint{8.400204in}{1.351470in}}%
\pgfpathlineto{\pgfqpoint{8.401719in}{1.354438in}}%
\pgfpathlineto{\pgfqpoint{8.402224in}{1.355745in}}%
\pgfpathlineto{\pgfqpoint{8.403738in}{1.357169in}}%
\pgfpathlineto{\pgfqpoint{8.405168in}{1.358476in}}%
\pgfpathlineto{\pgfqpoint{8.406682in}{1.360969in}}%
\pgfpathlineto{\pgfqpoint{8.407776in}{1.362394in}}%
\pgfpathlineto{\pgfqpoint{8.409206in}{1.364175in}}%
\pgfpathlineto{\pgfqpoint{8.410048in}{1.365481in}}%
\pgfpathlineto{\pgfqpoint{8.411562in}{1.367737in}}%
\pgfpathlineto{\pgfqpoint{8.414086in}{1.368925in}}%
\pgfpathlineto{\pgfqpoint{8.415432in}{1.370706in}}%
\pgfpathlineto{\pgfqpoint{8.417114in}{1.372131in}}%
\pgfpathlineto{\pgfqpoint{8.418544in}{1.373556in}}%
\pgfpathlineto{\pgfqpoint{8.420311in}{1.374862in}}%
\pgfpathlineto{\pgfqpoint{8.421657in}{1.378068in}}%
\pgfpathlineto{\pgfqpoint{8.423508in}{1.379374in}}%
\pgfpathlineto{\pgfqpoint{8.424854in}{1.383173in}}%
\pgfpathlineto{\pgfqpoint{8.425948in}{1.384598in}}%
\pgfpathlineto{\pgfqpoint{8.427294in}{1.386736in}}%
\pgfpathlineto{\pgfqpoint{8.429313in}{1.388161in}}%
\pgfpathlineto{\pgfqpoint{8.430743in}{1.389942in}}%
\pgfpathlineto{\pgfqpoint{8.432594in}{1.391129in}}%
\pgfpathlineto{\pgfqpoint{8.433856in}{1.393148in}}%
\pgfpathlineto{\pgfqpoint{8.435286in}{1.394573in}}%
\pgfpathlineto{\pgfqpoint{8.436632in}{1.396116in}}%
\pgfpathlineto{\pgfqpoint{8.438651in}{1.397541in}}%
\pgfpathlineto{\pgfqpoint{8.440166in}{1.399203in}}%
\pgfpathlineto{\pgfqpoint{8.441596in}{1.400153in}}%
\pgfpathlineto{\pgfqpoint{8.443026in}{1.402884in}}%
\pgfpathlineto{\pgfqpoint{8.444204in}{1.404190in}}%
\pgfpathlineto{\pgfqpoint{8.445634in}{1.406565in}}%
\pgfpathlineto{\pgfqpoint{8.447316in}{1.407634in}}%
\pgfpathlineto{\pgfqpoint{8.448831in}{1.410721in}}%
\pgfpathlineto{\pgfqpoint{8.450177in}{1.412027in}}%
\pgfpathlineto{\pgfqpoint{8.451691in}{1.414521in}}%
\pgfpathlineto{\pgfqpoint{8.452196in}{1.415708in}}%
\pgfpathlineto{\pgfqpoint{8.453458in}{1.418083in}}%
\pgfpathlineto{\pgfqpoint{8.454888in}{1.418914in}}%
\pgfpathlineto{\pgfqpoint{8.455813in}{1.420577in}}%
\pgfpathlineto{\pgfqpoint{8.457748in}{1.422001in}}%
\pgfpathlineto{\pgfqpoint{8.459263in}{1.424376in}}%
\pgfpathlineto{\pgfqpoint{8.460693in}{1.425801in}}%
\pgfpathlineto{\pgfqpoint{8.461787in}{1.426751in}}%
\pgfpathlineto{\pgfqpoint{8.463048in}{1.428057in}}%
\pgfpathlineto{\pgfqpoint{8.464563in}{1.429957in}}%
\pgfpathlineto{\pgfqpoint{8.466245in}{1.431382in}}%
\pgfpathlineto{\pgfqpoint{8.467760in}{1.434707in}}%
\pgfpathlineto{\pgfqpoint{8.468685in}{1.436013in}}%
\pgfpathlineto{\pgfqpoint{8.470199in}{1.437675in}}%
\pgfpathlineto{\pgfqpoint{8.473060in}{1.439100in}}%
\pgfpathlineto{\pgfqpoint{8.474406in}{1.441119in}}%
\pgfpathlineto{\pgfqpoint{8.475836in}{1.442543in}}%
\pgfpathlineto{\pgfqpoint{8.477266in}{1.443375in}}%
\pgfpathlineto{\pgfqpoint{8.478276in}{1.443968in}}%
\pgfpathlineto{\pgfqpoint{8.479790in}{1.447412in}}%
\pgfpathlineto{\pgfqpoint{8.481304in}{1.448837in}}%
\pgfpathlineto{\pgfqpoint{8.482819in}{1.450855in}}%
\pgfpathlineto{\pgfqpoint{8.483576in}{1.452043in}}%
\pgfpathlineto{\pgfqpoint{8.485090in}{1.455486in}}%
\pgfpathlineto{\pgfqpoint{8.486604in}{1.456792in}}%
\pgfpathlineto{\pgfqpoint{8.487950in}{1.458811in}}%
\pgfpathlineto{\pgfqpoint{8.489717in}{1.460236in}}%
\pgfpathlineto{\pgfqpoint{8.491231in}{1.462729in}}%
\pgfpathlineto{\pgfqpoint{8.492325in}{1.464035in}}%
\pgfpathlineto{\pgfqpoint{8.493839in}{1.466410in}}%
\pgfpathlineto{\pgfqpoint{8.495438in}{1.467835in}}%
\pgfpathlineto{\pgfqpoint{8.496784in}{1.470091in}}%
\pgfpathlineto{\pgfqpoint{8.498887in}{1.471397in}}%
\pgfpathlineto{\pgfqpoint{8.500401in}{1.473416in}}%
\pgfpathlineto{\pgfqpoint{8.501327in}{1.474841in}}%
\pgfpathlineto{\pgfqpoint{8.502841in}{1.477572in}}%
\pgfpathlineto{\pgfqpoint{8.504776in}{1.478997in}}%
\pgfpathlineto{\pgfqpoint{8.506290in}{1.480303in}}%
\pgfpathlineto{\pgfqpoint{8.508225in}{1.481609in}}%
\pgfpathlineto{\pgfqpoint{8.509740in}{1.482796in}}%
\pgfpathlineto{\pgfqpoint{8.511843in}{1.484102in}}%
\pgfpathlineto{\pgfqpoint{8.513189in}{1.485408in}}%
\pgfpathlineto{\pgfqpoint{8.514619in}{1.486833in}}%
\pgfpathlineto{\pgfqpoint{8.516049in}{1.488852in}}%
\pgfpathlineto{\pgfqpoint{8.517311in}{1.490277in}}%
\pgfpathlineto{\pgfqpoint{8.518741in}{1.491939in}}%
\pgfpathlineto{\pgfqpoint{8.520340in}{1.493364in}}%
\pgfpathlineto{\pgfqpoint{8.521265in}{1.493720in}}%
\pgfpathlineto{\pgfqpoint{8.524210in}{1.495145in}}%
\pgfpathlineto{\pgfqpoint{8.525640in}{1.497757in}}%
\pgfpathlineto{\pgfqpoint{8.527575in}{1.499182in}}%
\pgfpathlineto{\pgfqpoint{8.529089in}{1.500963in}}%
\pgfpathlineto{\pgfqpoint{8.532034in}{1.502388in}}%
\pgfpathlineto{\pgfqpoint{8.533548in}{1.504526in}}%
\pgfpathlineto{\pgfqpoint{8.535399in}{1.505950in}}%
\pgfpathlineto{\pgfqpoint{8.536913in}{1.507732in}}%
\pgfpathlineto{\pgfqpoint{8.538680in}{1.509156in}}%
\pgfpathlineto{\pgfqpoint{8.540194in}{1.510581in}}%
\pgfpathlineto{\pgfqpoint{8.541793in}{1.512006in}}%
\pgfpathlineto{\pgfqpoint{8.542718in}{1.512956in}}%
\pgfpathlineto{\pgfqpoint{8.545242in}{1.514381in}}%
\pgfpathlineto{\pgfqpoint{8.546756in}{1.516162in}}%
\pgfpathlineto{\pgfqpoint{8.548775in}{1.517587in}}%
\pgfpathlineto{\pgfqpoint{8.550121in}{1.519487in}}%
\pgfpathlineto{\pgfqpoint{8.552477in}{1.520912in}}%
\pgfpathlineto{\pgfqpoint{8.553907in}{1.522930in}}%
\pgfpathlineto{\pgfqpoint{8.556179in}{1.524236in}}%
\pgfpathlineto{\pgfqpoint{8.557693in}{1.527324in}}%
\pgfpathlineto{\pgfqpoint{8.560553in}{1.528749in}}%
\pgfpathlineto{\pgfqpoint{8.562068in}{1.530292in}}%
\pgfpathlineto{\pgfqpoint{8.563077in}{1.531598in}}%
\pgfpathlineto{\pgfqpoint{8.564423in}{1.532904in}}%
\pgfpathlineto{\pgfqpoint{8.566947in}{1.534329in}}%
\pgfpathlineto{\pgfqpoint{8.568377in}{1.536110in}}%
\pgfpathlineto{\pgfqpoint{8.570228in}{1.537535in}}%
\pgfpathlineto{\pgfqpoint{8.571658in}{1.539554in}}%
\pgfpathlineto{\pgfqpoint{8.572920in}{1.540860in}}%
\pgfpathlineto{\pgfqpoint{8.574350in}{1.542641in}}%
\pgfpathlineto{\pgfqpoint{8.575865in}{1.543947in}}%
\pgfpathlineto{\pgfqpoint{8.577211in}{1.545253in}}%
\pgfpathlineto{\pgfqpoint{8.579819in}{1.546678in}}%
\pgfpathlineto{\pgfqpoint{8.580997in}{1.547391in}}%
\pgfpathlineto{\pgfqpoint{8.583605in}{1.548816in}}%
\pgfpathlineto{\pgfqpoint{8.584951in}{1.550122in}}%
\pgfpathlineto{\pgfqpoint{8.586970in}{1.551547in}}%
\pgfpathlineto{\pgfqpoint{8.588484in}{1.553446in}}%
\pgfpathlineto{\pgfqpoint{8.589578in}{1.554871in}}%
\pgfpathlineto{\pgfqpoint{8.590924in}{1.555465in}}%
\pgfpathlineto{\pgfqpoint{8.593868in}{1.556890in}}%
\pgfpathlineto{\pgfqpoint{8.594962in}{1.557959in}}%
\pgfpathlineto{\pgfqpoint{8.597317in}{1.559265in}}%
\pgfpathlineto{\pgfqpoint{8.598159in}{1.560096in}}%
\pgfpathlineto{\pgfqpoint{8.600767in}{1.561521in}}%
\pgfpathlineto{\pgfqpoint{8.601608in}{1.562708in}}%
\pgfpathlineto{\pgfqpoint{8.603627in}{1.564133in}}%
\pgfpathlineto{\pgfqpoint{8.605057in}{1.564727in}}%
\pgfpathlineto{\pgfqpoint{8.606908in}{1.566033in}}%
\pgfpathlineto{\pgfqpoint{8.608254in}{1.567220in}}%
\pgfpathlineto{\pgfqpoint{8.610189in}{1.568645in}}%
\pgfpathlineto{\pgfqpoint{8.611535in}{1.569833in}}%
\pgfpathlineto{\pgfqpoint{8.614648in}{1.571257in}}%
\pgfpathlineto{\pgfqpoint{8.616078in}{1.573038in}}%
\pgfpathlineto{\pgfqpoint{8.618181in}{1.574463in}}%
\pgfpathlineto{\pgfqpoint{8.619696in}{1.575770in}}%
\pgfpathlineto{\pgfqpoint{8.622472in}{1.577194in}}%
\pgfpathlineto{\pgfqpoint{8.623734in}{1.578026in}}%
\pgfpathlineto{\pgfqpoint{8.625500in}{1.579450in}}%
\pgfpathlineto{\pgfqpoint{8.626762in}{1.580875in}}%
\pgfpathlineto{\pgfqpoint{8.628781in}{1.582300in}}%
\pgfpathlineto{\pgfqpoint{8.630296in}{1.583844in}}%
\pgfpathlineto{\pgfqpoint{8.632231in}{1.585269in}}%
\pgfpathlineto{\pgfqpoint{8.633745in}{1.586575in}}%
\pgfpathlineto{\pgfqpoint{8.635175in}{1.588000in}}%
\pgfpathlineto{\pgfqpoint{8.636690in}{1.590493in}}%
\pgfpathlineto{\pgfqpoint{8.637363in}{1.591799in}}%
\pgfpathlineto{\pgfqpoint{8.638625in}{1.593343in}}%
\pgfpathlineto{\pgfqpoint{8.639802in}{1.594530in}}%
\pgfpathlineto{\pgfqpoint{8.641233in}{1.595718in}}%
\pgfpathlineto{\pgfqpoint{8.642747in}{1.597143in}}%
\pgfpathlineto{\pgfqpoint{8.644009in}{1.598330in}}%
\pgfpathlineto{\pgfqpoint{8.646112in}{1.599755in}}%
\pgfpathlineto{\pgfqpoint{8.647290in}{1.600705in}}%
\pgfpathlineto{\pgfqpoint{8.649729in}{1.602130in}}%
\pgfpathlineto{\pgfqpoint{8.650991in}{1.603555in}}%
\pgfpathlineto{\pgfqpoint{8.653599in}{1.604861in}}%
\pgfpathlineto{\pgfqpoint{8.654945in}{1.605573in}}%
\pgfpathlineto{\pgfqpoint{8.656123in}{1.606879in}}%
\pgfpathlineto{\pgfqpoint{8.657638in}{1.609017in}}%
\pgfpathlineto{\pgfqpoint{8.659993in}{1.610442in}}%
\pgfpathlineto{\pgfqpoint{8.660666in}{1.611748in}}%
\pgfpathlineto{\pgfqpoint{8.661087in}{1.611748in}}%
\pgfpathlineto{\pgfqpoint{8.663779in}{1.613173in}}%
\pgfpathlineto{\pgfqpoint{8.665209in}{1.614004in}}%
\pgfpathlineto{\pgfqpoint{8.709713in}{1.644045in}}%
\pgfpathlineto{\pgfqpoint{8.711985in}{1.645470in}}%
\pgfpathlineto{\pgfqpoint{8.713499in}{1.646063in}}%
\pgfpathlineto{\pgfqpoint{8.716948in}{1.648557in}}%
\pgfpathlineto{\pgfqpoint{8.718294in}{1.649863in}}%
\pgfpathlineto{\pgfqpoint{8.719556in}{1.651288in}}%
\pgfpathlineto{\pgfqpoint{8.721239in}{1.652238in}}%
\pgfpathlineto{\pgfqpoint{8.722669in}{1.655206in}}%
\pgfpathlineto{\pgfqpoint{8.724351in}{1.656631in}}%
\pgfpathlineto{\pgfqpoint{8.725697in}{1.658887in}}%
\pgfpathlineto{\pgfqpoint{8.730661in}{1.660312in}}%
\pgfpathlineto{\pgfqpoint{8.732175in}{1.661737in}}%
\pgfpathlineto{\pgfqpoint{8.733606in}{1.662806in}}%
\pgfpathlineto{\pgfqpoint{8.735120in}{1.664112in}}%
\pgfpathlineto{\pgfqpoint{8.736550in}{1.665299in}}%
\pgfpathlineto{\pgfqpoint{8.738064in}{1.667080in}}%
\pgfpathlineto{\pgfqpoint{8.741514in}{1.668505in}}%
\pgfpathlineto{\pgfqpoint{8.742944in}{1.669693in}}%
\pgfpathlineto{\pgfqpoint{8.746898in}{1.671118in}}%
\pgfpathlineto{\pgfqpoint{8.748412in}{1.672305in}}%
\pgfpathlineto{\pgfqpoint{8.750936in}{1.673730in}}%
\pgfpathlineto{\pgfqpoint{8.751777in}{1.674324in}}%
\pgfpathlineto{\pgfqpoint{8.756068in}{1.675748in}}%
\pgfpathlineto{\pgfqpoint{8.757330in}{1.676580in}}%
\pgfpathlineto{\pgfqpoint{8.758592in}{1.677886in}}%
\pgfpathlineto{\pgfqpoint{8.760106in}{1.679311in}}%
\pgfpathlineto{\pgfqpoint{8.761200in}{1.680736in}}%
\pgfpathlineto{\pgfqpoint{8.762546in}{1.681448in}}%
\pgfpathlineto{\pgfqpoint{8.766920in}{1.682873in}}%
\pgfpathlineto{\pgfqpoint{8.768435in}{1.683704in}}%
\pgfpathlineto{\pgfqpoint{8.770538in}{1.685129in}}%
\pgfpathlineto{\pgfqpoint{8.771716in}{1.686673in}}%
\pgfpathlineto{\pgfqpoint{8.774660in}{1.687979in}}%
\pgfpathlineto{\pgfqpoint{8.775081in}{1.689404in}}%
\pgfpathlineto{\pgfqpoint{8.776090in}{1.689404in}}%
\pgfpathlineto{\pgfqpoint{8.778362in}{1.690710in}}%
\pgfpathlineto{\pgfqpoint{8.778362in}{1.690828in}}%
\pgfpathlineto{\pgfqpoint{8.779876in}{1.692847in}}%
\pgfpathlineto{\pgfqpoint{8.782568in}{1.694272in}}%
\pgfpathlineto{\pgfqpoint{8.783241in}{1.694984in}}%
\pgfpathlineto{\pgfqpoint{8.785849in}{1.696409in}}%
\pgfpathlineto{\pgfqpoint{8.787280in}{1.697359in}}%
\pgfpathlineto{\pgfqpoint{8.791654in}{1.698665in}}%
\pgfpathlineto{\pgfqpoint{8.793168in}{1.700565in}}%
\pgfpathlineto{\pgfqpoint{8.796281in}{1.701871in}}%
\pgfpathlineto{\pgfqpoint{8.797291in}{1.703296in}}%
\pgfpathlineto{\pgfqpoint{8.800908in}{1.704721in}}%
\pgfpathlineto{\pgfqpoint{8.802423in}{1.705315in}}%
\pgfpathlineto{\pgfqpoint{8.804610in}{1.706621in}}%
\pgfpathlineto{\pgfqpoint{8.804610in}{1.706740in}}%
\pgfpathlineto{\pgfqpoint{8.809153in}{1.710658in}}%
\pgfpathlineto{\pgfqpoint{8.810920in}{1.712083in}}%
\pgfpathlineto{\pgfqpoint{8.813780in}{1.714220in}}%
\pgfpathlineto{\pgfqpoint{8.816051in}{1.715645in}}%
\pgfpathlineto{\pgfqpoint{8.817061in}{1.716476in}}%
\pgfpathlineto{\pgfqpoint{8.820174in}{1.717901in}}%
\pgfpathlineto{\pgfqpoint{8.821604in}{1.718970in}}%
\pgfpathlineto{\pgfqpoint{8.824969in}{1.720395in}}%
\pgfpathlineto{\pgfqpoint{8.825642in}{1.720988in}}%
\pgfpathlineto{\pgfqpoint{8.826399in}{1.720988in}}%
\pgfpathlineto{\pgfqpoint{8.829260in}{1.722176in}}%
\pgfpathlineto{\pgfqpoint{8.830690in}{1.723838in}}%
\pgfpathlineto{\pgfqpoint{8.832961in}{1.725144in}}%
\pgfpathlineto{\pgfqpoint{8.832961in}{1.725263in}}%
\pgfpathlineto{\pgfqpoint{8.835401in}{1.726213in}}%
\pgfpathlineto{\pgfqpoint{8.839019in}{1.727638in}}%
\pgfpathlineto{\pgfqpoint{8.840533in}{1.729181in}}%
\pgfpathlineto{\pgfqpoint{8.842973in}{1.730606in}}%
\pgfpathlineto{\pgfqpoint{8.844487in}{1.732981in}}%
\pgfpathlineto{\pgfqpoint{8.848020in}{1.734168in}}%
\pgfpathlineto{\pgfqpoint{8.848609in}{1.734881in}}%
\pgfpathlineto{\pgfqpoint{8.849535in}{1.734881in}}%
\pgfpathlineto{\pgfqpoint{8.852731in}{1.736306in}}%
\pgfpathlineto{\pgfqpoint{8.853573in}{1.736662in}}%
\pgfpathlineto{\pgfqpoint{8.854246in}{1.736662in}}%
\pgfpathlineto{\pgfqpoint{8.857779in}{1.738087in}}%
\pgfpathlineto{\pgfqpoint{8.859209in}{1.739630in}}%
\pgfpathlineto{\pgfqpoint{8.862743in}{1.741055in}}%
\pgfpathlineto{\pgfqpoint{8.864257in}{1.741530in}}%
\pgfpathlineto{\pgfqpoint{8.866529in}{1.742955in}}%
\pgfpathlineto{\pgfqpoint{8.867706in}{1.743549in}}%
\pgfpathlineto{\pgfqpoint{8.869978in}{1.744974in}}%
\pgfpathlineto{\pgfqpoint{8.871408in}{1.746042in}}%
\pgfpathlineto{\pgfqpoint{8.874184in}{1.747349in}}%
\pgfpathlineto{\pgfqpoint{8.874184in}{1.747467in}}%
\pgfpathlineto{\pgfqpoint{8.877465in}{1.749723in}}%
\pgfpathlineto{\pgfqpoint{8.879905in}{1.751148in}}%
\pgfpathlineto{\pgfqpoint{8.881335in}{1.752217in}}%
\pgfpathlineto{\pgfqpoint{8.884532in}{1.753642in}}%
\pgfpathlineto{\pgfqpoint{8.885962in}{1.754710in}}%
\pgfpathlineto{\pgfqpoint{8.887476in}{1.756135in}}%
\pgfpathlineto{\pgfqpoint{8.888318in}{1.756848in}}%
\pgfpathlineto{\pgfqpoint{8.893534in}{1.758035in}}%
\pgfpathlineto{\pgfqpoint{8.895048in}{1.759223in}}%
\pgfpathlineto{\pgfqpoint{8.899507in}{1.760529in}}%
\pgfpathlineto{\pgfqpoint{8.900853in}{1.762072in}}%
\pgfpathlineto{\pgfqpoint{8.903461in}{1.763378in}}%
\pgfpathlineto{\pgfqpoint{8.904975in}{1.764566in}}%
\pgfpathlineto{\pgfqpoint{8.908172in}{1.765991in}}%
\pgfpathlineto{\pgfqpoint{8.910948in}{1.767416in}}%
\pgfpathlineto{\pgfqpoint{8.913640in}{1.768722in}}%
\pgfpathlineto{\pgfqpoint{8.914902in}{1.769315in}}%
\pgfpathlineto{\pgfqpoint{8.917342in}{1.770740in}}%
\pgfpathlineto{\pgfqpoint{8.918856in}{1.771572in}}%
\pgfpathlineto{\pgfqpoint{8.921717in}{1.772996in}}%
\pgfpathlineto{\pgfqpoint{8.923231in}{1.773709in}}%
\pgfpathlineto{\pgfqpoint{8.927353in}{1.774777in}}%
\pgfpathlineto{\pgfqpoint{8.928363in}{1.775965in}}%
\pgfpathlineto{\pgfqpoint{8.928784in}{1.775965in}}%
\pgfpathlineto{\pgfqpoint{8.932149in}{1.777390in}}%
\pgfpathlineto{\pgfqpoint{8.933326in}{1.778102in}}%
\pgfpathlineto{\pgfqpoint{8.936019in}{1.779527in}}%
\pgfpathlineto{\pgfqpoint{8.937365in}{1.780714in}}%
\pgfpathlineto{\pgfqpoint{8.940309in}{1.782139in}}%
\pgfpathlineto{\pgfqpoint{8.940814in}{1.782971in}}%
\pgfpathlineto{\pgfqpoint{8.941823in}{1.782971in}}%
\pgfpathlineto{\pgfqpoint{8.944347in}{1.784395in}}%
\pgfpathlineto{\pgfqpoint{8.945693in}{1.784752in}}%
\pgfpathlineto{\pgfqpoint{8.949227in}{1.786177in}}%
\pgfpathlineto{\pgfqpoint{8.950741in}{1.787364in}}%
\pgfpathlineto{\pgfqpoint{8.953097in}{1.788789in}}%
\pgfpathlineto{\pgfqpoint{8.954611in}{1.789739in}}%
\pgfpathlineto{\pgfqpoint{8.956967in}{1.791164in}}%
\pgfpathlineto{\pgfqpoint{8.958229in}{1.792113in}}%
\pgfpathlineto{\pgfqpoint{8.958481in}{1.792113in}}%
\pgfpathlineto{\pgfqpoint{8.960921in}{1.792945in}}%
\pgfpathlineto{\pgfqpoint{8.965211in}{1.794132in}}%
\pgfpathlineto{\pgfqpoint{8.966641in}{1.795319in}}%
\pgfpathlineto{\pgfqpoint{8.966725in}{1.795319in}}%
\pgfpathlineto{\pgfqpoint{8.969586in}{1.796626in}}%
\pgfpathlineto{\pgfqpoint{8.970680in}{1.797338in}}%
\pgfpathlineto{\pgfqpoint{8.971016in}{1.797338in}}%
\pgfpathlineto{\pgfqpoint{8.974129in}{1.798763in}}%
\pgfpathlineto{\pgfqpoint{8.976569in}{1.799713in}}%
\pgfpathlineto{\pgfqpoint{8.978335in}{1.801138in}}%
\pgfpathlineto{\pgfqpoint{8.979092in}{1.801613in}}%
\pgfpathlineto{\pgfqpoint{8.979261in}{1.801613in}}%
\pgfpathlineto{\pgfqpoint{8.983635in}{1.803038in}}%
\pgfpathlineto{\pgfqpoint{8.985739in}{1.803394in}}%
\pgfpathlineto{\pgfqpoint{8.991123in}{1.804700in}}%
\pgfpathlineto{\pgfqpoint{8.992553in}{1.807075in}}%
\pgfpathlineto{\pgfqpoint{8.996086in}{1.808262in}}%
\pgfpathlineto{\pgfqpoint{8.996086in}{1.808500in}}%
\pgfpathlineto{\pgfqpoint{8.999031in}{1.810162in}}%
\pgfpathlineto{\pgfqpoint{9.004079in}{1.811468in}}%
\pgfpathlineto{\pgfqpoint{9.005593in}{1.812655in}}%
\pgfpathlineto{\pgfqpoint{9.007864in}{1.814080in}}%
\pgfpathlineto{\pgfqpoint{9.008958in}{1.814793in}}%
\pgfpathlineto{\pgfqpoint{9.009379in}{1.814793in}}%
\pgfpathlineto{\pgfqpoint{9.012323in}{1.816218in}}%
\pgfpathlineto{\pgfqpoint{9.013585in}{1.817049in}}%
\pgfpathlineto{\pgfqpoint{9.018128in}{1.818355in}}%
\pgfpathlineto{\pgfqpoint{9.019642in}{1.819899in}}%
\pgfpathlineto{\pgfqpoint{9.022334in}{1.821086in}}%
\pgfpathlineto{\pgfqpoint{9.022334in}{1.821323in}}%
\pgfpathlineto{\pgfqpoint{9.024606in}{1.822036in}}%
\pgfpathlineto{\pgfqpoint{9.029065in}{1.823342in}}%
\pgfpathlineto{\pgfqpoint{9.030495in}{1.824411in}}%
\pgfpathlineto{\pgfqpoint{9.033944in}{1.825836in}}%
\pgfpathlineto{\pgfqpoint{9.039917in}{1.828329in}}%
\pgfpathlineto{\pgfqpoint{9.044628in}{1.829754in}}%
\pgfpathlineto{\pgfqpoint{9.045722in}{1.830229in}}%
\pgfpathlineto{\pgfqpoint{9.046059in}{1.830229in}}%
\pgfpathlineto{\pgfqpoint{9.049087in}{1.831654in}}%
\pgfpathlineto{\pgfqpoint{9.050854in}{1.832485in}}%
\pgfpathlineto{\pgfqpoint{9.056070in}{1.833910in}}%
\pgfpathlineto{\pgfqpoint{9.057584in}{1.835097in}}%
\pgfpathlineto{\pgfqpoint{9.061370in}{1.836522in}}%
\pgfpathlineto{\pgfqpoint{9.062464in}{1.836878in}}%
\pgfpathlineto{\pgfqpoint{9.062884in}{1.836878in}}%
\pgfpathlineto{\pgfqpoint{9.067091in}{1.838303in}}%
\pgfpathlineto{\pgfqpoint{9.068521in}{1.838778in}}%
\pgfpathlineto{\pgfqpoint{9.071970in}{1.840203in}}%
\pgfpathlineto{\pgfqpoint{9.075167in}{1.842459in}}%
\pgfpathlineto{\pgfqpoint{9.079710in}{1.843884in}}%
\pgfpathlineto{\pgfqpoint{9.081224in}{1.844597in}}%
\pgfpathlineto{\pgfqpoint{9.086188in}{1.846021in}}%
\pgfpathlineto{\pgfqpoint{9.087618in}{1.847209in}}%
\pgfpathlineto{\pgfqpoint{9.091404in}{1.848634in}}%
\pgfpathlineto{\pgfqpoint{9.092750in}{1.849821in}}%
\pgfpathlineto{\pgfqpoint{9.092834in}{1.849821in}}%
\pgfpathlineto{\pgfqpoint{9.094853in}{1.851246in}}%
\pgfpathlineto{\pgfqpoint{9.096620in}{1.851483in}}%
\pgfpathlineto{\pgfqpoint{9.101583in}{1.852908in}}%
\pgfpathlineto{\pgfqpoint{9.102845in}{1.853739in}}%
\pgfpathlineto{\pgfqpoint{9.106210in}{1.855164in}}%
\pgfpathlineto{\pgfqpoint{9.107725in}{1.855996in}}%
\pgfpathlineto{\pgfqpoint{9.111595in}{1.857420in}}%
\pgfpathlineto{\pgfqpoint{9.113109in}{1.858370in}}%
\pgfpathlineto{\pgfqpoint{9.114960in}{1.859676in}}%
\pgfpathlineto{\pgfqpoint{9.118241in}{1.861101in}}%
\pgfpathlineto{\pgfqpoint{9.119587in}{1.862051in}}%
\pgfpathlineto{\pgfqpoint{9.119755in}{1.862051in}}%
\pgfpathlineto{\pgfqpoint{9.121942in}{1.862645in}}%
\pgfpathlineto{\pgfqpoint{9.125897in}{1.863951in}}%
\pgfpathlineto{\pgfqpoint{9.127243in}{1.864426in}}%
\pgfpathlineto{\pgfqpoint{9.127327in}{1.864426in}}%
\pgfpathlineto{\pgfqpoint{9.132627in}{1.865851in}}%
\pgfpathlineto{\pgfqpoint{9.135403in}{1.867276in}}%
\pgfpathlineto{\pgfqpoint{9.139778in}{1.868582in}}%
\pgfpathlineto{\pgfqpoint{9.141124in}{1.869769in}}%
\pgfpathlineto{\pgfqpoint{9.142722in}{1.871194in}}%
\pgfpathlineto{\pgfqpoint{9.144994in}{1.871788in}}%
\pgfpathlineto{\pgfqpoint{9.148611in}{1.873213in}}%
\pgfpathlineto{\pgfqpoint{9.149873in}{1.873688in}}%
\pgfpathlineto{\pgfqpoint{9.150041in}{1.873688in}}%
\pgfpathlineto{\pgfqpoint{9.152229in}{1.875113in}}%
\pgfpathlineto{\pgfqpoint{9.153154in}{1.875825in}}%
\pgfpathlineto{\pgfqpoint{9.153743in}{1.875825in}}%
\pgfpathlineto{\pgfqpoint{9.159127in}{1.877131in}}%
\pgfpathlineto{\pgfqpoint{9.160642in}{1.877962in}}%
\pgfpathlineto{\pgfqpoint{9.164175in}{1.879387in}}%
\pgfpathlineto{\pgfqpoint{9.166951in}{1.880812in}}%
\pgfpathlineto{\pgfqpoint{9.171999in}{1.882237in}}%
\pgfpathlineto{\pgfqpoint{9.173513in}{1.883543in}}%
\pgfpathlineto{\pgfqpoint{9.177215in}{1.884968in}}%
\pgfpathlineto{\pgfqpoint{9.180244in}{1.886868in}}%
\pgfpathlineto{\pgfqpoint{9.184366in}{1.888293in}}%
\pgfpathlineto{\pgfqpoint{9.189666in}{1.890193in}}%
\pgfpathlineto{\pgfqpoint{9.194041in}{1.891617in}}%
\pgfpathlineto{\pgfqpoint{9.194041in}{1.891736in}}%
\pgfpathlineto{\pgfqpoint{9.195555in}{1.891736in}}%
\pgfpathlineto{\pgfqpoint{9.197490in}{1.893161in}}%
\pgfpathlineto{\pgfqpoint{9.206323in}{1.897554in}}%
\pgfpathlineto{\pgfqpoint{9.209268in}{1.898979in}}%
\pgfpathlineto{\pgfqpoint{9.212128in}{1.900523in}}%
\pgfpathlineto{\pgfqpoint{9.216082in}{1.901829in}}%
\pgfpathlineto{\pgfqpoint{9.217176in}{1.902304in}}%
\pgfpathlineto{\pgfqpoint{9.217428in}{1.902304in}}%
\pgfpathlineto{\pgfqpoint{9.220878in}{1.903729in}}%
\pgfpathlineto{\pgfqpoint{9.223654in}{1.905035in}}%
\pgfpathlineto{\pgfqpoint{9.225841in}{1.906460in}}%
\pgfpathlineto{\pgfqpoint{9.227692in}{1.906697in}}%
\pgfpathlineto{\pgfqpoint{9.253183in}{1.917147in}}%
\pgfpathlineto{\pgfqpoint{9.254361in}{1.917978in}}%
\pgfpathlineto{\pgfqpoint{9.254697in}{1.917978in}}%
\pgfpathlineto{\pgfqpoint{9.258819in}{1.919284in}}%
\pgfpathlineto{\pgfqpoint{9.260081in}{1.919996in}}%
\pgfpathlineto{\pgfqpoint{9.260250in}{1.919996in}}%
\pgfpathlineto{\pgfqpoint{9.264372in}{1.921184in}}%
\pgfpathlineto{\pgfqpoint{9.264372in}{1.921302in}}%
\pgfpathlineto{\pgfqpoint{9.266223in}{1.922609in}}%
\pgfpathlineto{\pgfqpoint{9.269167in}{1.923915in}}%
\pgfpathlineto{\pgfqpoint{9.275982in}{1.927002in}}%
\pgfpathlineto{\pgfqpoint{9.282291in}{1.930327in}}%
\pgfpathlineto{\pgfqpoint{9.289106in}{1.933533in}}%
\pgfpathlineto{\pgfqpoint{9.293901in}{1.934958in}}%
\pgfpathlineto{\pgfqpoint{9.296761in}{1.936620in}}%
\pgfpathlineto{\pgfqpoint{9.300715in}{1.937926in}}%
\pgfpathlineto{\pgfqpoint{9.305427in}{1.939470in}}%
\pgfpathlineto{\pgfqpoint{9.309885in}{1.940895in}}%
\pgfpathlineto{\pgfqpoint{9.317962in}{1.943032in}}%
\pgfpathlineto{\pgfqpoint{9.323767in}{1.944457in}}%
\pgfpathlineto{\pgfqpoint{9.334451in}{1.947900in}}%
\pgfpathlineto{\pgfqpoint{9.339246in}{1.949325in}}%
\pgfpathlineto{\pgfqpoint{9.342023in}{1.950750in}}%
\pgfpathlineto{\pgfqpoint{9.346902in}{1.952056in}}%
\pgfpathlineto{\pgfqpoint{9.348837in}{1.953362in}}%
\pgfpathlineto{\pgfqpoint{9.351445in}{1.954668in}}%
\pgfpathlineto{\pgfqpoint{9.354305in}{1.956093in}}%
\pgfpathlineto{\pgfqpoint{9.358259in}{1.957518in}}%
\pgfpathlineto{\pgfqpoint{9.366756in}{1.961318in}}%
\pgfpathlineto{\pgfqpoint{9.367766in}{1.961793in}}%
\pgfpathlineto{\pgfqpoint{9.368186in}{1.961793in}}%
\pgfpathlineto{\pgfqpoint{9.375926in}{1.965592in}}%
\pgfpathlineto{\pgfqpoint{9.387957in}{1.969511in}}%
\pgfpathlineto{\pgfqpoint{9.391995in}{1.970817in}}%
\pgfpathlineto{\pgfqpoint{9.394855in}{1.972123in}}%
\pgfpathlineto{\pgfqpoint{9.397631in}{1.973429in}}%
\pgfpathlineto{\pgfqpoint{9.401922in}{1.974854in}}%
\pgfpathlineto{\pgfqpoint{9.411344in}{1.979722in}}%
\pgfpathlineto{\pgfqpoint{9.419000in}{1.983285in}}%
\pgfpathlineto{\pgfqpoint{9.421103in}{1.984235in}}%
\pgfpathlineto{\pgfqpoint{9.422786in}{1.985659in}}%
\pgfpathlineto{\pgfqpoint{9.427749in}{1.987084in}}%
\pgfpathlineto{\pgfqpoint{9.431535in}{1.988272in}}%
\pgfpathlineto{\pgfqpoint{9.434648in}{1.989578in}}%
\pgfpathlineto{\pgfqpoint{9.456605in}{1.998246in}}%
\pgfpathlineto{\pgfqpoint{9.459886in}{1.999671in}}%
\pgfpathlineto{\pgfqpoint{9.466028in}{2.002639in}}%
\pgfpathlineto{\pgfqpoint{9.475534in}{2.006914in}}%
\pgfpathlineto{\pgfqpoint{9.479488in}{2.008101in}}%
\pgfpathlineto{\pgfqpoint{9.490762in}{2.012495in}}%
\pgfpathlineto{\pgfqpoint{9.495389in}{2.013920in}}%
\pgfpathlineto{\pgfqpoint{9.505989in}{2.017838in}}%
\pgfpathlineto{\pgfqpoint{9.511625in}{2.019500in}}%
\pgfpathlineto{\pgfqpoint{9.518356in}{2.020688in}}%
\pgfpathlineto{\pgfqpoint{9.539809in}{2.030543in}}%
\pgfpathlineto{\pgfqpoint{9.547128in}{2.033037in}}%
\pgfpathlineto{\pgfqpoint{9.585827in}{2.044673in}}%
\pgfpathlineto{\pgfqpoint{9.591716in}{2.047523in}}%
\pgfpathlineto{\pgfqpoint{9.609719in}{2.054529in}}%
\pgfpathlineto{\pgfqpoint{9.620740in}{2.057735in}}%
\pgfpathlineto{\pgfqpoint{9.630162in}{2.061653in}}%
\pgfpathlineto{\pgfqpoint{9.647998in}{2.067234in}}%
\pgfpathlineto{\pgfqpoint{9.662047in}{2.073171in}}%
\pgfpathlineto{\pgfqpoint{9.673657in}{2.076970in}}%
\pgfpathlineto{\pgfqpoint{9.703438in}{2.088132in}}%
\pgfpathlineto{\pgfqpoint{9.722115in}{2.095138in}}%
\pgfpathlineto{\pgfqpoint{9.737510in}{2.101668in}}%
\pgfpathlineto{\pgfqpoint{9.761150in}{2.109149in}}%
\pgfpathlineto{\pgfqpoint{9.761150in}{2.109149in}}%
\pgfusepath{stroke}%
\end{pgfscope}%
\begin{pgfscope}%
\pgfpathrectangle{\pgfqpoint{8.150541in}{0.639583in}}{\pgfqpoint{1.687305in}{1.539545in}}%
\pgfusepath{clip}%
\pgfsetrectcap%
\pgfsetroundjoin%
\pgfsetlinewidth{1.505625pt}%
\definecolor{currentstroke}{rgb}{0.501961,0.501961,0.501961}%
\pgfsetstrokecolor{currentstroke}%
\pgfsetdash{}{0pt}%
\pgfpathmoveto{\pgfqpoint{8.227236in}{0.709562in}}%
\pgfpathlineto{\pgfqpoint{9.761150in}{2.109149in}}%
\pgfusepath{stroke}%
\end{pgfscope}%
\begin{pgfscope}%
\pgfsetrectcap%
\pgfsetmiterjoin%
\pgfsetlinewidth{0.803000pt}%
\definecolor{currentstroke}{rgb}{0.000000,0.000000,0.000000}%
\pgfsetstrokecolor{currentstroke}%
\pgfsetdash{}{0pt}%
\pgfpathmoveto{\pgfqpoint{8.150541in}{0.639583in}}%
\pgfpathlineto{\pgfqpoint{8.150541in}{2.179128in}}%
\pgfusepath{stroke}%
\end{pgfscope}%
\begin{pgfscope}%
\pgfsetrectcap%
\pgfsetmiterjoin%
\pgfsetlinewidth{0.803000pt}%
\definecolor{currentstroke}{rgb}{0.000000,0.000000,0.000000}%
\pgfsetstrokecolor{currentstroke}%
\pgfsetdash{}{0pt}%
\pgfpathmoveto{\pgfqpoint{9.837846in}{0.639583in}}%
\pgfpathlineto{\pgfqpoint{9.837846in}{2.179128in}}%
\pgfusepath{stroke}%
\end{pgfscope}%
\begin{pgfscope}%
\pgfsetrectcap%
\pgfsetmiterjoin%
\pgfsetlinewidth{0.803000pt}%
\definecolor{currentstroke}{rgb}{0.000000,0.000000,0.000000}%
\pgfsetstrokecolor{currentstroke}%
\pgfsetdash{}{0pt}%
\pgfpathmoveto{\pgfqpoint{8.150541in}{0.639583in}}%
\pgfpathlineto{\pgfqpoint{9.837846in}{0.639583in}}%
\pgfusepath{stroke}%
\end{pgfscope}%
\begin{pgfscope}%
\pgfsetrectcap%
\pgfsetmiterjoin%
\pgfsetlinewidth{0.803000pt}%
\definecolor{currentstroke}{rgb}{0.000000,0.000000,0.000000}%
\pgfsetstrokecolor{currentstroke}%
\pgfsetdash{}{0pt}%
\pgfpathmoveto{\pgfqpoint{8.150541in}{2.179128in}}%
\pgfpathlineto{\pgfqpoint{9.837846in}{2.179128in}}%
\pgfusepath{stroke}%
\end{pgfscope}%
\begin{pgfscope}%
\definecolor{textcolor}{rgb}{0.000000,0.000000,0.000000}%
\pgfsetstrokecolor{textcolor}%
\pgfsetfillcolor{textcolor}%
\pgftext[x=8.994193in,y=2.262462in,,base]{\color{textcolor}\rmfamily\fontsize{20.000000}{24.000000}\selectfont Healthy}%
\end{pgfscope}%
\begin{pgfscope}%
\pgfsetbuttcap%
\pgfsetmiterjoin%
\definecolor{currentfill}{rgb}{1.000000,1.000000,1.000000}%
\pgfsetfillcolor{currentfill}%
\pgfsetfillopacity{0.800000}%
\pgfsetlinewidth{1.003750pt}%
\definecolor{currentstroke}{rgb}{0.800000,0.800000,0.800000}%
\pgfsetstrokecolor{currentstroke}%
\pgfsetstrokeopacity{0.800000}%
\pgfsetdash{}{0pt}%
\pgfpathmoveto{\pgfqpoint{8.628740in}{0.709028in}}%
\pgfpathlineto{\pgfqpoint{9.740624in}{0.709028in}}%
\pgfpathquadraticcurveto{\pgfqpoint{9.768402in}{0.709028in}}{\pgfqpoint{9.768402in}{0.736805in}}%
\pgfpathlineto{\pgfqpoint{9.768402in}{0.916589in}}%
\pgfpathquadraticcurveto{\pgfqpoint{9.768402in}{0.944367in}}{\pgfqpoint{9.740624in}{0.944367in}}%
\pgfpathlineto{\pgfqpoint{8.628740in}{0.944367in}}%
\pgfpathquadraticcurveto{\pgfqpoint{8.600962in}{0.944367in}}{\pgfqpoint{8.600962in}{0.916589in}}%
\pgfpathlineto{\pgfqpoint{8.600962in}{0.736805in}}%
\pgfpathquadraticcurveto{\pgfqpoint{8.600962in}{0.709028in}}{\pgfqpoint{8.628740in}{0.709028in}}%
\pgfpathclose%
\pgfusepath{stroke,fill}%
\end{pgfscope}%
\begin{pgfscope}%
\pgfsetrectcap%
\pgfsetroundjoin%
\pgfsetlinewidth{1.505625pt}%
\definecolor{currentstroke}{rgb}{0.000000,0.501961,0.000000}%
\pgfsetstrokecolor{currentstroke}%
\pgfsetdash{}{0pt}%
\pgfpathmoveto{\pgfqpoint{8.656518in}{0.840200in}}%
\pgfpathlineto{\pgfqpoint{8.934295in}{0.840200in}}%
\pgfusepath{stroke}%
\end{pgfscope}%
\begin{pgfscope}%
\definecolor{textcolor}{rgb}{0.000000,0.000000,0.000000}%
\pgfsetstrokecolor{textcolor}%
\pgfsetfillcolor{textcolor}%
\pgftext[x=9.045407in,y=0.791589in,left,base]{\color{textcolor}\rmfamily\fontsize{10.000000}{12.000000}\selectfont AUC 0.736}%
\end{pgfscope}%
\end{pgfpicture}%
\makeatother%
\endgroup%
}
%     \end{center}
% \end{figure}
