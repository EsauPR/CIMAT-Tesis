En este trabajo se ha desarrollado dos modelos de redes neuronales de aprendizaje profundo para ayudar
con el diagnóstico de 15 patologías pulmonares a través de imágenes de rayos-X.
Basados en un modelo previo que detectaba 14 de estas
patologías, se ha incorporado datos para la detección de COVID-19 y se ha desarrollado dos modelos
que pueden clasificar dichas patologías, así como aquellas que son saludables.

Uno de los objetivos principales de este trabajo es la inclusión de la detección de COVID-19 a través
de radiografías de rayos-X. Siendo una de las criticas a los sistemas actuales que detectan COVID-19
la limitante de estos a la detección de la patología contra casos de neumonía y pacientes saludables.
Esto provoca incertidumbre acerca del rendimiento de los modelos con imágenes con patologías que no son
incluidas en las bases de datos usadas para su entrenamiento.

En esta tesis se ataca dicho problema presentando dos modelos que pueden detectar 15 patologías, entre
ellas la de COVID-19. También se muestra a través de los experimentos que los modelos aquí presentados
tienen un rendimiento comprable con el estado del arte (ChestNet y otros modelos de comparación) en las
patologías y bases de datos de ChestRay-14, y en particular, para COVID-19 tiene el mejor rendimiento.
Los resultados experimentales muestran que en general los modelos propuestos compiten con el estado
del arte, en especial el modelo basado en ResNet50 muestra una mejoría en rendimiento y en
particular ambos modelos son altamente precisos para COVID-19. Creemos que realizar modelos más
robustos como los presentados pueden ayudar a discriminar mucho mejor
las afecciones, y brindar un panorama más completo a los radiólogos que utilizan estas
herramientas durante el proceso de un análisis clínico.

Los modelos desarrollados están bajo la arquitectura de \textit{Transformers} usado mecanismos de atención
y redes convoluciones respectivamente. Usando datos de diferentes fuentes y regiones geográficas, y
entrenándose bajo una estrategia muy popular en la implementación de redes neuronales de aprendizaje
profundo, \textit{Transfer Learning}. Construimos dos modelos distintos usando ResNet50 y ViT
pre-entrenados con el conjunto de datos de ImageNet. La ventaja de usar esta estrategia es que podemos
entrenar los modelos en distintas etapas provenientes de un proceso de \textit{fine-tuning} hasta el
entrenamiento completo del modelo \textit{full-tuning}.

Además, se demuestra que los modelos puede ser fácilmente extendidos para detectar otras enfermedades
pulmonares. Usando muestras de tuberculosis como ejemplo del proceso, se realiza un entrenamiento
de una bifurcación de la etapa clasificadora, con resultados comparables con otros métodos de deep learning
específicos para esta enfermedad. Los resultados obtenidos dieron lugar a un clasificador binario que tiene un
rendimiento comparable con el estado del arte de clasificadores desarrollados especialmente para la
detección de tuberculosis. Por otro lado cuando se evaluaron los datos correspondiente a tuberculosis
en los modelos de 15 patologías, estos fueron detectados como neumonía (unos pocos como COVID-19, otro
tipo de neumonía) y ninguno en alguna otra patología. Con esto se puede mostrar que el entrenamiento
del modelo fue adecuado y puede ser la base de implementación de nuevos detectores de otras patologías.

Estos hallazgos sugieren que el Transformer y la atención son
arquitecturas muy versátiles y potentes comparables con las redes convolucionales para el análisis
de imágenes médicas, y que en un futuro puedan superar a estas.

Este trabajo aporta una contribución original e innovadora al campo de la detección de enfermedades
pulmonares mediante imágenes de rayos X, al introducir el uso del Transformer y la atención como
arquitecturas alternativas a las redes convolucionales. Además, el trabajo presenta una base de
datos amplia y diversa de imágenes de rayos X de diferentes patologías y regiones, que puede servir
como referencia para futuros trabajos en este ámbito. Sin embargo, el trabajo también reconoce algunas
limitaciones importantes, como la calidad y la representatividad de los datos utilizados, que pueden
afectar a la generalización y la robustez de los modelos, la falta de validación y explicabilidad de
los modelos, que dificulta su interpretación y su aplicación clínica, y la escasez de recursos
computacionales y humanos para el análisis y etiquetado de las imágenes, que limita el alcance y la
profundidad del trabajo.


El trabajo plantea algunas líneas de investigación futuras, como el uso de otras arquitecturas y
técnicas de aprendizaje profundo, que puedan mejorar el rendimiento y la eficiencia de los modelos,
el aumento y la diversificación de las bases de datos de imágenes de rayos X, que puedan cubrir más
patologías y más variaciones, la incorporación de otras modalidades de datos, como la tomografía
computarizada, que puedan ofrecer más información y más resolución, la regulación y la evaluación de
los modelos, que puedan garantizar su seguridad y su calidad para su uso clínico, y la generación de
mapas de calor, que puedan indicar las regiones de interés en las imágenes y facilitar el diagnóstico
y el tratamiento.
