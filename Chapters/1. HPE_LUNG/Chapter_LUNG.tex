\section{Detección de patologías pulmonares con rayos-X}

A pesar de diversos esfuerzos para desarrollar métodos basados en aprendizaje automático para el
análisis de imágenes de Rayos-X y Tomografías Computarizadas, aún no están listos para uso clínico.
Las limitaciones incluyen sesgos presentados por el uso de bases de datos pequeñas o recopilaciones
de diversas fuentes sin un tratamiento o normalización adecuados, así como enfoques de detección en
enfermedades específicas, dejando de lado la posible contribución de la inclusión de otras
enfermedades en los modelos. Por ello, el trabajo presentado en este escrito se concentra en
desarrollar un modelo basado en aprendizaje profundo que ataque estos problemas. El modelo
desarrollado está entrenado para la detección de 15 patologías pulmonares, incluyendo \textit{COVID-19}.


El padecimiento por \textit{COVID-19} es una enfermedad contagiosa causada por el \textit{Síndrome
Respiratorio Agudo Severo Coronavirus 2} o \textit{SARS-CoV-2} por sus siglas en inglés
(\textit{Severe Acute Respiratory Syndrome Coronavirus 2}), reportada por primera vez en diciembre
del año 2019 como un nuevo tipo de neumonía viral \cite{huang2020clinical}. Pocos meses después,
en marzo del 2020, el \textit{COVID-19} fue declarado como pandemia a nivel mundial por la
Organización Mundial de la Salud (\textit{WHO}) \cite{world2020director}. Los métodos más eficaces
de detección de \textit{COVID-19} son la prueba clínica de \textit{Reacción en Cadena de Polimerasa con
Transcripción Inversa} (\textit{RT-PCR}), también llamada genéricamente \textit{Molecular Photocopying Test},
pues es usada para amplificar y copiar pequeños segmentos de \textit{DNA} y detectar material genético
de un organismo en específico como el virus \textit{SARS-CoV-2}, y mediante la búsqueda de anticuerpos
desarrollados por el organismo como respuesta a la enfermedad con la \textit{Prueba Rápida de Anticuerpos}
(\textit{RAT}) \cite{Gupta2021, Apra2021, pub.1136450856, LIU2021112817}. Puesto que los anticuerpos
tardan en generarse entre los 10 y 20 días después de la infección \cite{lou2020serology,o2021age,VABRET2020910},
las pruebas tipo \textit{PCR} son preferidas como método de detección temprana. En ausencia de prueba
\textit{PCR}, los pacientes con sintomatología similar a la provocada por \textit{COVID-19} solo pueden
ser diagnosticados con neumonía atípica. Por ello, diversos métodos de análisis de imágenes basados en
técnicas de Inteligencia Artificial han sido desarrollados para la detección de \textit{COVID-19} usando
imágenes de rayos X y tomografías computarizadas. Reportes clínicos indican que imágenes de rayos X y
tomografías computarizadas de pecho pueden mostrar efectos de afectaciones por \textit{COVID-19}. Dichos
efectos pueden ser apreciados en los pulmones incluso en casos donde la prueba \textit{PCR} resulta en
falso negativo \cite{ai2020correlation, wong2020frequency}. Así, la principal motivación es desarrollar
métodos alternativos que ayuden a la detección de \textit{COVID-19} (conjuntamente con otras patologías)
dada la limitada disponibilidad, creciente demanda, costo asociado y la obtención de resultados
inmediatos de la aplicación de técnicas como \textit{PCR} en todo el mundo.


Un práctico y exitoso enfoque en la implementación de \textit{Deep Neural Networks} (Redes Neuronales
Profundas) de dominio
específico \cite{liu2017survey}, basados en técnicas de clasificación, es usar \textit{Deep Transfer Learning},
\cite{long2016unsupervised,oquab2014learning,tan2018survey}.
Esta técnica fue inicialmente desarrollada para implementar modelos de dominio específico en
situaciones en las que se cuenta con una cantidad de datos limitada, pero también se ha visto que
usar técnicas de \textit{Deep Transfer Learning} resulta en un método efectivo para entrenar modelos
de dominio específico aun cuando se tiene suficientes datos. Particularmente, en problemas de
clasificación, el \textit{Deep Transfer Learning} consiste en reutilizar modelos de Aprendizaje Profundo entrenados en problemas de
dominio general en donde los conjuntos de datos son lo suficientemente grandes. Dado que la tarea de
clasificación contiene un gran número de clases y datos, los modelos entrenados pueden generalizar y
extraer mejores características de bajo nivel. Por lo tanto, el uso de \textit{Deep Transfer Learning}
también ha sido efectivo en el desarrollo de modelos específicos para el análisis de imágenes médicas
\cite{deniz2018transfer,litjens2017survey,sufian2020survey}, tales como la detección de
\textit{COVID-19} a partir de imágenes de Rayos X, problema atacado ampliamente por la comunidad
usando Transferencia de Conocimiento en modelos basados en Aprendizaje Profundo
\cite{agrawal2021focuscovid,ai2020correlation,sufian2020survey}.


Un popular método para la detección de \textit{COVID-19} es \textit{CoroNet} \cite{wang2020covid}, un
modelo con arquitectura basada en redes neuronales profundas. \textit{CoroNet} clasifica imágenes de
radiografías de Rayos X en 4 clases: \textit{COVID-19}, neumonía bacteriana, neumonía viral e imágenes
sin ninguno de los tres padecimientos anteriores. \citeauthor{bressem2020comparing} presenta un estudio
sobre cuáles arquitecturas usadas como configuración inicial de la red \textit{backbone}, encargada
principalmente de tareas de extracción de características, son más adecuadas para desarrollar modelos
de clasificación con imágenes radiográficas. Limitan el estudio a la detección de solo 5 patologías:
\textit{cardiomegalia, edema, consolidación, atelectasia y derrame pleural}, encontrando que es mucho más
importante el tamaño de lote que la red usada como \textit{backbone}. Es decir, la mejoría de los
modelos dependía mucho más del proceso de entrenamiento que de los datos y la arquitectura.
En \cite{zhong2021deep} se reporta un ejemplo de clasificador de \textit{COVID-19}, neumonía y
no \textit{COVID-19}. Otro reciente estudio compara redes neuronales para la clasificación de
radiografías de pecho entre \textit{COVID-19} y neumonía \cite{shazia2021comparative}. Para ello,
construyen clasificadores binarios usando redes preentrenadas con la base de datos ImageNet,
considerando los modelos VGG16 y VGG19, DenseNet121, Inception-ResNet-V2, InceptionV3, ResNet50 y
Xception. Con base en los resultados, los modelos basados en DenseNet121 tuvieron el mejor
\textit{accuracy} ($99.48\%$), seguidos muy de cerca por ResNet50 y VGG19 con $99.32\%$ y $99.18\%$,
respectivamente. Los resultados anteriores son muy cercanos a los reportados por
\textit{InstaCovNet-19}, un clasificador binario de \textit{COVID-19} y no \textit{COVID-19}
\cite{gupta2021instacovnet}. Por otro lado, \citeauthor{bassi2021deep} proponen una estrategia de
Transferencia de Conocimiento que toma los modelos DenseNet121 y DenseNet201 (los sufijos 121 y 201
hacen referencia al número de capas del modelo DenseNet) como \textit{backbone}, y posteriormente reemplazan
las neuronas correspondientes a la capa de salida para clasificar radiografías de Rayos X entre
\textit{COVID-19}, neumonía y sin ninguna de las patologías anteriores. De acuerdo con
\citeauthor{shoeibi2020automated}, la mayoría de los modelos de redes neuronales basados en aprendizaje
profundo tienen un \textit{accuracy} dentro del $90\%$ al $100\%$ para clasificación binaria entre
\textit{COVID-19} y no \textit{COVID-19}.


\subsection{Datos}

\subsubsection{CXR8 Dataset}

El \textit{dataset} es extraído de las bases de datos del hospital clínico de investigación \textit{NIH
Clinical Center}, perteneciente al instituto \textit{National Institutes of Health}. Está formado por
alrededor del 60 por ciento de imágenes de rayos X frontales de pecho capturadas en dicho hospital y
es considerado como uno de los conjuntos de datos más representativos. Contiene
112,120 imágenes de Rayos X frontales (\textit{frontal-view}) de un total de 30,805 pacientes únicos. Cada
imagen contiene etiquetas correspondientes a 14 enfermedades (más de una etiqueta puede estar
asociada a una imagen) extraídas usando técnicas de \textit{Natural Language Processing, NLP}
(Procesamiento de Lenguaje Natural) de los reportes médicos realizados por radiólogos asociados
con un \textit{accuracy} mayor al 90 por ciento \cite{8099852}. Las 14 patologías incluidas son las
siguientes: atelectasia (\textit{atelectasis}), consolidación (\textit{consolidation}), infiltración (\textit{infiltration}),
neumotórax (\textit{pneumothorax}), edema (\textit{edema}), enfisema (\textit{emphysema}), fibrosis (\textit{fibrosis}),
efusión (\textit{effusion}), neumonía (\textit{pneumonia}), derrame pleural (\textit{pleural-thickening}), cardiomegalia (\textit{cardiomegaly}),
nódulo (\textit{nodule}), masa (\textit{mass}) y hernia (\textit{hernia}).


\subsubsection{COVIDx Dataset}

Es una base de datos de uso público construida a través de las contribuciones de la comunidad
científica con la finalidad de construir métodos que ayuden a combatir el padecimiento y propagación
de \textit{COVID-19} \cite{Wang2020}. Contiene 13,975 imágenes de Rayos X de un total de 13,870 casos de
pacientes (las cifras pueden variar debido a su constante actualización) recopiladas a través de 5
repositorios de datos: \textit{COVID-19 Image Data Collection} \cite{cohen2020covid19},
\textit{Figure 1 COVID-19 Chest X-ray Dataset Initiative} \cite{figure1_2020covid19},
\textit{ActualMed COVID-19 Chest X-ray Dataset Initiative} \cite{actualmed_2020covid19},
\textit{RSNA Pneumonia Detection Challenge dataset} \cite{rsna_det_challlenge},
\textit{COVID-19 radiography database} \cite{rsna_det_challenge2}. Los conjuntos de datos
contienen casos de pacientes con y sin padecimientos de neumonía causada por \textit{COVID-19}.


\subsubsection{SIIM-FISABIO-RSNA COVID-19 Detection Dataset}

Es un conjunto de datos creado, organizado y liberado por la \textit{Society for Imaging Informatics
in Medicine (SIIM)} en conjunto con la
\textit{Foundation for the Promotion of Health and Biomedical Research of Valencia Region (FISABIO)},
\textit{Medical Imaging Databank of the Valencia Region (BIMCV)} y la
\textit{Radiological Society of North America (RSNA)} \cite{siim_det_challenge} a través de un
concurso en la plataforma \textit{Kaggle}. Esta competición está enfocada en localizar
e identificar anormalidades en radiografías de pecho y clasificarlas en cuatro casos de neumonía
causada por \textit{COVID-19} \cite{00005382-202011000-00004}:

\begin{itemize}
    \item \textit{Typical appearance}: Contiene observaciones típicas de neumonía causada por
        \textit{COVID-19}. Sin embargo, estas pueden presentarse en conjunto con otras infecciones,
        reacciones a medicamentos u otras causas de lesiones agudas en los pulmones.
    \item \textit{Indeterminate appearance}: Presenta observaciones indeterminadas de neumonía
        causada por \textit{COVID-19} que pueden ocurrir en conjunto con una variedad de
        condiciones infecciosas y no infecciosas.
    \item \textit{Atypical appearance}: Observaciones atípicas a las reportadas en neumonía causada
        por \textit{COVID-19}, y otros diagnósticos alternativos deben ser considerados.
    \item \textit{Negative for pneumonia}: No se encuentran observaciones de neumonía causada por
        \textit{COVID-19}. Sin embargo, las radiografías pueden no presentar aún elementos visibles
        en etapas tempranas de neumonía por \textit{COVID-19}.
  \end{itemize}

\subsubsection{Tuberculosis X-ray (TBX11K) dataset}

Es uno de los \textit{datasets} más grandes actualmente con imágenes de Rayos X con Tuberculosis. Este
\textit{dataset} contiene 11,200 imágenes de Rayos X con anotaciones marcando los \textit{bounding boxes} de las
áreas con observaciones de este padecimiento. Se clasifican en 4 distintas categorías:
saludable, Tuberculosis activa (\textit{active TB}), Tuberculosis latente (\textit{latent TB}) y con
padecimientos otros a
Tuberculosis (\textit{unhealthy but non-TB}). Las imágenes son de calidad mucho mayor a la mayoría de los
\textit{datasets} anteriores, con una resolución de 3000x3000 píxeles. \citeauthor{9156613} mencionan que
usar una resolución de 500x500 es suficiente para entrenar modelos de aprendizaje profundo de
detección y clasificación de Tuberculosis.


% \subsection{Métricas}
