\section{Detección de Patologías en Pulmones Mediante Imágenes de Rayos-X}

A pesar de diversos esfuerzos para desarrollar métodos basados en aprendizaje máquina basados en
análisis de imágenes de Rayos-X y Tomografías Computarizadas aún no están listos para uso clínico.
Limitaciones como sesgos debido a bases de datos pequeñas o recopilaciones de diversas fuentes sin
un tratamiento o normalización entre estos, así como enfoques de detección en enfermedades
específicas dejando de lado la posible contribución a los modelos la inclusión de otras enfermedades.
Por ello el trabajo presentado en este escrito se concentra en desarrollar un modelo basado en
aprendizaje profundo atacando estás problemas. El modelo desarrollado es entrenado para la detección
de 15 patologías de pulmones, incluyendo \textit{COVID-19}.

El padecimiento por \textit{COVID-19} es una enfermedad contagiosa causada por el \textit{Síndrome
Respiratorio Agudo Severo Coronavirus 2} o \textit{SARS-CoV-2} por sus siglas en inglés
(\textit{Severe Acute Respiratory Syndrome Coronavirus 2}) reportada por primera vez en diciembre
del año 2019 como un nuevo tipo de pneumonia viral \cite{huang2020clinical}. Pocos meses después,
en marzo del 2020 el \textit{COVID-19} fue declarado como pandemia a nivel mundial por la
Organización Mundial de la Salud (\textit{WHO}) \cite{world2020director}. Los métodos más eficaces
de detección de \textit{COVID-19} son la prueba clínica de Reacción en Cadena de Polimerasa con
Transcripción Inversa
(\textit{RT-PCR}) también llamada genéricamente \textit{molecular photocopying test} pues es usada
para amplificar-copiar pequeños
segmente de \textit{DNA} y detectar material genético de un organismo en específico como el virus
\textit{SARS-CoV-2} y mediante la búsqueda de anticuerpos desarrollados por el organismo como
respuesta a la enfermedad con la Prueba Rápida de Anticuerpos (\textit{RAT})
\cite{Gupta2021, Apra2021, pub.1136450856, LIU2021112817}. Puesto que los
anticuerpos tardan en generarse entre los 10 y 20 días después de la infección
\cite{lou2020serology,o2021age,VABRET2020910}, la pruebas tipo \textit{PCR} es preferida como
método de detección temprana. En la ausencia de prueba \textit{PCR}, los pacientes con sintomatología
similar a la provocada por \textit{COVID-19} solo pueden ser diagnosticados con pneumonia atípica
como padecimiento. Por ello, diversos métodos de análisis de imágenes basados en técnicas
de Inteligencia Artificial han sido desarrollados para la detección de \textit{COVID-19} usando
imágenes de Rayos X y Tomografías Computarizadas. La principal motivación es desarrollar métodos
alternativos que ayuden a la detección de \textit{COVID-19} dada la limitada disponibilidad y
creciente demanda en la aplicación de técnicas como \textit{PCR} en todo el mundo.
