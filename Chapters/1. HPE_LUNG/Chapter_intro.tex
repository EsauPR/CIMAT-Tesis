\chapter{Introduction}
La pandemia de \textit{COVID-19} ha tenido un impacto sin precedentes en la salud y el bienestar de las
personas, así como en el funcionamiento de los sistemas de salud a nivel mundial. La alta demanda
de atención médica por parte de los pacientes infectados ha provocado la saturación de los hospitales,
lo que afecta la calidad y la seguridad de los servicios de salud para otras enfermedades.
Además, la pandemia ha evidenciado la desprotección y la vulnerabilidad de los trabajadores de la
salud, que se enfrentan a condiciones de trabajo precarias, falta de equipo de protección personal
y riesgo de contagio \cite{demanda-hospital-mexico}.


La historia del COVID-19 inicia en el 2019 en Wuhan, China, reportada como un nuevo tipo de neumonía
viral \cite{huang2020clinical} y pocos meses después, en marzo del 2020 el \textit{COVID-19} fue
declarado como pandemia a nivel mundial por la Organización Mundial de la Salud (\textit{WHO})
\cite{world2020director}. Este escenario, fue suficiente para mostrar cuan vulnerables somos ante
eventos de esta magnitud y que hay mucho trabajo por delante. La ciencia y tecnología son nuestras
más grandes herramientas para construir  métodos que puedan apoyar y agilizar el diagnóstico y
tratamiento de las enfermedades pulmonares, incluyendo el COVID-19.

El análisis de imágenes de rayos X y \textit{Tomografías Computarizadas} (TC) es una técnica valiosa
para el diagnóstico de enfermedades pulmonares, incluyendo el COVID-19, que ha causado una grave
crisis sanitaria a nivel mundial. Mediante el análisis de imágenes de rayos X usando modelos de
Inteligencia Artificial se puede ayudar a los médicos a identificar las patologías presentes en las
imágenes, así como a visualizar las regiones de interés que indican la severidad de la enfermedad.
De esta manera, se mejorar la precisión y la rapidez del diagnóstico, lo que puede traducirse en una
mejor atención al paciente y una optimización de los recursos hospitalarios.

Sin embargo, a pesar de los diversos esfuerzos para desarrollar
métodos basados en aprendizaje máquina para la detección automática de COVID-19 y otras patologías
en estas imágenes, aún no están listos para su uso clínico \cite{roberts2021common}. Algunas de las
limitaciones que enfrentan estos métodos son:

\begin{itemize}
    \item Sesgos debido a bases de datos pequeñas o recopilaciones de diversas fuentes sin un
          tratamiento o normalización entre estos, lo que afecta la calidad y la representatividad
          de los datos.
    \item Enfoques de detección en enfermedades específicas dejando de lado la posible contribución
          a los modelos de la inclusión de otras enfermedades, lo que reduce la capacidad de
          diagnóstico múltiple y holístico.
    \item Falta de regulación, validación y explicabilidad de los modelos, lo que dificulta su
          interpretación y su confianza por parte de los médicos y los pacientes.
    \item Escasez de recursos computacionales y humanos para el pre-procesamiento, análisis y
          etiquetado de las imágenes, lo que limita la rapidez y la precisión de los resultados.
\end{itemize}

Estos desafíos requieren de soluciones innovadoras que puedan aprovechar el potencial del aprendizaje
máquina para el diagnóstico de enfermedades pulmonares mediante imágenes de Rayos X y TC, y que al
mismo tiempo puedan superar las barreras técnicas, éticas y sociales que impiden su aplicación clínica.

Hay un trabajo continuo en la comunidad científica para poder aumentar y diversificar las bases de
datos de imágenes de rayos X y TC, asegurar su calidad, su anonimato y su accesibilidad para los
investigadores y los médicos. Así en este trabajo se presenta un modelo basado en aprendizaje profundo
que puede detectar 15 patologías
pulmonares, incluyendo COVID-19, usando imágenes de rayos X de diferentes fuentes y regiones.
El modelo se basa en los transformers, una arquitectura reciente que usa mecanismos de atención para
procesar secuencias de datos, y que ha mostrado resultados sobresalientes en tareas de procesamiento
de lenguaje natural y visión artificial. El modelo se compara con otros modelos basados en redes
convolucionales y transfer learning, y se evalúa usando métricas de clasificación y visualización
de las regiones de interés en las imágenes.


\chapter{Marco Teórico}

\section{Planteamiento del Problema}

La detección de \textit{COVID-19} y otras patologías pulmonares mediante imágenes de rayos X es un
problema relevante y desafiante, que requiere de métodos de inteligencia artificial eficaces y confiables.
El diagnóstico temprano y preciso de estas enfermedades puede mejorar el pronóstico y el tratamiento
de los pacientes, así como reducir el riesgo de contagio y la carga sobre el sistema de salud.

Sin embargo, los métodos existentes basados en redes neuronales profundas tienen limitaciones como
el sesgo de los datos, el enfoque en enfermedades específicas, y la falta de generalización y
explicabilidad.

En este trabajo se propone desarrollar algunos modelos basado en \textit{Aprendizaje Profundo} que
puedan detectar distintas patologías pulmonares, incluyendo COVID-19. En su totalidad se cuenta con
datos de 15 patologías con información obtenida a través de imágenes de rayos X provenientes de
diferentes fuentes y regiones.

Uno de los modelos se basa en los \textit{Transformers}, una arquitectura reciente que usa
\textit{Mecanismos de Atención} para procesar secuencias de datos, y que ha mostrado resultados
sobresalientes en tareas de procesamiento de lenguaje natural y visión artificial.
El modelo se compara con otros modelos basados en \textit{Redes Convolucionales} aplicando
técnicas de \emph{Transfer Learning} (Transferencia de conocimiento en español, en este trabajo
usamos el término en inglés debido a su amplia difusión en la comunidad) en ambos. Se evalúan usando
métricas de clasificación y visualización de las regiones de interés en las imágenes.

El modelo propuesto tiene varias ventajas sobre los métodos existentes. Primero, es capaz de
detectar múltiples patologías pulmonares simultáneamente, lo que permite un diagnóstico más
completo y holístico. Segundo, es robusto a la variabilidad de los datos, ya que puede adaptarse a
imágenes de rayos X de diferentes calidades, resoluciones, ángulos y contrastes.
Tercero, es interpretable, ya que puede generar mapas de calor que indican las zonas más relevantes
para la clasificación, lo que facilita la comprensión y la validación de las predicciones. Es
importante mencionar que los modelos propuestos no pretende reemplazar un análisis profesional
realizado por un médico, sino que busca ser una herramienta de apoyo para el análisis exploratorio
de las radiografías.

Durante la pandemia, era crítico detectar oportunamente el padecimiento y estos modelos ayudan a
acelerar la decisión del médico radiólogo. Además, el modelo puede ser útil para la detección de
otras enfermedades pulmonares que pueden afectar la salud y la calidad de vida de las personas. A
pesar de los esfuerzos en desarrollar diversos métodos de aprendizaje máquina para la detección de
\textit{COVID-19} basados en Rayos X y Tomografías Computarizadas, los resultados obtenidos y
reportados hasta el momento aún no están listos para uso clínico \cite{roberts2021common}.
\citeauthor{shuja2021covid} presenta una relación de los conjuntos de datos abiertos (datasets) de
\textit{COVID-19} categorizándolos por tipo (imágenes biomédicas,
datos textuales y de audio), aplicaciones y métodos aplicados de \textit{IA}, \textit{Big Data} y
estadísticos. Sin embargo, las imágenes en los datasets mencionados son reducidos y limitados a
regiones específicas alrededor del mundo. Por otro lado, \citeauthor{greenspan2020position} mencionan
que la mayoría de los modelos reportados fueron probados en bajo en esquema de diagnóstico bastante
estrecho, puesto que los modelos deberían ser capaces de detectar \textit{COVID-19} en conjunto
con una amplia variedad de patologías. De acuerdo a \citeauthor{roberts2021common} los fallos comunes
son, entre otros, el sesgo en datasets pequeños o datasets no normalizados recolectados de una larga
variedad de fuentes. \citeauthor{roberts2021common} también argumenta la importancia de desarrollar
modelos no solo para la clasificación binaria de \textit{COVID-19}, sino además poder distinguirlo
de otros tipos de neumonías virales y bacteriales. El trabajo descrito a continuación se centra en
atacar el problema en los términos anteriores, no solamente haciendo la distinción entre
\textit{COVID-19} y otros tipos de neumonías sino a través de diversas enfermedades en afán de ayudar
a clínicos en el diagnóstico de otras patologías mas allá de neumonía.
