% 07-Conclusiones_TrabajoFuturo_Presentacion_Tesis.tex
% Diapositivas de la sección Conclusiones y Trabajo Futuro

\section{Conclusiones y Trabajo Futuro}

\begin{frame}
\frametitle{Conclusiones}
\begin{itemize}
    \item Se desarrollaron dos modelos de aprendizaje profundo para el diagnóstico de 15 patologías pulmonares, incluyendo COVID-19, a partir de imágenes de rayos X.
    \item Ambos modelos muestran rendimiento comparable o superior al estado del arte, especialmente en la detección de COVID-19.
    \item El uso de Transfer Learning y arquitecturas como ResNet50 y Vision Transformer permite extender fácilmente la detección a nuevas patologías.
\end{itemize}
\end{frame}

\begin{frame}
\frametitle{Ventajas y Limitaciones}
\begin{itemize}
    \item Modelos robustos y precisos, útiles como herramienta de apoyo para radiólogos.
    \item Facilidad de extensión a otras enfermedades pulmonares, como la tuberculosis.
    \item Limitaciones: calidad y representatividad de los datos, falta de validación clínica, y recursos computacionales limitados.
\end{itemize}
\end{frame}

\begin{frame}
\frametitle{Trabajo Futuro}
\begin{itemize}
    \item Mejorar la calidad y diversidad de las bases de datos de imágenes.
    \item Explorar nuevas arquitecturas y técnicas de aprendizaje profundo.
    \item Incorporar otras modalidades de imagen, como tomografía computarizada.
    \item Desarrollar herramientas visuales y mapas de calor para facilitar el diagnóstico.
    \item Validación clínica y regulación de los modelos para su uso real.
\end{itemize}
\end{frame}
