% 06-Conclusiones_TrabajoFuturo_Presentacion_Tesis.tex
% Diapositivas de la sección Conclusiones y Trabajo Futuro

\section{Conclusiones y Trabajo Futuro}

\begin{frame}
\frametitle{Logros Principales}
\begin{itemize}
    \item \textbf{Desarrollo exitoso} de dos modelos de aprendizaje profundo para 15 patologías pulmonares
    \begin{itemize}
        \item ResNet50: AUC-ROC Global-15 = 0.852 (nuevo estándar incluyendo COVID-19)
        \item ViT: AUC-ROC Global-15 = 0.791 (alto rendimiento en COVID-19)
    \end{itemize}
    \item \textbf{Desempeño destacado en COVID-19}
    \begin{itemize}
        \item ResNet50: AUC-ROC = 0.991, F1-Score = 0.799
        \item ViT: AUC-ROC = 0.982, F1-Score = 0.801
    \end{itemize}
    \item \textbf{Superación del estado del arte} en múltiples patologías
    \item \textbf{Capacidad de extensión} demostrada con tuberculosis
\end{itemize}
\end{frame}

\begin{frame}
\frametitle{Contribuciones Científicas}
\begin{itemize}
    \item \textbf{Introducción de Vision Transformers} en análisis de imágenes médicas
    \begin{itemize}
        \item Demostración de viabilidad vs CNNs tradicionales
        \item Requiere mayor ajuste pero ofrece atención global
    \end{itemize}
    \item \textbf{Estrategia de Transfer Learning} de tres etapas
    \begin{itemize}
        \item Entrenamiento inicial (backbone congelado)
        \item Fine-tuning (últimas capas)
        \item Full-tuning (modelo completo)
    \end{itemize}
    \item \textbf{Preservación del rendimiento} en patologías originales
    \item \textbf{Incorporación exitosa} de COVID-19 sin degradación
\end{itemize}
\end{frame}

\begin{frame}
\frametitle{Capacidad de Extensión Demostrada}
\begin{itemize}
    \item \textbf{Extensión a tuberculosis} exitosa
    \begin{itemize}
        \item F1-Score: 0.707
        \item Accuracy: 0.846
        \item Verdaderos positivos: 388/488 casos
    \end{itemize}
    \item \textbf{Comparabilidad} con métodos específicos para tuberculosis
    \item \textbf{Plataforma versátil} para futuras patologías
    \item \textbf{Proceso de extensión} simple y eficiente
    \begin{itemize}
        \item Reutilización del backbone pre-entrenado
        \item Adición de rama clasificadora específica
        \item Entrenamiento solo de nuevas capas
    \end{itemize}
\end{itemize}
\end{frame}


\begin{frame}
\frametitle{Limitaciones Identificadas}
\begin{itemize}
    \item \textbf{Calidad y representatividad de datos}
    \begin{itemize}
        \item Diversidad geográfica limitada
        \item Variabilidad en dispositivos de adquisición
        \item Posibles sesgos en etiquetado
    \end{itemize}
    \item \textbf{Falta de validación clínica}
    \begin{itemize}
        \item No probado en entornos hospitalarios reales
        \item Ausencia de estudios prospectivos
        \item Necesidad de certificación médica
    \end{itemize}
    \item \textbf{Recursos computacionales}
    \begin{itemize}
        \item ViT limitado a 384x384 píxeles
        \item Consumo de memoria alto
        \item Tiempo de entrenamiento extenso
    \end{itemize}
\end{itemize}
\end{frame}

\begin{frame}
\frametitle{Líneas de Trabajo Futuro}
\begin{itemize}
    \item \textbf{Exploración de nuevas arquitecturas}
    \begin{itemize}
        \item EfficientNet y DenseNet avanzados
        \item Variantes mejoradas de Vision Transformers
        \item Arquitecturas híbridas CNN-Transformer
        \item Arquitecturas de atención más sofisticadas
    \end{itemize}
    \item \textbf{Datos y Validación}
    \begin{itemize}
        \item Mejora de bases de datos. Diversificación geográfica y demográfica, inclusion de más patologías, datos de alta calidad y anotación precisa.
        \item Validación clínica rigurosa.
        \item Regulación y certificación para uso clínico.
    \end{itemize}
\end{itemize}
\end{frame}

\begin{frame}
\frametitle{Líneas de Trabajo Futuro - Multimodalidad}
\begin{itemize}
    \item \textbf{Integración de múltiples modalidades}
    \begin{itemize}
        \item Tomografía computarizada (CT)
        \item Resonancia magnética (MRI)
        \item Ultrasonido pulmonar
    \end{itemize}
    \item \textbf{Modelos multimodales}
    \begin{itemize}
        \item Fusión de información de diferentes fuentes
        \item Arquitecturas que procesen múltiples tipos de imagen
        \item Técnicas de alineación multimodal
    \end{itemize}
    \item \textbf{Datos clínicos adicionales}
    \begin{itemize}
        \item Historiales médicos
        \item Síntomas del paciente
        \item Resultados de laboratorio
    \end{itemize}
\end{itemize}
\end{frame}

\begin{frame}
\frametitle{Líneas de Trabajo Futuro - Herramientas Visuales}
\begin{itemize}
    % \item \textbf{Mejora de técnicas de interpretabilidad}
    % \begin{itemize}
    %     \item GradCAM avanzado
    %     \item SHAP (SHapley Additive exPlanations)
    %     \item LIME (Local Interpretable Model-agnostic Explanations)
    % \end{itemize}
    \item \textbf{Interfaces de usuario clínicas}
    \begin{itemize}
        \item Dashboards para radiólogos
        \item Sistemas de reporte automatizado
        \item Integración con PACS (Picture Archiving and Communication System)
    \end{itemize}
    \item \textbf{Visualizaciones interactivas}
    \begin{itemize}
        \item Exploración de regiones de interés
        \item Comparación de múltiples modelos
    \end{itemize}
\end{itemize}
\end{frame}

% \begin{frame}
% \frametitle{Impacto Clínico y Social}
% \begin{itemize}
%     \item \textbf{Acceso a diagnóstico especializado}
%     \begin{itemize}
%         \item Regiones con recursos limitados
%         \item Reducción de tiempos de espera
%         \item Mejora de eficiencia del sistema de salud
%     \end{itemize}
%     \item \textbf{Herramientas de triaje}
%     \begin{itemize}
%         \item Clasificación automática de urgencias
%         \item Priorización de casos críticos
%         \item Apoyo en emergencias médicas
%     \end{itemize}
%     \item \textbf{Adaptabilidad epidemiológica}
%     \begin{itemize}
%         \item Extensión a nuevas patologías emergentes
%         \item Adaptación a contextos locales
%         \item Respuesta rápida a crisis sanitarias
%     \end{itemize}
% \end{itemize}
% \end{frame}

\begin{frame}
\frametitle{Conclusiones Finales}
\begin{itemize}
    \item \textbf{Nuevos estándares establecidos}
    \begin{itemize}
        \item Rendimiento superior al estado del arte
        \item Viabilidad de Vision Transformers en medicina
        \item Capacidad de extensión demostrada
    \end{itemize}
    \item \textbf{Contribución al campo}
    \begin{itemize}
        \item Metodología reproducible con base de datos diversa
        \item Plataforma para futuras investigaciones
    \end{itemize}
    \item \textbf{Impacto potencial}
    \begin{itemize}
        \item Herramientas valiosas para diagnóstico
        \item Mejora y avance a atención médica asistida por computadora
    \end{itemize}
    \item \textbf{Próximos pasos}
    \begin{itemize}
        \item Validación clínica rigurosa
        \item Desarrollo de herramientas visuales
        \item Integración en práctica médica como herramienta de apoyo
    \end{itemize}
\end{itemize}
\end{frame}
