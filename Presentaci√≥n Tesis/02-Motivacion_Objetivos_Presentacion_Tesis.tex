% 02-Motivacion_Objetivos_Presentacion_Tesis.tex
% Diapositivas de la sección Motivación y Objetivos

\section{Motivación y Objetivos}

\begin{frame}
\frametitle{Motivación: Problema Relevante}
\begin{itemize}
    \item La detección de COVID-19 y otras patologías pulmonares mediante imágenes de rayos X es un reto relevante y desafiante.
    \item El diagnóstico temprano y preciso mejora el pronóstico y tratamiento de los pacientes.
    \item Reduce el riesgo de contagio y la carga sobre el sistema de salud.
\end{itemize}
\end{frame}

\begin{frame}
\frametitle{Limitaciones de los Métodos Existentes}
\begin{itemize}
    \item Sesgo de los datos y enfoque en enfermedades específicas.
    \item Falta de generalización y explicabilidad en los modelos actuales.
    \item Necesidad de métodos robustos y confiables para el diagnóstico automatizado.
    \item Escasez de recursos para el procesamiento y etiquetado de imágenes.
\end{itemize}
\end{frame}

\begin{frame}
\frametitle{Propuesta de la Tesis}
\begin{itemize}
    \item Desarrollo de modelos de Deep Learning capaces de detectar múltiples patologías pulmonares, incluyendo COVID-19.
    \item Uso de imágenes de rayos X provenientes de diferentes fuentes y regiones.
    \item Comparación de modelos basados en Transformers y CNNs, ambos con técnicas de Transfer Learning.
    \item Evaluación mediante métricas de clasificación y visualización de regiones de interés.
\end{itemize}
\end{frame}

\begin{frame}
\frametitle{Ventajas del Modelo Propuesto}
\begin{itemize}
    \item \textbf{Diagnóstico múltiple y holístico}: detección simultánea de varias patologías.
    \item \textbf{Robustez ante variabilidad de datos}: diferentes calidades, resoluciones y contrastes.
    \item \textbf{Interpretabilidad}: generación de mapas de calor para validar las predicciones.
    \item \textbf{Herramienta de apoyo}: para el análisis exploratorio, no reemplazo del médico.
\end{itemize}
\end{frame}

\begin{frame}
\frametitle{Objetivos Específicos}
\begin{itemize}
    \item Desarrollar modelos de clasificación multiclase para 15 patologías pulmonares.
    \item Implementar y comparar arquitecturas basadas en Transformers y CNNs.
    \item Evaluar la robustez de los modelos ante diferentes fuentes de datos.
    \item Generar visualizaciones interpretables para validación clínica.
\end{itemize}
\end{frame}
