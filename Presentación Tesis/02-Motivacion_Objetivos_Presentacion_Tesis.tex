% 02-Motivacion_Objetivos_Presentacion_Tesis.tex
% Diapositivas de la sección Motivación y Objetivos

\section{Motivación y Objetivos}

\begin{frame}
\frametitle{Motivación: Problema Relevante}
\begin{itemize}
    \item La detección de COVID-19 y otras patologías pulmonares mediante imágenes de rayos X es un reto relevante y desafiante.
    \item El diagnóstico temprano y preciso mejora el pronóstico y tratamiento de los pacientes.
    \item Reduce el riesgo de contagio y la carga sobre el sistema de salud.
\end{itemize}
\end{frame}

\begin{frame}
\frametitle{Limitaciones de los Métodos Existentes}
\begin{itemize}
    \item Sesgo de los datos y enfoque en enfermedades específicas.
    \item Falta de generalización y explicabilidad en los modelos actuales.
    \item Necesidad de métodos robustos y confiables para el diagnóstico automatizado.
\end{itemize}
\end{frame}

\begin{frame}
\frametitle{Propuesta de la Tesis}
\begin{itemize}
    \item Desarrollar modelos de Deep Learning capaces de detectar múltiples patologías pulmonares, incluyendo COVID-19.
    \item Uso de imágenes de rayos X provenientes de diferentes fuentes y regiones.
    \item Comparar modelos basados en Transformers y CNNs, ambos con técnicas de Transfer Learning.
\end{itemize}
\end{frame}

\begin{frame}
\frametitle{Ventajas del Modelo Propuesto}
\begin{itemize}
    \item Diagnóstico múltiple y holístico: detección simultánea de varias patologías.
    \item Robustez ante variabilidad de datos: diferentes calidades, resoluciones y contrastes.
    \item Interpretabilidad: generación de mapas de calor para validar las predicciones.
    \item Herramienta de apoyo para el análisis exploratorio, no reemplazo del médico.
\end{itemize}
\end{frame}
